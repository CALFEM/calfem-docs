%% Generated by Sphinx.
\def\sphinxdocclass{report}
\documentclass[a5paper,10pt,english]{sphinxmanual}
\ifdefined\pdfpxdimen
   \let\sphinxpxdimen\pdfpxdimen\else\newdimen\sphinxpxdimen
\fi \sphinxpxdimen=.75bp\relax
\ifdefined\pdfimageresolution
    \pdfimageresolution= \numexpr \dimexpr1in\relax/\sphinxpxdimen\relax
\fi
%% let collapsible pdf bookmarks panel have high depth per default
\PassOptionsToPackage{bookmarksdepth=5}{hyperref}

\PassOptionsToPackage{booktabs}{sphinx}
\PassOptionsToPackage{colorrows}{sphinx}

\PassOptionsToPackage{warn}{textcomp}
\usepackage[utf8]{inputenc}
\ifdefined\DeclareUnicodeCharacter
% support both utf8 and utf8x syntaxes
  \ifdefined\DeclareUnicodeCharacterAsOptional
    \def\sphinxDUC#1{\DeclareUnicodeCharacter{"#1}}
  \else
    \let\sphinxDUC\DeclareUnicodeCharacter
  \fi
  \sphinxDUC{00A0}{\nobreakspace}
  \sphinxDUC{2500}{\sphinxunichar{2500}}
  \sphinxDUC{2502}{\sphinxunichar{2502}}
  \sphinxDUC{2514}{\sphinxunichar{2514}}
  \sphinxDUC{251C}{\sphinxunichar{251C}}
  \sphinxDUC{2572}{\textbackslash}
\fi
\usepackage{cmap}
\usepackage[T1]{fontenc}
\usepackage{amsmath,amssymb,amstext}
\usepackage{babel}



\usepackage{tgtermes}
\usepackage{tgheros}
\renewcommand{\ttdefault}{txtt}



\usepackage[Bjarne]{fncychap}
\usepackage{sphinx}
\sphinxsetup{
        hmargin=1.5cm, 
        vmargin=2cm,
    }
\fvset{fontsize=auto}
\usepackage{geometry}


% Include hyperref last.
\usepackage{hyperref}
% Fix anchor placement for figures with captions.
\usepackage{hypcap}% it must be loaded after hyperref.
% Set up styles of URL: it should be placed after hyperref.
\urlstyle{same}

\addto\captionsenglish{\renewcommand{\contentsname}{Contents:}}

\usepackage{sphinxmessages}
\setcounter{tocdepth}{1}



\title{CALFEM \sphinxhyphen{} A Finite Element Toolbox}
\date{May 25, 2025}
\release{0.1}
\author{...\@{}}
\newcommand{\sphinxlogo}{\vbox{}}
\renewcommand{\releasename}{Release}
\makeindex
\begin{document}

\ifdefined\shorthandoff
  \ifnum\catcode`\=\string=\active\shorthandoff{=}\fi
  \ifnum\catcode`\"=\active\shorthandoff{"}\fi
\fi

\pagestyle{empty}
\sphinxmaketitle
\pagestyle{plain}
\sphinxtableofcontents
\pagestyle{normal}
\phantomsection\label{\detokenize{index::doc}}


\sphinxAtStartPar
Welcome to the documentation page for CALFEM for Python and CALFEM for MATLAB. On this page you will find examples of how to use CALFEM as well as reference documentation for the different modules in the CALFEM toolbox.

\sphinxstepscope


\chapter{Introduction}
\label{\detokenize{introduction:introduction}}\label{\detokenize{introduction::doc}}
\sphinxAtStartPar
The computer program CALFEM is a MATLAB toolbox for finite element applications. This manual concerns mainly the finite element functions, but it also contains descriptions of some often\sphinxhyphen{}used MATLAB functions.

\sphinxAtStartPar
The finite element analysis can be carried out either interactively or in a batch\sphinxhyphen{}oriented fashion. In the interactive mode, the functions are evaluated one by one in the MATLAB command window. In the batch\sphinxhyphen{}oriented mode, a sequence of functions is written in a file named \sphinxtitleref{.m} file and evaluated by writing the file name in the command window. The batch\sphinxhyphen{}oriented mode is a more flexible way of performing finite element analysis because the \sphinxtitleref{.m} file can be written in an ordinary editor. This way of using CALFEM is recommended because it gives a structured organization of the functions. Changes and reruns are also easily executed in the batch\sphinxhyphen{}oriented mode.

\sphinxAtStartPar
A command line typically consists of functions for vector and matrix operations, calls to functions in the CALFEM finite element library, or commands for workspace operations. An example of a command line for a matrix operation is:
\begin{equation*}
\begin{split}C = A + B'\end{split}
\end{equation*}
\sphinxAtStartPar
where two matrices \(A\) and \(B'\) are added together, and the result is stored in matrix \(C\). The matrix \(B'\) is the transpose of  \(B\).

\sphinxAtStartPar
An example of a call to the element library is:
\begin{equation*}
\begin{split}Ke = spring1e(k)\end{split}
\end{equation*}
\sphinxAtStartPar
where the two\sphinxhyphen{}by\sphinxhyphen{}two element stiffness matrix \(K^e\) is computed for a spring element with spring stiffness \(k\), and is stored in the variable \sphinxtitleref{Ke}.  The input argument is given within parentheses \sphinxtitleref{( )} after the name of the function. Some functions have multiple input arguments and/or multiple output arguments. For example:
\begin{equation*}
\begin{split}[lambda, X] = eigen(K, M)\end{split}
\end{equation*}
\sphinxAtStartPar
computes the eigenvalues and eigenvectors of a pair of matrices
\(K\) and \(M\). The output variables \sphinxhyphen{} the eigenvalues stored in the vector \(\lambda\) and the corresponding eigenvectors stored in the matrix \(X\) \sphinxhyphen{} are surrounded by brackets \sphinxtitleref{{[} {]}} and separated by commas. The input arguments are given inside the parentheses and also separated by commas.

\sphinxAtStartPar
The statement:

\begin{sphinxVerbatim}[commandchars=\\\{\}]
\PYG{n+nb}{help}\PYG{+w}{ }\PYG{k}{function}
\end{sphinxVerbatim}

\sphinxAtStartPar
provides information about the purpose and syntax for the specified function.

\sphinxAtStartPar
The available functions are organized in groups as follows.  Each group is described in a separate chapter.

\sphinxstepscope


\chapter{General purpose functions}
\label{\detokenize{general_purpose_functions:general-purpose-functions}}\label{\detokenize{general_purpose_functions::doc}}
\sphinxstepscope


\chapter{Matrix functions}
\label{\detokenize{matrix_functions:matrix-functions}}\label{\detokenize{matrix_functions::doc}}
\sphinxstepscope


\chapter{Material functions}
\label{\detokenize{material_functions:material-functions}}\label{\detokenize{material_functions:id1}}\label{\detokenize{material_functions::doc}}
\sphinxAtStartPar
The group of material functions comprises functions for constitutive models. The available models can treat linear elastic and isotropic hardening von Mises material.


\section{hooke}
\label{\detokenize{material_functions:hooke}}\begin{quote}\begin{description}
\sphinxlineitem{Purpose}
\sphinxAtStartPar
Compute material matrix for a linear elastic and isotropic material.

\sphinxlineitem{Syntax}
\begin{sphinxVerbatim}[commandchars=\\\{\}]
\PYG{n}{D}\PYG{+w}{ }\PYG{p}{=}\PYG{+w}{ }\PYG{n}{hooke}\PYG{p}{(}\PYG{n}{ptype}\PYG{p}{,}\PYG{+w}{ }\PYG{n}{E}\PYG{p}{,}\PYG{+w}{ }\PYG{n}{v}\PYG{p}{)}
\end{sphinxVerbatim}

\sphinxlineitem{Description}
\sphinxAtStartPar
The function \sphinxcode{\sphinxupquote{hooke}} computes the material matrix \(\mathbf{D}\) for a linear elastic and isotropic material.

\sphinxAtStartPar
The variable \(\mathrm{ptype}\) is used to define the type of analysis:
\begin{equation*}
\begin{split}\mathrm{ptype} = \left\{
    \begin{array}{ll}
        1 & \text{plane stress} \\
        2 & \text{plane strain} \\
        3 & \text{axisymmetry} \\
        4 & \text{three dimensional analysis}
    \end{array}
\right.\end{split}
\end{equation*}
\sphinxAtStartPar
The material parameters \(E\) and \(\nu\) define the modulus of elasticity and the Poisson’s ratio, respectively.

\sphinxlineitem{Theory}
\sphinxAtStartPar
For plane stress (\(\mathrm{ptype}=1\)), \(\mathbf{D}\) is formed as
\begin{equation*}
\begin{split}\mathbf{D} = \frac{E}{1-\nu^2}
\begin{bmatrix}
    1 & \nu & 0 \\
    \nu & 1 & 0 \\
    0 & 0 & \frac{1-\nu}{2}
\end{bmatrix}\end{split}
\end{equation*}
\sphinxAtStartPar
For plane strain (\(\mathrm{ptype}=2\)) and axisymmetry (\(\mathrm{ptype}=3\)), \(\mathbf{D}\) is formed as
\begin{equation*}
\begin{split}\mathbf{D} = \frac{E}{(1+\nu)(1-2\nu)}
\begin{bmatrix}
    1-\nu & \nu & \nu & 0 \\
    \nu & 1-\nu & \nu & 0 \\
    \nu & \nu & 1-\nu & 0 \\
    0 & 0 & 0 & \frac{1}{2}(1-2\nu)
\end{bmatrix}\end{split}
\end{equation*}
\sphinxAtStartPar
For the three dimensional case (\(\mathrm{ptype}=4\)), \(\mathbf{D}\) is formed as
\begin{equation*}
\begin{split}\mathbf{D} = \frac{E}{(1+\nu)(1-2\nu)}
\begin{bmatrix}
    1-\nu & \nu & \nu & 0 & 0 & 0 \\
    \nu & 1-\nu & \nu & 0 & 0 & 0 \\
    \nu & \nu & 1-\nu & 0 & 0 & 0 \\
    0 & 0 & 0 & \frac{1}{2}(1-2\nu) & 0 & 0 \\
    0 & 0 & 0 & 0 & \frac{1}{2}(1-2\nu) & 0 \\
    0 & 0 & 0 & 0 & 0 & \frac{1}{2}(1-2\nu)
\end{bmatrix}\end{split}
\end{equation*}
\end{description}\end{quote}


\section{mises}
\label{\detokenize{material_functions:mises}}\begin{quote}\begin{description}
\sphinxlineitem{Purpose}
\sphinxAtStartPar
Compute stresses and plastic strains for an elasto\sphinxhyphen{}plastic isotropic hardening von Mises material.

\sphinxlineitem{Syntax}
\begin{sphinxVerbatim}[commandchars=\\\{\}]
\PYG{p}{[}\PYG{n}{es}\PYG{p}{,}\PYG{+w}{ }\PYG{n}{deps}\PYG{p}{,}\PYG{+w}{ }\PYG{n}{st}\PYG{p}{]}\PYG{+w}{ }\PYG{p}{=}\PYG{+w}{ }\PYG{n}{mises}\PYG{p}{(}\PYG{n}{ptype}\PYG{p}{,}\PYG{+w}{ }\PYG{n}{mp}\PYG{p}{,}\PYG{+w}{ }\PYG{n}{est}\PYG{p}{,}\PYG{+w}{ }\PYG{n}{st}\PYG{p}{)}
\end{sphinxVerbatim}

\sphinxlineitem{Description}
\sphinxAtStartPar
The \(\texttt{mises}\) function computes updated stresses (\(\texttt{es}\)), plastic strain increments (\(\texttt{deps}\)), and state variables (\(\texttt{st}\)) for an elasto\sphinxhyphen{}plastic isotropic hardening von Mises material.

\sphinxAtStartPar
The input variable \(\texttt{ptype}\) defines the type of analysis, see also \(\texttt{hooke}\). The vector \(\texttt{mp}\) contains the material constants:

\sphinxAtStartPar
\(\mathbf{mp} = [\, E\;\nu\;h\,]\)

\sphinxAtStartPar
where \(E\) is the modulus of elasticity, \(\nu\) is the Poisson’s ratio, and \(h\) is the plastic modulus.

\sphinxAtStartPar
The input matrix \(\texttt{est}\) contains trial stresses obtained by using the elastic material matrix \(\texttt{D}\) in \(\texttt{plants}\) or a similar \(\texttt{s}\)\sphinxhyphen{}function. The input vector \(\texttt{st}\) contains the state parameters:

\sphinxAtStartPar
\(\mathbf{st} = [\, yi\;\sigma_y\;\varepsilon_{eff}^p\,]\)

\sphinxAtStartPar
at the beginning of the step. The scalar \(yi\) indicates whether the material behaviour is elasto\sphinxhyphen{}plastic (\(yi = 1\)) or elastic (\(yi = 0\)). The current yield stress is denoted by \(\sigma_y\) and the effective plastic strain by \(\varepsilon_{eff}^p\).

\sphinxAtStartPar
The output variables \(\texttt{es}\) and \(\texttt{st}\) contain updated values obtained by integration of the constitutive equations over the actual displacement step. The increments of the plastic strains are stored in the vector \(\texttt{deps}\).

\sphinxAtStartPar
If \(\texttt{es}\) and \(\texttt{st}\) contain more than one row, then every row will be treated by the command.

\sphinxlineitem{Note}
\sphinxAtStartPar
It is not necessary to check whether the material behaviour is elastic or elasto\sphinxhyphen{}plastic; this test is performed by the function. The computation is based on an Euler\sphinxhyphen{}Backward method, i.e., the radial return method.

\sphinxAtStartPar
Only the cases \(\texttt{ptype} = 2, 3, 4\) are implemented.

\end{description}\end{quote}


\section{dmises}
\label{\detokenize{material_functions:dmises}}\begin{quote}\begin{description}
\sphinxlineitem{Purpose}
\sphinxAtStartPar
Form the elasto\sphinxhyphen{}plastic continuum matrix for an isotropic hardening von Mises material.

\sphinxlineitem{Syntax}
\begin{sphinxVerbatim}[commandchars=\\\{\}]
\PYG{n}{D}\PYG{+w}{ }\PYG{p}{=}\PYG{+w}{ }\PYG{n}{dmises}\PYG{p}{(}\PYG{n}{ptype}\PYG{p}{,}\PYG{+w}{ }\PYG{n}{mp}\PYG{p}{,}\PYG{+w}{ }\PYG{n}{es}\PYG{p}{,}\PYG{+w}{ }\PYG{n}{st}\PYG{p}{)}
\end{sphinxVerbatim}

\sphinxlineitem{Description}
\sphinxAtStartPar
\(\text{dmises}\) forms the elasto\sphinxhyphen{}plastic continuum matrix for an isotropic hardening von Mises material.

\sphinxAtStartPar
The input variable \(\text{ptype}\) is used to define the type of analysis, cf. \(\text{hooke}\).

\sphinxAtStartPar
The vector \(\mathbf{mp}\) contains the material constants:
\begin{equation*}
\begin{split}\mathbf{mp} = [\, E\;\nu\;h\,]\end{split}
\end{equation*}
\sphinxAtStartPar
where \(E\) is the modulus of elasticity, \(\nu\) is the Poisson’s ratio, and \(h\) is the plastic modulus.

\sphinxAtStartPar
The matrix \(\text{es}\) contains current stresses obtained from \(\text{plants}\) or some similar \(s\)\sphinxhyphen{}function, and the vector \(\text{st}\) contains the current state parameters:
\begin{equation*}
\begin{split}\mathbf{st} = [\, yi\;\sigma_y\;\varepsilon_{eff}^p\,]\end{split}
\end{equation*}
\sphinxAtStartPar
where \(yi = 1\) if the material behaviour is elasto\sphinxhyphen{}plastic, and \(yi = 0\) if the material behaviour is elastic. The current yield stress is denoted by \(\sigma_y\), and the current effective plastic strain by \(\varepsilon_{eff}^p\).

\sphinxlineitem{Note}
\sphinxAtStartPar
Only the case \(\text{ptype} = 2\) is implemented.

\end{description}\end{quote}

\sphinxstepscope


\chapter{Element functions}
\label{\detokenize{element_functions:element-functions}}\label{\detokenize{element_functions::doc}}
\sphinxAtStartPar
The group of element functions contains functions for computation of element
matrices and element forces for different element types. The element functions
have been divided into the following groups:
\begin{itemize}
\item {} 
\sphinxAtStartPar
Spring elements

\item {} 
\sphinxAtStartPar
Bar elements

\item {} 
\sphinxAtStartPar
Heat flow elements

\item {} 
\sphinxAtStartPar
Solid elements

\item {} 
\sphinxAtStartPar
Beam elements

\item {} 
\sphinxAtStartPar
Plate elements

\end{itemize}

\sphinxAtStartPar
For each element type, there is a function for computation of the element stiffness matrix \(K^e\). For most of the elements, an element load vector \(f^e\) can also be computed. These functions are identified by their last letter \sphinxtitleref{\sphinxhyphen{}e}.

\sphinxAtStartPar
Using the function \sphinxtitleref{assem()}, the element stiffness matrices and element load vectors are assembled into a global stiffness matrix \(K\) and a load vector \(f\). Unknown nodal values of temperatures or displacements \(a\) are computed by solving the system of equations \(Ka = f\) using the function \sphinxtitleref{solveq()}. A vector of nodal values of temperatures or displacements for a specific element is formed by the function \sphinxtitleref{extract()}.

\sphinxAtStartPar
When the element nodal values have been computed, the element flux or element stresses can be calculated using functions specific to the element type concerned. These functions are identified by their last letter \sphinxtitleref{\sphinxhyphen{}s}.

\sphinxAtStartPar
For some elements, a function for computing the internal force vector is also available. These functions are identified by their last letter \sphinxtitleref{\sphinxhyphen{}f}.

\sphinxstepscope


\chapter{Spring element functions}
\label{\detokenize{spring_functions:spring-element-functions}}\label{\detokenize{spring_functions::doc}}
\sphinxAtStartPar
The spring element, shown below, can be used for the analysis of
one\sphinxhyphen{}dimensional spring systems and for a variety of analogous physical problems.

\begin{figure}[htbp]
\centering
\capstart

\noindent\sphinxincludegraphics[width=0.700\linewidth]{{SPRING1}.png}
\caption{Spring element}\label{\detokenize{spring_functions:id1}}\end{figure}

\sphinxAtStartPar
Quantities corresponding to the variables
of the spring are listed in Table 1.


\begin{savenotes}\sphinxattablestart
\sphinxthistablewithglobalstyle
\centering
\sphinxcapstartof{table}
\sphinxthecaptionisattop
\sphinxcaption{Analogous quantities}\label{\detokenize{spring_functions:id2}}\label{\detokenize{spring_functions:tanalog}}
\sphinxaftertopcaption
\begin{tabular}[t]{\X{20}{100}\X{20}{100}\X{20}{100}\X{20}{100}\X{20}{100}}
\sphinxtoprule
\sphinxstyletheadfamily 
\sphinxAtStartPar
Problem type
&\sphinxstyletheadfamily 
\sphinxAtStartPar
Spring stiffness
&\sphinxstyletheadfamily 
\sphinxAtStartPar
Nodal displacement
&\sphinxstyletheadfamily 
\sphinxAtStartPar
Element force
&\sphinxstyletheadfamily 
\sphinxAtStartPar
Spring force
\\
\sphinxmidrule
\sphinxtableatstartofbodyhook
\sphinxAtStartPar
Spring
&
\sphinxAtStartPar
\(k\)
&
\sphinxAtStartPar
\(u\)
&
\sphinxAtStartPar
\(P\)
&
\sphinxAtStartPar
\(N\)
\\
\sphinxhline
\sphinxAtStartPar
Bar
&
\sphinxAtStartPar
\(\frac{EA}{L}\)
&
\sphinxAtStartPar
\(u\)
&
\sphinxAtStartPar
\(P\)
&
\sphinxAtStartPar
\(N\)
\\
\sphinxhline
\sphinxAtStartPar
Thermal conduction
&
\sphinxAtStartPar
\(\frac{\lambda A}{L}\)
&
\sphinxAtStartPar
\(T\)
&
\sphinxAtStartPar
\(\bar{H}\)
&
\sphinxAtStartPar
\(H\)
\\
\sphinxhline
\sphinxAtStartPar
Diffusion
&
\sphinxAtStartPar
\(\frac{D A}{L}\)
&
\sphinxAtStartPar
\(c\)
&
\sphinxAtStartPar
\(\bar{H}\)
&
\sphinxAtStartPar
\(H\)
\\
\sphinxhline
\sphinxAtStartPar
Electrical circuit
&
\sphinxAtStartPar
\(\frac{1}{R}\)
&
\sphinxAtStartPar
\(U\)
&
\sphinxAtStartPar
\(\bar{I}\)
&
\sphinxAtStartPar
\(I\)
\\
\sphinxhline
\sphinxAtStartPar
Groundwater flow
&
\sphinxAtStartPar
\(\frac{kA}{L}\)
&
\sphinxAtStartPar
\(\phi\)
&
\sphinxAtStartPar
\(\bar{H}\)
&
\sphinxAtStartPar
\(H\)
\\
\sphinxhline
\sphinxAtStartPar
Pipe network
&
\sphinxAtStartPar
\(\frac{\pi D^4}{128{\mu}L}\)
&
\sphinxAtStartPar
\(p\)
&
\sphinxAtStartPar
\(\bar{H}\)
&
\sphinxAtStartPar
\(H\)
\\
\sphinxbottomrule
\end{tabular}
\sphinxtableafterendhook\par
\sphinxattableend\end{savenotes}


\begin{savenotes}\sphinxattablestart
\sphinxthistablewithglobalstyle
\centering
\sphinxcapstartof{table}
\sphinxthecaptionisattop
\sphinxcaption{Quantities used in different types of problems}\label{\detokenize{spring_functions:id3}}
\sphinxaftertopcaption
\begin{tabular}[t]{\X{20}{100}\X{30}{100}\X{10}{100}\X{40}{100}}
\sphinxtoprule
\sphinxstyletheadfamily 
\sphinxAtStartPar
Problem type
&\sphinxstyletheadfamily 
\sphinxAtStartPar
Quantities
&\sphinxstyletheadfamily 
\sphinxAtStartPar
Designations
&\sphinxstyletheadfamily 
\sphinxAtStartPar
Description
\\
\sphinxmidrule
\sphinxtableatstartofbodyhook
\sphinxAtStartPar
Spring
&
\noindent\sphinxincludegraphics[width=61mm]{{ANA1}.png}
&
\sphinxAtStartPar
\(k\), \(u\), \(P\), \(N\)
&
\sphinxAtStartPar
spring stiffness, displacement, element force, spring force
\\
\sphinxhline
\sphinxAtStartPar
Bar
&
\noindent\sphinxincludegraphics[width=61mm]{{ANA2}.png}
&
\sphinxAtStartPar
\(L\), \(E\), \(A\), \(u\), \(P\), \(N\)
&
\sphinxAtStartPar
length, modulus of elasticity, area of cross section, displacement, element force, normal force
\\
\sphinxhline
\sphinxAtStartPar
Thermal conduction
&
\noindent\sphinxincludegraphics[width=61mm]{{ANA3}.png}
&
\sphinxAtStartPar
\(L\), \(\lambda\), \(T\), \(\bar{H}\), \(H\)
&
\sphinxAtStartPar
length, thermal conductivity, temperature, element heat flow, internal heat flow
\\
\sphinxhline
\sphinxAtStartPar
Diffusion
&
\noindent\sphinxincludegraphics[width=61mm]{{ana7}.png}
&
\sphinxAtStartPar
\(L\), \(D\), \(c\), \(\bar{H}\), \(H\)
&
\sphinxAtStartPar
length, diffusivity, nodal concentration, nodal mass flow, element mass flow
\\
\sphinxhline
\sphinxAtStartPar
Electrical circuit
&
\noindent\sphinxincludegraphics[width=61mm]{{ANA4}.png}
&
\sphinxAtStartPar
\(R\), \(U\), \(\bar{I}\), \(I\)
&
\sphinxAtStartPar
resistance, potential, element current, internal current
\\
\sphinxhline
\sphinxAtStartPar
Groundwater flow
&
\noindent\sphinxincludegraphics[width=61mm]{{ANA5}.png}
&
\sphinxAtStartPar
\(L\), \(k\), \(\phi\), \(\bar{H}\), \(H\)
&
\sphinxAtStartPar
length, permeability, piezometric head, element water flow, internal water flow
\\
\sphinxhline
\sphinxAtStartPar
Pipe network (laminar flow)
&
\noindent\sphinxincludegraphics[width=61mm]{{ANA6}.png}
&
\sphinxAtStartPar
\(L\), \(D\), \(\mu\), \(p\), \(\bar{H}\), \(H\)
&
\sphinxAtStartPar
length, pipe diameter, viscosity, pressure, element fluid flow, internal fluid flow
\\
\sphinxbottomrule
\end{tabular}
\sphinxtableafterendhook\par
\sphinxattableend\end{savenotes}


\section{spring1e}
\label{\detokenize{spring_functions:spring1e}}\begin{quote}\begin{description}
\sphinxlineitem{Purpose}
\sphinxAtStartPar
Compute element stiffness matrix for a spring element.

\begin{figure}[htbp]
\centering

\noindent\sphinxincludegraphics[width=0.700\linewidth]{{SPRING1}.png}
\end{figure}

\sphinxlineitem{Syntax}
\begin{sphinxVerbatim}[commandchars=\\\{\}]
\PYG{n}{Ke}\PYG{+w}{ }\PYG{p}{=}\PYG{+w}{ }\PYG{n}{spring1e}\PYG{p}{(}\PYG{n}{ep}\PYG{p}{)}
\end{sphinxVerbatim}

\sphinxlineitem{Description}
\sphinxAtStartPar
\(\mathtt{spring1e}\) provides the element stiffness matrix \(\mathtt{Ke}\) for a spring element.

\sphinxAtStartPar
The input variable

\sphinxAtStartPar
\(\mathtt{ep} = [\,k\,]\)

\sphinxAtStartPar
supplies the spring stiffness \(k\) or the analog quantity defined in Table {\hyperref[\detokenize{spring_functions:tanalog}]{\sphinxcrossref{\DUrole{std}{\DUrole{std-ref}{Analogous quantities}}}}}.

\sphinxlineitem{Theory}
\sphinxAtStartPar
The element stiffness matrix \(\mathbf{K}^e\), stored in \(\mathtt{Ke}\), is computed according to
\begin{equation*}
\begin{split}\mathbf{K}^e = \begin{bmatrix}
    k & -k \\
    -k & k
\end{bmatrix}\end{split}
\end{equation*}
\sphinxAtStartPar
where \(k\) is defined by \(\mathtt{ep}\).

\end{description}\end{quote}


\section{spring1s}
\label{\detokenize{spring_functions:spring1s}}\begin{quote}\begin{description}
\sphinxlineitem{Purpose}
\sphinxAtStartPar
Compute spring force in a spring element.

\begin{figure}[htbp]
\centering

\noindent\sphinxincludegraphics[width=0.700\linewidth]{{SPRING3}.png}
\end{figure}

\sphinxlineitem{Syntax}
\begin{sphinxVerbatim}[commandchars=\\\{\}]
\PYG{n}{es}\PYG{+w}{ }\PYG{p}{=}\PYG{+w}{ }\PYG{n}{spring1s}\PYG{p}{(}\PYG{n}{ep}\PYG{p}{,}\PYG{+w}{ }\PYG{n}{ed}\PYG{p}{)}
\end{sphinxVerbatim}

\sphinxlineitem{Description}
\sphinxAtStartPar
\(\mathrm{spring1s}\) computes the spring force \(\mathrm{es}\) in a spring element.

\sphinxAtStartPar
The input variable \(\mathrm{ep}\) is defined in \(\mathrm{spring1e}\) and the
element nodal displacements \(\mathrm{ed}\) are obtained by the function \(\mathrm{extract}\).

\sphinxAtStartPar
The output variable
\begin{equation*}
\begin{split}\mathrm{es} = \left[\,N\,\right]\end{split}
\end{equation*}
\sphinxAtStartPar
contains the spring force \(N\), or the analog quantity.

\sphinxlineitem{Theory}
\sphinxAtStartPar
The spring force \(N\), or analog quantity, is computed according to
\begin{equation*}
\begin{split}N = k \left(u_2 - u_1\right)\end{split}
\end{equation*}
\end{description}\end{quote}

\sphinxstepscope


\chapter{Bar element functions}
\label{\detokenize{bar_functions:bar-element-functions}}\label{\detokenize{bar_functions::doc}}

\section{1D Bar elements}
\label{\detokenize{bar_functions:d-bar-elements}}

\subsection{bar1e}
\label{\detokenize{bar_functions:bar1e}}
\index{bar1e@\spxentry{bar1e}}\ignorespaces \begin{quote}\begin{description}
\sphinxlineitem{Purpose}
\sphinxAtStartPar
Compute element stiffness matrix for a one dimensional bar element.

\begin{figure}[htbp]
\centering

\noindent\sphinxincludegraphics[width=0.700\linewidth]{{bar1e_1}.png}
\end{figure}

\sphinxlineitem{Syntax}
\begin{sphinxVerbatim}[commandchars=\\\{\}]
\PYG{n}{Ke}\PYG{+w}{ }\PYG{p}{=}\PYG{+w}{ }\PYG{n}{bar1e}\PYG{p}{(}\PYG{n}{ex}\PYG{p}{,}\PYG{+w}{ }\PYG{n}{ep}\PYG{p}{)}
\PYG{p}{[}\PYG{n}{Ke}\PYG{p}{,}\PYG{+w}{ }\PYG{n}{fe}\PYG{p}{]}\PYG{+w}{ }\PYG{p}{=}\PYG{+w}{ }\PYG{n}{bar1e}\PYG{p}{(}\PYG{n}{ex}\PYG{p}{,}\PYG{+w}{ }\PYG{n}{ep}\PYG{p}{,}\PYG{+w}{ }\PYG{n+nb}{eq}\PYG{p}{)}
\end{sphinxVerbatim}

\sphinxlineitem{Description}
\sphinxAtStartPar
\sphinxcode{\sphinxupquote{bar1e}} provides the element stiffness matrix \sphinxcode{\sphinxupquote{Ke}} for a one dimensional bar element.
The input variables
\begin{equation*}
\begin{split}\text{ex} = [x_1 \;\; x_2]
\qquad
\text{ep} = [E \; A]\end{split}
\end{equation*}
\sphinxAtStartPar
supply the element nodal coordinates \(x_1\) and \(x_2\), the modulus of elasticity \(E\),
and the cross section area \(A\).

\sphinxAtStartPar
The element load vector \sphinxcode{\sphinxupquote{fe}} can also be computed if a uniformly distributed load is applied to the element.
The optional input variable
\begin{equation*}
\begin{split}\text{eq} = [q_{\bar{x}}]\end{split}
\end{equation*}
\sphinxAtStartPar
contains the distributed load per unit length, \(q_{\bar{x}}\).

\begin{figure}[htbp]
\centering

\noindent\sphinxincludegraphics[width=0.700\linewidth]{{bar1e_2}.png}
\end{figure}

\sphinxlineitem{Theory}
\sphinxAtStartPar
The element stiffness matrix \(\bar{\mathbf{K}}^e\), stored in \sphinxcode{\sphinxupquote{Ke}}, is computed according to
\begin{equation*}
\begin{split}\bar{\mathbf{K}}^e = \frac{D_{EA}}{L}
\begin{bmatrix}
1 & -1 \\
-1 & 1
\end{bmatrix}\end{split}
\end{equation*}
\sphinxAtStartPar
where the axial stiffness \(D_{EA}\) and the length \(L\) are given by
\begin{equation*}
\begin{split}D_{EA} = EA; \quad L = x_2 - x_1\end{split}
\end{equation*}
\sphinxAtStartPar
The element load vector \(\bar{\mathbf{f}}_l^e\), stored in \sphinxcode{\sphinxupquote{fe}}, is computed according to
\begin{equation*}
\begin{split}\bar{\mathbf{f}}_l^e = \frac{q_{\bar{x}} L}{2}
\begin{bmatrix}
1 \\
1
\end{bmatrix}\end{split}
\end{equation*}
\end{description}\end{quote}


\subsection{bar1s}
\label{\detokenize{bar_functions:bar1s}}
\index{bar1s@\spxentry{bar1s}}\ignorespaces \begin{quote}\begin{description}
\sphinxlineitem{Purpose}
\sphinxAtStartPar
Compute normal force in a one dimensional bar element.

\begin{figure}[htbp]
\centering

\noindent\sphinxincludegraphics[width=0.700\linewidth]{{bar1s}.png}
\end{figure}

\sphinxlineitem{Syntax}
\begin{sphinxVerbatim}[commandchars=\\\{\}]
\PYG{n}{es}\PYG{+w}{ }\PYG{p}{=}\PYG{+w}{ }\PYG{n}{bar1s}\PYG{p}{(}\PYG{n}{ex}\PYG{p}{,}\PYG{+w}{ }\PYG{n}{ep}\PYG{p}{,}\PYG{+w}{ }\PYG{n}{ed}\PYG{p}{)}
\PYG{n}{es}\PYG{+w}{ }\PYG{p}{=}\PYG{+w}{ }\PYG{n}{bar1s}\PYG{p}{(}\PYG{n}{ex}\PYG{p}{,}\PYG{+w}{ }\PYG{n}{ep}\PYG{p}{,}\PYG{+w}{ }\PYG{n}{ed}\PYG{p}{,}\PYG{+w}{ }\PYG{n+nb}{eq}\PYG{p}{)}
\PYG{p}{[}\PYG{n}{es}\PYG{p}{,}\PYG{+w}{ }\PYG{n}{edi}\PYG{p}{]}\PYG{+w}{ }\PYG{p}{=}\PYG{+w}{ }\PYG{n}{bar1s}\PYG{p}{(}\PYG{n}{ex}\PYG{p}{,}\PYG{+w}{ }\PYG{n}{ep}\PYG{p}{,}\PYG{+w}{ }\PYG{n}{ed}\PYG{p}{,}\PYG{+w}{ }\PYG{n+nb}{eq}\PYG{p}{,}\PYG{+w}{ }\PYG{n}{n}\PYG{p}{)}
\PYG{p}{[}\PYG{n}{es}\PYG{p}{,}\PYG{+w}{ }\PYG{n}{edi}\PYG{p}{,}\PYG{+w}{ }\PYG{n}{eci}\PYG{p}{]}\PYG{+w}{ }\PYG{p}{=}\PYG{+w}{ }\PYG{n}{bar1s}\PYG{p}{(}\PYG{n}{ex}\PYG{p}{,}\PYG{+w}{ }\PYG{n}{ep}\PYG{p}{,}\PYG{+w}{ }\PYG{n}{ed}\PYG{p}{,}\PYG{+w}{ }\PYG{n+nb}{eq}\PYG{p}{,}\PYG{+w}{ }\PYG{n}{n}\PYG{p}{)}
\end{sphinxVerbatim}

\sphinxlineitem{Description}
\sphinxAtStartPar
\sphinxcode{\sphinxupquote{bar1s}} computes the normal force in the one dimensional bar element \sphinxcode{\sphinxupquote{bar1e}}.

\sphinxAtStartPar
The input variables \sphinxcode{\sphinxupquote{ex}} and \sphinxcode{\sphinxupquote{ep}} are defined in \sphinxcode{\sphinxupquote{bar1e}} and the element nodal displacements, stored in \sphinxcode{\sphinxupquote{ed}}, are obtained by the function \sphinxcode{\sphinxupquote{extract}}. If distributed load is applied to the element, the variable \sphinxcode{\sphinxupquote{eq}} must be included.

\sphinxAtStartPar
The number of evaluation points for normal force and displacement are determined by \sphinxcode{\sphinxupquote{n}}. If \sphinxcode{\sphinxupquote{n}} is omitted, only the ends of the bar are evaluated.

\sphinxAtStartPar
The output variables
\begin{equation*}
\begin{split}\mathrm{es} =
\begin{bmatrix}
N(0) \\
N(\bar{x}_2) \\
\vdots \\
N(\bar{x}_{n-1}) \\
N(L)
\end{bmatrix}
\qquad
\mathrm{edi} =
\begin{bmatrix}
u(0) \\
u(\bar{x}_2) \\
\vdots \\
u(\bar{x}_{n-1}) \\
u(L)
\end{bmatrix}
\qquad
\mathrm{eci} =
\begin{bmatrix}
0 \\
\bar{x}_2 \\
\vdots \\
\bar{x}_{n-1} \\
L
\end{bmatrix}\end{split}
\end{equation*}
\sphinxAtStartPar
contain the normal force, the displacement, and the evaluation points on the local \(\bar{x}\)\sphinxhyphen{}axis. \(L\) is the length of the bar element.

\sphinxlineitem{Theory}
\sphinxAtStartPar
The nodal displacements in local coordinates are given by
\begin{equation*}
\begin{split}\mathbf{\bar{a}}^e = \begin{bmatrix} \bar{u}_1 \\ \bar{u}_2 \end{bmatrix}\end{split}
\end{equation*}
\sphinxAtStartPar
The transpose of \(\mathbf{\bar{a}}^e\) is stored in \sphinxcode{\sphinxupquote{ed}}.

\sphinxAtStartPar
The displacement \(u(\bar{x})\) and the normal force \(N(\bar{x})\) are computed from
\begin{equation*}
\begin{split}u(\bar{x}) = \mathbf{N} \mathbf{\bar{a}}^e + u_p(\bar{x})\end{split}
\end{equation*}\begin{equation*}
\begin{split}N(\bar{x}) = D_{EA} \mathbf{B} \mathbf{\bar{a}}^e + N_p(\bar{x})\end{split}
\end{equation*}
\sphinxAtStartPar
where
\begin{equation*}
\begin{split}\mathbf{N} = \begin{bmatrix} 1 & \bar{x} \end{bmatrix} \mathbf{C}^{-1} = \begin{bmatrix} 1 - \frac{\bar{x}}{L} & \frac{\bar{x}}{L} \end{bmatrix}\end{split}
\end{equation*}\begin{equation*}
\begin{split}\mathbf{B} = \begin{bmatrix} 0 & 1 \end{bmatrix} \mathbf{C}^{-1} = \frac{1}{L} \begin{bmatrix} -1 & 1 \end{bmatrix}\end{split}
\end{equation*}\begin{equation*}
\begin{split}u_p(\bar{x}) = -\frac{q_{\bar{x}}}{D_{EA}} \left( \frac{\bar{x}^2}{2} - \frac{L\bar{x}}{2} \right)\end{split}
\end{equation*}\begin{equation*}
\begin{split}N_p(\bar{x}) = -q_{\bar{x}} \left( \bar{x} - \frac{L}{2} \right)\end{split}
\end{equation*}
\sphinxAtStartPar
in which \(D_{EA}\), \(L\), and \(q_{\bar{x}}\) are defined in \sphinxcode{\sphinxupquote{bar1e}} and
\begin{equation*}
\begin{split}\mathbf{C}^{-1} = \begin{bmatrix} 1 & 0 \\ -\frac{1}{L} & \frac{1}{L} \end{bmatrix}\end{split}
\end{equation*}
\end{description}\end{quote}


\subsection{bar1we}
\label{\detokenize{bar_functions:bar1we}}\begin{quote}\begin{description}
\sphinxlineitem{Purpose}
\sphinxAtStartPar
Compute element stiffness matrix for a one dimensional bar element with elastic support.

\begin{figure}[htbp]
\centering

\noindent\sphinxincludegraphics[width=0.700\linewidth]{{bar1w_1}.png}
\end{figure}

\sphinxlineitem{Syntax}
\begin{sphinxVerbatim}[commandchars=\\\{\}]
\PYG{n}{Ke}\PYG{+w}{ }\PYG{p}{=}\PYG{+w}{ }\PYG{n}{bar1we}\PYG{p}{(}\PYG{n}{ex}\PYG{p}{,}\PYG{+w}{ }\PYG{n}{ep}\PYG{p}{)}
\PYG{p}{[}\PYG{n}{Ke}\PYG{p}{,}\PYG{+w}{ }\PYG{n}{fe}\PYG{p}{]}\PYG{+w}{ }\PYG{p}{=}\PYG{+w}{ }\PYG{n}{bar1we}\PYG{p}{(}\PYG{n}{ex}\PYG{p}{,}\PYG{+w}{ }\PYG{n}{ep}\PYG{p}{,}\PYG{+w}{ }\PYG{n+nb}{eq}\PYG{p}{)}
\end{sphinxVerbatim}

\sphinxlineitem{Description}
\sphinxAtStartPar
\sphinxcode{\sphinxupquote{bar1we}} provides the element stiffness matrix \sphinxcode{\sphinxupquote{Ke}} for a one dimensional bar element with elastic support.

\sphinxAtStartPar
The input variables
\begin{equation*}
\begin{split}\mathrm{ex} = [x_1\;\; x_2]
\qquad
\mathrm{ep} = [E\; A\; k_{\bar{x}}]\end{split}
\end{equation*}
\sphinxAtStartPar
supply the element nodal coordinates \(x_1\) and \(x_2\), the modulus of elasticity \(E\), the cross section area \(A\) and the stiffness of the axial springs \(k_{\bar{x}}\).

\sphinxAtStartPar
The element load vector \sphinxcode{\sphinxupquote{fe}} can also be computed if a uniformly distributed load is applied to the element.

\sphinxAtStartPar
The optional input variable
\begin{equation*}
\begin{split}\mathrm{eq} = [q_{\bar{x}}]\end{split}
\end{equation*}
\sphinxAtStartPar
then contains the distributed load per unit length, \(q_{\bar{x}}\).

\begin{figure}[htbp]
\centering

\noindent\sphinxincludegraphics[width=0.700\linewidth]{{bar1e_2}.png}
\end{figure}

\sphinxAtStartPar
Bar element with distributed load

\sphinxlineitem{Theory}
\sphinxAtStartPar
The element stiffness matrix \(\bar{\mathbf{K}}^e\), stored in \sphinxcode{\sphinxupquote{Ke}}, is computed according to
\begin{equation*}
\begin{split}\bar{\mathbf{K}}^e = \bar{\mathbf{K}}^e_0 + \bar{\mathbf{K}}^e_s\end{split}
\end{equation*}
\sphinxAtStartPar
where
\begin{equation*}
\begin{split}\bar{\mathbf{K}}^e_0 = \frac{D_{EA}}{L}
\begin{bmatrix}
    1 & -1 \\
  -1 &  1
\end{bmatrix}\end{split}
\end{equation*}\begin{equation*}
\begin{split}\bar{\mathbf{K}}^e_s = k_{\bar{x}} L
\begin{bmatrix}
    \frac{1}{3} & \frac{1}{6} \\
    \frac{1}{6} & \frac{1}{3}
\end{bmatrix}\end{split}
\end{equation*}
\sphinxAtStartPar
where the axial stiffness \(D_{EA}\) and the length \(L\) are given by
\begin{equation*}
\begin{split}D_{EA} = EA; \qquad L = x_2 - x_1\end{split}
\end{equation*}
\sphinxAtStartPar
The element load vector \(\bar{\mathbf{f}}_l^e\), stored in \sphinxcode{\sphinxupquote{fe}}, is computed according to
\begin{equation*}
\begin{split}\bar{\mathbf{f}}_l^e = \frac{q_{\bar{x}} L}{2}
\begin{bmatrix}
    1 \\
    1
\end{bmatrix}\end{split}
\end{equation*}
\end{description}\end{quote}


\subsection{bar1ws}
\label{\detokenize{bar_functions:bar1ws}}\begin{quote}\begin{description}
\sphinxlineitem{Purpose}
\sphinxAtStartPar
Compute normal force in a one dimensional bar element with elastic support.

\begin{figure}[htbp]
\centering

\noindent\sphinxincludegraphics[width=0.700\linewidth]{{bar1s}.png}
\end{figure}

\sphinxlineitem{Syntax}
\begin{sphinxVerbatim}[commandchars=\\\{\}]
\PYG{n}{es}\PYG{+w}{ }\PYG{p}{=}\PYG{+w}{ }\PYG{n}{bar1ws}\PYG{p}{(}\PYG{n}{ex}\PYG{p}{,}\PYG{+w}{ }\PYG{n}{ep}\PYG{p}{,}\PYG{+w}{ }\PYG{n}{ed}\PYG{p}{)}
\PYG{n}{es}\PYG{+w}{ }\PYG{p}{=}\PYG{+w}{ }\PYG{n}{bar1ws}\PYG{p}{(}\PYG{n}{ex}\PYG{p}{,}\PYG{+w}{ }\PYG{n}{ep}\PYG{p}{,}\PYG{+w}{ }\PYG{n}{ed}\PYG{p}{,}\PYG{+w}{ }\PYG{n+nb}{eq}\PYG{p}{)}
\PYG{p}{[}\PYG{n}{es}\PYG{p}{,}\PYG{+w}{ }\PYG{n}{edi}\PYG{p}{]}\PYG{+w}{ }\PYG{p}{=}\PYG{+w}{ }\PYG{n}{bar1ws}\PYG{p}{(}\PYG{n}{ex}\PYG{p}{,}\PYG{+w}{ }\PYG{n}{ep}\PYG{p}{,}\PYG{+w}{ }\PYG{n}{ed}\PYG{p}{,}\PYG{+w}{ }\PYG{n+nb}{eq}\PYG{p}{,}\PYG{+w}{ }\PYG{n}{n}\PYG{p}{)}
\PYG{p}{[}\PYG{n}{es}\PYG{p}{,}\PYG{+w}{ }\PYG{n}{edi}\PYG{p}{,}\PYG{+w}{ }\PYG{n}{eci}\PYG{p}{]}\PYG{+w}{ }\PYG{p}{=}\PYG{+w}{ }\PYG{n}{bar1ws}\PYG{p}{(}\PYG{n}{ex}\PYG{p}{,}\PYG{+w}{ }\PYG{n}{ep}\PYG{p}{,}\PYG{+w}{ }\PYG{n}{ed}\PYG{p}{,}\PYG{+w}{ }\PYG{n+nb}{eq}\PYG{p}{,}\PYG{+w}{ }\PYG{n}{n}\PYG{p}{)}
\end{sphinxVerbatim}

\sphinxlineitem{Description}
\sphinxAtStartPar
\sphinxcode{\sphinxupquote{bar1ws}} computes the normal force in the one dimensional bar element \sphinxcode{\sphinxupquote{bar1we}}.

\sphinxAtStartPar
The input variables \sphinxcode{\sphinxupquote{ex}} and \sphinxcode{\sphinxupquote{ep}} are defined in \sphinxcode{\sphinxupquote{bar1we}} and the element nodal displacements, stored in \sphinxcode{\sphinxupquote{ed}}, are obtained by the function \sphinxcode{\sphinxupquote{extract}}. If distributed load is applied to the element, the variable \sphinxcode{\sphinxupquote{eq}} must be included.

\sphinxAtStartPar
The number of evaluation points for normal force and displacement are determined by \sphinxcode{\sphinxupquote{n}}. If \sphinxcode{\sphinxupquote{n}} is omitted, only the ends of the bar are evaluated.

\sphinxAtStartPar
The output variables are:
\begin{equation*}
\begin{split}\mathrm{es} =
\begin{bmatrix}
N(0) \\
N(\bar{x}_2) \\
\vdots \\
N(\bar{x}_{n-1}) \\
N(L)
\end{bmatrix}
\qquad
\mathrm{edi} =
\begin{bmatrix}
u(0) \\
u(\bar{x}_2) \\
\vdots \\
u(\bar{x}_{n-1}) \\
u(L)
\end{bmatrix}
\qquad
\mathrm{eci} =
\begin{bmatrix}
0 \\
\bar{x}_2 \\
\vdots \\
\bar{x}_{n-1} \\
L
\end{bmatrix}\end{split}
\end{equation*}
\sphinxAtStartPar
These contain the normal force, the displacement, and the evaluation points on the local \(\bar{x}\)\sphinxhyphen{}axis. \(L\) is the length of the bar element.

\sphinxlineitem{Theory}
\sphinxAtStartPar
The nodal displacements in local coordinates are given by
\begin{equation*}
\begin{split}\mathbf{\bar{a}}^e = \begin{bmatrix} \bar{u}_1 \\ \bar{u}_2 \end{bmatrix}\end{split}
\end{equation*}
\sphinxAtStartPar
The transpose of \(\mathbf{\bar{a}}^e\) is stored in \sphinxcode{\sphinxupquote{ed}}.

\sphinxAtStartPar
The displacement \(u(\bar{x})\) and the normal force \(N(\bar{x})\) are computed from
\begin{equation*}
\begin{split}u(\bar{x}) = \mathbf{N} \mathbf{\bar{a}}^e + u_p(\bar{x})\end{split}
\end{equation*}\begin{equation*}
\begin{split}N(\bar{x}) = D_{EA} \mathbf{B} \mathbf{\bar{a}}^e + N_p(\bar{x})\end{split}
\end{equation*}
\sphinxAtStartPar
where
\begin{equation*}
\begin{split}\mathbf{N} = \begin{bmatrix} 1 & \bar{x} \end{bmatrix} \mathbf{C}^{-1} = \begin{bmatrix} 1 - \frac{\bar{x}}{L} & \frac{\bar{x}}{L} \end{bmatrix}\end{split}
\end{equation*}\begin{equation*}
\begin{split}\mathbf{B} = \begin{bmatrix} 0 & 1 \end{bmatrix} \mathbf{C}^{-1} = \frac{1}{L} \begin{bmatrix} -1 & 1 \end{bmatrix}\end{split}
\end{equation*}\begin{equation*}
\begin{split}u_p(\bar{x}) = \frac{k_{\bar{x}}}{D_{EA}} \left[ \frac{\bar{x}^2 - L\bar{x}}{2} \quad \frac{\bar{x}^3 - L^2\bar{x}}{6} \right] \mathbf{C}^{-1} \mathbf{\bar{a}}^e
- \frac{q_{\bar{x}}}{D_{EA}} \left( \frac{\bar{x}^2}{2} - \frac{L\bar{x}}{2} \right)\end{split}
\end{equation*}\begin{equation*}
\begin{split}N_p(\bar{x}) = k_{\bar{x}} \left[ \frac{2\bar{x} - L}{2} \quad \frac{3\bar{x}^2 - L^2}{6} \right] \mathbf{C}^{-1} \mathbf{\bar{a}}^e
- q_{\bar{x}} \left( \bar{x} - \frac{L}{2} \right)\end{split}
\end{equation*}
\sphinxAtStartPar
in which \(D_{EA}\), \(L\), \(k_{\bar{x}}\) and \(q_{\bar{x}}\) are defined in \sphinxcode{\sphinxupquote{bar1we}} and
\begin{equation*}
\begin{split}\mathbf{C}^{-1} = \begin{bmatrix} 1 & 0 \\ -\frac{1}{L} & \frac{1}{L} \end{bmatrix}\end{split}
\end{equation*}
\end{description}\end{quote}


\section{2D bar elements}
\label{\detokenize{bar_functions:id1}}

\subsection{bar2e}
\label{\detokenize{bar_functions:bar2e}}\begin{quote}\begin{description}
\sphinxlineitem{Purpose}
\sphinxAtStartPar
Compute element stiffness matrix for a two dimensional bar element.

\begin{figure}[htbp]
\centering

\noindent\sphinxincludegraphics[width=0.700\linewidth]{{bar2e}.png}
\end{figure}

\sphinxlineitem{Syntax}
\begin{sphinxVerbatim}[commandchars=\\\{\}]
\PYG{n}{Ke}\PYG{+w}{ }\PYG{p}{=}\PYG{+w}{ }\PYG{n}{bar2e}\PYG{p}{(}\PYG{n}{ex}\PYG{p}{,}\PYG{+w}{ }\PYG{n}{ey}\PYG{p}{,}\PYG{+w}{ }\PYG{n}{ep}\PYG{p}{)}
\PYG{p}{[}\PYG{n}{Ke}\PYG{p}{,}\PYG{+w}{ }\PYG{n}{fe}\PYG{p}{]}\PYG{+w}{ }\PYG{p}{=}\PYG{+w}{ }\PYG{n}{bar2e}\PYG{p}{(}\PYG{n}{ex}\PYG{p}{,}\PYG{+w}{ }\PYG{n}{ey}\PYG{p}{,}\PYG{+w}{ }\PYG{n}{ep}\PYG{p}{,}\PYG{+w}{ }\PYG{n+nb}{eq}\PYG{p}{)}
\end{sphinxVerbatim}

\sphinxlineitem{Description}
\sphinxAtStartPar
\sphinxcode{\sphinxupquote{bar2e}} provides the global element stiffness matrix \sphinxcode{\sphinxupquote{Ke}} for a two dimensional bar element.

\sphinxAtStartPar
The input variables
\begin{equation*}
\begin{split}\begin{array}{l}
\mathrm{ex} = [\, x_1 \;\; x_2 \,] \\
\mathrm{ey} = [\, y_1 \;\; y_2 \,]
\end{array}
\qquad
\mathrm{ep} = [\, E \; A \,]\end{split}
\end{equation*}
\sphinxAtStartPar
supply the element nodal coordinates \(x_1\), \(y_1\), \(x_2\), and \(y_2\), the modulus of elasticity \(E\), and the cross section area \(A\).

\sphinxAtStartPar
The element load vector \sphinxcode{\sphinxupquote{fe}} can also be computed if a uniformly distributed axial load is applied to the element.
The optional input variable
\begin{equation*}
\begin{split}\mathrm{eq} = [\, q_{\bar{x}} \,]\end{split}
\end{equation*}
\sphinxAtStartPar
then contains the distributed load per unit length, \(q_{\bar{x}}\).

\begin{figure}[htbp]
\centering

\noindent\sphinxincludegraphics[width=0.700\linewidth]{{bar2e_2}.png}
\end{figure}

\sphinxlineitem{Theory}
\sphinxAtStartPar
The element stiffness matrix \(\mathbf{K}^e\), stored in \sphinxcode{\sphinxupquote{Ke}}, is computed according to
\begin{equation*}
\begin{split}\mathbf{K}^e = \mathbf{G}^T \; \bar{\mathbf{K}}^e \; \mathbf{G}\end{split}
\end{equation*}
\sphinxAtStartPar
where
\begin{equation*}
\begin{split}\bar{\mathbf{K}}^e = \frac{D_{EA}}{L}
\begin{bmatrix}
1 & -1 \\
-1 & 1
\end{bmatrix}
\qquad
\mathbf{G} =
\begin{bmatrix}
n_{x\bar{x}} & n_{y\bar{x}} & 0 & 0 \\
0 & 0 & n_{x\bar{x}} & n_{y\bar{x}}
\end{bmatrix}\end{split}
\end{equation*}
\sphinxAtStartPar
where the axial stiffness \(D_{EA}\) and the length \(L\) are given by
\begin{equation*}
\begin{split}D_{EA} = EA; \qquad L = \sqrt{(x_2 - x_1)^2 + (y_2 - y_1)^2}\end{split}
\end{equation*}
\sphinxAtStartPar
and the transformation matrix \(\mathbf{G}\) contains the direction cosines
\begin{equation*}
\begin{split}n_{x\bar{x}} = \frac{x_2 - x_1}{L} \qquad n_{y\bar{x}} = \frac{y_2 - y_1}{L}\end{split}
\end{equation*}
\sphinxAtStartPar
The element load vector \(\mathbf{f}_l^e\), stored in \sphinxcode{\sphinxupquote{fe}}, is computed according to
\begin{equation*}
\begin{split}\mathbf{f}_l^e = \mathbf{G}^T \; \bar{\mathbf{f}}_l^e\end{split}
\end{equation*}
\sphinxAtStartPar
where
\begin{equation*}
\begin{split}\bar{\mathbf{f}}_l^e = \frac{q_{\bar{x}} L}{2}
\begin{bmatrix}
1 \\
1
\end{bmatrix}\end{split}
\end{equation*}
\end{description}\end{quote}


\subsection{bar2s}
\label{\detokenize{bar_functions:bar2s}}\begin{quote}\begin{description}
\sphinxlineitem{Purpose}
\sphinxAtStartPar
Compute normal force in a two dimensional bar element.

\begin{figure}[htbp]
\centering

\noindent\sphinxincludegraphics[width=0.700\linewidth]{{bar2s}.png}
\end{figure}

\sphinxlineitem{Syntax}
\begin{sphinxVerbatim}[commandchars=\\\{\}]
\PYG{n}{es}\PYG{+w}{ }\PYG{p}{=}\PYG{+w}{ }\PYG{n}{bar2s}\PYG{p}{(}\PYG{n}{ex}\PYG{p}{,}\PYG{+w}{ }\PYG{n}{ey}\PYG{p}{,}\PYG{+w}{ }\PYG{n}{ep}\PYG{p}{,}\PYG{+w}{ }\PYG{n}{ed}\PYG{p}{)}
\PYG{n}{es}\PYG{+w}{ }\PYG{p}{=}\PYG{+w}{ }\PYG{n}{bar2s}\PYG{p}{(}\PYG{n}{ex}\PYG{p}{,}\PYG{+w}{ }\PYG{n}{ey}\PYG{p}{,}\PYG{+w}{ }\PYG{n}{ep}\PYG{p}{,}\PYG{+w}{ }\PYG{n}{ed}\PYG{p}{,}\PYG{+w}{ }\PYG{n+nb}{eq}\PYG{p}{)}
\PYG{p}{[}\PYG{n}{es}\PYG{p}{,}\PYG{+w}{ }\PYG{n}{edi}\PYG{p}{]}\PYG{+w}{ }\PYG{p}{=}\PYG{+w}{ }\PYG{n}{bar2s}\PYG{p}{(}\PYG{n}{ex}\PYG{p}{,}\PYG{+w}{ }\PYG{n}{ey}\PYG{p}{,}\PYG{+w}{ }\PYG{n}{ep}\PYG{p}{,}\PYG{+w}{ }\PYG{n}{ed}\PYG{p}{,}\PYG{+w}{ }\PYG{n+nb}{eq}\PYG{p}{,}\PYG{+w}{ }\PYG{n}{n}\PYG{p}{)}
\PYG{p}{[}\PYG{n}{es}\PYG{p}{,}\PYG{+w}{ }\PYG{n}{edi}\PYG{p}{,}\PYG{+w}{ }\PYG{n}{eci}\PYG{p}{]}\PYG{+w}{ }\PYG{p}{=}\PYG{+w}{ }\PYG{n}{bar2s}\PYG{p}{(}\PYG{n}{ex}\PYG{p}{,}\PYG{+w}{ }\PYG{n}{ey}\PYG{p}{,}\PYG{+w}{ }\PYG{n}{ep}\PYG{p}{,}\PYG{+w}{ }\PYG{n}{ed}\PYG{p}{,}\PYG{+w}{ }\PYG{n+nb}{eq}\PYG{p}{,}\PYG{+w}{ }\PYG{n}{n}\PYG{p}{)}
\end{sphinxVerbatim}

\sphinxlineitem{Description}
\sphinxAtStartPar
\sphinxcode{\sphinxupquote{bar2s}} computes the normal force in the two dimensional bar element \sphinxcode{\sphinxupquote{bar2e}}.

\sphinxAtStartPar
The input variables \sphinxcode{\sphinxupquote{ex}}, \sphinxcode{\sphinxupquote{ey}}, and \sphinxcode{\sphinxupquote{ep}} are defined in \sphinxcode{\sphinxupquote{bar2e}} and the element nodal displacements, stored in \sphinxcode{\sphinxupquote{ed}}, are obtained by the function \sphinxcode{\sphinxupquote{extract}}. If distributed loads are applied to the element, the variable \sphinxcode{\sphinxupquote{eq}} must be included.
The number of evaluation points for section forces and displacements are determined by \sphinxcode{\sphinxupquote{n}}. If \sphinxcode{\sphinxupquote{n}} is omitted, only the ends of the bar are evaluated.

\sphinxAtStartPar
The output variables
\begin{equation*}
\begin{split}\mathrm{es} =
\begin{bmatrix}
N(0)    \\
N(\bar{x}_{2})\\
\vdots  \\
N(\bar{x}_{n-1})\\
N(L)
\end{bmatrix}
\qquad
\mathrm{edi} =
\begin{bmatrix}
u(0)    \\
u(\bar{x}_{2})\\
\vdots  \\
u(\bar{x}_{n-1})\\
u(L)
\end{bmatrix}
\qquad
\mathrm{eci} =
\begin{bmatrix}
0  \\
\bar x_{2} \\
\vdots   \\
\bar x_{n-1} \\
L
\end{bmatrix}\end{split}
\end{equation*}
\sphinxAtStartPar
contain the normal force, the displacement, and the evaluation points on the local \(\bar{x}\)\sphinxhyphen{}axis.
\(L\) is the length of the bar element.

\sphinxlineitem{Theory}
\sphinxAtStartPar
The nodal displacements in global coordinates
\begin{equation*}
\begin{split}\mathbf{a}^e = \begin{bmatrix} u_1 & u_2 & u_3 & u_4 \end{bmatrix}^T\end{split}
\end{equation*}
\sphinxAtStartPar
are also shown in \sphinxcode{\sphinxupquote{bar2e}}. The transpose of \(\mathbf{a}^e\) is stored in \sphinxcode{\sphinxupquote{ed}}.

\sphinxAtStartPar
The nodal displacements in local coordinates are given by
\begin{equation*}
\begin{split}\mathbf{\bar{a}}^e = \mathbf{G} \mathbf{a}^e\end{split}
\end{equation*}
\sphinxAtStartPar
where the transformation matrix \(\mathbf{G}\) is defined in \sphinxcode{\sphinxupquote{bar2e}}.

\sphinxAtStartPar
The displacement \(u(\bar{x})\) and the normal force \(N(\bar{x})\) are computed from
\begin{equation*}
\begin{split}u(\bar{x}) = \mathbf{N} \mathbf{\bar{a}}^e + u_p(\bar{x})\end{split}
\end{equation*}\begin{equation*}
\begin{split}N(\bar{x}) = D_{EA} \mathbf{B} \mathbf{\bar{a}}^e + N_p(\bar{x})\end{split}
\end{equation*}
\sphinxAtStartPar
where
\begin{equation*}
\begin{split}\mathbf{N} = \begin{bmatrix} 1 & \bar{x} \end{bmatrix} \mathbf{C}^{-1}
= \begin{bmatrix} 1-\frac{\bar{x}}{L} & \frac{\bar{x}}{L} \end{bmatrix}\end{split}
\end{equation*}\begin{equation*}
\begin{split}\mathbf{B} = \begin{bmatrix} 0 & 1 \end{bmatrix} \mathbf{C}^{-1}
= \frac{1}{L} \begin{bmatrix} -1 & 1 \end{bmatrix}\end{split}
\end{equation*}\begin{equation*}
\begin{split}u_p(\bar{x}) = -\frac{q_{\bar{x}}}{D_{EA}}\left(\frac{\bar{x}^2}{2}-\frac{L\bar{x}}{2}\right)\end{split}
\end{equation*}\begin{equation*}
\begin{split}N_p(\bar{x}) = -q_{\bar{x}}\left(\bar{x}-\frac{L}{2}\right)\end{split}
\end{equation*}
\sphinxAtStartPar
where \(D_{EA}\), \(L\), \(q_{\bar{x}}\) are defined in \sphinxcode{\sphinxupquote{bar2e}} and
\begin{equation*}
\begin{split}\mathbf{C}^{-1} = \begin{bmatrix} 1 & 0 \\ -\frac{1}{L} & \frac{1}{L} \end{bmatrix}\end{split}
\end{equation*}
\end{description}\end{quote}


\subsection{bar2ge}
\label{\detokenize{bar_functions:bar2ge}}\begin{quote}\begin{description}
\sphinxlineitem{Purpose}
\sphinxAtStartPar
Compute element stiffness matrix for a two dimensional geometric nonlinear bar.

\begin{figure}[htbp]
\centering

\noindent\sphinxincludegraphics[width=0.700\linewidth]{{bar2g}.png}
\end{figure}

\sphinxlineitem{Syntax}
\begin{sphinxVerbatim}[commandchars=\\\{\}]
\PYG{n}{Ke}\PYG{+w}{ }\PYG{p}{=}\PYG{+w}{ }\PYG{n}{bar2ge}\PYG{p}{(}\PYG{n}{ex}\PYG{p}{,}\PYG{+w}{ }\PYG{n}{ey}\PYG{p}{,}\PYG{+w}{ }\PYG{n}{ep}\PYG{p}{,}\PYG{+w}{ }\PYG{n}{Qx}\PYG{p}{)}
\end{sphinxVerbatim}

\sphinxlineitem{Description}
\sphinxAtStartPar
\sphinxcode{\sphinxupquote{bar2ge}} provides the element stiffness matrix \sphinxcode{\sphinxupquote{Ke}} for a two dimensional
geometric nonlinear bar element.

\sphinxAtStartPar
The input variables:

\begin{sphinxVerbatim}[commandchars=\\\{\}]
\PYG{n}{ex} \PYG{o}{=} \PYG{p}{[}\PYG{n}{x1}\PYG{p}{,} \PYG{n}{x2}\PYG{p}{]}
\PYG{n}{ey} \PYG{o}{=} \PYG{p}{[}\PYG{n}{y1}\PYG{p}{,} \PYG{n}{y2}\PYG{p}{]}
\PYG{n}{ep} \PYG{o}{=} \PYG{p}{[}\PYG{n}{E}\PYG{p}{,} \PYG{n}{A}\PYG{p}{]}
\end{sphinxVerbatim}

\sphinxAtStartPar
supply the element nodal coordinates
\(x_1\), \(y_1\), \(x_2\), and \(y_2\), the modulus of elasticity \(E\),
and the cross section area \(A\).

\sphinxAtStartPar
The input variable:

\begin{sphinxVerbatim}[commandchars=\\\{\}]
\PYG{n}{Qx} \PYG{o}{=} \PYG{p}{[}\PYG{n}{Q\PYGZus{}}\PYG{p}{\PYGZob{}}\PYGZbs{}\PYG{n}{bar}\PYG{p}{\PYGZob{}}\PYG{n}{x}\PYG{p}{\PYGZcb{}}\PYG{p}{\PYGZcb{}}\PYG{p}{]}
\end{sphinxVerbatim}

\sphinxAtStartPar
contains the value of the axial force, which is positive in tension.

\sphinxlineitem{Theory}
\sphinxAtStartPar
The global element stiffness matrix \(\mathbf{K}^e\), stored in \sphinxcode{\sphinxupquote{Ke}},
is computed according to
\begin{equation*}
\begin{split}\mathbf{K}^e = \mathbf{G}^T\,\mathbf{\bar{K}}^e\,\mathbf{G}\end{split}
\end{equation*}
\sphinxAtStartPar
where
\begin{equation*}
\begin{split}\mathbf{\bar{K}}^e = \frac{D_{EA}}{L}
\begin{bmatrix}
1 & 0 & -1 & 0 \\
0 & 0 & 0 & 0 \\
-1 & 0 & 1 & 0 \\
0 & 0 & 0 & 0
\end{bmatrix}
+
\frac{Q_{\bar{x}}}{L}
\begin{bmatrix}
0 & 0 & 0 & 0 \\
0 & 1 & 0 & -1 \\
0 & 0 & 0 & 0 \\
0 & -1 & 0 & 1
\end{bmatrix}\end{split}
\end{equation*}\begin{equation*}
\begin{split}\mathbf{G} =
\begin{bmatrix}
n_{x\bar{x}} & n_{y\bar{x}} & 0 & 0 \\
n_{x\bar{y}} & n_{y\bar{y}} & 0 & 0 \\
0 & 0 & n_{x\bar{x}} & n_{y\bar{x}} \\
0 & 0 & n_{x\bar{y}} & n_{y\bar{y}}
\end{bmatrix}\end{split}
\end{equation*}
\sphinxAtStartPar
where the axial stiffness \(D_{EA}\) and the length \(L\) are given by
\begin{equation*}
\begin{split}D_{EA} = EA \qquad L = \sqrt{(x_2 - x_1)^2 + (y_2 - y_1)^2}\end{split}
\end{equation*}
\sphinxAtStartPar
and the transformation matrix \(\mathbf{G}\) contains the direction cosines
\begin{equation*}
\begin{split}n_{x\bar{x}} = n_{y\bar{y}} = \frac{x_2 - x_1}{L} \qquad
n_{y\bar{x}} = -n_{x\bar{y}} = \frac{y_2 - y_1}{L}\end{split}
\end{equation*}
\end{description}\end{quote}


\subsection{bar2gs}
\label{\detokenize{bar_functions:bar2gs}}\begin{quote}\begin{description}
\sphinxlineitem{Purpose}\begin{quote}

\sphinxAtStartPar
Compute axial force and normal force in a two dimensional bar element.
\end{quote}

\begin{figure}[htbp]
\centering

\noindent\sphinxincludegraphics[width=0.700\linewidth]{{bar2s}.png}
\end{figure}

\sphinxlineitem{Syntax}
\begin{sphinxVerbatim}[commandchars=\\\{\}]
\PYG{p}{[}\PYG{n}{es}\PYG{p}{,}\PYG{+w}{ }\PYG{n}{Qx}\PYG{p}{]}\PYG{+w}{ }\PYG{p}{=}\PYG{+w}{ }\PYG{n}{bar2gs}\PYG{p}{(}\PYG{n}{ex}\PYG{p}{,}\PYG{+w}{ }\PYG{n}{ey}\PYG{p}{,}\PYG{+w}{ }\PYG{n}{ep}\PYG{p}{,}\PYG{+w}{ }\PYG{n}{ed}\PYG{p}{)}
\PYG{p}{[}\PYG{n}{es}\PYG{p}{,}\PYG{+w}{ }\PYG{n}{Qx}\PYG{p}{]}\PYG{+w}{ }\PYG{p}{=}\PYG{+w}{ }\PYG{n}{bar2gs}\PYG{p}{(}\PYG{n}{ex}\PYG{p}{,}\PYG{+w}{ }\PYG{n}{ey}\PYG{p}{,}\PYG{+w}{ }\PYG{n}{ep}\PYG{p}{,}\PYG{+w}{ }\PYG{n}{ed}\PYG{p}{,}\PYG{+w}{ }\PYG{n+nb}{eq}\PYG{p}{)}
\PYG{p}{[}\PYG{n}{es}\PYG{p}{,}\PYG{+w}{ }\PYG{n}{Qx}\PYG{p}{,}\PYG{+w}{ }\PYG{n}{edi}\PYG{p}{]}\PYG{+w}{ }\PYG{p}{=}\PYG{+w}{ }\PYG{n}{bar2gs}\PYG{p}{(}\PYG{n}{ex}\PYG{p}{,}\PYG{+w}{ }\PYG{n}{ey}\PYG{p}{,}\PYG{+w}{ }\PYG{n}{ep}\PYG{p}{,}\PYG{+w}{ }\PYG{n}{ed}\PYG{p}{,}\PYG{+w}{ }\PYG{n+nb}{eq}\PYG{p}{,}\PYG{+w}{ }\PYG{n}{n}\PYG{p}{)}
\PYG{p}{[}\PYG{n}{es}\PYG{p}{,}\PYG{+w}{ }\PYG{n}{Qx}\PYG{p}{,}\PYG{+w}{ }\PYG{n}{edi}\PYG{p}{,}\PYG{+w}{ }\PYG{n}{eci}\PYG{p}{]}\PYG{+w}{ }\PYG{p}{=}\PYG{+w}{ }\PYG{n}{bar2gs}\PYG{p}{(}\PYG{n}{ex}\PYG{p}{,}\PYG{+w}{ }\PYG{n}{ey}\PYG{p}{,}\PYG{+w}{ }\PYG{n}{ep}\PYG{p}{,}\PYG{+w}{ }\PYG{n}{ed}\PYG{p}{,}\PYG{+w}{ }\PYG{n+nb}{eq}\PYG{p}{,}\PYG{+w}{ }\PYG{n}{n}\PYG{p}{)}
\end{sphinxVerbatim}

\sphinxlineitem{Description}
\sphinxAtStartPar
\sphinxcode{\sphinxupquote{bar2gs}} computes the normal force in the two dimensional bar elements \sphinxcode{\sphinxupquote{bar2g}}.

\sphinxAtStartPar
The input variables \sphinxcode{\sphinxupquote{ex}}, \sphinxcode{\sphinxupquote{ey}}, and \sphinxcode{\sphinxupquote{ep}} are defined in \sphinxcode{\sphinxupquote{bar2ge}} and the element nodal displacements, stored in \sphinxcode{\sphinxupquote{ed}}, are obtained by the function \sphinxcode{\sphinxupquote{extract}}.
The number of evaluation points for section forces and displacements are determined by \sphinxcode{\sphinxupquote{n}}. If \sphinxcode{\sphinxupquote{n}} is omitted, only the ends of the bar are evaluated.

\sphinxAtStartPar
The output variable \sphinxcode{\sphinxupquote{Qx}} contains the axial force \(Q_{\bar{x}}\) and the output variables
\begin{equation*}
\begin{split}\mathsf{es} =
\left[
\begin{array}{c}
N(0)    \\
N(\bar{x}_{2})\\
\vdots  \\
N(\bar{x}_{n-1})\\
N(L)
\end{array}
  \right]
\qquad
\mathsf{edi} =
\left[
\begin{array}{c}
{u}(0)    \\
{u}(\bar{x}_{2})\\
\vdots  \\
{u}(\bar{x}_{n-1})\\
{u}(L)
\end{array}
  \right]
\qquad
\mathsf{eci} =
\left[
\begin{array}{c}
0  \\
\bar x_{2} \\
\vdots   \\
\bar x_{n-1} \\
L
\end{array}
  \right]\end{split}
\end{equation*}
\sphinxAtStartPar
contain the normal force, the displacement, and the evaluation points on the local \(\bar{x}\)\sphinxhyphen{}axis.
\(L\) is the length of the bar element.

\sphinxlineitem{Theory}
\sphinxAtStartPar
The nodal displacements in global coordinates are given by
\begin{equation*}
\begin{split}\mathbf{a}^e = \left[\; u_1\;\; u_2\;\; u_3\;\; u_4 \;\right]^T\end{split}
\end{equation*}
\sphinxAtStartPar
The transpose of \(\mathbf{a}^e\) is stored in \sphinxcode{\sphinxupquote{ed}}.
The nodal displacements in local coordinates are given by
\begin{equation*}
\begin{split}\mathbf{\bar{a}}^e = \mathbf{G} \mathbf{a}^e\end{split}
\end{equation*}
\sphinxAtStartPar
where the transformation matrix \(\mathbf{G}\) is defined in \sphinxcode{\sphinxupquote{bar2ge}}.
The displacements associated with bar action are determined as
\begin{equation*}
\begin{split}{\mathbf{\bar{a}}}^e_{\text{bar}} = \left[ \begin{array}{r} \bar{u}_1 \\ \bar{u}_3 \end{array}\right]\end{split}
\end{equation*}
\sphinxAtStartPar
The displacement \(u(\bar{x})\) and the normal force \(N(\bar{x})\) are computed from
\begin{equation*}
\begin{split}u(\bar{x}) = {\mathbf{N}} \mathbf{\bar{a}}^e_{\text{bar}}\end{split}
\end{equation*}\begin{equation*}
\begin{split}N(\bar{x}) = D_{EA} \mathbf{B} \mathbf{\bar{a}}^e_{\text{bar}}\end{split}
\end{equation*}
\sphinxAtStartPar
where
\begin{equation*}
\begin{split}\mathbf{N} = \left[\begin{array}{rr} 1 & \bar{x} \end{array}\right] \mathbf{C}^{-1} = \left[\begin{array}{rr} 1-\frac{\bar{x}}{L} & \frac{\bar{x}}{L} \end{array}\right]\end{split}
\end{equation*}\begin{equation*}
\begin{split}\mathbf{B} = \left[\begin{array}{rr} 0 & 1 \end{array}\right] \mathbf{C}^{-1} = \frac{1}{L}\left[\begin{array}{rr} -1 & 1 \end{array}\right]\end{split}
\end{equation*}
\sphinxAtStartPar
where \(D_{EA}\) and \(L\) are defined in \sphinxcode{\sphinxupquote{bar2ge}} and
\begin{equation*}
\begin{split}\mathbf{C}^{-1} = \left[ \begin{array}{rr} 1 & 0 \\ -\frac{1}{L} & \frac{1}{L} \end{array}\right]\end{split}
\end{equation*}
\sphinxAtStartPar
An updated value of the axial force is computed as
\begin{equation*}
\begin{split}Q_{\bar{x}} = N(0)\end{split}
\end{equation*}
\end{description}\end{quote}


\section{3D bar elements}
\label{\detokenize{bar_functions:id2}}

\subsection{bar3e}
\label{\detokenize{bar_functions:bar3e}}
\index{bar3e@\spxentry{bar3e}}\ignorespaces \begin{quote}\begin{description}
\sphinxlineitem{Purpose}
\sphinxAtStartPar
Compute element stiffness matrix for a three dimensional bar element.

\begin{figure}[htbp]
\centering

\noindent\sphinxincludegraphics[width=0.700\linewidth]{{bar3e}.png}
\end{figure}

\sphinxlineitem{Syntax}
\begin{sphinxVerbatim}[commandchars=\\\{\}]
\PYG{n}{Ke}\PYG{+w}{ }\PYG{p}{=}\PYG{+w}{ }\PYG{n}{bar3e}\PYG{p}{(}\PYG{n}{ex}\PYG{p}{,}\PYG{+w}{ }\PYG{n}{ey}\PYG{p}{,}\PYG{+w}{ }\PYG{n}{ez}\PYG{p}{,}\PYG{+w}{ }\PYG{n}{ep}\PYG{p}{)}
\PYG{p}{[}\PYG{n}{Ke}\PYG{p}{,}\PYG{+w}{ }\PYG{n}{fe}\PYG{p}{]}\PYG{+w}{ }\PYG{p}{=}\PYG{+w}{ }\PYG{n}{bar3e}\PYG{p}{(}\PYG{n}{ex}\PYG{p}{,}\PYG{+w}{ }\PYG{n}{ey}\PYG{p}{,}\PYG{+w}{ }\PYG{n}{ez}\PYG{p}{,}\PYG{+w}{ }\PYG{n}{ep}\PYG{p}{,}\PYG{+w}{ }\PYG{n+nb}{eq}\PYG{p}{)}
\end{sphinxVerbatim}

\sphinxlineitem{Description}
\sphinxAtStartPar
\sphinxcode{\sphinxupquote{bar3e}} provides the global element stiffness matrix \sphinxcode{\sphinxupquote{Ke}} for a three dimensional bar element.

\sphinxAtStartPar
The input variables
\begin{equation*}
\begin{split}\begin{array}{l}
\mathrm{ex} = [x_1 \;\; x_2] \\
\mathrm{ey} = [y_1 \;\; y_2] \\
\mathrm{ez} = [z_1 \;\; z_2]
\end{array}
\qquad
\mathrm{ep} = [E \;\; A]\end{split}
\end{equation*}
\sphinxAtStartPar
supply the element nodal coordinates \(x_1\), \(y_1\), \(z_1\), \(x_2\), \(y_2\), and \(z_2\), the modulus of elasticity \(E\), and the cross section area \(A\).

\sphinxAtStartPar
The element load vector \sphinxcode{\sphinxupquote{fe}} can also be computed if a uniformly distributed axial load is applied to the element. The optional input variable
\begin{equation*}
\begin{split}\mathrm{eq} = [q_{\bar{x}}]\end{split}
\end{equation*}
\sphinxAtStartPar
contains the distributed load per unit length, \(q_{\bar{x}}\).

\sphinxlineitem{Theory}
\sphinxAtStartPar
The element stiffness matrix \(\mathbf{K}^e\), stored in \sphinxcode{\sphinxupquote{Ke}}, is computed according to
\begin{equation*}
\begin{split}\mathbf{K}^e = \mathbf{G}^T \; \bar{\mathbf{K}}^e \; \mathbf{G}\end{split}
\end{equation*}
\sphinxAtStartPar
where
\begin{equation*}
\begin{split}\bar{\mathbf{K}}^e = \frac{D_{EA}}{L}
\begin{bmatrix}
1 & -1 \\
-1 & 1
\end{bmatrix}
\qquad
\mathbf{G} =
\begin{bmatrix}
n_{x\bar{x}} & n_{y\bar{x}} & n_{z\bar{x}} & 0 & 0 & 0 \\
0 & 0 & 0 & n_{x\bar{x}} & n_{y\bar{x}} & n_{z\bar{x}}
\end{bmatrix}\end{split}
\end{equation*}
\sphinxAtStartPar
where the axial stiffness \(D_{EA}\) and the length \(L\) are given by
\begin{equation*}
\begin{split}D_{EA} = EA \qquad
L = \sqrt{(x_2 - x_1)^2 + (y_2 - y_1)^2 + (z_2 - z_1)^2}\end{split}
\end{equation*}
\sphinxAtStartPar
and the transformation matrix \(\mathbf{G}\) contains the direction cosines
\begin{equation*}
\begin{split}n_{x\bar{x}} = \frac{x_2 - x_1}{L} \qquad
n_{y\bar{x}} = \frac{y_2 - y_1}{L} \qquad
n_{z\bar{x}} = \frac{z_2 - z_1}{L}\end{split}
\end{equation*}
\sphinxAtStartPar
The element load vector \(\mathbf{f}_l^e\), stored in \sphinxcode{\sphinxupquote{fe}}, is computed according to
\begin{equation*}
\begin{split}\mathbf{f}_l^e = \mathbf{G}^T \; \bar{\mathbf{f}}_l^e\end{split}
\end{equation*}
\sphinxAtStartPar
where
\begin{equation*}
\begin{split}\bar{\mathbf{f}}_l^e = \frac{q_{\bar{x}} L}{2}
\begin{bmatrix}
1 \\
1
\end{bmatrix}\end{split}
\end{equation*}
\end{description}\end{quote}


\subsection{bar3s}
\label{\detokenize{bar_functions:bar3s}}
\index{bar3s@\spxentry{bar3s}}\ignorespaces \begin{quote}\begin{description}
\sphinxlineitem{Purpose}
\sphinxAtStartPar
Compute normal force in a three dimensional bar element.

\begin{figure}[htbp]
\centering

\noindent\sphinxincludegraphics[width=0.700\linewidth]{{bar3s}.png}
\end{figure}

\sphinxlineitem{Syntax}
\begin{sphinxVerbatim}[commandchars=\\\{\}]
\PYG{n}{es}\PYG{+w}{ }\PYG{p}{=}\PYG{+w}{ }\PYG{n}{bar3s}\PYG{p}{(}\PYG{n}{ex}\PYG{p}{,}\PYG{+w}{ }\PYG{n}{ey}\PYG{p}{,}\PYG{+w}{ }\PYG{n}{ez}\PYG{p}{,}\PYG{+w}{ }\PYG{n}{ep}\PYG{p}{,}\PYG{+w}{ }\PYG{n}{ed}\PYG{p}{)}
\PYG{n}{es}\PYG{+w}{ }\PYG{p}{=}\PYG{+w}{ }\PYG{n}{bar3s}\PYG{p}{(}\PYG{n}{ex}\PYG{p}{,}\PYG{+w}{ }\PYG{n}{ey}\PYG{p}{,}\PYG{+w}{ }\PYG{n}{ez}\PYG{p}{,}\PYG{+w}{ }\PYG{n}{ep}\PYG{p}{,}\PYG{+w}{ }\PYG{n}{ed}\PYG{p}{,}\PYG{+w}{ }\PYG{n+nb}{eq}\PYG{p}{)}
\PYG{p}{[}\PYG{n}{es}\PYG{p}{,}\PYG{+w}{ }\PYG{n}{edi}\PYG{p}{]}\PYG{+w}{ }\PYG{p}{=}\PYG{+w}{ }\PYG{n}{bar3s}\PYG{p}{(}\PYG{n}{ex}\PYG{p}{,}\PYG{+w}{ }\PYG{n}{ey}\PYG{p}{,}\PYG{+w}{ }\PYG{n}{ez}\PYG{p}{,}\PYG{+w}{ }\PYG{n}{ep}\PYG{p}{,}\PYG{+w}{ }\PYG{n}{ed}\PYG{p}{,}\PYG{+w}{ }\PYG{n+nb}{eq}\PYG{p}{,}\PYG{+w}{ }\PYG{n}{n}\PYG{p}{)}
\PYG{p}{[}\PYG{n}{es}\PYG{p}{,}\PYG{+w}{ }\PYG{n}{edi}\PYG{p}{,}\PYG{+w}{ }\PYG{n}{eci}\PYG{p}{]}\PYG{+w}{ }\PYG{p}{=}\PYG{+w}{ }\PYG{n}{bar3s}\PYG{p}{(}\PYG{n}{ex}\PYG{p}{,}\PYG{+w}{ }\PYG{n}{ey}\PYG{p}{,}\PYG{+w}{ }\PYG{n}{ez}\PYG{p}{,}\PYG{+w}{ }\PYG{n}{ep}\PYG{p}{,}\PYG{+w}{ }\PYG{n}{ed}\PYG{p}{,}\PYG{+w}{ }\PYG{n+nb}{eq}\PYG{p}{,}\PYG{+w}{ }\PYG{n}{n}\PYG{p}{)}
\end{sphinxVerbatim}

\sphinxlineitem{Description}
\sphinxAtStartPar
\sphinxcode{\sphinxupquote{bar3s}} computes the normal force in a three dimensional bar element (see \sphinxcode{\sphinxupquote{bar3e}}).

\sphinxAtStartPar
The input variables \sphinxcode{\sphinxupquote{ex}}, \sphinxcode{\sphinxupquote{ey}}, and \sphinxcode{\sphinxupquote{ep}} are defined in \sphinxcode{\sphinxupquote{bar3e}} and the element nodal displacements, stored in \sphinxcode{\sphinxupquote{ed}}, are obtained by the function \sphinxcode{\sphinxupquote{extract}}.
The number of evaluation points for section forces and displacements are determined by \sphinxcode{\sphinxupquote{n}}. If \sphinxcode{\sphinxupquote{n}} is omitted, only the ends of the bar are evaluated.

\sphinxAtStartPar
The output variables:
\begin{equation*}
\begin{split}\mathrm{es} =
\begin{bmatrix}
N(0) \\
N(\bar{x}_2) \\
\vdots \\
N(\bar{x}_{n-1}) \\
N(L)
\end{bmatrix}
\qquad
\mathrm{edi} =
\begin{bmatrix}
u(0) \\
u(\bar{x}_2) \\
\vdots \\
u(\bar{x}_{n-1}) \\
u(L)
\end{bmatrix}
\qquad
\mathrm{eci} =
\begin{bmatrix}
0 \\
\bar{x}_2 \\
\vdots \\
\bar{x}_{n-1} \\
L
\end{bmatrix}\end{split}
\end{equation*}
\sphinxAtStartPar
contain the normal force, the displacement, and the evaluation points on the local \(\bar{x}\)\sphinxhyphen{}axis.
\(L\) is the length of the bar element.

\sphinxlineitem{Theory}
\sphinxAtStartPar
The nodal displacements in global coordinates are given by
\begin{equation*}
\begin{split}\mathbf{a}^e = \begin{bmatrix} u_1 & u_2 & u_3 & u_4 & u_5 & u_6 \end{bmatrix}^T\end{split}
\end{equation*}
\sphinxAtStartPar
The transpose of \(\mathbf{a}^e\) is stored in \sphinxcode{\sphinxupquote{ed}}.

\sphinxAtStartPar
The nodal displacements in local coordinates are given by
\begin{equation*}
\begin{split}\mathbf{\bar{a}}^e = \mathbf{G} \mathbf{a}^e\end{split}
\end{equation*}
\sphinxAtStartPar
where the transformation matrix \(\mathbf{G}\) is defined in \sphinxcode{\sphinxupquote{bar3e}}.

\sphinxAtStartPar
The displacement \(u(\bar{x})\) and the normal force \(N(\bar{x})\) are computed from
\begin{equation*}
\begin{split}u(\bar{x}) = \mathbf{N} \mathbf{\bar{a}}^e + u_p(\bar{x})\end{split}
\end{equation*}\begin{equation*}
\begin{split}N(\bar{x}) = D_{EA} \mathbf{B} \mathbf{\bar{a}}^e + N_p(\bar{x})\end{split}
\end{equation*}
\sphinxAtStartPar
where
\begin{equation*}
\begin{split}\mathbf{N} = \begin{bmatrix} 1 & \bar{x} \end{bmatrix} \mathbf{C}^{-1} = \begin{bmatrix} 1-\frac{\bar{x}}{L} & \frac{\bar{x}}{L} \end{bmatrix}\end{split}
\end{equation*}\begin{equation*}
\begin{split}\mathbf{B} = \begin{bmatrix} 0 & 1 \end{bmatrix} \mathbf{C}^{-1} = \frac{1}{L} \begin{bmatrix} -1 & 1 \end{bmatrix}\end{split}
\end{equation*}\begin{equation*}
\begin{split}u_p(\bar{x}) = -\frac{q_{\bar{x}}}{D_{EA}} \left( \frac{\bar{x}^2}{2} - \frac{L\bar{x}}{2} \right)\end{split}
\end{equation*}\begin{equation*}
\begin{split}N_p(\bar{x}) = -q_{\bar{x}} \left( \bar{x} - \frac{L}{2} \right)\end{split}
\end{equation*}
\sphinxAtStartPar
where \(D_{EA}\), \(L\), \(q_{\bar{x}}\) are defined in \sphinxcode{\sphinxupquote{bar3e}} and
\begin{equation*}
\begin{split}\mathbf{C}^{-1} = \begin{bmatrix} 1 & 0 \\ -\frac{1}{L} & \frac{1}{L} \end{bmatrix}\end{split}
\end{equation*}
\end{description}\end{quote}

\sphinxstepscope


\chapter{Heat Flow Elements}
\label{\detokenize{heat_functions:heat-flow-elements}}\label{\detokenize{heat_functions::doc}}
\sphinxAtStartPar
Heat flow elements are available for one, two, and three dimensional analysis.
For one dimensional heat flow the spring element \sphinxcode{\sphinxupquote{spring1}} is used.

\sphinxAtStartPar
A variety of important physical phenomena are described by
the same differential equation as the heat flow problem. The heat flow element
is thus applicable in modelling different physical applications.
Table below shows the relation between the primary variable \sphinxstylestrong{a},
the constitutive matrix \sphinxstylestrong{D}, and the load vector \sphinxstylestrong{f\_l}
for a chosen set of two dimensional physical problems.


\begin{savenotes}\sphinxattablestart
\sphinxthistablewithglobalstyle
\centering
\sphinxcapstartof{table}
\sphinxthecaptionisattop
\sphinxcaption{Problem dependent parameters}\label{\detokenize{heat_functions:id2}}
\sphinxaftertopcaption
\begin{tabular}[t]{\X{20}{100}\X{15}{100}\X{20}{100}\X{15}{100}\X{30}{100}}
\sphinxtoprule
\sphinxstyletheadfamily 
\sphinxAtStartPar
Problem type
&\sphinxstyletheadfamily 
\sphinxAtStartPar
\sphinxstylestrong{a}
&\sphinxstyletheadfamily 
\sphinxAtStartPar
\sphinxstylestrong{D}
&\sphinxstyletheadfamily 
\sphinxAtStartPar
\sphinxstylestrong{f\_l}
&\sphinxstyletheadfamily 
\sphinxAtStartPar
Designation
\\
\sphinxmidrule
\sphinxtableatstartofbodyhook
\sphinxAtStartPar
Heat flow
&
\sphinxAtStartPar
\(T\)
&
\sphinxAtStartPar
\(\lambda_x\), \(\lambda_y\)
&
\sphinxAtStartPar
\(Q\)
&\begin{description}
\sphinxlineitem{\(T\) = temperature}
\sphinxAtStartPar
\(\lambda_x\), \(\lambda_y\) = thermal conductivity
\(Q\) = heat supply

\end{description}
\\
\sphinxhline
\sphinxAtStartPar
Groundwater flow
&
\sphinxAtStartPar
\(\phi\)
&
\sphinxAtStartPar
\(k_x\), \(k_y\)
&
\sphinxAtStartPar
\(Q\)
&\begin{description}
\sphinxlineitem{\(\phi\) = piezometric head}
\sphinxAtStartPar
\(k_x\), \(k_y\) = permeabilities
\(Q\) = fluid supply

\end{description}
\\
\sphinxhline
\sphinxAtStartPar
St. Venant torsion
&
\sphinxAtStartPar
\(\phi\)
&
\sphinxAtStartPar
\(1/G_{zy}\), \(1/G_{zx}\)
&
\sphinxAtStartPar
\(2\Theta\)
&\begin{description}
\sphinxlineitem{\(\phi\) = stress function}
\sphinxAtStartPar
\(G_{zy}\), \(G_{zx}\) = shear moduli
\(\Theta\) = angle of torsion per unit length

\end{description}
\\
\sphinxbottomrule
\end{tabular}
\sphinxtableafterendhook\par
\sphinxattableend\end{savenotes}


\section{Element types}
\label{\detokenize{heat_functions:element-types}}

\begin{savenotes}\sphinxattablestart
\sphinxthistablewithglobalstyle
\centering
\begin{tabular}[t]{\X{50}{100}\X{50}{100}}
\sphinxtoprule
\sphinxtableatstartofbodyhook
\noindent{\hspace*{\fill}\sphinxincludegraphics[width=0.500\linewidth]{{FLW2T}.png}\hspace*{\fill}}

\sphinxAtStartPar
\sphinxcode{\sphinxupquote{flw2te}}
&
\noindent{\hspace*{\fill}\sphinxincludegraphics[width=0.500\linewidth]{{FLW2I4}.png}\hspace*{\fill}}

\sphinxAtStartPar
\sphinxcode{\sphinxupquote{flw2qe}}, \sphinxcode{\sphinxupquote{flw2i4e}}
\\
\sphinxhline
\noindent{\hspace*{\fill}\sphinxincludegraphics[width=0.500\linewidth]{{FLW2I8}.png}\hspace*{\fill}}

\sphinxAtStartPar
\sphinxcode{\sphinxupquote{flw2i8e}}
&
\noindent{\hspace*{\fill}\sphinxincludegraphics[width=0.500\linewidth]{{FLW3I8}.png}\hspace*{\fill}}

\sphinxAtStartPar
\sphinxcode{\sphinxupquote{flw3i8e}}
\\
\sphinxbottomrule
\end{tabular}
\sphinxtableafterendhook\par
\sphinxattableend\end{savenotes}


\section{2D Heat Flow Functions}
\label{\detokenize{heat_functions:d-heat-flow-functions}}

\subsection{flw2te}
\label{\detokenize{heat_functions:flw2te}}\begin{quote}\begin{description}
\sphinxlineitem{Purpose}
\sphinxAtStartPar
Compute element stiffness matrix for a triangular heat flow element.

\begin{figure}[htbp]
\centering

\noindent\sphinxincludegraphics[width=0.700\linewidth]{{FLW2TE}.png}
\end{figure}

\sphinxlineitem{Syntax}
\begin{sphinxVerbatim}[commandchars=\\\{\}]
\PYG{n}{Ke}\PYG{+w}{ }\PYG{p}{=}\PYG{+w}{ }\PYG{n}{flw2te}\PYG{p}{(}\PYG{n}{ex}\PYG{p}{,}\PYG{+w}{ }\PYG{n}{ey}\PYG{p}{,}\PYG{+w}{ }\PYG{n}{ep}\PYG{p}{,}\PYG{+w}{ }\PYG{n}{D}\PYG{p}{)}
\PYG{p}{[}\PYG{n}{Ke}\PYG{p}{,}\PYG{+w}{ }\PYG{n}{fe}\PYG{p}{]}\PYG{+w}{ }\PYG{p}{=}\PYG{+w}{ }\PYG{n}{flw2te}\PYG{p}{(}\PYG{n}{ex}\PYG{p}{,}\PYG{+w}{ }\PYG{n}{ey}\PYG{p}{,}\PYG{+w}{ }\PYG{n}{ep}\PYG{p}{,}\PYG{+w}{ }\PYG{n}{D}\PYG{p}{,}\PYG{+w}{ }\PYG{n+nb}{eq}\PYG{p}{)}
\end{sphinxVerbatim}

\sphinxlineitem{Description}
\sphinxAtStartPar
\sphinxcode{\sphinxupquote{flw2te}} provides the element stiffness (conductivity) matrix \sphinxcode{\sphinxupquote{Ke}} and
the element load vector \sphinxcode{\sphinxupquote{fe}} for a triangular heat flow element.

\sphinxAtStartPar
The element nodal coordinates \(x_1\), \(y_1\), \(x_2\) etc,
are supplied to the function by \sphinxcode{\sphinxupquote{ex}} and \sphinxcode{\sphinxupquote{ey}}, the element thickness \(t\)
is supplied by \sphinxcode{\sphinxupquote{ep}} and the thermal conductivities (or corresponding quantities)
\(k_{xx}\), \(k_{xy}\) etc are supplied by \sphinxcode{\sphinxupquote{D}}.
\begin{equation*}
\begin{split}\begin{array}{l}
\mathbf{ex} = [\, x_1 \;\; x_2 \;\; x_3\,] \\
\mathbf{ey} = [\, y_1 \;\; y_2 \;\; y_3\,]
\end{array}
\qquad
\mathbf{ep} = [\, t \,]
\qquad
\mathbf{D} = \begin{bmatrix}
    k_{xx} & k_{xy} \\
    k_{yx} & k_{yy}
\end{bmatrix}\end{split}
\end{equation*}
\sphinxAtStartPar
If the scalar variable \sphinxcode{\sphinxupquote{eq}} is given in the function, the
element load vector \(\mathbf{fe}\)
is computed, using
\begin{equation*}
\begin{split}\mathbf{eq} = [\, Q \,]\end{split}
\end{equation*}
\sphinxAtStartPar
where \(Q\) is the heat supply per unit volume.

\sphinxlineitem{Theory}
\sphinxAtStartPar
The element stiffness matrix \(\mathbf{K}^e\) and the
element load vector \(\mathbf{f}_l^e\), stored in \sphinxcode{\sphinxupquote{Ke}} and \sphinxcode{\sphinxupquote{fe}},
respectively, are computed according to
\begin{equation*}
\begin{split}\mathbf{K}^e = (\mathbf{C}^{-1})^T \int_A \bar{\mathbf{B}}^T
\mathbf{D} \bar{\mathbf{B}}\, t\, dA\, \mathbf{C}^{-1}\end{split}
\end{equation*}\begin{equation*}
\begin{split}\mathbf{f}_l^e = (\mathbf{C}^{-1})^T \int_A \bar{\mathbf{N}}^T Q\, t\, dA\end{split}
\end{equation*}
\sphinxAtStartPar
with the constitutive matrix \(\mathbf{D}\) defined by \sphinxcode{\sphinxupquote{D}}.

\sphinxAtStartPar
The evaluation of the integrals for the triangular element is based on the linear
temperature approximation \(T(x, y)\) and is expressed
in terms of the nodal variables \(T_1\), \(T_2\) and \(T_3\) as
\begin{equation*}
\begin{split}T(x, y) = \mathbf{N}^e \mathbf{a}^e = \bar{\mathbf{N}}\, \mathbf{C}^{-1} \mathbf{a}^e\end{split}
\end{equation*}
\sphinxAtStartPar
where
\begin{equation*}
\begin{split}\bar{\mathbf{N}} = [\, 1 \;\; x \;\; y\,]
\qquad
\mathbf{C} = \begin{bmatrix}
    1 & x_1 & y_1 \\
    1 & x_2 & y_2 \\
    1 & x_3 & y_3
\end{bmatrix}
\qquad
\mathbf{a}^e = \begin{bmatrix}
    T_1 \\
    T_2 \\
    T_3
\end{bmatrix}\end{split}
\end{equation*}
\sphinxAtStartPar
and hence it follows that
\begin{equation*}
\begin{split}\bar{\mathbf{B}} = \nabla \bar{\mathbf{N}} = \begin{bmatrix}
    0 & 1 & 0 \\
    0 & 0 & 1
\end{bmatrix}
\qquad
\nabla = \begin{bmatrix}
    \dfrac{\partial}{\partial x} \\
    \dfrac{\partial}{\partial y}
\end{bmatrix}\end{split}
\end{equation*}
\sphinxAtStartPar
Evaluation of the integrals for the triangular element yields
\begin{equation*}
\begin{split}\mathbf{K}^e = (\mathbf{C}^{-1})^T \bar{\mathbf{B}}^T
\mathbf{D} \bar{\mathbf{B}}
\mathbf{C}^{-1} t A\end{split}
\end{equation*}\begin{equation*}
\begin{split}\mathbf{f}_l^e = \frac{Q A t}{3} \begin{bmatrix} 1 \\ 1 \\ 1 \end{bmatrix}\end{split}
\end{equation*}
\sphinxAtStartPar
where the element area \(A\) is determined as
\begin{equation*}
\begin{split}A = \frac{1}{2} \det \mathbf{C}\end{split}
\end{equation*}
\end{description}\end{quote}


\subsection{flw2ts}
\label{\detokenize{heat_functions:flw2ts}}\begin{quote}\begin{description}
\sphinxlineitem{Purpose}
\sphinxAtStartPar
Compute heat flux and temperature gradients in a triangular heat flow element.

\sphinxlineitem{Syntax}
\begin{sphinxVerbatim}[commandchars=\\\{\}]
\PYG{p}{[}\PYG{n}{es}\PYG{p}{,}\PYG{+w}{ }\PYG{n}{et}\PYG{p}{]}\PYG{+w}{ }\PYG{p}{=}\PYG{+w}{ }\PYG{n}{flw2ts}\PYG{p}{(}\PYG{n}{ex}\PYG{p}{,}\PYG{+w}{ }\PYG{n}{ey}\PYG{p}{,}\PYG{+w}{ }\PYG{n}{D}\PYG{p}{,}\PYG{+w}{ }\PYG{n}{ed}\PYG{p}{)}
\end{sphinxVerbatim}

\sphinxlineitem{Description}
\sphinxAtStartPar
\sphinxcode{\sphinxupquote{flw2ts}} computes the heat flux vector \sphinxcode{\sphinxupquote{es}} and the temperature gradient \sphinxcode{\sphinxupquote{et}} (or corresponding quantities) in a triangular heat flow element.

\sphinxAtStartPar
The input variables \sphinxcode{\sphinxupquote{ex}}, \sphinxcode{\sphinxupquote{ey}} and the matrix \sphinxcode{\sphinxupquote{D}} are defined in \sphinxcode{\sphinxupquote{flw2te}}. The vector \sphinxcode{\sphinxupquote{ed}} contains the nodal temperatures \(\mathbf{a}^e\) of the element and is obtained by the function \sphinxcode{\sphinxupquote{extract}} as
\begin{equation*}
\begin{split}\mathbf{ed} = (\mathbf{a}^e)^T = [\;T_1\;\; T_2\;\; T_3\;]\end{split}
\end{equation*}
\sphinxAtStartPar
The output variables
\begin{equation*}
\begin{split}\mathbf{es} = \mathbf{q}^T = \left[\; q_x \; q_y \;\right]\end{split}
\end{equation*}\begin{equation*}
\begin{split}\mathbf{et} = (\nabla T)^T = \left[\begin{array}{l}
\frac{\partial T}{\partial x}\;\;\frac{\partial T}{\partial y}
\end{array} \right]\end{split}
\end{equation*}
\sphinxAtStartPar
contain the components of the heat flux and the temperature gradient computed in the directions of the coordinate axis.

\sphinxlineitem{Theory}
\sphinxAtStartPar
The temperature gradient and the heat flux are computed according to
\begin{equation*}
\begin{split}\nabla T = \bar{\mathbf{B}}\;\mathbf{C}^{-1}\;\mathbf{a}^e\end{split}
\end{equation*}\begin{equation*}
\begin{split}\mathbf{q} = - \mathbf{D} \nabla T\end{split}
\end{equation*}
\sphinxAtStartPar
where the matrices \(\mathbf{D}\), \(\bar{\mathbf{B}}\), and \(\mathbf{C}\) are described in \sphinxcode{\sphinxupquote{flw2te}}. Note that both the temperature gradient and the heat flux are constant in the element.

\end{description}\end{quote}


\subsection{flw2qe}
\label{\detokenize{heat_functions:flw2qe}}\begin{quote}\begin{description}
\sphinxlineitem{Purpose}\begin{quote}

\sphinxAtStartPar
Compute element stiffness matrix for a quadrilateral heat flow element.
\end{quote}

\begin{figure}[htbp]
\centering

\noindent\sphinxincludegraphics[width=0.700\linewidth]{{f14}.png}
\end{figure}

\sphinxlineitem{Syntax}
\begin{sphinxVerbatim}[commandchars=\\\{\}]
\PYG{n}{Ke}\PYG{+w}{ }\PYG{p}{=}\PYG{+w}{ }\PYG{n}{flw2qe}\PYG{p}{(}\PYG{n}{ex}\PYG{p}{,}\PYG{+w}{ }\PYG{n}{ey}\PYG{p}{,}\PYG{+w}{ }\PYG{n}{ep}\PYG{p}{,}\PYG{+w}{ }\PYG{n}{D}\PYG{p}{)}
\PYG{p}{[}\PYG{n}{Ke}\PYG{p}{,}\PYG{+w}{ }\PYG{n}{fe}\PYG{p}{]}\PYG{+w}{ }\PYG{p}{=}\PYG{+w}{ }\PYG{n}{flw2qe}\PYG{p}{(}\PYG{n}{ex}\PYG{p}{,}\PYG{+w}{ }\PYG{n}{ey}\PYG{p}{,}\PYG{+w}{ }\PYG{n}{ep}\PYG{p}{,}\PYG{+w}{ }\PYG{n}{D}\PYG{p}{,}\PYG{+w}{ }\PYG{n+nb}{eq}\PYG{p}{)}
\end{sphinxVerbatim}

\sphinxlineitem{Description}
\sphinxAtStartPar
\sphinxcode{\sphinxupquote{flw2qe}} provides the element stiffness (conductivity) matrix \sphinxcode{\sphinxupquote{Ke}} and
the element load vector \sphinxcode{\sphinxupquote{fe}} for a quadrilateral heat flow element.

\sphinxAtStartPar
The element nodal coordinates \(x_1\), \(y_1\), \(x_2\) etc,
are supplied to the function by \sphinxcode{\sphinxupquote{ex}} and \sphinxcode{\sphinxupquote{ey}}, the element thickness \(t\)
is supplied by \sphinxcode{\sphinxupquote{ep}} and the thermal conductivities (or corresponding quantities)
\(k_{xx}\), \(k_{xy}\) etc are supplied by \sphinxcode{\sphinxupquote{D}}.
\begin{equation*}
\begin{split}\begin{array}{l}
\mathbf{ex} = [\, x_1 \;\; x_2 \;\; x_3 \;\; x_4 \,] \\
\mathbf{ey} = [\, y_1 \;\; y_2 \;\; y_3 \;\; y_4 \,]
\end{array}
\qquad
\mathbf{ep} = \left[\, t \,\right]
\qquad
\mathbf{D} = \left[
      \begin{array}{cc}
           k_{xx} & k_{xy} \\
           k_{yx} & k_{yy}
      \end{array}
\right]\end{split}
\end{equation*}
\sphinxAtStartPar
If the scalar variable \sphinxcode{\sphinxupquote{eq}} is given in the function, the element load
vector \(\mathbf{fe}\) is computed, using
\begin{equation*}
\begin{split}\mathbf{eq} = \left[\, Q \,\right]\end{split}
\end{equation*}
\sphinxAtStartPar
where \(Q\) is the heat supply per unit volume.

\sphinxlineitem{Theory}
\sphinxAtStartPar
In computing the element matrices, a fifth degree of freedom is introduced.
The location of this extra degree of freedom is defined by the mean value of the coordinates in
the corner points. Four sets of element matrices are calculated using
\sphinxcode{\sphinxupquote{flw2te}}. These matrices are then assembled and the fifth degree of freedom is eliminated by static condensation.

\end{description}\end{quote}


\subsection{flw2qs}
\label{\detokenize{heat_functions:flw2qs}}\begin{quote}\begin{description}
\sphinxlineitem{Purpose}
\sphinxAtStartPar
Compute heat flux and temperature gradients in a quadrilateral heat flow element.

\sphinxlineitem{Syntax}
\begin{sphinxVerbatim}[commandchars=\\\{\}]
\PYG{p}{[}\PYG{n}{es}\PYG{p}{,}\PYG{+w}{ }\PYG{n}{et}\PYG{p}{]}\PYG{+w}{ }\PYG{p}{=}\PYG{+w}{ }\PYG{n}{flw2qs}\PYG{p}{(}\PYG{n}{ex}\PYG{p}{,}\PYG{+w}{ }\PYG{n}{ey}\PYG{p}{,}\PYG{+w}{ }\PYG{n}{ep}\PYG{p}{,}\PYG{+w}{ }\PYG{n}{D}\PYG{p}{,}\PYG{+w}{ }\PYG{n}{ed}\PYG{p}{)}
\PYG{p}{[}\PYG{n}{es}\PYG{p}{,}\PYG{+w}{ }\PYG{n}{et}\PYG{p}{]}\PYG{+w}{ }\PYG{p}{=}\PYG{+w}{ }\PYG{n}{flw2qs}\PYG{p}{(}\PYG{n}{ex}\PYG{p}{,}\PYG{+w}{ }\PYG{n}{ey}\PYG{p}{,}\PYG{+w}{ }\PYG{n}{ep}\PYG{p}{,}\PYG{+w}{ }\PYG{n}{D}\PYG{p}{,}\PYG{+w}{ }\PYG{n}{ed}\PYG{p}{,}\PYG{+w}{ }\PYG{n+nb}{eq}\PYG{p}{)}
\end{sphinxVerbatim}

\sphinxlineitem{Description}
\sphinxAtStartPar
\sphinxcode{\sphinxupquote{flw2qs}} computes the heat flux vector \sphinxcode{\sphinxupquote{es}} and the temperature gradient \sphinxcode{\sphinxupquote{et}} (or corresponding quantities) in a quadrilateral heat flow element.

\sphinxAtStartPar
The input variables \sphinxcode{\sphinxupquote{ex}}, \sphinxcode{\sphinxupquote{ey}}, \sphinxcode{\sphinxupquote{eq}} and the matrix \sphinxcode{\sphinxupquote{D}} are defined in \sphinxcode{\sphinxupquote{flw2qe}}.
The vector \sphinxcode{\sphinxupquote{ed}} contains the nodal temperatures \(\mathbf{a}^e\) of the element and is obtained by the function \sphinxcode{\sphinxupquote{extract}} as
\begin{equation*}
\begin{split}\mathbf{ed} = (\mathbf{a}^e)^T = [\;T_1\;\; T_2\;\; T_3\;\; T_4\;]\end{split}
\end{equation*}
\sphinxAtStartPar
The output variables
\begin{equation*}
\begin{split}\mathbf{es} = \mathbf{q}^T = \left[\; q_x \; q_y \;\right]\end{split}
\end{equation*}\begin{equation*}
\begin{split}\mathbf{et} = (\nabla T)^T = \left[\begin{array}{l}
\frac{\partial T}{\partial x}\;\;\frac{\partial T}{\partial y}
\end{array} \right]\end{split}
\end{equation*}
\sphinxAtStartPar
contain the components of the heat flux and the temperature gradient computed in the directions of the coordinate axis.

\sphinxlineitem{Theory}
\sphinxAtStartPar
By assembling four triangular elements as described in \sphinxcode{\sphinxupquote{flw2te}} a system of equations containing 5 degrees of freedom is obtained. From this system of equations the unknown temperature at the center of the element is computed.
Then according to the description in \sphinxcode{\sphinxupquote{flw2ts}} the temperature gradient and the heat flux in each of the four triangular elements are produced.
Finally the temperature gradient and the heat flux of the quadrilateral element are computed as area weighted mean values from the values of the four triangular elements. If heat is supplied to the element, the element load vector \sphinxcode{\sphinxupquote{eq}} is needed for the calculations.

\begin{sphinxadmonition}{note}{Note}

\sphinxAtStartPar
If the input variables are given for a number of identical (\sphinxcode{\sphinxupquote{nie}}) elements, i.e. \sphinxcode{\sphinxupquote{Ex}}, \sphinxcode{\sphinxupquote{Ey}}, and \sphinxcode{\sphinxupquote{Ed}} are matrices, then the output variables are defined as
\begin{equation*}
\begin{split}\mathrm{Es} =
\left[
\begin{array}{cc}
q^1_x & q^1_y  \\
q^2_x & q^2_y  \\
\vdots  &  \vdots   \\
q^{nie}_x & q^{nie}_y
\end{array}
\right]
\qquad
\mathrm{Et} = \left[
\begin{array}{cc}
\frac{\partial T}{\partial x}^1 & \frac{\partial T}{\partial y}^1 \\
\frac{\partial T}{\partial x}^2 & \frac{\partial T}{\partial y}^2 \\
\vdots & \vdots \\
\frac{\partial T}{\partial x}^{nie} & \frac{\partial T}{\partial y}^{nie}
\end{array}
\right]\end{split}
\end{equation*}
\sphinxAtStartPar
where \(\mathbf{q}^i\) and \(\nabla T^i\) are computed from the nodal values located in column \sphinxcode{\sphinxupquote{i}} of \sphinxcode{\sphinxupquote{Ed}}.
\end{sphinxadmonition}

\end{description}\end{quote}


\subsection{flw2i4e}
\label{\detokenize{heat_functions:flw2i4e}}\begin{quote}\begin{description}
\sphinxlineitem{Purpose}
\sphinxAtStartPar
Compute element stiffness matrix for a 4 node isoparametric heat flow element.

\begin{figure}[htbp]
\centering

\noindent\sphinxincludegraphics[width=70mm]{{FLW2I4E}.png}
\end{figure}

\sphinxlineitem{Syntax}
\begin{sphinxVerbatim}[commandchars=\\\{\}]
\PYG{n}{Ke}\PYG{+w}{ }\PYG{p}{=}\PYG{+w}{ }\PYG{n}{flw2i4e}\PYG{p}{(}\PYG{n}{ex}\PYG{p}{,}\PYG{+w}{ }\PYG{n}{ey}\PYG{p}{,}\PYG{+w}{ }\PYG{n}{ep}\PYG{p}{,}\PYG{+w}{ }\PYG{n}{D}\PYG{p}{)}
\PYG{p}{[}\PYG{n}{Ke}\PYG{p}{,}\PYG{+w}{ }\PYG{n}{fe}\PYG{p}{]}\PYG{+w}{ }\PYG{p}{=}\PYG{+w}{ }\PYG{n}{flw2i4e}\PYG{p}{(}\PYG{n}{ex}\PYG{p}{,}\PYG{+w}{ }\PYG{n}{ey}\PYG{p}{,}\PYG{+w}{ }\PYG{n}{ep}\PYG{p}{,}\PYG{+w}{ }\PYG{n}{D}\PYG{p}{,}\PYG{+w}{ }\PYG{n+nb}{eq}\PYG{p}{)}
\end{sphinxVerbatim}

\sphinxlineitem{Description}
\sphinxAtStartPar
\sphinxcode{\sphinxupquote{flw2i4e}} provides the element stiffness (conductivity) matrix \sphinxcode{\sphinxupquote{Ke}} and
the element load vector \sphinxcode{\sphinxupquote{fe}} for a 4 node isoparametric heat flow element.

\sphinxAtStartPar
The element nodal coordinates \(x_1\), \(y_1\), \(x_2\) etc,
are supplied to the function by \sphinxcode{\sphinxupquote{ex}} and \sphinxcode{\sphinxupquote{ey}}. The element thickness
\(t\) and the number of Gauss points \(n\)
(\(n \times n\) integration points, \(n=1,2,3\))
are supplied to the function by \sphinxcode{\sphinxupquote{ep}} and the thermal conductivities (or corresponding quantities)
\(k_{xx}\), \(k_{xy}\) etc are supplied by \sphinxcode{\sphinxupquote{D}}.
\begin{equation*}
\begin{split}\begin{array}{l}
\mathbf{ex} = [\, x_1 \;\; x_2 \;\; x_3 \;\; x_4 \,] \\
\mathbf{ey} = [\, y_1 \;\; y_2 \;\; y_3 \;\; y_4 \,]
\end{array}
\qquad
\mathbf{ep} = [\, t \;\; n \,]
\qquad
\mathbf{D} = \begin{bmatrix}
    k_{xx} & k_{xy} \\
    k_{yx} & k_{yy}
\end{bmatrix}\end{split}
\end{equation*}
\sphinxAtStartPar
If the scalar variable \sphinxcode{\sphinxupquote{eq}} is given in the function, the element load
vector \(fe\) is computed, using
\begin{equation*}
\begin{split}\mathbf{eq} = [\, Q \,]\end{split}
\end{equation*}
\sphinxAtStartPar
where \(Q\) is the heat supply per unit volume.

\sphinxlineitem{Theory}
\sphinxAtStartPar
The element stiffness matrix \(\mathbf{K}^e\) and the element load vector
\(\mathbf{f}_l^e\), stored in \sphinxcode{\sphinxupquote{Ke}} and \sphinxcode{\sphinxupquote{fe}}, respectively, are computed
according to
\begin{equation*}
\begin{split}\mathbf{K}^e = \int_A \mathbf{B}^{eT} \mathbf{D} \mathbf{B}^e t\, dA\end{split}
\end{equation*}\begin{equation*}
\begin{split}\mathbf{f}_l^e = \int_A \mathbf{N}^{eT} Q t\, dA\end{split}
\end{equation*}
\sphinxAtStartPar
with the constitutive matrix \(\mathbf{D}\) defined by \sphinxcode{\sphinxupquote{D}}.

\sphinxAtStartPar
The evaluation of the integrals for the isoparametric 4 node element
is based on a temperature approximation \(T(\xi, \eta)\), expressed in a
local coordinate system in terms of the nodal
variables \(T_1\), \(T_2\), \(T_3\) and \(T_4\) as
\begin{equation*}
\begin{split}T(\xi, \eta) = \mathbf{N}^e \mathbf{a}^e\end{split}
\end{equation*}
\sphinxAtStartPar
where
\begin{equation*}
\begin{split}\mathbf{N}^e = [\, N_1^e \;\; N_2^e \;\; N_3^e \;\; N_4^e \,]
\qquad
\mathbf{a}^e = [\, T_1 \;\; T_2 \;\; T_3 \;\; T_4 \,]^T\end{split}
\end{equation*}
\sphinxAtStartPar
The element shape functions are given by
\begin{equation*}
\begin{split}N_1^e = \frac{1}{4}(1-\xi)(1-\eta) \qquad N_2^e = \frac{1}{4}(1+\xi)(1-\eta) \\
N_3^e = \frac{1}{4}(1+\xi)(1+\eta) \qquad N_4^e = \frac{1}{4}(1-\xi)(1+\eta)\end{split}
\end{equation*}
\sphinxAtStartPar
The \(\mathbf{B}^e\)\sphinxhyphen{}matrix is given by
\begin{equation*}
\begin{split}\mathbf{B}^e = \nabla \mathbf{N}^e
= \begin{bmatrix}
    \frac{\partial}{\partial x} \\
    \frac{\partial}{\partial y}
\end{bmatrix} \mathbf{N}^e
= (\mathbf{J}^T)^{-1}
\begin{bmatrix}
    \frac{\partial}{\partial \xi} \\
    \frac{\partial}{\partial \eta}
\end{bmatrix} \mathbf{N}^e\end{split}
\end{equation*}
\sphinxAtStartPar
where \(\mathbf{J}\) is the Jacobian matrix
\begin{equation*}
\begin{split}\mathbf{J} =
\begin{bmatrix}
    \frac{\partial x}{\partial \xi} & \frac{\partial x}{\partial \eta} \\
    \frac{\partial y}{\partial \xi} & \frac{\partial y}{\partial \eta}
\end{bmatrix}\end{split}
\end{equation*}
\sphinxAtStartPar
Evaluation of the integrals is done by Gauss integration.

\end{description}\end{quote}


\subsection{flw2i4s}
\label{\detokenize{heat_functions:flw2i4s}}\begin{quote}\begin{description}
\sphinxlineitem{Purpose}
\sphinxAtStartPar
Compute heat flux and temperature gradients in a 4 node isoparametric heat flow element.

\sphinxlineitem{Syntax}
\sphinxAtStartPar
\sphinxcode{\sphinxupquote{{[}es, et, eci{]} = flw2i4s(ex, ey, ep, D, ed)}}

\sphinxlineitem{Description}
\sphinxAtStartPar
\sphinxcode{\sphinxupquote{flw2i4s}} computes the heat flux vector \sphinxcode{\sphinxupquote{es}} and the temperature gradient \sphinxcode{\sphinxupquote{et}} (or corresponding quantities) in a 4 node isoparametric heat flow element.

\sphinxAtStartPar
The input variables \(\mathbf{ex}\), \(\mathbf{ey}\), \(\mathbf{ep}\) and the matrix \(\mathbf{D}\) are defined in \sphinxcode{\sphinxupquote{flw2i4e}}. The vector \(\mathbf{ed}\) contains the nodal temperatures \(\mathbf{a}^e\) of the element and is obtained by \sphinxcode{\sphinxupquote{extract}} as
\begin{equation*}
\begin{split}\mathbf{ed} = (\mathbf{a}^e)^T = [\;T_1\;\; T_2\;\; T_3\;\; T_4\;]\end{split}
\end{equation*}
\sphinxAtStartPar
The output variables
\begin{equation*}
\begin{split}\mathbf{es} = \bar{\mathbf{q}}^T = \left[
\begin{array}{cc}
q^1_x & q^1_y  \\
q^2_x & q^2_y  \\
\vdots  &  \vdots   \\
q^{n^2}_x & q^{n^2}_y
\end{array}
\right]\end{split}
\end{equation*}\begin{align*}\!\begin{aligned}
\mathbf{et} = (\bar {\nabla} T)^T = \left[
\begin{array}{cc}
\frac{\partial T}{\partial x}^1 & \frac{\partial T}{\partial y}^1 \\
\frac{\partial T}{\partial x}^2 & \frac{\partial T}{\partial y}^2 \\
\vdots & \vdots \\
\frac{\partial T}{\partial x}^{n^2} & \frac{\partial T}{\partial y}^{n^2}
\end{array}
\right]\\
\qquad
\mathbf{eci} = \left[
\begin{array}{cc}
x_1 & y_1 \\
x_2 & y_2 \\
\vdots  &  \vdots \\
x_{n^2} & y_{n^2}
\end{array}
\right]\\
\end{aligned}\end{align*}
\sphinxAtStartPar
contain the heat flux, the temperature gradient, and the coordinates of the integration points. The index \(n\) denotes the number of integration points used within the element, cf. \sphinxcode{\sphinxupquote{flw2i4e}}.

\sphinxlineitem{Theory}
\sphinxAtStartPar
The temperature gradient and the heat flux are computed according to
\begin{equation*}
\begin{split}\nabla T = \mathbf{B}^e\,\mathbf{a}^e\end{split}
\end{equation*}\begin{equation*}
\begin{split}\mathbf{q} = - \mathbf{D} \nabla T\end{split}
\end{equation*}
\sphinxAtStartPar
where the matrices \(\mathbf{D}\), \(\mathbf{B}^e\), and \(\mathbf{a}^e\) are described in \sphinxcode{\sphinxupquote{flw2i4e}}, and where the integration points are chosen as evaluation points.

\end{description}\end{quote}


\subsection{flw2i8e}
\label{\detokenize{heat_functions:flw2i8e}}\begin{quote}\begin{description}
\sphinxlineitem{Purpose}
\sphinxAtStartPar
Compute element stiffness matrix for an 8 node isoparametric heat flow element.

\begin{figure}[htbp]
\centering

\noindent\sphinxincludegraphics[width=0.700\linewidth]{{FLW2I8E}.png}
\end{figure}

\sphinxlineitem{Syntax}
\begin{sphinxVerbatim}[commandchars=\\\{\}]
\PYG{n}{Ke}\PYG{+w}{ }\PYG{p}{=}\PYG{+w}{ }\PYG{n}{flw2i8e}\PYG{p}{(}\PYG{n}{ex}\PYG{p}{,}\PYG{+w}{ }\PYG{n}{ey}\PYG{p}{,}\PYG{+w}{ }\PYG{n}{ep}\PYG{p}{,}\PYG{+w}{ }\PYG{n}{D}\PYG{p}{)}
\PYG{p}{[}\PYG{n}{Ke}\PYG{p}{,}\PYG{+w}{ }\PYG{n}{fe}\PYG{p}{]}\PYG{+w}{ }\PYG{p}{=}\PYG{+w}{ }\PYG{n}{flw2i8e}\PYG{p}{(}\PYG{n}{ex}\PYG{p}{,}\PYG{+w}{ }\PYG{n}{ey}\PYG{p}{,}\PYG{+w}{ }\PYG{n}{ep}\PYG{p}{,}\PYG{+w}{ }\PYG{n}{D}\PYG{p}{,}\PYG{+w}{ }\PYG{n+nb}{eq}\PYG{p}{)}
\end{sphinxVerbatim}

\sphinxlineitem{Description}
\sphinxAtStartPar
\sphinxcode{\sphinxupquote{flw2i8e}} provides the element stiffness (conductivity) matrix \sphinxcode{\sphinxupquote{Ke}} and the element load vector \sphinxcode{\sphinxupquote{fe}} for an 8 node isoparametric heat flow element.

\sphinxAtStartPar
The element nodal coordinates \(x_1\), \(y_1\), \(x_2\) etc, are supplied to the function by \(\mathbf{ex}\) and \(\mathbf{ey}\). The element thickness \(t\) and the number of Gauss points \(n\) (\(n \times n\) integration points, \(n=1,2,3\)) are supplied to the function by \(\mathbf{ep}\) and the thermal conductivities (or corresponding quantities) \(k_{xx}\), \(k_{xy}\) etc are supplied by \sphinxcode{\sphinxupquote{D}}.
\begin{equation*}
\begin{split}\begin{array}{l}
\mathbf{ex} = [\, x_1 \;\; x_2 \;\; x_3 \;\; \dots \;\; x_8 \,] \\
\mathbf{ey} = [\, y_1 \;\; y_2 \;\; y_3 \;\; \dots \;\; y_8 \,]
\end{array}
\qquad
\mathbf{ep} = [\, t \;\; n \,]
\qquad
\mathbf{D} = \begin{bmatrix}
    k_{xx} & k_{xy} \\
    k_{yx} & k_{yy}
\end{bmatrix}\end{split}
\end{equation*}
\sphinxAtStartPar
If the scalar variable \sphinxcode{\sphinxupquote{eq}} is given in the function, the vector \(\mathbf{fe}\) is computed, using
\begin{equation*}
\begin{split}\mathbf{eq} = [\, Q \,]\end{split}
\end{equation*}
\sphinxAtStartPar
where \(Q\) is the heat supply per unit volume.

\sphinxlineitem{Theory}
\sphinxAtStartPar
The element stiffness matrix \(\mathbf{K}^e\) and the element load vector \(\mathbf{f}_l^e\), stored in \sphinxcode{\sphinxupquote{Ke}} and \sphinxcode{\sphinxupquote{fe}}, respectively, are computed according to
\begin{equation*}
\begin{split}\mathbf{K}^e = \int_A \mathbf{B}^{eT} \mathbf{D} \mathbf{B}^e t\, dA\end{split}
\end{equation*}\begin{equation*}
\begin{split}\mathbf{f}_l^e = \int_A \mathbf{N}^{eT} Q t\, dA\end{split}
\end{equation*}
\sphinxAtStartPar
with the constitutive matrix \(\mathbf{D}\) defined by \sphinxcode{\sphinxupquote{D}}.

\sphinxAtStartPar
The evaluation of the integrals for the 2D isoparametric 8 node element is based on a temperature approximation \(T(\xi, \eta)\), expressed in a local coordinates system in terms of the nodal variables \(T_1\) to \(T_8\) as
\begin{equation*}
\begin{split}T(\xi, \eta) = \mathbf{N}^e \mathbf{a}^e\end{split}
\end{equation*}
\sphinxAtStartPar
where
\begin{equation*}
\begin{split}\mathbf{N}^e = [\, N_1^e \;\; N_2^e \;\; N_3^e \;\; \dots \;\; N_8^e \,]
\qquad
\mathbf{a}^e = [\, T_1 \;\; T_2 \;\; T_3 \;\; \dots \;\; T_8 \,]^T\end{split}
\end{equation*}
\sphinxAtStartPar
The element shape functions are given by
\begin{equation*}
\begin{split}\begin{aligned}
N_1^e &= -\frac{1}{4}(1-\xi)(1-\eta)(1+\xi+\eta)
& N_5^e &= \frac{1}{2}(1-\xi^2)(1-\eta) \\
N_2^e &= -\frac{1}{4}(1+\xi)(1-\eta)(1-\xi+\eta)
& N_6^e &= \frac{1}{2}(1+\xi)(1-\eta^2) \\
N_3^e &= -\frac{1}{4}(1+\xi)(1+\eta)(1-\xi-\eta)
& N_7^e &= \frac{1}{2}(1-\xi^2)(1+\eta) \\
N_4^e &= -\frac{1}{4}(1-\xi)(1+\eta)(1+\xi-\eta)
& N_8^e &= \frac{1}{2}(1-\xi)(1-\eta^2)
\end{aligned}\end{split}
\end{equation*}
\sphinxAtStartPar
The \(\mathbf{B}^e\)\sphinxhyphen{}matrix is given by
\begin{equation*}
\begin{split}\mathbf{B}^e = \nabla \mathbf{N}^e
= \begin{bmatrix}
    \frac{\partial}{\partial x} \\
    \frac{\partial}{\partial y}
\end{bmatrix} \mathbf{N}^e
= (\mathbf{J}^T)^{-1}
\begin{bmatrix}
    \frac{\partial}{\partial \xi} \\
    \frac{\partial}{\partial \eta}
\end{bmatrix} \mathbf{N}^e\end{split}
\end{equation*}
\sphinxAtStartPar
where \(\mathbf{J}\) is the Jacobian matrix
\begin{equation*}
\begin{split}\mathbf{J} =
\begin{bmatrix}
    \frac{\partial x}{\partial \xi} & \frac{\partial x}{\partial \eta} \\
    \frac{\partial y}{\partial \xi} & \frac{\partial y}{\partial \eta}
\end{bmatrix}\end{split}
\end{equation*}
\sphinxAtStartPar
Evaluation of the integrals is done by Gauss integration.

\end{description}\end{quote}


\subsection{flw2i8s}
\label{\detokenize{heat_functions:flw2i8s}}\begin{quote}\begin{description}
\sphinxlineitem{Purpose}
\sphinxAtStartPar
Compute heat flux and temperature gradients in an 8 node isoparametric heat flow element.

\sphinxlineitem{Syntax}
\sphinxAtStartPar
\sphinxcode{\sphinxupquote{{[}es, et, eci{]} = flw2i8s(ex, ey, ep, D, ed)}}

\sphinxlineitem{Description}
\sphinxAtStartPar
\sphinxcode{\sphinxupquote{flw2i8s}} computes the heat flux vector \sphinxcode{\sphinxupquote{es}} and the temperature gradient \sphinxcode{\sphinxupquote{et}} (or corresponding quantities) in an 8 node isoparametric heat flow element.

\sphinxAtStartPar
The input variables \sphinxcode{\sphinxupquote{ex}}, \sphinxcode{\sphinxupquote{ey}}, \sphinxcode{\sphinxupquote{ep}} and the matrix \sphinxcode{\sphinxupquote{D}} are defined in \sphinxcode{\sphinxupquote{flw2i8e}}. The vector \sphinxcode{\sphinxupquote{ed}} contains the nodal temperatures \(\mathbf{a}^e\) of the element and is obtained by the function \sphinxcode{\sphinxupquote{extract}} as
\begin{equation*}
\begin{split}\mathbf{ed} = (\mathbf{a}^e)^T = [\;T_1\;\; T_2\;\; T_3\;\;\dots\;\, T_8\;]\end{split}
\end{equation*}
\sphinxAtStartPar
The output variables
\begin{equation*}
\begin{split}\mathbf{es} = \bar{\mathbf{q}}^T = \left[
\begin{array}{cc}
q^1_x & q^1_y  \\
q^2_x & q^2_y  \\
\vdots  &  \vdots   \\
q^{n^2}_x & q^{n^2}_y
\end{array}
\right]\end{split}
\end{equation*}\begin{equation*}
\begin{split}\mathbf{et} = (\bar {\nabla} T)^T = \left[
\begin{array}{cc}
\frac{\partial T}{\partial x}^1 & \frac{\partial T}{\partial y}^1 \\
\frac{\partial T}{\partial x}^2 & \frac{\partial T}{\partial y}^2 \\
\vdots & \vdots \\
\frac{\partial T}{\partial x}^{n^2} & \frac{\partial T}{\partial y}^{n^2}
\end{array}
\right]\end{split}
\end{equation*}\begin{equation*}
\begin{split}\mathbf{eci} = \left[
\begin{array}{cc}
x_1 & y_1 \\
x_2 & y_2 \\
\vdots  &  \vdots \\
x_{n^2} & y_{n^2}
\end{array}
\right]\end{split}
\end{equation*}
\sphinxAtStartPar
contain the heat flux, the temperature gradient, and the coordinates of the integration points. The index \(n\) denotes the number of integration points used within the element, see \sphinxcode{\sphinxupquote{flw2i8e}}.

\sphinxlineitem{Theory}
\sphinxAtStartPar
The temperature gradient and the heat flux are computed according to
\begin{equation*}
\begin{split}\nabla T = \mathbf{B}^e\,\mathbf{a}^e\end{split}
\end{equation*}\begin{equation*}
\begin{split}\mathbf{q} = - \mathbf{D} \nabla T\end{split}
\end{equation*}
\sphinxAtStartPar
where the matrices \(\mathbf{D}\), \(\mathbf{B}^e\), and \(\mathbf{a}^e\) are described in \sphinxcode{\sphinxupquote{flw2i8e}}, and where the integration points are chosen as evaluation points.

\end{description}\end{quote}


\section{3D Heat Flow Functions}
\label{\detokenize{heat_functions:id1}}

\subsection{flw3i8e}
\label{\detokenize{heat_functions:flw3i8e}}\begin{quote}\begin{description}
\sphinxlineitem{Purpose}
\sphinxAtStartPar
Compute element stiffness matrix for an 8 node isoparametric element.

\begin{figure}[htbp]
\centering

\noindent\sphinxincludegraphics[width=70mm]{{FLW3I8E}.png}
\end{figure}

\sphinxlineitem{Syntax}
\begin{sphinxVerbatim}[commandchars=\\\{\}]
\PYG{n}{Ke}\PYG{+w}{ }\PYG{p}{=}\PYG{+w}{ }\PYG{n}{flw3i8e}\PYG{p}{(}\PYG{n}{ex}\PYG{p}{,}\PYG{+w}{ }\PYG{n}{ey}\PYG{p}{,}\PYG{+w}{ }\PYG{n}{ez}\PYG{p}{,}\PYG{+w}{ }\PYG{n}{ep}\PYG{p}{,}\PYG{+w}{ }\PYG{n}{D}\PYG{p}{)}
\PYG{p}{[}\PYG{n}{Ke}\PYG{p}{,}\PYG{+w}{ }\PYG{n}{fe}\PYG{p}{]}\PYG{+w}{ }\PYG{p}{=}\PYG{+w}{ }\PYG{n}{flw3i8e}\PYG{p}{(}\PYG{n}{ex}\PYG{p}{,}\PYG{+w}{ }\PYG{n}{ey}\PYG{p}{,}\PYG{+w}{ }\PYG{n}{ez}\PYG{p}{,}\PYG{+w}{ }\PYG{n}{ep}\PYG{p}{,}\PYG{+w}{ }\PYG{n}{D}\PYG{p}{,}\PYG{+w}{ }\PYG{n+nb}{eq}\PYG{p}{)}
\end{sphinxVerbatim}

\sphinxlineitem{Description}
\sphinxAtStartPar
\sphinxcode{\sphinxupquote{flw3i8e}} provides the element stiffness (conductivity) matrix \sphinxcode{\sphinxupquote{Ke}} and
the element load vector \sphinxcode{\sphinxupquote{fe}} for an 8 node isoparametric heat flow element.

\sphinxAtStartPar
The element nodal coordinates \(x_1\), \(y_1\), \(z_1\), \(x_2\) etc,
are supplied to the function by \sphinxcode{\sphinxupquote{ex}}, \sphinxcode{\sphinxupquote{ey}} and \sphinxcode{\sphinxupquote{ez}}.
The number of Gauss points \(n\) (\(n \times n \times n\) integration points, \(n=1,2,3\))
are supplied to the function by \sphinxcode{\sphinxupquote{ep}} and the thermal conductivities (or corresponding quantities)
\(k_{xx}\), \(k_{xy}\) etc are supplied by \sphinxcode{\sphinxupquote{D}}.
\begin{equation*}
\begin{split}\begin{array}{l}
\mathbf{ex} = [\, x_1 \;\; x_2 \;\; x_3 \;\; \dots \;\; x_8 \,] \\
\mathbf{ey} = [\, y_1 \;\; y_2 \;\; y_3 \;\; \dots \;\; y_8 \,] \\
\mathbf{ez} = [\, z_1 \;\; z_2 \;\; z_3 \;\; \dots \;\; z_8 \,]
\end{array}
\qquad
\mathbf{ep} = [\, n \,]
\qquad
\mathbf{D} =
\begin{bmatrix}
    k_{xx} & k_{xy} & k_{xz} \\
    k_{yx} & k_{yy} & k_{yz} \\
    k_{zx} & k_{zy} & k_{zz}
\end{bmatrix}\end{split}
\end{equation*}
\sphinxAtStartPar
If the scalar variable \sphinxcode{\sphinxupquote{eq}} is given in the function, the element load
vector \(\mathbf{fe}\) is computed, using
\begin{equation*}
\begin{split}\mathbf{eq} = [\, Q \,]\end{split}
\end{equation*}
\sphinxAtStartPar
where \(Q\) is the heat supply per unit volume.

\sphinxlineitem{Theory}
\sphinxAtStartPar
The element stiffness matrix \(\mathbf{K}^e\) and the element load vector
\(\mathbf{f}_l^e\), stored in \sphinxcode{\sphinxupquote{Ke}} and \sphinxcode{\sphinxupquote{fe}}, respectively, are computed
according to
\begin{align*}\!\begin{aligned}
\mathbf{K}^e = \int_V \mathbf{B}^{eT} \mathbf{D} \mathbf{B}^e \, dV\\
\mathbf{f}_l^e = \int_V \mathbf{N}^{eT} Q \, dV\\
\end{aligned}\end{align*}
\sphinxAtStartPar
with the constitutive matrix \(\mathbf{D}\) defined by \sphinxcode{\sphinxupquote{D}}.

\sphinxAtStartPar
The evaluation of the integrals for the 3D isoparametric 8 node element
is based on a temperature approximation \(T(\xi, \eta, \zeta)\), expressed in a
local coordinate system in terms of the nodal variables \(T_1\) to \(T_8\) as
\begin{align*}\!\begin{aligned}
T(\xi, \eta, \zeta) = \mathbf{N}^e \mathbf{a}^e\\
\mathbf{N}^e = [\, N_1^e \;\; N_2^e \;\; N_3^e \;\; \dots \;\; N_8^e \,]
\qquad
\mathbf{a}^e = [\, T_1 \;\; T_2 \;\; T_3 \;\; \dots \;\; T_8 \,]^T\\
\end{aligned}\end{align*}
\sphinxAtStartPar
The element shape functions are given by
\begin{equation*}
\begin{split}N_1^e = \frac{1}{8}(1-\xi)(1-\eta)(1-\zeta) \qquad
N_2^e = \frac{1}{8}(1+\xi)(1-\eta)(1-\zeta) \\
N_3^e = \frac{1}{8}(1+\xi)(1+\eta)(1-\zeta) \qquad
N_4^e = \frac{1}{8}(1-\xi)(1+\eta)(1-\zeta) \\
N_5^e = \frac{1}{8}(1-\xi)(1-\eta)(1+\zeta) \qquad
N_6^e = \frac{1}{8}(1+\xi)(1-\eta)(1+\zeta) \\
N_7^e = \frac{1}{8}(1+\xi)(1+\eta)(1+\zeta) \qquad
N_8^e = \frac{1}{8}(1-\xi)(1+\eta)(1+\zeta)\end{split}
\end{equation*}
\sphinxAtStartPar
The \(\mathbf{B}^e\)\sphinxhyphen{}matrix is given by
\begin{equation*}
\begin{split}\mathbf{B}^e = \nabla \mathbf{N}^e
= \begin{bmatrix}
    \frac{\partial}{\partial x} \\
    \frac{\partial}{\partial y} \\
    \frac{\partial}{\partial z}
\end{bmatrix} \mathbf{N}^e
= (\mathbf{J}^T)^{-1}
\begin{bmatrix}
    \frac{\partial}{\partial \xi} \\
    \frac{\partial}{\partial \eta} \\
    \frac{\partial}{\partial \zeta}
\end{bmatrix} \mathbf{N}^e\end{split}
\end{equation*}
\sphinxAtStartPar
where \(\mathbf{J}\) is the Jacobian matrix
\begin{equation*}
\begin{split}\mathbf{J} =
\begin{bmatrix}
    \frac{\partial x}{\partial \xi} & \frac{\partial x}{\partial \eta} & \frac{\partial x}{\partial \zeta} \\
    \frac{\partial y}{\partial \xi} & \frac{\partial y}{\partial \eta} & \frac{\partial y}{\partial \zeta} \\
    \frac{\partial z}{\partial \xi} & \frac{\partial z}{\partial \eta} & \frac{\partial z}{\partial \zeta}
\end{bmatrix}\end{split}
\end{equation*}
\sphinxAtStartPar
Evaluation of the integrals is done by Gauss integration.

\end{description}\end{quote}


\subsection{flw3i8s}
\label{\detokenize{heat_functions:flw3i8s}}\begin{quote}\begin{description}
\sphinxlineitem{Purpose}
\sphinxAtStartPar
Compute heat flux and temperature gradients in an
8 node isoparametric heat flow element.

\sphinxlineitem{Syntax}
\begin{sphinxVerbatim}[commandchars=\\\{\}]
\PYG{p}{[}\PYG{n}{es}\PYG{p}{,}\PYG{+w}{ }\PYG{n}{et}\PYG{p}{,}\PYG{+w}{ }\PYG{n}{eci}\PYG{p}{]}\PYG{+w}{ }\PYG{p}{=}\PYG{+w}{ }\PYG{n}{flw3i8s}\PYG{p}{(}\PYG{n}{ex}\PYG{p}{,}\PYG{+w}{ }\PYG{n}{ey}\PYG{p}{,}\PYG{+w}{ }\PYG{n}{ez}\PYG{p}{,}\PYG{+w}{ }\PYG{n}{ep}\PYG{p}{,}\PYG{+w}{ }\PYG{n}{D}\PYG{p}{,}\PYG{+w}{ }\PYG{n}{ed}\PYG{p}{)}
\end{sphinxVerbatim}

\sphinxlineitem{Description}
\sphinxAtStartPar
\sphinxcode{\sphinxupquote{flw3i8s}} computes the heat flux vector \sphinxcode{\sphinxupquote{es}} and the temperature
gradient \sphinxcode{\sphinxupquote{et}} (or corresponding quantities)
in an 8 node isoparametric heat flow element.

\sphinxAtStartPar
The input variables \(\mathbf{ex}\), \(\mathbf{ey}\), \(\mathbf{ez}\), \(\mathbf{ep}\) and
the matrix \(\mathbf{D}\) are defined in \sphinxcode{\sphinxupquote{flw3i8e}}.
The vector \(\mathbf{ed}\) contains the nodal temperatures \(\mathbf{a}^e\)
of the element and is obtained by the function \sphinxcode{\sphinxupquote{extract}} as
\begin{equation*}
\begin{split}\mathbf{ed} = (\mathbf{a}^e)^T = [\,T_1\;\; T_2\;\; T_3\;\;\dots\;\; T_8\,]\end{split}
\end{equation*}
\sphinxAtStartPar
The output variables
\begin{equation*}
\begin{split}\mathbf{es} = \bar{\mathbf{q}}^T =
\begin{bmatrix}
q^1_x & q^1_y  & q^1_z \\
q^2_x & q^2_y  & q^2_z \\
\vdots  &  \vdots   &  \vdots\\
q^{n^3}_x  & q^{n^3}_y  & q^{n^3}_z
\end{bmatrix}\end{split}
\end{equation*}\begin{equation*}
\begin{split}\mathbf{et} = (\bar {\nabla} T)^T =
\begin{bmatrix}
\frac{\partial T}{\partial x}^1 &
\frac{\partial T}{\partial y}^1 &
\frac{\partial T}{\partial z}^1\\
\frac{\partial T}{\partial x}^2 &
\frac{\partial T}{\partial y}^2 &
\frac{\partial T}{\partial z}^2\\
\vdots & \vdots & \vdots \\
\frac{\partial T}{\partial x}^{n^3} &
\frac{\partial T}{\partial y}^{n^3} &
\frac{\partial T}{\partial z}^{n^3}
\end{bmatrix}
\qquad
\mathbf{eci} =
\begin{bmatrix}
x_1 & y_1 & z_1 \\
x_2 & y_2 & z_2 \\
\vdots  &  \vdots &  \vdots \\
x_{n^3} & y_{n^3} & z_{n^3}
\end{bmatrix}\end{split}
\end{equation*}
\sphinxAtStartPar
contain the heat flux, the temperature gradient,
and the coordinates of the integration points.
The index \(n\) denotes the number of integration points
used within the element, see \sphinxcode{\sphinxupquote{flw3i8e}}.

\sphinxlineitem{Theory}
\sphinxAtStartPar
The temperature gradient and the heat flux are computed according to
\begin{equation*}
\begin{split}\nabla T = \mathbf{B}^e\,\mathbf{a}^e\end{split}
\end{equation*}\begin{equation*}
\begin{split}\mathbf{q} = - \mathbf{D} \nabla T\end{split}
\end{equation*}
\sphinxAtStartPar
where the matrices \(\mathbf{D}\), \(\mathbf{B}^e\), and \(\mathbf{a}^e\)
are described in \sphinxcode{\sphinxupquote{flw3i8e}}, and where the integration points
are chosen as evaluation points.

\end{description}\end{quote}

\sphinxstepscope


\chapter{Solid elements functions}
\label{\detokenize{solid_functions:solid-elements-functions}}\label{\detokenize{solid_functions::doc}}
\sphinxAtStartPar
Solid elements are available for two dimensional analysis in plane stress (panels) and plane strain, and for general three dimensional analysis. In the two dimensional case there are a triangular three node element, a quadrilateral four node element, two rectangular four node elements, and quadrilateral isoparametric four and eight node elements. For three dimensional analysis there is an eight node isoparametric element.

\sphinxAtStartPar
The elements are able to deal with both isotropic and anisotropic materials. The triangular element
and the three isoparametric elements can also be used together with a nonlinear material model.

\sphinxAtStartPar
The material properties are specified by supplying the constitutive matrix
\(\mathbf{D}\) as an input variable to the element functions. This matrix can
be formed by the functions described in Section {\hyperref[\detokenize{material_functions:material-functions}]{\sphinxcrossref{\DUrole{std}{\DUrole{std-ref}{Material functions}}}}}.


\begin{savenotes}\sphinxattablestart
\sphinxthistablewithglobalstyle
\centering
\sphinxcapstartof{table}
\sphinxthecaptionisattop
\sphinxcaption{\sphinxstylestrong{Solid elements}}\label{\detokenize{solid_functions:id2}}
\sphinxaftertopcaption
\begin{tabular}[t]{\X{50}{100}\X{50}{100}}
\sphinxtoprule
\sphinxtableatstartofbodyhook
\noindent\sphinxincludegraphics[width=0.700\linewidth]{{PLANT}.png}

\begin{center}\sphinxcode{\sphinxupquote{plante}}
\end{center}&
\noindent\sphinxincludegraphics[width=0.700\linewidth]{{PLANQ}.png}

\begin{center}\sphinxcode{\sphinxupquote{planqe}}
\end{center}\\
\sphinxhline
\noindent\sphinxincludegraphics[width=0.700\linewidth]{{PLANTR}.png}

\begin{center}\sphinxcode{\sphinxupquote{planre}}
\sphinxcode{\sphinxupquote{plantce}}
\end{center}&
\noindent\sphinxincludegraphics[width=0.700\linewidth]{{PLANI4}.png}

\begin{center}\sphinxcode{\sphinxupquote{plani4e}}
\end{center}\\
\sphinxhline
\noindent\sphinxincludegraphics[width=0.700\linewidth]{{PLANI8}.png}

\begin{center}\sphinxcode{\sphinxupquote{plani8e}}
\end{center}&
\noindent\sphinxincludegraphics[width=0.700\linewidth]{{SOLI8}.png}

\begin{center}\sphinxcode{\sphinxupquote{soli8e}}
\end{center}\\
\sphinxbottomrule
\end{tabular}
\sphinxtableafterendhook\par
\sphinxattableend\end{savenotes}


\section{2D Solid Functions}
\label{\detokenize{solid_functions:d-solid-functions}}

\subsection{plante}
\label{\detokenize{solid_functions:plante}}\begin{quote}\begin{description}
\sphinxlineitem{Purpose}
\sphinxAtStartPar
Compute element matrices for a triangular element in plane strain or plane stress.

\begin{figure}[htbp]
\centering

\noindent\sphinxincludegraphics[width=0.700\linewidth]{{PLANTE}.png}
\end{figure}

\sphinxlineitem{Syntax}
\begin{sphinxVerbatim}[commandchars=\\\{\}]
\PYG{n}{Ke}\PYG{+w}{ }\PYG{p}{=}\PYG{+w}{ }\PYG{n}{plante}\PYG{p}{(}\PYG{n}{ex}\PYG{p}{,}\PYG{+w}{ }\PYG{n}{ey}\PYG{p}{,}\PYG{+w}{ }\PYG{n}{ep}\PYG{p}{,}\PYG{+w}{ }\PYG{n}{D}\PYG{p}{)}
\PYG{p}{[}\PYG{n}{Ke}\PYG{p}{,}\PYG{+w}{ }\PYG{n}{fe}\PYG{p}{]}\PYG{+w}{ }\PYG{p}{=}\PYG{+w}{ }\PYG{n}{plante}\PYG{p}{(}\PYG{n}{ex}\PYG{p}{,}\PYG{+w}{ }\PYG{n}{ey}\PYG{p}{,}\PYG{+w}{ }\PYG{n}{ep}\PYG{p}{,}\PYG{+w}{ }\PYG{n}{D}\PYG{p}{,}\PYG{+w}{ }\PYG{n+nb}{eq}\PYG{p}{)}
\end{sphinxVerbatim}

\sphinxlineitem{Description}
\sphinxAtStartPar
\sphinxcode{\sphinxupquote{plante}} provides an element stiffness matrix \sphinxcode{\sphinxupquote{Ke}} and an element load vector \sphinxcode{\sphinxupquote{fe}} for a triangular element in plane strain or plane stress.

\sphinxAtStartPar
The element nodal coordinates \(x_1, y_1, x_2, \ldots\) are supplied to the function by \(\mathbf{ex}\) and \(\mathbf{ey}\). The type of analysis \sphinxcode{\sphinxupquote{ptype}} and the element thickness \sphinxcode{\sphinxupquote{t}} are supplied by \(\mathbf{ep}\),
\begin{equation*}
\begin{split}\begin{array}{l}
ptype=1 \quad \text{plane stress} \\
ptype=2 \quad \text{plane strain}
\end{array}\end{split}
\end{equation*}
\sphinxAtStartPar
and the material properties are supplied by the constitutive matrix \sphinxcode{\sphinxupquote{D}}. Any arbitrary \(\mathbf{D}\)\sphinxhyphen{}matrix with dimensions from \(3 \times 3\) to \(6 \times 6\) may be given. For an isotropic elastic material the constitutive matrix can be formed by the function \sphinxcode{\sphinxupquote{hooke}}, see Section {\hyperref[\detokenize{material_functions:material-functions}]{\sphinxcrossref{\DUrole{std}{\DUrole{std-ref}{Material functions}}}}}.
\begin{equation*}
\begin{split}\mathbf{ex} = [\,x_1 \;\; x_2 \;\; x_3\,] \\
\mathbf{ey} = [\,y_1 \;\; y_2 \;\; y_3\,] \\
\mathbf{ep} = [\,ptype \;\; t\,]\end{split}
\end{equation*}\begin{equation*}
\begin{split}\mathbf{D} = \begin{bmatrix}
    D_{11} & D_{12} & D_{13} \\
    D_{21} & D_{22} & D_{23} \\
    D_{31} & D_{32} & D_{33}
\end{bmatrix}
\quad \text{or} \quad
\mathbf{D} = \begin{bmatrix}
    D_{11} & D_{12} & D_{13} & D_{14} & [D_{15}] & [D_{16}] \\
    D_{21} & D_{22} & D_{23} & D_{24} & [D_{25}] & [D_{26}] \\
    D_{31} & D_{32} & D_{33} & D_{34} & [D_{35}] & [D_{36}] \\
    D_{41} & D_{42} & D_{43} & D_{44} & [D_{45}] & [D_{46}] \\
    [D_{51}] & [D_{52}] & [D_{53}] & [D_{54}] & [D_{55}] & [D_{56}] \\
    [D_{61}] & [D_{62}] & [D_{63}] & [D_{64}] & [D_{65}] & [D_{66}]
\end{bmatrix}\end{split}
\end{equation*}
\sphinxAtStartPar
If uniformly distributed loads are applied to the element, the element load vector \sphinxcode{\sphinxupquote{fe}} is computed. The input variable
\begin{equation*}
\begin{split}\text{eq} = \begin{bmatrix}
    b_x \\
    b_y
\end{bmatrix}\end{split}
\end{equation*}
\sphinxAtStartPar
containing loads per unit volume, \(b_x\) and \(b_y\), is then given.

\sphinxlineitem{Theory}
\sphinxAtStartPar
The element stiffness matrix \(\mathbf{K}^e\) and the element load vector \(\mathbf{f}_l^e\), stored in \sphinxcode{\sphinxupquote{Ke}} and \sphinxcode{\sphinxupquote{fe}}, respectively, are computed according to
\begin{equation*}
\begin{split}\mathbf{K}^e = (\mathbf{C}^{-1})^T \int_A \bar{\mathbf{B}}^T \mathbf{D} \bar{\mathbf{B}}\, t\, dA\, \mathbf{C}^{-1}\end{split}
\end{equation*}\begin{equation*}
\begin{split}\mathbf{f}_l^e = (\mathbf{C}^{-1})^T \int_A \bar{\mathbf{N}}^T \mathbf{b}\, t\, dA\end{split}
\end{equation*}
\sphinxAtStartPar
with the constitutive matrix \(\mathbf{D}\) defined by \sphinxcode{\sphinxupquote{D}}, and the body force vector \(\mathbf{b}\) defined by \sphinxcode{\sphinxupquote{eq}}.

\sphinxAtStartPar
The evaluation of the integrals for the triangular element is based on a linear displacement approximation \(\mathbf{u}(x, y)\) and is expressed in terms of the nodal variables \(u_1, u_2, \ldots, u_6\) as
\begin{equation*}
\begin{split}\mathbf{u}(x, y) = \mathbf{N}^e \mathbf{a}^e = \bar{\mathbf{N}}\, \mathbf{C}^{-1} \mathbf{a}^e\end{split}
\end{equation*}
\sphinxAtStartPar
where
\begin{equation*}
\begin{split}\mathbf{u} = \begin{bmatrix}
    u_x \\
    u_y
\end{bmatrix}
\quad
\bar{\mathbf{N}} = \begin{bmatrix}
    1 & x & y & 0 & 0 & 0 \\
    0 & 0 & 0 & 1 & x & y
\end{bmatrix}\end{split}
\end{equation*}\begin{equation*}
\begin{split}\mathbf{C} = \begin{bmatrix}
    1 & x_1 & y_1 & 0 & 0 & 0 \\
    0 & 0 & 0 & 1 & x_1 & y_1 \\
    1 & x_2 & y_2 & 0 & 0 & 0 \\
    0 & 0 & 0 & 1 & x_2 & y_2 \\
    1 & x_3 & y_3 & 0 & 0 & 0 \\
    0 & 0 & 0 & 1 & x_3 & y_3 \\
\end{bmatrix}
\quad
\mathbf{a}^e = \begin{bmatrix}
    u_1 \\ u_2 \\ u_3 \\ u_4 \\ u_5 \\ u_6
\end{bmatrix}\end{split}
\end{equation*}
\sphinxAtStartPar
The matrix \(\bar{\mathbf{B}}\) is obtained as
\begin{equation*}
\begin{split}\bar{\mathbf{B}} = \tilde{\nabla} \bar{\mathbf{N}}
\quad \text{where} \quad
\tilde{\nabla} = \begin{bmatrix}
    \dfrac{\partial}{\partial x} & 0 \\
    0 & \dfrac{\partial}{\partial y} \\
    \dfrac{\partial}{\partial y} & \dfrac{\partial}{\partial x}
\end{bmatrix}\end{split}
\end{equation*}
\sphinxAtStartPar
If a larger \(\mathbf{D}\)\sphinxhyphen{}matrix than \(3 \times 3\) is used for plane stress (\(ptype=1\)), the \(\mathbf{D}\)\sphinxhyphen{}matrix is reduced to a \(3 \times 3\) matrix by static condensation using \(\sigma_{zz} = \sigma_{xz} = \sigma_{yz} = 0\). These stress components are connected with the rows 3, 5 and 6 in the D\sphinxhyphen{}matrix respectively.

\sphinxAtStartPar
If a larger \(\mathbf{D}\)\sphinxhyphen{}matrix than \(3 \times 3\) is used for plane strain (\(ptype=2\)), the \(\mathbf{D}\)\sphinxhyphen{}matrix is reduced to a \(3 \times 3\) matrix using \(\varepsilon_{zz} = \gamma_{xz} = \gamma_{yz} = 0\). This implies that a \(3 \times 3\) \(\mathbf{D}\)\sphinxhyphen{}matrix is created by the rows and the columns 1, 2 and 4 from the original D\sphinxhyphen{}matrix.

\sphinxAtStartPar
Evaluation of the integrals for the triangular element yields
\begin{equation*}
\begin{split}\mathbf{K}^e = (\mathbf{C}^{-1})^T \bar{\mathbf{B}}^T \mathbf{D} \bar{\mathbf{B}}\, \mathbf{C}^{-1}\, t\, A\end{split}
\end{equation*}\begin{equation*}
\begin{split}\mathbf{f}_l^e = \frac{A t}{3} \begin{bmatrix} b_x & b_y & b_x & b_y & b_x & b_y \end{bmatrix}^T\end{split}
\end{equation*}
\sphinxAtStartPar
where the element area \(A\) is determined as
\begin{equation*}
\begin{split}A = \frac{1}{2} \det \begin{bmatrix}
    1 & x_1 & y_1 \\
    1 & x_2 & y_2 \\
    1 & x_3 & y_3
\end{bmatrix}\end{split}
\end{equation*}
\end{description}\end{quote}


\subsection{plants}
\label{\detokenize{solid_functions:plants}}\begin{quote}\begin{description}
\sphinxlineitem{Purpose}
\sphinxAtStartPar
Compute stresses and strains in a triangular element in plane strain or plane stress.

\noindent{\hspace*{\fill}\sphinxincludegraphics[width=0.700\linewidth]{{PLANTS}.png}\hspace*{\fill}}

\sphinxlineitem{Syntax}
\begin{sphinxVerbatim}[commandchars=\\\{\}]
\PYG{p}{[}\PYG{n}{es}\PYG{p}{,}\PYG{+w}{ }\PYG{n}{et}\PYG{p}{]}\PYG{+w}{ }\PYG{p}{=}\PYG{+w}{ }\PYG{n}{plants}\PYG{p}{(}\PYG{n}{ex}\PYG{p}{,}\PYG{+w}{ }\PYG{n}{ey}\PYG{p}{,}\PYG{+w}{ }\PYG{n}{ep}\PYG{p}{,}\PYG{+w}{ }\PYG{n}{D}\PYG{p}{,}\PYG{+w}{ }\PYG{n}{ed}\PYG{p}{)}
\end{sphinxVerbatim}

\sphinxlineitem{Description}
\sphinxAtStartPar
\sphinxcode{\sphinxupquote{plants}} computes the stresses \sphinxcode{\sphinxupquote{es}} and the strains \sphinxcode{\sphinxupquote{et}} in a triangular element in plane strain or plane stress.

\sphinxAtStartPar
The input variables \(\mathbf{ex}\), \(\mathbf{ey}\), \(\mathbf{ep}\) and \(\mathbf{D}\) are defined in \sphinxcode{\sphinxupquote{plante}}.
The vector \(\mathbf{ed}\) contains the nodal displacements \(\mathbf{a}^e\) of the element and is obtained by the function \sphinxcode{\sphinxupquote{extract}} as
\begin{equation*}
\begin{split}\mathbf{ed} = (\mathbf{a}^e)^T = [\, u_1\;\; u_2\;\; \dots \;\; u_6\,]\end{split}
\end{equation*}
\sphinxAtStartPar
The output variables
\begin{equation*}
\begin{split}\mathrm{es} = \boldsymbol{\sigma}^T = \left[\, \sigma_{xx}\; \sigma_{yy}\; [\sigma_{zz}]\; \sigma_{xy}\; [\sigma_{xz}]\; [\sigma_{yz}]\, \right]\end{split}
\end{equation*}\begin{equation*}
\begin{split}\mathrm{et} = \boldsymbol{\varepsilon}^T = [\, \varepsilon_{xx}\; \varepsilon_{yy}\; [\varepsilon_{zz}]\; \gamma_{xy}\; [\gamma_{xz}]\; [\gamma_{yz}]\,]\end{split}
\end{equation*}
\sphinxAtStartPar
contain the stress and strain components. The size of \sphinxcode{\sphinxupquote{es}} and \sphinxcode{\sphinxupquote{et}} follows the size of \sphinxcode{\sphinxupquote{D}}.
Note that for plane stress \(\varepsilon_{zz} \neq 0\), and for plane strain \(\sigma_{zz} \neq 0\).

\sphinxlineitem{Theory}
\sphinxAtStartPar
The strains and stresses are computed according to
\begin{equation*}
\begin{split}\boldsymbol{\varepsilon} = \bar{\mathbf{B}}\, \mathbf{C}^{-1}\, \mathbf{a}^e\end{split}
\end{equation*}\begin{equation*}
\begin{split}\boldsymbol{\sigma} = \mathbf{D}\, \boldsymbol{\varepsilon}\end{split}
\end{equation*}
\sphinxAtStartPar
where the matrices \(\mathbf{D}\), \(\bar{\mathbf{B}}\), \(\mathbf{C}\) and \(\mathbf{a}^e\) are described in \sphinxcode{\sphinxupquote{plante}}.
Note that both the strains and the stresses are constant in the element.

\end{description}\end{quote}


\subsection{plantf}
\label{\detokenize{solid_functions:plantf}}\begin{quote}\begin{description}
\sphinxlineitem{Purpose}
\sphinxAtStartPar
Compute internal element force vector in a triangular element in plane strain or plane stress.

\sphinxlineitem{Syntax}
\begin{sphinxVerbatim}[commandchars=\\\{\}]
\PYG{n}{ef}\PYG{+w}{ }\PYG{p}{=}\PYG{+w}{ }\PYG{n}{plantf}\PYG{p}{(}\PYG{n}{ex}\PYG{p}{,}\PYG{+w}{ }\PYG{n}{ey}\PYG{p}{,}\PYG{+w}{ }\PYG{n}{ep}\PYG{p}{,}\PYG{+w}{ }\PYG{n}{es}\PYG{p}{)}
\end{sphinxVerbatim}

\sphinxlineitem{Description}
\sphinxAtStartPar
\sphinxcode{\sphinxupquote{plantf}} computes the internal element forces \sphinxcode{\sphinxupquote{ef}} in a triangular element in plane strain or plane stress.

\sphinxAtStartPar
The input variables \(\mathbf{ex}\), \(\mathbf{ey}\) and \(\mathbf{ep}\) are defined in \sphinxcode{\sphinxupquote{plante}}, and the input variable \(\mathbf{es}\) is defined in \sphinxcode{\sphinxupquote{plants}}.

\sphinxAtStartPar
The output variable
\begin{equation*}
\begin{split}\mathrm{ef} = \mathbf{f}_i^{eT} = \left[\, f_{i1}\; f_{i2}\; \dots \; f_{i6}\; \right]\end{split}
\end{equation*}
\sphinxAtStartPar
contains the components of the internal force vector.

\sphinxlineitem{Theory}
\sphinxAtStartPar
The internal force vector is computed according to
\begin{equation*}
\begin{split}\mathbf{f}_i^e = (\mathbf{C}^{-1})^T \int_A \bar{\mathbf{B}}^T \boldsymbol{\sigma}\; t\; dA\end{split}
\end{equation*}
\sphinxAtStartPar
where the matrices \(\bar{\mathbf{B}}\) and \(\mathbf{C}\) are defined in \sphinxcode{\sphinxupquote{plante}} and \(\boldsymbol{\sigma}\) is defined in \sphinxcode{\sphinxupquote{plants}}.

\sphinxAtStartPar
Evaluation of the integral for the triangular element yields
\begin{equation*}
\begin{split}\mathbf{f}_i^e = (\mathbf{C}^{-1})^T \bar{\mathbf{B}}^T\,\boldsymbol{\sigma}\; t\; A\end{split}
\end{equation*}
\end{description}\end{quote}


\subsection{planqe}
\label{\detokenize{solid_functions:planqe}}\begin{quote}\begin{description}
\sphinxlineitem{Purpose}
\sphinxAtStartPar
Compute element matrices for a quadrilateral element in plane strain or plane stress.

\begin{figure}[htbp]
\centering

\noindent\sphinxincludegraphics[width=0.700\linewidth]{{PLANI4E}.png}
\end{figure}

\sphinxlineitem{Syntax}
\begin{sphinxVerbatim}[commandchars=\\\{\}]
\PYG{n}{Ke}\PYG{+w}{ }\PYG{p}{=}\PYG{+w}{ }\PYG{n}{planqe}\PYG{p}{(}\PYG{n}{ex}\PYG{p}{,}\PYG{+w}{ }\PYG{n}{ey}\PYG{p}{,}\PYG{+w}{ }\PYG{n}{ep}\PYG{p}{,}\PYG{+w}{ }\PYG{n}{D}\PYG{p}{)}
\PYG{p}{[}\PYG{n}{Ke}\PYG{p}{,}\PYG{+w}{ }\PYG{n}{fe}\PYG{p}{]}\PYG{+w}{ }\PYG{p}{=}\PYG{+w}{ }\PYG{n}{planqe}\PYG{p}{(}\PYG{n}{ex}\PYG{p}{,}\PYG{+w}{ }\PYG{n}{ey}\PYG{p}{,}\PYG{+w}{ }\PYG{n}{ep}\PYG{p}{,}\PYG{+w}{ }\PYG{n}{D}\PYG{p}{,}\PYG{+w}{ }\PYG{n+nb}{eq}\PYG{p}{)}
\end{sphinxVerbatim}

\sphinxlineitem{Description}
\sphinxAtStartPar
\sphinxcode{\sphinxupquote{planqe}} provides an element stiffness matrix \(Ke\) and an element load vector \(fe\) for a quadrilateral element in plane strain or plane stress.

\sphinxAtStartPar
The element nodal coordinates \(x_1\), \(y_1\), \(x_2\), etc. are supplied to the function by \(\mathbf{ex}\) and \(\mathbf{ey}\). The type of analysis \(ptype\) and the element thickness \(t\) are supplied by \(\mathbf{ep}\):
\begin{equation*}
\begin{split}\begin{array}{lll}
ptype=1 & \quad \text{plane stress} \\
ptype=2 & \quad \text{plane strain}
\end{array}\end{split}
\end{equation*}
\sphinxAtStartPar
The material properties are supplied by the constitutive matrix \(D\). Any arbitrary \(\mathbf{D}\)\sphinxhyphen{}matrix with dimensions from \(3 \times 3\) to \(6 \times 6\) may be given. For an isotropic elastic material the constitutive matrix can be formed by the function \sphinxcode{\sphinxupquote{hooke}}, see Section {\hyperref[\detokenize{material_functions:material-functions}]{\sphinxcrossref{\DUrole{std}{\DUrole{std-ref}{Material functions}}}}}.
\begin{equation*}
\begin{split}\mathbf{ex} = [\,x_1 \;\, x_2 \;\; x_3\;\, x_4\,] \\
\mathbf{ey} = [\,y_1 \;\,\, y_2 \;\; y_3\;\,\, y_4\,] \\
\mathbf{ep} = [\,ptype \;\, t\,]\end{split}
\end{equation*}\begin{equation*}
\begin{split}\mathbf{D} =
\begin{bmatrix}
    D_{11} & D_{12} & D_{13} \\
    D_{21} & D_{22} & D_{23} \\
    D_{31} & D_{32} & D_{33}
\end{bmatrix}
\quad \text{or} \quad
\mathbf{D} =
\begin{bmatrix}
    D_{11} & D_{12} & D_{13} & D_{14} & D_{15} & D_{16} \\
    D_{21} & D_{22} & D_{23} & D_{24} & D_{25} & D_{26} \\
    D_{31} & D_{32} & D_{33} & D_{34} & D_{35} & D_{36} \\
    D_{41} & D_{42} & D_{43} & D_{44} & D_{45} & D_{46} \\
    D_{51} & D_{52} & D_{53} & D_{54} & D_{55} & D_{56} \\
    D_{61} & D_{62} & D_{63} & D_{64} & D_{65} & D_{66}
\end{bmatrix}\end{split}
\end{equation*}
\sphinxAtStartPar
If uniformly distributed loads are applied on the element, the element load vector \(fe\) is computed. The input variable
\begin{equation*}
\begin{split}eq = \begin{bmatrix}
    b_x \\
    b_y
\end{bmatrix}\end{split}
\end{equation*}
\sphinxAtStartPar
contains loads per unit volume, \(b_x\) and \(b_y\).

\sphinxlineitem{Theory}
\sphinxAtStartPar
In computing the element matrices, two more degrees of freedom are introduced. The location of these two degrees of freedom is defined by the mean value of the coordinates at the corner points. Four sets of element matrices are calculated using \sphinxcode{\sphinxupquote{plante}}. These matrices are then assembled and the two extra degrees of freedom are eliminated by static condensation.

\end{description}\end{quote}


\subsection{planqs}
\label{\detokenize{solid_functions:planqs}}\begin{quote}\begin{description}
\sphinxlineitem{Purpose}
\sphinxAtStartPar
Compute stresses and strains in a quadrilateral element in plane strain or plane stress.

\begin{figure}[htbp]
\centering

\noindent\sphinxincludegraphics[width=0.700\linewidth]{{PLANI4S}.png}
\end{figure}

\sphinxlineitem{Syntax}
\begin{sphinxVerbatim}[commandchars=\\\{\}]
\PYG{p}{[}\PYG{n}{es}\PYG{p}{,}\PYG{n}{et}\PYG{p}{]}\PYG{p}{=}\PYG{n}{planqs}\PYG{p}{(}\PYG{n}{ex}\PYG{p}{,}\PYG{n}{ey}\PYG{p}{,}\PYG{n}{ep}\PYG{p}{,}\PYG{n}{D}\PYG{p}{,}\PYG{n}{ed}\PYG{p}{)}
\PYG{p}{[}\PYG{n}{es}\PYG{p}{,}\PYG{n}{et}\PYG{p}{]}\PYG{p}{=}\PYG{n}{planqs}\PYG{p}{(}\PYG{n}{ex}\PYG{p}{,}\PYG{n}{ey}\PYG{p}{,}\PYG{n}{ep}\PYG{p}{,}\PYG{n}{D}\PYG{p}{,}\PYG{n}{ed}\PYG{p}{,}\PYG{n+nb}{eq}\PYG{p}{)}
\end{sphinxVerbatim}

\sphinxlineitem{Description}
\sphinxAtStartPar
\sphinxcode{\sphinxupquote{planqs}} computes the stresses \sphinxcode{\sphinxupquote{es}} and the strains \sphinxcode{\sphinxupquote{et}} in a quadrilateral element in plane strain or plane stress.

\sphinxAtStartPar
The input variables \(\mathbf{ex}\), \(\mathbf{ey}\), \(\mathbf{ep}\), \(\mathbf{D}\) and \(\mathbf{eq}\) are defined in \sphinxcode{\sphinxupquote{planqe}}.
The vector \(\mathbf{ed}\) contains the nodal displacements \(\mathbf{a}^e\) of the element and is obtained by the function \sphinxcode{\sphinxupquote{extract}} as
\begin{equation*}
\begin{split}\mathbf{ed} = (\mathbf{a}^e)^T = [\,u_1\;\; u_2\;\; \dots  \;\; u_8\,]\end{split}
\end{equation*}
\sphinxAtStartPar
If body forces are applied to the element the variable \(\mathbf{eq}\) must be included.

\sphinxAtStartPar
The output variables
\begin{equation*}
\begin{split}\mathrm{es} = \boldsymbol{\sigma}^T = [\, \sigma_{xx}\; \sigma_{yy}\; [\sigma_{zz}]\; \sigma_{xy}\; [\sigma_{xz}]\; [\sigma_{yz}]\,]\end{split}
\end{equation*}\begin{equation*}
\begin{split}\mathrm{et} = \boldsymbol{\varepsilon}^T = [\,\varepsilon_{xx}\;\varepsilon_{yy}\;[\varepsilon_{zz}]\;\gamma_{xy}\;[\gamma_{xz}]\;[\gamma_{yz}]\,]\end{split}
\end{equation*}
\sphinxAtStartPar
contain the stress and strain components. The size of \sphinxcode{\sphinxupquote{es}} and \sphinxcode{\sphinxupquote{et}} follows the size of \sphinxcode{\sphinxupquote{D}}.
Note that for plane stress \(\varepsilon_{zz} \neq 0\), and for plane strain \(\sigma_{zz} \neq 0\).

\sphinxlineitem{Theory}
\sphinxAtStartPar
By assembling triangular elements as described in \sphinxcode{\sphinxupquote{planqe}} a system of equations containing 10 degrees of freedom is obtained. From this system of equations the two unknown displacements at the center of the element are computed.
Then according to the description in \sphinxcode{\sphinxupquote{plants}} the strain and stress components in each of the four triangular elements are produced.
Finally the quadrilateral element strains and stresses are computed as area weighted mean values from the values of the four triangular elements.
If uniformly distributed loads are applied on the element, the element load vector \sphinxcode{\sphinxupquote{eq}} is needed for the calculations.

\end{description}\end{quote}


\subsection{planre}
\label{\detokenize{solid_functions:planre}}\begin{quote}\begin{description}
\sphinxlineitem{Purpose}
\sphinxAtStartPar
Compute element matrices for a rectangular (Melosh) element in plane strain or plane stress.

\begin{figure}[htbp]
\centering

\noindent\sphinxincludegraphics[width=0.700\linewidth]{{PLANTRE}.png}
\end{figure}

\sphinxlineitem{Syntax}
\begin{sphinxVerbatim}[commandchars=\\\{\}]
\PYG{n}{Ke}\PYG{+w}{ }\PYG{p}{=}\PYG{+w}{ }\PYG{n}{planre}\PYG{p}{(}\PYG{n}{ex}\PYG{p}{,}\PYG{+w}{ }\PYG{n}{ey}\PYG{p}{,}\PYG{+w}{ }\PYG{n}{ep}\PYG{p}{,}\PYG{+w}{ }\PYG{n}{D}\PYG{p}{)}
\PYG{p}{[}\PYG{n}{Ke}\PYG{p}{,}\PYG{+w}{ }\PYG{n}{fe}\PYG{p}{]}\PYG{+w}{ }\PYG{p}{=}\PYG{+w}{ }\PYG{n}{planre}\PYG{p}{(}\PYG{n}{ex}\PYG{p}{,}\PYG{+w}{ }\PYG{n}{ey}\PYG{p}{,}\PYG{+w}{ }\PYG{n}{ep}\PYG{p}{,}\PYG{+w}{ }\PYG{n}{D}\PYG{p}{,}\PYG{+w}{ }\PYG{n+nb}{eq}\PYG{p}{)}
\end{sphinxVerbatim}

\sphinxlineitem{Description}
\sphinxAtStartPar
planre provides an element stiffness matrix Ke and an element load vector fe for a rectangular (Melosh) element in plane strain or plane stress. This element can only be used if the element edges are parallel to the coordinate axis.

\sphinxAtStartPar
The element nodal coordinates \((x_1, y_1)\) and \((x_3, y_3)\) are supplied to the function by ex and ey. The type of analysis ptype and the element thickness t are supplied by ep,
\begin{equation*}
\begin{split}\begin{array}{lll}
ptype=1 \quad \mbox{plane stress} \\
ptype=2 \quad \mbox{plane strain}
\end{array}\end{split}
\end{equation*}
\sphinxAtStartPar
and the material properties are supplied by the constitutive matrix D. Any arbitrary D\sphinxhyphen{}matrix with dimensions from \(3 \times 3\) to \(6 \times 6\) may be given. For an isotropic elastic material the constitutive matrix can be formed by the function hooke, see Section Material.
\begin{equation*}
\begin{split}\begin{array}{l}
\mathbf{ex} = [\,x_1 \;\, x_3\,]  \\
\mathbf{ey} = [\,y_1 \;\, y_3\,]
\end{array}
\qquad
\mathbf{ep} = [\,ptype \;\, t\,]\end{split}
\end{equation*}\begin{equation*}
\begin{split}\mathbf{D} = \begin{bmatrix}
D_{11} & D_{12} & D_{13} \\
D_{21} & D_{22} & D_{23} \\
D_{31} & D_{32} & D_{33}
\end{bmatrix}
\quad \text{or} \quad
\mathbf{D} = \begin{bmatrix}
D_{11} & D_{12} & D_{13} & D_{14} & [D_{15}] & [D_{16}] \\
D_{21} & D_{22} & D_{23} & D_{24} & [D_{25}] & [D_{26}] \\
D_{31} & D_{32} & D_{33} & D_{34} & [D_{35}] & [D_{36}] \\
D_{41} & D_{42} & D_{43} & D_{44} & [D_{45}] & [D_{46}] \\
[D_{51}] & [D_{52}] & [D_{53}] & [D_{54}] & [D_{55}] & [D_{56}] \\
[D_{61}] & [D_{62}] & [D_{63}] & [D_{64}] & [D_{65}] & [D_{66}]
\end{bmatrix}\end{split}
\end{equation*}
\sphinxAtStartPar
If uniformly distributed loads are applied on the element, the element load vector fe is computed. The input variable
\begin{equation*}
\begin{split}eq = \begin{bmatrix}
b_x \\
b_y
\end{bmatrix}\end{split}
\end{equation*}
\sphinxAtStartPar
containing loads per unit volume, \(b_x\) and \(b_y\), is then given.

\sphinxlineitem{Theory}
\sphinxAtStartPar
The element stiffness matrix \(\mathbf{K}^e\) and the element load vector \(\mathbf{f}_l^e\), stored in Ke and fe, respectively, are computed according to
\begin{equation*}
\begin{split}\mathbf{K}^e = \int_A \mathbf{B}^{eT} \mathbf{D} \mathbf{B}^e t\, dA\end{split}
\end{equation*}\begin{equation*}
\begin{split}\mathbf{f}_l^e = \int_A \mathbf{N}^{eT} \mathbf{b} t\, dA\end{split}
\end{equation*}
\sphinxAtStartPar
with the constitutive matrix \(\mathbf{D}\) defined by D, and the body force vector \(\mathbf{b}\) defined by eq.

\sphinxAtStartPar
The evaluation of the integrals for the rectangular element is based on a bilinear displacement approximation \(\mathbf{u}(x,y)\) and is expressed in terms of the nodal variables \(u_1, u_2, \dots, u_8\) as
\begin{equation*}
\begin{split}\mathbf{u}(x, y) = \mathbf{N}^e \mathbf{a}^e\end{split}
\end{equation*}
\sphinxAtStartPar
where
\begin{equation*}
\begin{split}\mathbf{u} = \begin{bmatrix} u_x \\ u_y \end{bmatrix} \quad
\mathbf{N}^e = \begin{bmatrix}
N^e_1 & 0 & N^e_2 & 0 & N^e_3 & 0 & N^e_4 & 0 \\
0 & N^e_1 & 0 & N^e_2 & 0 & N^e_3 & 0 & N^e_4
\end{bmatrix}
\quad
\mathbf{a}^e = \begin{bmatrix}
u_1 \\ u_2 \\ \vdots \\ u_8
\end{bmatrix}\end{split}
\end{equation*}
\sphinxAtStartPar
With a local coordinate system located at the center of the element, the element shape functions \(N^e_1\)\textendash{}\(N^e_4\) are obtained as
\begin{equation*}
\begin{split}N^e_1 = \frac{1}{4ab}(x - x_2)(y - y_4) \\
N^e_2 = -\frac{1}{4ab}(x - x_1)(y - y_3) \\
N^e_3 = \frac{1}{4ab}(x - x_4)(y - y_2) \\
N^e_4 = -\frac{1}{4ab}(x - x_3)(y - y_1)\end{split}
\end{equation*}
\sphinxAtStartPar
where
\begin{equation*}
\begin{split}a = \frac{1}{2}(x_3 - x_1) \quad \text{and} \quad b = \frac{1}{2}(y_3 - y_1)\end{split}
\end{equation*}
\sphinxAtStartPar
The matrix \(\mathbf{B}^e\) is obtained as
\begin{equation*}
\begin{split}\mathbf{B}^e = \tilde{\nabla} \mathbf{N}^e \qquad
\text{where} \quad \tilde{\nabla} = \begin{bmatrix}
\dfrac{\partial}{\partial x} & 0 \\
0 & \dfrac{\partial}{\partial y} \\
\dfrac{\partial}{\partial y} & \dfrac{\partial}{\partial x}
\end{bmatrix}\end{split}
\end{equation*}
\sphinxAtStartPar
If a larger D\sphinxhyphen{}matrix than \(3 \\times 3\) is used for plane stress (\(ptype=1\)), the D\sphinxhyphen{}matrix is reduced to a \(3 \\times 3\) matrix by static condensation using \(\sigma_{zz} = \sigma_{xz} = \sigma_{yz} = 0\). These stress components are connected with the rows 3, 5 and 6 in the D\sphinxhyphen{}matrix respectively.

\sphinxAtStartPar
If a larger D\sphinxhyphen{}matrix than \(3 \\times 3\) is used for plane strain (\(ptype=2\)), the D\sphinxhyphen{}matrix is reduced to a \(3 \\times 3\) matrix using \(\varepsilon_{zz} = \gamma_{xz} = \gamma_{yz} = 0\). This implies that a \(3 \\times 3\) D\sphinxhyphen{}matrix is created by the rows and the columns 1, 2 and 4 from the original D\sphinxhyphen{}matrix.

\sphinxAtStartPar
Evaluation of the integrals for the rectangular element can be done either analytically or numerically by use of a \(2 \\times 2\) point Gauss integration. The element load vector \(\mathbf{f}_l^e\) yields
\begin{equation*}
\begin{split}\mathbf{f}_l^e = abt \begin{bmatrix}
b_x \\ b_y \\ b_x \\ b_y \\
b_x \\ b_y \\ b_x \\ b_y
\end{bmatrix}\end{split}
\end{equation*}
\end{description}\end{quote}


\subsection{planrs}
\label{\detokenize{solid_functions:planrs}}\begin{quote}\begin{description}
\sphinxlineitem{Purpose}
\sphinxAtStartPar
Compute stresses and strains in a rectangular (Melosh) element in plane strain or plane stress.

\begin{figure}[htbp]
\centering

\noindent\sphinxincludegraphics[width=0.700\linewidth]{{PLANTRS}.png}
\end{figure}

\sphinxlineitem{Syntax}
\begin{sphinxVerbatim}[commandchars=\\\{\}]
\PYG{p}{[}\PYG{n}{es}\PYG{p}{,}\PYG{n}{et}\PYG{p}{]}\PYG{p}{=}\PYG{n}{planrs}\PYG{p}{(}\PYG{n}{ex}\PYG{p}{,}\PYG{n}{ey}\PYG{p}{,}\PYG{n}{ep}\PYG{p}{,}\PYG{n}{D}\PYG{p}{,}\PYG{n}{ed}\PYG{p}{)}
\end{sphinxVerbatim}

\sphinxlineitem{Description}
\sphinxAtStartPar
\sphinxcode{\sphinxupquote{planrs}} computes the stresses \sphinxcode{\sphinxupquote{es}} and the strains \sphinxcode{\sphinxupquote{et}} in a rectangular (Melosh) element in plane strain or plane stress. The stress and strain components are computed at the center of the element.

\sphinxAtStartPar
The input variables \(\mathbf{ex}\), \(\mathbf{ey}\), \(\mathbf{ep}\) and \(\mathbf{D}\) are defined in \sphinxcode{\sphinxupquote{planre}}. The vector \(\mathbf{ed}\) contains the nodal displacements \(\mathbf{a}^e\) of the element and is obtained by the function \sphinxcode{\sphinxupquote{extract}} as
\begin{equation*}
\begin{split}\mathbf{ed} = (\mathbf{a}^e)^T = [\,u_1\;\; u_2\;\; \dots \;\; u_8\,]\end{split}
\end{equation*}
\sphinxAtStartPar
The output variables
\begin{equation*}
\begin{split}\mathrm{es} = \boldsymbol{\sigma}^T = [\, \sigma_{xx}\; \sigma_{yy}\; [\sigma_{zz}]\; \sigma_{xy}\; [\sigma_{xz}]\; [\sigma_{yz}]\,]\end{split}
\end{equation*}\begin{equation*}
\begin{split}\mathrm{et} = \boldsymbol{\varepsilon}^T = [\, \varepsilon_{xx}\; \varepsilon_{yy}\; [\varepsilon_{zz}]\; \gamma_{xy}\; [\gamma_{xz}]\; [\gamma_{yz}]\,]\end{split}
\end{equation*}
\sphinxAtStartPar
contain the stress and strain components. The size of \sphinxcode{\sphinxupquote{es}} and \sphinxcode{\sphinxupquote{et}} follows the size of \sphinxcode{\sphinxupquote{D}}. Note that for plane stress \(\varepsilon_{zz} \neq 0\), and for plane strain \(\sigma_{zz} \neq 0\).

\sphinxlineitem{Theory}
\sphinxAtStartPar
The strains and stresses are computed according to
\begin{equation*}
\begin{split}\boldsymbol{\varepsilon} = \mathbf{B}^e\,\mathbf{a}^e\end{split}
\end{equation*}\begin{equation*}
\begin{split}\boldsymbol{\sigma} = \mathbf{D}\,\boldsymbol{\varepsilon}\end{split}
\end{equation*}
\sphinxAtStartPar
where the matrices \(\mathbf{D}\), \(\mathbf{B}^e\), and \(\mathbf{a}^e\) are described in \sphinxcode{\sphinxupquote{planre}}, and where the evaluation point \((x, y)\) is chosen to be at the center of the element.

\end{description}\end{quote}


\subsection{plantce}
\label{\detokenize{solid_functions:plantce}}\begin{quote}\begin{description}
\sphinxlineitem{Purpose}
\sphinxAtStartPar
Compute element matrices for a rectangular (Turner\sphinxhyphen{}Clough)
element in plane strain or plane stress.

\begin{figure}[htbp]
\centering

\noindent\sphinxincludegraphics[width=0.700\linewidth]{{PLANTRE}.png}
\end{figure}

\sphinxlineitem{Syntax}
\begin{sphinxVerbatim}[commandchars=\\\{\}]
\PYG{n}{Ke}\PYG{+w}{ }\PYG{p}{=}\PYG{+w}{ }\PYG{n}{plantce}\PYG{p}{(}\PYG{n}{ex}\PYG{p}{,}\PYG{+w}{ }\PYG{n}{ey}\PYG{p}{,}\PYG{+w}{ }\PYG{n}{ep}\PYG{p}{)}
\PYG{p}{[}\PYG{n}{Ke}\PYG{p}{,}\PYG{+w}{ }\PYG{n}{fe}\PYG{p}{]}\PYG{+w}{ }\PYG{p}{=}\PYG{+w}{ }\PYG{n}{plantce}\PYG{p}{(}\PYG{n}{ex}\PYG{p}{,}\PYG{+w}{ }\PYG{n}{ey}\PYG{p}{,}\PYG{+w}{ }\PYG{n}{ep}\PYG{p}{,}\PYG{+w}{ }\PYG{n+nb}{eq}\PYG{p}{)}
\end{sphinxVerbatim}

\sphinxlineitem{Description}
\sphinxAtStartPar
\sphinxcode{\sphinxupquote{plantce}} provides an element stiffness matrix \sphinxcode{\sphinxupquote{Ke}} and an element
load vector \sphinxcode{\sphinxupquote{fe}} for a rectangular (Turner\sphinxhyphen{}Clough) element in plane strain
or plane stress. This element can only be used if the material is isotropic and
if the element edges are parallel to the coordinate axis.

\sphinxAtStartPar
The element nodal coordinates \((x_1, y_1)\) and
\((x_3, y_3)\) are supplied to the function
by \(\mathbf{ex}\) and \(\mathbf{ey}\). The state of stress \sphinxcode{\sphinxupquote{ptype}}, the element thickness \(t\)
and the material properties \(E\) and \(\nu\) are supplied by \(\mathbf{ep}\).
For plane stress \(ptype=1\) and for plane strain \(ptype=2\).
\begin{equation*}
\begin{split}\begin{array}{l}
\mathbf{ex} = [\, x_1 \;\; x_3\,] \\
\mathbf{ey} = [\, y_1 \;\; y_3\,]
\end{array}
\qquad
\mathbf{ep} = [\, ptype \;\; t \;\; E \;\; \nu \,]\end{split}
\end{equation*}
\sphinxAtStartPar
If uniformly distributed loads are applied to the element,
the element load vector \sphinxcode{\sphinxupquote{fe}} is computed. The input variable
\begin{equation*}
\begin{split}eq = \begin{bmatrix}
    b_x \\
    b_y
\end{bmatrix}\end{split}
\end{equation*}
\sphinxAtStartPar
containing loads per unit volume, \(b_x\) and \(b_y\), is then given.

\sphinxlineitem{Theory}
\sphinxAtStartPar
The element stiffness matrix \(\mathbf{K}^e\) and the
element load vector \(\mathbf{f}_l^e\), stored in \sphinxcode{\sphinxupquote{Ke}} and \sphinxcode{\sphinxupquote{fe}},
respectively, are computed according to
\begin{equation*}
\begin{split}\mathbf{K}^e = \int_A \mathbf{B}^{eT} \mathbf{D} \mathbf{B}^e t\, dA\end{split}
\end{equation*}\begin{equation*}
\begin{split}\mathbf{f}_l^e = \int_A \mathbf{N}^{eT} \mathbf{b} t\, dA\end{split}
\end{equation*}
\sphinxAtStartPar
where the constitutive matrix \(\mathbf{D}\) is described in \sphinxcode{\sphinxupquote{hooke}},
see Section {\hyperref[\detokenize{material_functions:material-functions}]{\sphinxcrossref{\DUrole{std}{\DUrole{std-ref}{Material functions}}}}}, and the body force vector \(\mathbf{b}\)
is defined by \sphinxcode{\sphinxupquote{eq}}.

\sphinxAtStartPar
The evaluation of the integrals for the Turner\sphinxhyphen{}Clough element is based on a
displacement field \(\mathbf{u}(x, y)\) built up of a bilinear displacement
approximation superposed by bubble functions in order to create a linear stress field over the element.
The displacement field is expressed in terms of the nodal variables \(u_1, u_2, \dots, u_8\) as
\begin{equation*}
\begin{split}\mathbf{u}(x, y) = \mathbf{N}^e \mathbf{a}^e\end{split}
\end{equation*}
\sphinxAtStartPar
where
\begin{equation*}
\begin{split}\mathbf{u} = \begin{bmatrix}
    u_x \\
    u_y
\end{bmatrix}
\quad
\mathbf{N}^e = \begin{bmatrix}
    N^e_1 & N^e_5 & N^e_2 & -N^e_5 & N^e_3 & N^e_5 & N^e_4 & -N^e_5 \\
    N^e_6 & N^e_1 & -N^e_6 & N^e_2 & N^e_6 & N^e_3 & -N^e_6 & N^e_4
\end{bmatrix}
\quad
\mathbf{a}^e = \begin{bmatrix}
    u_1 \\ u_2 \\ \vdots \\ u_8
\end{bmatrix}\end{split}
\end{equation*}
\sphinxAtStartPar
With a local coordinate system located at the center of the element,
the element shape functions \(N^e_1\)\textendash{}\(N^e_6\) are obtained as
\begin{equation*}
\begin{split}N^e_1 = \frac{1}{4ab}(a-x)(b-y) \\
N^e_2 = \frac{1}{4ab}(a+x)(b-y) \\
N^e_3 = \frac{1}{4ab}(a+x)(b+y) \\
N^e_4 = \frac{1}{4ab}(a-x)(b+y) \\
N^e_5 = \frac{1}{8ab}\left[ (b^2-y^2) + \nu (a^2-x^2) \right] \\
N^e_6 = \frac{1}{8ab}\left[ (a^2-x^2) + \nu (b^2-y^2) \right]\end{split}
\end{equation*}
\sphinxAtStartPar
where
\begin{equation*}
\begin{split}a = \frac{1}{2}(x_3 - x_1) \qquad b = \frac{1}{2}(y_3 - y_1)\end{split}
\end{equation*}
\sphinxAtStartPar
The matrix \(\mathbf{B}^e\) is obtained as
\begin{equation*}
\begin{split}\mathbf{B}^e = \tilde{\nabla} \mathbf{N}^e \qquad
\text{where} \quad \tilde{\nabla} = \begin{bmatrix}
    \dfrac{\partial}{\partial x} & 0 \\
    0 & \dfrac{\partial}{\partial y} \\
    \dfrac{\partial}{\partial y} & \dfrac{\partial}{\partial x}
\end{bmatrix}\end{split}
\end{equation*}
\sphinxAtStartPar
Evaluation of the integrals for the Turner\sphinxhyphen{}Clough element can be done either
analytically or numerically by use of a \(2 \times 2\) point Gauss integration.
The element load vector \(\mathbf{f}_l^e\) yields
\begin{equation*}
\begin{split}\mathbf{f}_l^e = abt \begin{bmatrix}
    b_x \\ b_y \\ b_x \\ b_y \\
    b_x \\ b_y \\ b_x \\ b_y
\end{bmatrix}\end{split}
\end{equation*}
\end{description}\end{quote}


\subsection{plantcs}
\label{\detokenize{solid_functions:plantcs}}\begin{quote}\begin{description}
\sphinxlineitem{Purpose}
\sphinxAtStartPar
Compute stresses and strains in a Turner\sphinxhyphen{}Clough element in plane strain or plane stress.

\begin{figure}[htbp]
\centering

\noindent\sphinxincludegraphics[width=0.700\linewidth]{{PLANTRS}.png}
\end{figure}

\sphinxlineitem{Syntax}
\begin{sphinxVerbatim}[commandchars=\\\{\}]
\PYG{p}{[}\PYG{n}{es}\PYG{p}{,}\PYG{+w}{ }\PYG{n}{et}\PYG{p}{]}\PYG{+w}{ }\PYG{p}{=}\PYG{+w}{ }\PYG{n}{plantcs}\PYG{p}{(}\PYG{n}{ex}\PYG{p}{,}\PYG{+w}{ }\PYG{n}{ey}\PYG{p}{,}\PYG{+w}{ }\PYG{n}{ep}\PYG{p}{,}\PYG{+w}{ }\PYG{n}{ed}\PYG{p}{)}
\end{sphinxVerbatim}

\sphinxlineitem{Description}
\sphinxAtStartPar
\sphinxcode{\sphinxupquote{plantcs}} computes the stresses \sphinxcode{\sphinxupquote{es}} and the strains \sphinxcode{\sphinxupquote{et}} in a rectangular Turner\sphinxhyphen{}Clough element in plane strain or plane stress. The stress and strain components are computed at the center of the element.

\sphinxAtStartPar
The input variables \(\mathbf{ex}\), \(\mathbf{ey}\), and \(\mathbf{ep}\) are defined in \sphinxcode{\sphinxupquote{plantce}}. The vector \(\mathbf{ed}\) contains the nodal displacements \(\mathbf{a}^e\) of the element and is obtained by the function \sphinxcode{\sphinxupquote{extract}} as
\begin{equation*}
\begin{split}\mathbf{ed} = (\mathbf{a}^e)^T = [\, u_1\;\; u_2\;\; \dots \;\; u_8\,]\end{split}
\end{equation*}
\sphinxAtStartPar
The output variables
\begin{equation*}
\begin{split}\mathrm{es} = \boldsymbol{\sigma}^T = [\, \sigma_{xx}\; \sigma_{yy}\; [\sigma_{zz}]\; \sigma_{xy}\; [\sigma_{xz}]\; [\sigma_{yz}]\,]\end{split}
\end{equation*}\begin{equation*}
\begin{split}\mathrm{et} = \boldsymbol{\varepsilon}^T = [\, \varepsilon_{xx}\; \varepsilon_{yy}\; [\varepsilon_{zz}]\; \gamma_{xy}\; [\gamma_{xz}]\; [\gamma_{yz}]\,]\end{split}
\end{equation*}
\sphinxAtStartPar
contain the stress and strain components. The size of \sphinxcode{\sphinxupquote{es}} and \sphinxcode{\sphinxupquote{et}} follows the size of \sphinxcode{\sphinxupquote{D}}. Note that for plane stress \(\varepsilon_{zz} \neq 0\), and for plane strain \(\sigma_{zz} \neq 0\).

\sphinxlineitem{Theory}
\sphinxAtStartPar
The strains and stresses are computed according to
\begin{equation*}
\begin{split}\boldsymbol{\varepsilon} = \mathbf{B}^e\,\mathbf{a}^e\end{split}
\end{equation*}\begin{equation*}
\begin{split}\boldsymbol{\sigma} = \mathbf{D}\;\boldsymbol{\varepsilon}\end{split}
\end{equation*}
\sphinxAtStartPar
where the matrices \(\mathbf{D}\), \(\mathbf{B}^e\), and \(\mathbf{a}^e\) are described in \sphinxcode{\sphinxupquote{plantce}}, and where the evaluation point \((x, y)\) is chosen to be at the center of the element.

\end{description}\end{quote}


\subsection{plani4e}
\label{\detokenize{solid_functions:plani4e}}\begin{quote}\begin{description}
\sphinxlineitem{Purpose}
\sphinxAtStartPar
Compute element matrices for a 4 node isoparametric element in plane strain or plane stress.

\begin{figure}[htbp]
\centering

\noindent\sphinxincludegraphics[width=0.700\linewidth]{{PLANI4E}.png}
\end{figure}

\sphinxlineitem{Syntax}
\begin{sphinxVerbatim}[commandchars=\\\{\}]
\PYG{n}{Ke}\PYG{+w}{ }\PYG{p}{=}\PYG{+w}{ }\PYG{n}{plani4e}\PYG{p}{(}\PYG{n}{ex}\PYG{p}{,}\PYG{+w}{ }\PYG{n}{ey}\PYG{p}{,}\PYG{+w}{ }\PYG{n}{ep}\PYG{p}{,}\PYG{+w}{ }\PYG{n}{D}\PYG{p}{)}
\PYG{p}{[}\PYG{n}{Ke}\PYG{p}{,}\PYG{+w}{ }\PYG{n}{fe}\PYG{p}{]}\PYG{+w}{ }\PYG{p}{=}\PYG{+w}{ }\PYG{n}{plani4e}\PYG{p}{(}\PYG{n}{ex}\PYG{p}{,}\PYG{+w}{ }\PYG{n}{ey}\PYG{p}{,}\PYG{+w}{ }\PYG{n}{ep}\PYG{p}{,}\PYG{+w}{ }\PYG{n}{D}\PYG{p}{,}\PYG{+w}{ }\PYG{n+nb}{eq}\PYG{p}{)}
\end{sphinxVerbatim}

\sphinxlineitem{Description}
\sphinxAtStartPar
\sphinxcode{\sphinxupquote{plani4e}} provides an element stiffness matrix \sphinxcode{\sphinxupquote{Ke}} and an element load vector \sphinxcode{\sphinxupquote{fe}} for a 4 node isoparametric element in plane strain or plane stress.

\sphinxAtStartPar
The element nodal coordinates \(x_1, y_1, x_2, \ldots\) are supplied to the function by \(\mathbf{ex}\) and \(\mathbf{ey}\). The type of analysis \(ptype\), the element thickness \(t\), and the number of Gauss points \(n\) are supplied by \(\mathbf{ep}\):
\begin{equation*}
\begin{split}\begin{array}{lll}
ptype=1 & \text{plane stress} & (n \times n) \text{ integration points} \\
ptype=2 & \text{plane strain} & n=1,2,3
\end{array}\end{split}
\end{equation*}
\sphinxAtStartPar
The material properties are supplied by the constitutive matrix \(D\). Any arbitrary \(\mathbf{D}\)\sphinxhyphen{}matrix with dimensions from \(3 \times 3\) to \(6 \times 6\) may be given. For an isotropic elastic material the constitutive matrix can be formed by the function \sphinxcode{\sphinxupquote{hooke}}, see Section {\hyperref[\detokenize{material_functions:material-functions}]{\sphinxcrossref{\DUrole{std}{\DUrole{std-ref}{Material functions}}}}}.
\begin{equation*}
\begin{split}\mathbf{ex} = [\,x_1 \;\, x_2 \;\; x_3\;\, x_4\,] \\
\mathbf{ey} = [\,y_1 \;\,\, y_2 \;\; y_3\;\,\, y_4\,] \\
\mathbf{ep} = [\,ptype \;\, t\;\, n\,]\end{split}
\end{equation*}\begin{equation*}
\begin{split}\mathbf{D} = \begin{bmatrix}
D_{11} & D_{12} & D_{13} \\
D_{21} & D_{22} & D_{23} \\
D_{31} & D_{32} & D_{33}
\end{bmatrix}
\quad \text{or} \quad
\mathbf{D} = \begin{bmatrix}
D_{11} & D_{12} & D_{13} & D_{14} & [D_{15}] & [D_{16}] \\
D_{21} & D_{22} & D_{23} & D_{24} & [D_{25}] & [D_{26}] \\
D_{31} & D_{32} & D_{33} & D_{34} & [D_{35}] & [D_{36}] \\
D_{41} & D_{42} & D_{43} & D_{44} & [D_{45}] & [D_{46}] \\
[D_{51}] & [D_{52}] & [D_{53}] & [D_{54}] & [D_{55}] & [D_{56}] \\
[D_{61}] & [D_{62}] & [D_{63}] & [D_{64}] & [D_{65}] & [D_{66}]
\end{bmatrix}\end{split}
\end{equation*}
\sphinxAtStartPar
If different \(\mathbf{D}_i\) matrices are used in the Gauss points, these \(\mathbf{D}_i\) matrices are stored in a global vector \sphinxcode{\sphinxupquote{D}}. For numbering of the Gauss points, see \sphinxcode{\sphinxupquote{eci}} in \sphinxcode{\sphinxupquote{plani4s}}.
\begin{equation*}
\begin{split}\mathbf{D} = \begin{bmatrix}
\mathbf{D}_1 \\
\mathbf{D}_2 \\
\vdots \\
\mathbf{D}_{n^2}
\end{bmatrix}\end{split}
\end{equation*}
\sphinxAtStartPar
If uniformly distributed loads are applied to the element, the element load vector \sphinxcode{\sphinxupquote{fe}} is computed. The input variable
\begin{equation*}
\begin{split}\mathrm{eq} = \begin{bmatrix}
b_x \\
b_y
\end{bmatrix}\end{split}
\end{equation*}
\sphinxAtStartPar
containing loads per unit volume, \(b_x\) and \(b_y\), is then given.

\sphinxlineitem{Theory}
\sphinxAtStartPar
The element stiffness matrix \(\mathbf{K}^e\) and the element load vector \(\mathbf{f}_l^e\), stored in \sphinxcode{\sphinxupquote{Ke}} and \sphinxcode{\sphinxupquote{fe}}, respectively, are computed according to
\begin{equation*}
\begin{split}\mathbf{K}^e = \int_A \mathbf{B}^{eT} \mathbf{D} \mathbf{B}^e t\, dA \\
\mathbf{f}_l^e = \int_A \mathbf{N}^{eT} \mathbf{b} t\, dA\end{split}
\end{equation*}
\sphinxAtStartPar
with the constitutive matrix \(\mathbf{D}\) defined by \sphinxcode{\sphinxupquote{D}}, and the body force vector \(\mathbf{b}\) defined by \sphinxcode{\sphinxupquote{eq}}.

\sphinxAtStartPar
The evaluation of the integrals for the isoparametric 4 node element is based on a displacement approximation \(\mathbf{u}(\xi, \eta)\), expressed in a local coordinate system in terms of the nodal variables \(u_1, u_2, \ldots, u_8\) as
\begin{equation*}
\begin{split}\mathbf{u}(\xi, \eta) = \mathbf{N}^e \mathbf{a}^e\end{split}
\end{equation*}
\sphinxAtStartPar
where
\begin{equation*}
\begin{split}\mathbf{u} = \begin{bmatrix} u_x \\ u_y \end{bmatrix} \qquad
\mathbf{N}^e = \begin{bmatrix}
N_1^e & 0 & N_2^e & 0 & N_3^e & 0 & N_4^e & 0 \\
0 & N_1^e & 0 & N_2^e & 0 & N_3^e & 0 & N_4^e
\end{bmatrix} \qquad
\mathbf{a}^e = \begin{bmatrix}
u_1 \\ u_2 \\ \vdots \\ u_8
\end{bmatrix}\end{split}
\end{equation*}
\sphinxAtStartPar
The element shape functions are given by
\begin{equation*}
\begin{split}N_1^e = \frac{1}{4}(1-\xi)(1-\eta) \qquad N_2^e = \frac{1}{4}(1+\xi)(1-\eta) \\
N_3^e = \frac{1}{4}(1+\xi)(1+\eta) \qquad N_4^e = \frac{1}{4}(1-\xi)(1+\eta)\end{split}
\end{equation*}
\sphinxAtStartPar
The matrix \(\mathbf{B}^e\) is obtained as
\begin{equation*}
\begin{split}\mathbf{B}^e = \tilde{\nabla} \mathbf{N}^e \qquad
\text{where} \quad \tilde{\nabla} = \begin{bmatrix}
\frac{\partial}{\partial x} & 0 \\
0 & \frac{\partial}{\partial y} \\
\frac{\partial}{\partial y} & \frac{\partial}{\partial x}
\end{bmatrix}\end{split}
\end{equation*}
\sphinxAtStartPar
and
\begin{equation*}
\begin{split}\begin{bmatrix}
\frac{\partial}{\partial x} \\
\frac{\partial}{\partial y}
\end{bmatrix}
= (\mathbf{J}^T)^{-1}
\begin{bmatrix}
\frac{\partial}{\partial \xi} \\
\frac{\partial}{\partial \eta}
\end{bmatrix}
\qquad
\mathbf{J} = \begin{bmatrix}
\frac{\partial x}{\partial \xi} & \frac{\partial x}{\partial \eta} \\
\frac{\partial y}{\partial \xi} & \frac{\partial y}{\partial \eta}
\end{bmatrix}\end{split}
\end{equation*}
\sphinxAtStartPar
If a larger \(\mathbf{D}\)\sphinxhyphen{}matrix than \(3 \times 3\) is used for plane stress (\(ptype=1\)), the \(\mathbf{D}\)\sphinxhyphen{}matrix is reduced to a \(3 \times 3\) matrix by static condensation using \(\sigma_{zz} = \sigma_{xz} = \sigma_{yz} = 0\). These stress components are connected with the rows 3, 5 and 6 in the D\sphinxhyphen{}matrix respectively.

\sphinxAtStartPar
If a larger \(\mathbf{D}\)\sphinxhyphen{}matrix than \(3 \times 3\) is used for plane strain (\(ptype=2\)), the \(\mathbf{D}\)\sphinxhyphen{}matrix is reduced to a \(3 \times 3\) matrix using \(\varepsilon_{zz} = \gamma_{xz} = \gamma_{yz} = 0\). This implies that a \(3 \times 3\) \(\mathbf{D}\)\sphinxhyphen{}matrix is created by the rows and the columns 1, 2 and 4 from the original D\sphinxhyphen{}matrix.

\sphinxAtStartPar
Evaluation of the integrals is done by Gauss integration.

\end{description}\end{quote}


\subsection{plani4s}
\label{\detokenize{solid_functions:plani4s}}\begin{quote}\begin{description}
\sphinxlineitem{Purpose}
\sphinxAtStartPar
Compute stresses and strains in a 4 node isoparametric element in plane strain or plane stress.

\begin{figure}[htbp]
\centering

\noindent\sphinxincludegraphics[width=0.700\linewidth]{{PLANI4S}.png}
\end{figure}

\sphinxlineitem{Syntax}
\begin{sphinxVerbatim}[commandchars=\\\{\}]
\PYG{p}{[}\PYG{n}{es}\PYG{p}{,}\PYG{n}{et}\PYG{p}{,}\PYG{n}{eci}\PYG{p}{]}\PYG{p}{=}\PYG{n}{plani4s}\PYG{p}{(}\PYG{n}{ex}\PYG{p}{,}\PYG{n}{ey}\PYG{p}{,}\PYG{n}{ep}\PYG{p}{,}\PYG{n}{D}\PYG{p}{,}\PYG{n}{ed}\PYG{p}{)}
\end{sphinxVerbatim}

\sphinxlineitem{Description}
\sphinxAtStartPar
\sphinxcode{\sphinxupquote{plani4s}} computes stresses \sphinxcode{\sphinxupquote{es}} and the strains \sphinxcode{\sphinxupquote{et}} in a 4 node isoparametric element in plane strain or plane stress.

\sphinxAtStartPar
The input variables \(\mathbf{ex}\), \(\mathbf{ey}\), \(\mathbf{ep}\) and the matrix \(\mathbf{D}\) are defined in \sphinxcode{\sphinxupquote{plani4e}}.
The vector \(\mathbf{ed}\) contains the nodal displacements \(\mathbf{a}^e\) of the element and is obtained by the function \sphinxcode{\sphinxupquote{extract}} as
\begin{equation*}
\begin{split}\mathbf{ed} = (\mathbf{a}^e)^T = [\,u_1\;\; u_2\;\; \dots \;\; u_8\,]\end{split}
\end{equation*}
\sphinxAtStartPar
The output variables
\begin{equation*}
\begin{split}\mathrm{es} = \boldsymbol{\sigma}^T =
\begin{bmatrix}
\sigma^1_{xx} & \sigma^1_{yy} & [\sigma^1_{zz}] & \sigma^1_{xy} & [\sigma^1_{xz}] & [\sigma^1_{yz}] \\
\sigma^2_{xx} & \sigma^2_{yy} & [\sigma^2_{zz}] & \sigma^2_{xy} & [\sigma^2_{xz}] & [\sigma^2_{yz}] \\
\vdots & \vdots & \vdots & \vdots & \vdots & \vdots \\
\sigma^{n^2}_{xx} & \sigma^{n^2}_{yy} & [\sigma^{n^2}_{zz}] & \sigma^{n^2}_{xy} & [\sigma^{n^2}_{xz}] & [\sigma^{n^2}_{yz}]
\end{bmatrix}\end{split}
\end{equation*}\begin{equation*}
\begin{split}\mathrm{et} = \boldsymbol{\varepsilon}^T =
\begin{bmatrix}
\varepsilon^1_{xx} & \varepsilon^1_{yy} & [\varepsilon^1_{zz}] & \gamma^1_{xy} & [\gamma^1_{xz}] & [\gamma^1_{yz}] \\
\varepsilon^2_{xx} & \varepsilon^2_{yy} & [\varepsilon^2_{zz}] & \gamma^2_{xy} & [\gamma^2_{xz}] & [\gamma^2_{yz}] \\
\vdots & \vdots & \vdots & \vdots & \vdots & \vdots \\
\varepsilon^{n^2}_{xx} & \varepsilon^{n^2}_{yy} & [\varepsilon^{n^2}_{zz}] & \gamma^{n^2}_{xy} & [\gamma^{n^2}_{xz}] & [\gamma^{n^2}_{yz}]
\end{bmatrix}\end{split}
\end{equation*}\begin{equation*}
\begin{split}\mathbf{eci} =
\begin{bmatrix}
x_1 & y_1 \\
x_2 & y_2 \\
\vdots & \vdots \\
x_{n^2} & y_{n^2}
\end{bmatrix}\end{split}
\end{equation*}
\sphinxAtStartPar
contain the stress and strain components, and the coordinates of the integration points.
The index \(n\) denotes the number of integration points used within the element, cf. \sphinxcode{\sphinxupquote{plani4e}}.
The number of columns in \sphinxcode{\sphinxupquote{es}} and \sphinxcode{\sphinxupquote{et}} follows the size of \sphinxcode{\sphinxupquote{D}}.
Note that for plane stress \(\varepsilon_{zz} \neq 0\), and for plane strain \(\sigma_{zz} \neq 0\).

\sphinxlineitem{Theory}
\sphinxAtStartPar
The strains and stresses are computed according to
\begin{equation*}
\begin{split}\boldsymbol{\varepsilon} = \mathbf{B}^e\,\mathbf{a}^e\end{split}
\end{equation*}\begin{equation*}
\begin{split}\boldsymbol{\sigma} = \mathbf{D}\;\boldsymbol{\varepsilon}\end{split}
\end{equation*}
\sphinxAtStartPar
where the matrices \(\mathbf{D}\), \(\mathbf{B}^e\), and \(\mathbf{a}^e\) are described in \sphinxcode{\sphinxupquote{plani4e}}, and where the integration points are chosen as evaluation points.

\end{description}\end{quote}


\subsection{plani4f}
\label{\detokenize{solid_functions:plani4f}}\begin{quote}\begin{description}
\sphinxlineitem{Purpose}
\sphinxAtStartPar
Compute internal element force vector in a 4 node isoparametric element in plane strain or plane stress.

\sphinxlineitem{Syntax}
\begin{sphinxVerbatim}[commandchars=\\\{\}]
\PYG{n}{ef}\PYG{+w}{ }\PYG{p}{=}\PYG{+w}{ }\PYG{n}{plani4f}\PYG{p}{(}\PYG{n}{ex}\PYG{p}{,}\PYG{+w}{ }\PYG{n}{ey}\PYG{p}{,}\PYG{+w}{ }\PYG{n}{ep}\PYG{p}{,}\PYG{+w}{ }\PYG{n}{es}\PYG{p}{)}
\end{sphinxVerbatim}

\sphinxlineitem{Description}
\sphinxAtStartPar
\sphinxcode{\sphinxupquote{plani4f}} computes the internal element forces \(\mathrm{ef}\) in a 4 node isoparametric element in plane strain or plane stress.

\sphinxAtStartPar
The input variables \(\mathbf{ex}\), \(\mathbf{ey}\) and \(\mathbf{ep}\) are defined in \sphinxcode{\sphinxupquote{plani4e}}, and the input variable \(\mathbf{es}\) is defined in \sphinxcode{\sphinxupquote{plani4s}}.

\sphinxAtStartPar
The output variable
\begin{equation*}
\begin{split}\mathrm{ef} = \mathbf{f}_i^{eT} = \left[\, f_{i1}\; f_{i2}\; \dots \; f_{i8}\, \right]\end{split}
\end{equation*}
\sphinxAtStartPar
contains the components of the internal force vector.

\sphinxlineitem{Theory}
\sphinxAtStartPar
The internal force vector is computed according to
\begin{equation*}
\begin{split}\mathbf{f}_i^e = \int_A \mathbf{B}^{eT} \boldsymbol{\sigma} \; t \; dA\end{split}
\end{equation*}
\sphinxAtStartPar
where the matrices \(\mathbf{B}^e\) and \(\boldsymbol{\sigma}\) are defined in \sphinxcode{\sphinxupquote{plani4e}} and \sphinxcode{\sphinxupquote{plani4s}}, respectively.

\sphinxAtStartPar
Evaluation of the integral is done by Gauss integration.

\end{description}\end{quote}


\subsection{plani8e}
\label{\detokenize{solid_functions:plani8e}}\begin{quote}\begin{description}
\sphinxlineitem{Purpose}
\sphinxAtStartPar
Compute element matrices for an 8 node isoparametric element in plane strain or plane stress.

\begin{figure}[htbp]
\centering

\noindent\sphinxincludegraphics[width=0.700\linewidth]{{PLANI8E}.png}
\end{figure}

\sphinxlineitem{Syntax}
\begin{sphinxVerbatim}[commandchars=\\\{\}]
\PYG{n}{Ke}\PYG{+w}{ }\PYG{p}{=}\PYG{+w}{ }\PYG{n}{plani8e}\PYG{p}{(}\PYG{n}{ex}\PYG{p}{,}\PYG{+w}{ }\PYG{n}{ey}\PYG{p}{,}\PYG{+w}{ }\PYG{n}{ep}\PYG{p}{,}\PYG{+w}{ }\PYG{n}{D}\PYG{p}{)}
\PYG{p}{[}\PYG{n}{Ke}\PYG{p}{,}\PYG{+w}{ }\PYG{n}{fe}\PYG{p}{]}\PYG{+w}{ }\PYG{p}{=}\PYG{+w}{ }\PYG{n}{plani8e}\PYG{p}{(}\PYG{n}{ex}\PYG{p}{,}\PYG{+w}{ }\PYG{n}{ey}\PYG{p}{,}\PYG{+w}{ }\PYG{n}{ep}\PYG{p}{,}\PYG{+w}{ }\PYG{n}{D}\PYG{p}{,}\PYG{+w}{ }\PYG{n+nb}{eq}\PYG{p}{)}
\end{sphinxVerbatim}

\sphinxlineitem{Description}
\sphinxAtStartPar
\sphinxcode{\sphinxupquote{plani8e}} provides an element stiffness matrix \sphinxcode{\sphinxupquote{Ke}} and an element load vector \sphinxcode{\sphinxupquote{fe}} for an 8 node isoparametric element in plane strain or plane stress.

\sphinxAtStartPar
The element nodal coordinates \(x_1, y_1, x_2, \ldots\) are supplied to the function by \(\mathbf{ex}\) and \(\mathbf{ey}\). The type of analysis \(ptype\), the element thickness \(t\), and the number of Gauss points \(n\) are supplied by \(\mathbf{ep}\):
\begin{equation*}
\begin{split}\begin{array}{lll}
ptype=1 & \text{plane stress} & (n \times n) \text{ integration points} \\
ptype=2 & \text{plane strain} & n=1,2,3
\end{array}\end{split}
\end{equation*}
\sphinxAtStartPar
The material properties are supplied by the constitutive matrix \(\mathbf{D}\). Any arbitrary \(\mathbf{D}\)\sphinxhyphen{}matrix with dimensions from \(3 \times 3\) to \(6 \times 6\) may be given. For an isotropic elastic material the constitutive matrix can be formed by the function \sphinxcode{\sphinxupquote{hooke}}.
\begin{equation*}
\begin{split}\mathbf{ex} = [x_1, x_2, \ldots, x_8] \\
\mathbf{ey} = [y_1, y_2, \ldots, y_8] \\
\mathbf{ep} = [ptype, t, n]\end{split}
\end{equation*}\begin{equation*}
\begin{split}\mathbf{D} = \begin{bmatrix}
D_{11} & D_{12} & D_{13} \\
D_{21} & D_{22} & D_{23} \\
D_{31} & D_{32} & D_{33}
\end{bmatrix}
\quad \text{or} \quad
\mathbf{D} = \begin{bmatrix}
D_{11} & D_{12} & D_{13} & D_{14} & [D_{15}] & [D_{16}] \\
D_{21} & D_{22} & D_{23} & D_{24} & [D_{25}] & [D_{26}] \\
D_{31} & D_{32} & D_{33} & D_{34} & [D_{35}] & [D_{36}] \\
D_{41} & D_{42} & D_{43} & D_{44} & [D_{45}] & [D_{46}] \\
[D_{51}] & [D_{52}] & [D_{53}] & [D_{54}] & [D_{55}] & [D_{56}] \\
[D_{61}] & [D_{62}] & [D_{63}] & [D_{64}] & [D_{65}] & [D_{66}]
\end{bmatrix}\end{split}
\end{equation*}
\sphinxAtStartPar
If different \(\mathbf{D}_i\) matrices are used in the Gauss points these \(\mathbf{D}_i\) matrices are stored in a global vector \sphinxcode{\sphinxupquote{D}}. For numbering of the Gauss points, see \sphinxcode{\sphinxupquote{eci}} in \sphinxcode{\sphinxupquote{plani8s}}.
\begin{equation*}
\begin{split}D = \begin{bmatrix}
D_1 \\
D_2 \\
\vdots \\
D_{n^2}
\end{bmatrix}\end{split}
\end{equation*}
\sphinxAtStartPar
If uniformly distributed loads are applied to the element, the element load vector \sphinxcode{\sphinxupquote{fe}} is computed. The input variable
\begin{equation*}
\begin{split}\mathbf{eq} = \begin{bmatrix}
b_x \\
b_y
\end{bmatrix}\end{split}
\end{equation*}
\sphinxAtStartPar
containing loads per unit volume, \(b_x\) and \(b_y\), is then given.

\sphinxlineitem{Theory}
\sphinxAtStartPar
The element stiffness matrix \(\mathbf{K}^e\) and the element load vector \(\mathbf{f}_l^e\), stored in \sphinxcode{\sphinxupquote{Ke}} and \sphinxcode{\sphinxupquote{fe}}, respectively, are computed according to
\begin{equation*}
\begin{split}\mathbf{K}^e = \int_A \mathbf{B}^{eT} \mathbf{D} \mathbf{B}^e t \, dA \\
\mathbf{f}_l^e = \int_A \mathbf{N}^{eT} \mathbf{b} t \, dA\end{split}
\end{equation*}
\sphinxAtStartPar
with the constitutive matrix \(\mathbf{D}\) defined by \sphinxcode{\sphinxupquote{D}}, and the body force vector \(\mathbf{b}\) defined by \sphinxcode{\sphinxupquote{eq}}.

\sphinxAtStartPar
The evaluation of the integrals for the isoparametric 8 node element is based on a displacement approximation \(\mathbf{u}(\xi, \eta)\), expressed in a local coordinate system in terms of the nodal variables \(u_1, u_2, \ldots, u_{16}\) as
\begin{equation*}
\begin{split}\mathbf{u}(\xi, \eta) = \mathbf{N}^e \mathbf{a}^e\end{split}
\end{equation*}
\sphinxAtStartPar
where
\begin{equation*}
\begin{split}\mathbf{u} = \begin{bmatrix} u_x \\ u_y \end{bmatrix}, \quad
\mathbf{N}^e = \begin{bmatrix}
N_1^e & 0 & N_2^e & 0 & \cdots & N_8^e & 0 \\
0 & N_1^e & 0 & N_2^e & \cdots & 0 & N_8^e
\end{bmatrix}, \quad
\mathbf{a}^e = \begin{bmatrix}
u_1 \\ u_2 \\ \vdots \\ u_{16}
\end{bmatrix}\end{split}
\end{equation*}
\sphinxAtStartPar
The element shape functions are given by
\begin{equation*}
\begin{split}N_1^e = -\frac{1}{4}(1-\xi)(1-\eta)(1+\xi+\eta) \qquad
N_5^e = \frac{1}{2}(1-\xi^2)(1-\eta) \\
N_2^e = -\frac{1}{4}(1+\xi)(1-\eta)(1-\xi+\eta) \qquad
N_6^e = \frac{1}{2}(1+\xi)(1-\eta^2) \\
N_3^e = -\frac{1}{4}(1+\xi)(1+\eta)(1-\xi-\eta) \qquad
N_7^e = \frac{1}{2}(1-\xi^2)(1+\eta) \\
N_4^e = -\frac{1}{4}(1-\xi)(1+\eta)(1+\xi-\eta) \qquad
N_8^e = \frac{1}{2}(1-\xi)(1-\eta^2)\end{split}
\end{equation*}
\sphinxAtStartPar
The matrix \(\mathbf{B}^e\) is obtained as
\begin{equation*}
\begin{split}\mathbf{B}^e = \tilde{\nabla} \mathbf{N}^e \qquad
\text{where} \quad \tilde{\nabla} = \begin{bmatrix}
\frac{\partial}{\partial x} & 0 \\
0 & \frac{\partial}{\partial y} \\
\frac{\partial}{\partial y} & \frac{\partial}{\partial x}
\end{bmatrix}\end{split}
\end{equation*}
\sphinxAtStartPar
and where
\begin{equation*}
\begin{split}\begin{bmatrix}
\frac{\partial}{\partial x} \\
\frac{\partial}{\partial y}
\end{bmatrix}
= (\mathbf{J}^T)^{-1}
\begin{bmatrix}
\frac{\partial}{\partial \xi} \\
\frac{\partial}{\partial \eta}
\end{bmatrix}
\qquad
\mathbf{J} = \begin{bmatrix}
\frac{\partial x}{\partial \xi} & \frac{\partial x}{\partial \eta} \\
\frac{\partial y}{\partial \xi} & \frac{\partial y}{\partial \eta}
\end{bmatrix}\end{split}
\end{equation*}
\sphinxAtStartPar
If a larger \(\mathbf{D}\)\sphinxhyphen{}matrix than \(3 \times 3\) is used for plane stress (\(ptype=1\)), the \(\mathbf{D}\)\sphinxhyphen{}matrix is reduced to a \(3 \times 3\) matrix by static condensation, setting
\begin{equation*}
\begin{split}\sigma_{zz} = 0 \\
\sigma_{xz} = 0 \\
\sigma_{yz} = 0\end{split}
\end{equation*}
\sphinxAtStartPar
These stress components correspond to rows and columns 3, 5, and 6 in the D\sphinxhyphen{}matrix, respectively.

\sphinxAtStartPar
If a larger \(\mathbf{D}\)\sphinxhyphen{}matrix than \(3 \times 3\) is used for plane strain (\(ptype=2\)), the \(\mathbf{D}\)\sphinxhyphen{}matrix is reduced to a \(3 \times 3\) matrix by setting
\begin{equation*}
\begin{split}\varepsilon_{zz} = 0 \\
\gamma_{xz} = 0 \\
\gamma_{yz} = 0\end{split}
\end{equation*}
\sphinxAtStartPar
This means that the resulting \(3 \times 3\) \(\mathbf{D}\)\sphinxhyphen{}matrix is formed by extracting rows and columns 1, 2, and 4 from the original D\sphinxhyphen{}matrix.

\sphinxAtStartPar
Evaluation of the integrals is done by Gauss integration.

\end{description}\end{quote}


\subsection{plani8s}
\label{\detokenize{solid_functions:plani8s}}\begin{quote}\begin{description}
\sphinxlineitem{Purpose}
\sphinxAtStartPar
Compute stresses and strains in an 8 node isoparametric element in plane strain or plane stress.

\begin{figure}[htbp]
\centering

\noindent\sphinxincludegraphics[width=0.700\linewidth]{{PLANI8S}.png}
\end{figure}

\sphinxlineitem{Syntax}
\begin{sphinxVerbatim}[commandchars=\\\{\}]
\PYG{p}{[}\PYG{n}{es}\PYG{p}{,}\PYG{n}{et}\PYG{p}{,}\PYG{n}{eci}\PYG{p}{]}\PYG{p}{=}\PYG{n}{plani8s}\PYG{p}{(}\PYG{n}{ex}\PYG{p}{,}\PYG{n}{ey}\PYG{p}{,}\PYG{n}{ep}\PYG{p}{,}\PYG{n}{D}\PYG{p}{,}\PYG{n}{ed}\PYG{p}{)}
\end{sphinxVerbatim}

\sphinxlineitem{Description}
\sphinxAtStartPar
\sphinxcode{\sphinxupquote{plani8s}} computes stresses \sphinxcode{\sphinxupquote{es}} and the strains \sphinxcode{\sphinxupquote{et}} in an 8 node isoparametric element in plane strain or plane stress.

\sphinxAtStartPar
The input variables \(\mathbf{ex}\), \(\mathbf{ey}\), \(\mathbf{ep}\) and the matrix \(\mathbf{D}\) are defined in \sphinxcode{\sphinxupquote{plani8e}}.
The vector \(\mathbf{ed}\) contains the nodal displacements \(\mathbf{a}^e\) of the element and is obtained by the function \sphinxcode{\sphinxupquote{extract}} as
\begin{equation*}
\begin{split}\mathbf{ed} = (\mathbf{a}^e)^T = [\,u_1\;\; u_2\;\; \dots \;\; u_{16}\,]\end{split}
\end{equation*}
\sphinxAtStartPar
The output variables
\begin{equation*}
\begin{split}\mathrm{es} = \boldsymbol{\sigma}^T =
\begin{bmatrix}
\sigma^1_{xx} & \sigma^1_{yy} & [\sigma^1_{zz}] & \sigma^1_{xy} & [\sigma^1_{xz}] & [\sigma^1_{yz}] \\
\sigma^2_{xx} & \sigma^2_{yy} & [\sigma^2_{zz}] & \sigma^2_{xy} & [\sigma^2_{xz}] & [\sigma^2_{yz}] \\
\vdots & \vdots & \vdots & \vdots & \vdots & \vdots \\
\sigma^{n^2}_{xx} & \sigma^{n^2}_{yy} & [\sigma^{n^2}_{zz}] & \sigma^{n^2}_{xy} & [\sigma^{n^2}_{xz}] & [\sigma^{n^2}_{yz}]
\end{bmatrix}\end{split}
\end{equation*}\begin{equation*}
\begin{split}\mathrm{et} = \boldsymbol{\varepsilon}^T =
\begin{bmatrix}
\varepsilon^1_{xx} & \varepsilon^1_{yy} & [\varepsilon^1_{zz}] & \gamma^1_{xy} & [\gamma^1_{xz}] & [\gamma^1_{yz}] \\
\varepsilon^2_{xx} & \varepsilon^2_{yy} & [\varepsilon^2_{zz}] & \gamma^2_{xy} & [\gamma^2_{xz}] & [\gamma^2_{yz}] \\
\vdots & \vdots & \vdots & \vdots & \vdots & \vdots \\
\varepsilon^{n^2}_{xx} & \varepsilon^{n^2}_{yy} & [\varepsilon^{n^2}_{zz}] & \gamma^{n^2}_{xy} & [\gamma^{n^2}_{xz}] & [\gamma^{n^2}_{yz}]
\end{bmatrix}
\qquad
\mathbf{eci} =
\begin{bmatrix}
x_1 & y_1 \\
x_2 & y_2 \\
\vdots & \vdots \\
x_{n^2} & y_{n^2}
\end{bmatrix}\end{split}
\end{equation*}
\sphinxAtStartPar
contain the stress and strain components, and the coordinates of the integration points. The index \(n\) denotes the number of integration points used within the element, cf. \sphinxcode{\sphinxupquote{plani8e}}. The number of columns in \sphinxcode{\sphinxupquote{es}} and \sphinxcode{\sphinxupquote{et}} follows the size of \sphinxcode{\sphinxupquote{D}}.
Note that for plane stress \(\varepsilon_{zz} \neq 0\), and for plane strain \(\sigma_{zz} \neq 0\).

\sphinxlineitem{Theory}
\sphinxAtStartPar
The strains and stresses are computed according to
\begin{equation*}
\begin{split}\boldsymbol{\varepsilon} = \mathbf{B}^e\,\mathbf{a}^e\end{split}
\end{equation*}\begin{equation*}
\begin{split}\boldsymbol{\sigma} = \mathbf{D}\;\boldsymbol{\varepsilon}\end{split}
\end{equation*}
\sphinxAtStartPar
where the matrices \(\mathbf{D}\), \(\mathbf{B}^e\), and \(\mathbf{a}^e\) are described in \sphinxcode{\sphinxupquote{plani8e}}, and where the integration points are chosen as evaluation points.

\end{description}\end{quote}


\subsection{plani8f}
\label{\detokenize{solid_functions:plani8f}}\begin{quote}\begin{description}
\sphinxlineitem{Purpose}
\sphinxAtStartPar
Compute internal element force vector in an 8 node isoparametric element in plane strain or plane stress.

\sphinxlineitem{Syntax}
\begin{sphinxVerbatim}[commandchars=\\\{\}]
\PYG{n}{ef}\PYG{+w}{ }\PYG{p}{=}\PYG{+w}{ }\PYG{n}{plani8f}\PYG{p}{(}\PYG{n}{ex}\PYG{p}{,}\PYG{+w}{ }\PYG{n}{ey}\PYG{p}{,}\PYG{+w}{ }\PYG{n}{ep}\PYG{p}{,}\PYG{+w}{ }\PYG{n}{es}\PYG{p}{)}
\end{sphinxVerbatim}

\sphinxlineitem{Description}
\sphinxAtStartPar
\sphinxcode{\sphinxupquote{plani8f}} computes the internal element forces \(\mathrm{ef}\) in an 8 node isoparametric element in plane strain or plane stress.

\sphinxAtStartPar
The input variables \(\mathbf{ex}\), \(\mathbf{ey}\) and \(\mathbf{ep}\) are defined in \sphinxcode{\sphinxupquote{plani8e}}, and the input variable \(\mathbf{es}\) is defined in \sphinxcode{\sphinxupquote{plani8s}}.

\sphinxAtStartPar
The output variable
\begin{equation*}
\begin{split}\mathrm{ef} = \mathbf{f}_i^{eT} = \left[\, f_{i1}\; f_{i2}\; \dots \; f_{i16}\; \right]\end{split}
\end{equation*}
\sphinxAtStartPar
contains the components of the internal force vector.

\sphinxlineitem{Theory}
\sphinxAtStartPar
The internal force vector is computed according to
\begin{equation*}
\begin{split}\mathbf{f}_i^e = \int_A \mathbf{B}^{eT} \boldsymbol{\sigma} \; t \; dA\end{split}
\end{equation*}
\sphinxAtStartPar
where the matrices \(\mathbf{B}^e\) and \(\boldsymbol{\sigma}\) are defined in \sphinxcode{\sphinxupquote{plani8e}} and \sphinxcode{\sphinxupquote{plani8s}}, respectively.

\sphinxAtStartPar
Evaluation of the integral is done by Gauss integration.

\end{description}\end{quote}


\section{3D Solid Functions}
\label{\detokenize{solid_functions:id1}}

\subsection{soli8e}
\label{\detokenize{solid_functions:soli8e}}\begin{quote}\begin{description}
\sphinxlineitem{Purpose}
\sphinxAtStartPar
Compute element matrices for an 8 node isoparametric solid element.

\begin{figure}[htbp]
\centering

\noindent\sphinxincludegraphics[width=0.700\linewidth]{{SOLI8E}.png}
\end{figure}

\sphinxlineitem{Syntax}
\begin{sphinxVerbatim}[commandchars=\\\{\}]
\PYG{n}{Ke}\PYG{+w}{ }\PYG{p}{=}\PYG{+w}{ }\PYG{n}{soli8e}\PYG{p}{(}\PYG{n}{ex}\PYG{p}{,}\PYG{+w}{ }\PYG{n}{ey}\PYG{p}{,}\PYG{+w}{ }\PYG{n}{ez}\PYG{p}{,}\PYG{+w}{ }\PYG{n}{ep}\PYG{p}{,}\PYG{+w}{ }\PYG{n}{D}\PYG{p}{)}
\PYG{p}{[}\PYG{n}{Ke}\PYG{p}{,}\PYG{+w}{ }\PYG{n}{fe}\PYG{p}{]}\PYG{+w}{ }\PYG{p}{=}\PYG{+w}{ }\PYG{n}{soli8e}\PYG{p}{(}\PYG{n}{ex}\PYG{p}{,}\PYG{+w}{ }\PYG{n}{ey}\PYG{p}{,}\PYG{+w}{ }\PYG{n}{ez}\PYG{p}{,}\PYG{+w}{ }\PYG{n}{ep}\PYG{p}{,}\PYG{+w}{ }\PYG{n}{D}\PYG{p}{,}\PYG{+w}{ }\PYG{n+nb}{eq}\PYG{p}{)}
\end{sphinxVerbatim}

\sphinxlineitem{Description}
\sphinxAtStartPar
\sphinxcode{\sphinxupquote{soli8e}} provides an element stiffness matrix \sphinxcode{\sphinxupquote{Ke}} and an element
load vector \sphinxcode{\sphinxupquote{fe}} for an 8 node isoparametric solid element.

\sphinxAtStartPar
The element nodal coordinates \(x_1, y_1, z_1, x_2\) etc. are supplied to the function
by \(\mathbf{ex}\), \(\mathbf{ey}\) and \(\mathbf{ez}\), and the number of Gauss points \(n\) are
supplied by \(\mathbf{ep}\).

\sphinxAtStartPar
\((n \times n \times n)\) integration points, \(n=1,2,3\)

\sphinxAtStartPar
The material properties are supplied by the
constitutive matrix \sphinxcode{\sphinxupquote{D}}. Any arbitrary \(\mathbf{D}\)\sphinxhyphen{}matrix with
dimensions \((6 \times 6)\) may be given.
For an isotropic elastic material the constitutive matrix can be formed
by the function \sphinxcode{\sphinxupquote{hooke}}, see Section {\hyperref[\detokenize{material_functions:material-functions}]{\sphinxcrossref{\DUrole{std}{\DUrole{std-ref}{Material functions}}}}}.
\begin{equation*}
\begin{split}\begin{aligned}
\mathbf{ex} &= [\,x_1,\, x_2,\, \dots,\, x_8\,] \\
\mathbf{ey} &= [\,y_1,\, y_2,\, \dots,\, y_8\,] \\
\mathbf{ez} &= [\,z_1,\, z_2,\, \dots,\, z_8\,]
\end{aligned}\end{split}
\end{equation*}\begin{equation*}
\begin{split}\mathbf{ep} = [\, n \,]\end{split}
\end{equation*}\begin{equation*}
\begin{split}\mathbf{D} =
\begin{bmatrix}
    D_{11} & D_{12} & \cdots & D_{16} \\
    D_{21} & D_{22} & \cdots & D_{26} \\
    \vdots & \vdots & \ddots & \vdots \\
    D_{61} & D_{62} & \cdots & D_{66}
\end{bmatrix}\end{split}
\end{equation*}
\sphinxAtStartPar
If different \(\mathbf{D}_i\) matrices are used in the Gauss points these
\(\mathbf{D}_i\) matrices are stored in a global vector \sphinxcode{\sphinxupquote{D}}. For
numbering of the Gauss points, see \sphinxcode{\sphinxupquote{eci}} in \sphinxcode{\sphinxupquote{soli8s}}.
\begin{equation*}
\begin{split}\mathbf{D} =
\begin{bmatrix}
    \mathbf{D}_1 \\
    \mathbf{D}_2 \\
    \vdots \\
    \mathbf{D}_{n^3}
\end{bmatrix}\end{split}
\end{equation*}
\sphinxAtStartPar
If uniformly distributed loads are applied to the element,
the element load vector \sphinxcode{\sphinxupquote{fe}} is computed. The input variable
\begin{equation*}
\begin{split}\mathrm{eq} =
\begin{bmatrix}
    b_x \\
    b_y \\
    b_z
\end{bmatrix}\end{split}
\end{equation*}
\sphinxAtStartPar
containing loads per unit volume, \(b_x\), \(b_y\), and \(b_z\), is then given.

\sphinxlineitem{Theory}
\sphinxAtStartPar
The element stiffness matrix \(\mathbf{K}^e\) and the
element load vector \(\mathbf{f}_l^e\), stored in \sphinxcode{\sphinxupquote{Ke}} and \sphinxcode{\sphinxupquote{fe}},
respectively, are computed according to
\begin{align*}\!\begin{aligned}
\mathbf{K}^e = \int_V \mathbf{B}^{eT} \mathbf{D} \mathbf{B}^e \, dV\\
\mathbf{f}_l^e = \int_V \mathbf{N}^{eT} \mathbf{b} \, dV\\
\end{aligned}\end{align*}
\sphinxAtStartPar
with the constitutive matrix \(\mathbf{D}\) defined by \sphinxcode{\sphinxupquote{D}}, and the body force
vector \(\mathbf{b}\) defined by \sphinxcode{\sphinxupquote{eq}}.

\sphinxAtStartPar
The evaluation of the integrals for the isoparametric 8 node solid element
is based on a displacement approximation \(\mathbf{u}(\xi, \eta, \zeta)\), expressed in a
local coordinate system in terms of the nodal
variables \(u_1, u_2, \dots, u_{24}\) as
\begin{align*}\!\begin{aligned}
\mathbf{u}(\xi, \eta, \zeta) = \mathbf{N}^e \mathbf{a}^e\\
\mathbf{u} =
\begin{bmatrix}
    u_x \\
    u_y \\
    u_z
\end{bmatrix}
\quad
\mathbf{N}^e =
\begin{bmatrix}
    N^e_1 & 0 & 0 & N^e_2 & 0 & 0 & \dots & N^e_8 & 0 & 0 \\
    0 & N^e_1 & 0 & 0 & N^e_2 & 0 & \dots & 0 & N^e_8 & 0 \\
    0 & 0 & N^e_1 & 0 & 0 & N^e_2 & \dots & 0 & 0 & N^e_8 \\
\end{bmatrix}
\quad
\mathbf{a}^e =
\begin{bmatrix}
    u_1 \\
    u_2 \\
    \vdots \\
    u_{24}
\end{bmatrix}\\
\end{aligned}\end{align*}
\sphinxAtStartPar
The element shape functions are given by
\begin{equation*}
\begin{split}\begin{aligned}
N_1^e &= \frac{1}{8}(1-\xi)(1-\eta)(1-\zeta) & N_5^e &= \frac{1}{8}(1-\xi)(1-\eta)(1+\zeta) \\
N_2^e &= \frac{1}{8}(1+\xi)(1-\eta)(1-\zeta) & N_6^e &= \frac{1}{8}(1+\xi)(1-\eta)(1+\zeta) \\
N_3^e &= \frac{1}{8}(1+\xi)(1+\eta)(1-\zeta) & N_7^e &= \frac{1}{8}(1+\xi)(1+\eta)(1+\zeta) \\
N_4^e &= \frac{1}{8}(1-\xi)(1+\eta)(1-\zeta) & N_8^e &= \frac{1}{8}(1-\xi)(1+\eta)(1+\zeta)
\end{aligned}\end{split}
\end{equation*}
\sphinxAtStartPar
The \(\mathbf{B}^e\)\sphinxhyphen{}matrix is obtained as
\begin{equation*}
\begin{split}\mathbf{B}^e = \tilde{\nabla} \mathbf{N}^e\end{split}
\end{equation*}
\sphinxAtStartPar
where
\begin{equation*}
\begin{split}\tilde{\nabla} =
\begin{bmatrix}
    \frac{\partial}{\partial x} & 0 & 0 \\
    0 & \frac{\partial}{\partial y} & 0 \\
    0 & 0 & \frac{\partial}{\partial z} \\
    \frac{\partial}{\partial y} & \frac{\partial}{\partial x} & 0 \\
    \frac{\partial}{\partial z} & 0 & \frac{\partial}{\partial x} \\
    0 & \frac{\partial}{\partial z} & \frac{\partial}{\partial y}
\end{bmatrix}
\qquad
\begin{bmatrix}
    \frac{\partial}{\partial x} \\
    \frac{\partial}{\partial y} \\
    \frac{\partial}{\partial z}
\end{bmatrix}
= (\mathbf{J}^T)^{-1}
\begin{bmatrix}
    \frac{\partial}{\partial \xi} \\
    \frac{\partial}{\partial \eta} \\
    \frac{\partial}{\partial \zeta}
\end{bmatrix}\end{split}
\end{equation*}\begin{equation*}
\begin{split}\mathbf{J} =
\begin{bmatrix}
    \frac{\partial x}{\partial \xi} & \frac{\partial x}{\partial \eta} & \frac{\partial x}{\partial \zeta} \\
    \frac{\partial y}{\partial \xi} & \frac{\partial y}{\partial \eta} & \frac{\partial y}{\partial \zeta} \\
    \frac{\partial z}{\partial \xi} & \frac{\partial z}{\partial \eta} & \frac{\partial z}{\partial \zeta}
\end{bmatrix}\end{split}
\end{equation*}
\sphinxAtStartPar
Evaluation of the integrals is done by Gauss integration.

\end{description}\end{quote}


\subsection{soli8s}
\label{\detokenize{solid_functions:soli8s}}\begin{quote}\begin{description}
\sphinxlineitem{Purpose}
\sphinxAtStartPar
Compute stresses and strains in an 8 node isoparametric solid element.

\begin{figure}[htbp]
\centering

\noindent\sphinxincludegraphics[width=0.700\linewidth]{{SOLI8S}.png}
\end{figure}

\sphinxlineitem{Syntax}
\begin{sphinxVerbatim}[commandchars=\\\{\}]
\PYG{p}{[}\PYG{n}{es}\PYG{p}{,}\PYG{n}{et}\PYG{p}{,}\PYG{n}{eci}\PYG{p}{]}\PYG{p}{=}\PYG{n}{soli8s}\PYG{p}{(}\PYG{n}{ex}\PYG{p}{,}\PYG{n}{ey}\PYG{p}{,}\PYG{n}{ez}\PYG{p}{,}\PYG{n}{ep}\PYG{p}{,}\PYG{n}{D}\PYG{p}{,}\PYG{n}{ed}\PYG{p}{)}
\end{sphinxVerbatim}

\sphinxlineitem{Description}
\sphinxAtStartPar
\sphinxcode{\sphinxupquote{soli8s}} computes stresses \sphinxcode{\sphinxupquote{es}} and the strains \sphinxcode{\sphinxupquote{et}}
in an 8 node isoparametric solid element.

\sphinxAtStartPar
The input variables \(\mathbf{ex}\), \(\mathbf{ey}\), \(\mathbf{ez}\), \(\mathbf{ep}\) and
the matrix \(\mathbf{D}\) are defined in \sphinxcode{\sphinxupquote{soli8e}}.
The vector \(\mathbf{ed}\) contains the nodal displacements \(\mathbf{a}^e\)
of the element and is obtained by the function \sphinxcode{\sphinxupquote{extract}} as
\begin{equation*}
\begin{split}\mathbf{ed} = (\mathbf{a}^e)^T = [\;u_1\;\; u_2\;\; \dots \;\; u_{24}\;]\end{split}
\end{equation*}
\sphinxAtStartPar
The output variables
\begin{equation*}
\begin{split}\mathrm{es} = \boldsymbol{\sigma}^T =
\begin{bmatrix}
\sigma^1_{xx} & \sigma^1_{yy} & \sigma^1_{zz} & \sigma^1_{xy} & \sigma^1_{xz} & \sigma^1_{yz} \\
\sigma^2_{xx} & \sigma^2_{yy} & \sigma^2_{zz} & \sigma^2_{xy} & \sigma^2_{xz} & \sigma^2_{yz} \\
\vdots & \vdots & \vdots & \vdots & \vdots & \vdots \\
\sigma^{n^3}_{xx} & \sigma^{n^3}_{yy} & \sigma^{n^3}_{zz} & \sigma^{n^3}_{xy} & \sigma^{n^3}_{xz} & \sigma^{n^3}_{yz}
\end{bmatrix}\end{split}
\end{equation*}\begin{equation*}
\begin{split}\mathrm{et} = \boldsymbol{\varepsilon}^T =
\begin{bmatrix}
\varepsilon^1_{xx} & \varepsilon^1_{yy} & \varepsilon^1_{zz} & \gamma^1_{xy} & \gamma^1_{xz} & \gamma^1_{yz} \\
\varepsilon^2_{xx} & \varepsilon^2_{yy} & \varepsilon^2_{zz} & \gamma^2_{xy} & \gamma^2_{xz} & \gamma^2_{yz} \\
\vdots & \vdots & \vdots & \vdots & \vdots & \vdots \\
\varepsilon^{n^3}_{xx} & \varepsilon^{n^3}_{yy} & \varepsilon^{n^3}_{zz} & \gamma^{n^3}_{xy} & \gamma^{n^3}_{xz} & \gamma^{n^3}_{yz}
\end{bmatrix}
\qquad
\mathbf{eci} =
\begin{bmatrix}
x_1 & y_1 & z_1 \\
x_2 & y_2 & z_2 \\
\vdots & \vdots & \vdots \\
x_{n^3} & y_{n^3} & z_{n^3}
\end{bmatrix}\end{split}
\end{equation*}
\sphinxAtStartPar
contain the stress and strain components, and the coordinates of
the integration points. The index \(n\) denotes the number of integration points
used within the element, cf. \sphinxcode{\sphinxupquote{soli8e}}.

\sphinxlineitem{Theory}
\sphinxAtStartPar
The strains and stresses are computed according to
\begin{equation*}
\begin{split}\boldsymbol{\varepsilon} = \mathbf{B}^e \mathbf{a}^e\end{split}
\end{equation*}\begin{equation*}
\begin{split}\boldsymbol{\sigma} = \mathbf{D} \boldsymbol{\varepsilon}\end{split}
\end{equation*}
\sphinxAtStartPar
where the matrices \(\mathbf{D}\), \(\mathbf{B}^e\), and \(\mathbf{a}^e\) are described
in \sphinxcode{\sphinxupquote{soli8e}}, and where the integration points
are chosen as evaluation points.

\end{description}\end{quote}


\subsection{soli8f}
\label{\detokenize{solid_functions:soli8f}}\begin{quote}\begin{description}
\sphinxlineitem{Purpose}
\sphinxAtStartPar
Compute internal element force vector in an 8 node isoparametric solid element.

\sphinxlineitem{Syntax}
\begin{sphinxVerbatim}[commandchars=\\\{\}]
\PYG{n}{ef}\PYG{+w}{ }\PYG{p}{=}\PYG{+w}{ }\PYG{n}{soli8f}\PYG{p}{(}\PYG{n}{ex}\PYG{p}{,}\PYG{+w}{ }\PYG{n}{ey}\PYG{p}{,}\PYG{+w}{ }\PYG{n}{ez}\PYG{p}{,}\PYG{+w}{ }\PYG{n}{ep}\PYG{p}{,}\PYG{+w}{ }\PYG{n}{es}\PYG{p}{)}
\end{sphinxVerbatim}

\sphinxlineitem{Description}
\sphinxAtStartPar
\sphinxcode{\sphinxupquote{soli8f}} computes the internal element forces \(\mathrm{ef}\) in an 8 node isoparametric solid element.

\sphinxAtStartPar
The input variables \(\mathbf{ex}\), \(\mathbf{ey}\), \(\mathbf{ez}\) and \(\mathbf{ep}\) are defined in \sphinxcode{\sphinxupquote{soli8e}}, and the input variable \(\mathbf{es}\) is defined in \sphinxcode{\sphinxupquote{soli8s}}.

\sphinxAtStartPar
The output variable
\begin{equation*}
\begin{split}\mathrm{ef} = \mathbf{f}_i^{eT} = \left[\, f_{i1}\; f_{i2}\; \dots \; f_{i24}\; \right]\end{split}
\end{equation*}
\sphinxAtStartPar
contains the components of the internal force vector.

\sphinxlineitem{Theory}
\sphinxAtStartPar
The internal force vector is computed according to
\begin{equation*}
\begin{split}\mathbf{f}_i^e = \int_V \mathbf{B}^{eT} \boldsymbol{\sigma} \; dV\end{split}
\end{equation*}
\sphinxAtStartPar
where the matrices \(\mathbf{B}\) and \(\boldsymbol{\sigma}\) are defined in \sphinxcode{\sphinxupquote{soli8e}} and \sphinxcode{\sphinxupquote{soli8s}}, respectively.

\sphinxAtStartPar
Evaluation of the integral is done by Gauss integration.

\end{description}\end{quote}

\sphinxstepscope


\chapter{Beam element functions}
\label{\detokenize{beam_functions:beam-element-functions}}\label{\detokenize{beam_functions::doc}}
\sphinxAtStartPar
Beam elements are available for one, two, and three dimensional linear static analysis.
Two dimensional beam elements for nonlinear geometric and dynamic analysis are also available.


\section{1D beam elements}
\label{\detokenize{beam_functions:d-beam-elements}}

\subsection{beam1e}
\label{\detokenize{beam_functions:beam1e}}\begin{quote}\begin{description}
\sphinxlineitem{Purpose}
\sphinxAtStartPar
Compute element stiffness matrix for a one dimensional beam element.

\begin{figure}[htbp]
\centering

\noindent\sphinxincludegraphics[width=0.700\linewidth]{{beam1e}.png}
\end{figure}

\sphinxlineitem{Syntax}
\begin{sphinxVerbatim}[commandchars=\\\{\}]
\PYG{n}{Ke}\PYG{+w}{ }\PYG{p}{=}\PYG{+w}{ }\PYG{n}{beam1e}\PYG{p}{(}\PYG{n}{ex}\PYG{p}{,}\PYG{+w}{ }\PYG{n}{ep}\PYG{p}{)}
\PYG{p}{[}\PYG{n}{Ke}\PYG{p}{,}\PYG{+w}{ }\PYG{n}{fe}\PYG{p}{]}\PYG{+w}{ }\PYG{p}{=}\PYG{+w}{ }\PYG{n}{beam1e}\PYG{p}{(}\PYG{n}{ex}\PYG{p}{,}\PYG{+w}{ }\PYG{n}{ep}\PYG{p}{,}\PYG{+w}{ }\PYG{n+nb}{eq}\PYG{p}{)}
\end{sphinxVerbatim}

\sphinxlineitem{Description}
\sphinxAtStartPar
\sphinxcode{\sphinxupquote{beam1e}} provides the global element stiffness matrix \sphinxcode{\sphinxupquote{Ke}} for a one dimensional beam element.

\sphinxAtStartPar
The input variables

\sphinxAtStartPar
\sphinxcode{\sphinxupquote{ex = {[}x1 x2{]}}}

\sphinxAtStartPar
\sphinxcode{\sphinxupquote{ep = {[}E I{]}}}

\sphinxAtStartPar
supply the element nodal coordinates \(x_1\) and \(x_2\), the modulus of elasticity \(E\) and the moment of inertia \(I\).

\sphinxAtStartPar
The element load vector \sphinxcode{\sphinxupquote{fe}} can also be computed if a uniformly distributed load is applied to the element.
The optional input variable

\sphinxAtStartPar
\sphinxcode{\sphinxupquote{eq = {[}q\_ybar{]}}}

\sphinxAtStartPar
then contains the distributed load per unit length, \(q_{\bar{y}}\).

\begin{figure}[htbp]
\centering

\noindent\sphinxincludegraphics[width=0.700\linewidth]{{beam1e_2}.png}
\end{figure}

\sphinxlineitem{Theory}
\sphinxAtStartPar
The element stiffness matrix \(\bar{\mathbf{K}}^e\), stored in \sphinxcode{\sphinxupquote{Ke}}, is computed according to
\begin{equation*}
\begin{split}\bar{\mathbf{K}}^e = \frac{D_{EI}}{L^3}
\begin{bmatrix}
12 & 6L & -12 & 6L \\
6L & 4L^2 & -6L & 2L^2 \\
-12 & -6L & 12 & -6L \\
6L & 2L^2 & -6L & 4L^2
\end{bmatrix}\end{split}
\end{equation*}
\sphinxAtStartPar
where the bending stiffness \(D_{EI}\) and the length \(L\) are given by
\begin{equation*}
\begin{split}D_{EI} = EI; \quad L = x_2 - x_1\end{split}
\end{equation*}
\sphinxAtStartPar
The element loads \(\bar{\mathbf{f}}_l^e\) stored in the variable \sphinxcode{\sphinxupquote{fe}} are computed according to
\begin{equation*}
\begin{split}\bar{\mathbf{f}}_l^e = q_{\bar{y}}
\begin{bmatrix}
\dfrac{L}{2} \\
\dfrac{L^2}{12} \\
\dfrac{L}{2} \\
-\dfrac{L^2}{12}
\end{bmatrix}\end{split}
\end{equation*}
\end{description}\end{quote}


\subsection{beam1s}
\label{\detokenize{beam_functions:beam1s}}\begin{quote}\begin{description}
\sphinxlineitem{Purpose}
\sphinxAtStartPar
Compute section forces in a one dimensional beam element.

\begin{figure}[htbp]
\centering

\noindent\sphinxincludegraphics[width=0.700\linewidth]{{beam1s}.png}
\end{figure}

\sphinxlineitem{Syntax}
\begin{sphinxVerbatim}[commandchars=\\\{\}]
\PYG{n}{es}\PYG{+w}{ }\PYG{p}{=}\PYG{+w}{ }\PYG{n}{beam1s}\PYG{p}{(}\PYG{n}{ex}\PYG{p}{,}\PYG{+w}{ }\PYG{n}{ep}\PYG{p}{,}\PYG{+w}{ }\PYG{n}{ed}\PYG{p}{)}
\PYG{n}{es}\PYG{+w}{ }\PYG{p}{=}\PYG{+w}{ }\PYG{n}{beam1s}\PYG{p}{(}\PYG{n}{ex}\PYG{p}{,}\PYG{+w}{ }\PYG{n}{ep}\PYG{p}{,}\PYG{+w}{ }\PYG{n}{ed}\PYG{p}{,}\PYG{+w}{ }\PYG{n+nb}{eq}\PYG{p}{)}
\PYG{p}{[}\PYG{n}{es}\PYG{p}{,}\PYG{+w}{ }\PYG{n}{edi}\PYG{p}{,}\PYG{+w}{ }\PYG{n}{eci}\PYG{p}{]}\PYG{+w}{ }\PYG{p}{=}\PYG{+w}{ }\PYG{n}{beam1s}\PYG{p}{(}\PYG{n}{ex}\PYG{p}{,}\PYG{+w}{ }\PYG{n}{ep}\PYG{p}{,}\PYG{+w}{ }\PYG{n}{ed}\PYG{p}{,}\PYG{+w}{ }\PYG{n+nb}{eq}\PYG{p}{,}\PYG{+w}{ }\PYG{n}{n}\PYG{p}{)}
\end{sphinxVerbatim}

\sphinxlineitem{Description}
\sphinxAtStartPar
\sphinxcode{\sphinxupquote{beam1s}} computes the section forces and displacements in local directions
along the beam element \sphinxcode{\sphinxupquote{beam1e}}.

\sphinxAtStartPar
The input variables \sphinxcode{\sphinxupquote{ex}}, \sphinxcode{\sphinxupquote{ep}} and \sphinxcode{\sphinxupquote{eq}} are defined in
\sphinxcode{\sphinxupquote{beam1e}}, and the element displacements, stored
in \sphinxcode{\sphinxupquote{ed}}, are obtained by the function \sphinxcode{\sphinxupquote{extract}}.
If distributed loads are applied to the element, the variable \sphinxcode{\sphinxupquote{eq}} must be
included.
The number of evaluation points for section forces and displacements are
determined by \sphinxcode{\sphinxupquote{n}}. If \sphinxcode{\sphinxupquote{n}} is omitted, only the ends of the
beam are evaluated.

\sphinxAtStartPar
The output variables
\begin{equation*}
\begin{split}\mathrm{es} =
\begin{bmatrix}
V(0)  & M(0) \\
V(\bar{x}_{2}) & M(\bar{x}_{2})  \\
\vdots & \vdots \\
V(\bar{x}_{n-1}) & M(\bar{x}_{n-1})\\
V(L) & M(L)
\end{bmatrix}
\qquad
\mathrm{edi} =
\begin{bmatrix}
v(0)   \\
v(\bar{x}_{2})   \\
\vdots \\
v(\bar{x}_{n-1})\\
v(L)
\end{bmatrix}
\qquad
\mathrm{eci} =
\begin{bmatrix}
0  \\
\bar x_{2} \\
\vdots   \\
\bar x_{n-1} \\
L
\end{bmatrix}\end{split}
\end{equation*}
\sphinxAtStartPar
contain the section forces, the displacements, and the evaluation points on the local \(\bar{x}\)\sphinxhyphen{}axis.
\(L\) is the length of the beam element.

\sphinxlineitem{Theory}
\sphinxAtStartPar
The nodal displacements in local coordinates are given by
\begin{equation*}
\begin{split}\mathbf{\bar{a}}^e = \begin{bmatrix} \bar{u}_1 \\ \bar{u}_2 \\ \bar{u}_3 \\ \bar{u}_4 \end{bmatrix}\end{split}
\end{equation*}
\sphinxAtStartPar
where the transpose of \(\mathbf{a}^e\) is stored in \sphinxcode{\sphinxupquote{ed}}.

\sphinxAtStartPar
The displacement \(v(\bar{x})\), the bending moment \(M(\bar{x})\) and the shear force \(V(\bar{x})\) are computed from
\begin{equation*}
\begin{split}v(\bar{x}) = \mathbf{N} \mathbf{\bar{a}}^e + v_p(\bar{x})\end{split}
\end{equation*}\begin{equation*}
\begin{split}M(\bar{x}) = D_{EI} \mathbf{B} \mathbf{\bar{a}}^e + M_p(\bar{x})\end{split}
\end{equation*}\begin{equation*}
\begin{split}V(\bar{x}) = -D_{EI} \frac{d\mathbf{B}}{dx} \mathbf{\bar{a}}^e + V_p(\bar{x})\end{split}
\end{equation*}
\sphinxAtStartPar
where
\begin{equation*}
\begin{split}\mathbf{N} = \begin{bmatrix} 1 & \bar{x} & \bar{x}^2 & \bar{x}^3 \end{bmatrix} \mathbf{C}^{-1}\end{split}
\end{equation*}\begin{equation*}
\begin{split}\mathbf{B} = \begin{bmatrix} 0 & 0 & 2 & 6\bar{x} \end{bmatrix} \mathbf{C}^{-1}\end{split}
\end{equation*}\begin{equation*}
\begin{split}\frac{d\mathbf{B}}{dx} = \begin{bmatrix} 0 & 0 & 0 & 6 \end{bmatrix} \mathbf{C}^{-1}\end{split}
\end{equation*}\begin{equation*}
\begin{split}v_p(\bar{x}) = \frac{q_{\bar{y}}}{D_{EI}} \left( \frac{\bar{x}^4}{24} - \frac{L \bar{x}^3}{12} + \frac{L^2 \bar{x}^2}{24} \right)\end{split}
\end{equation*}\begin{equation*}
\begin{split}M_p(\bar{x}) = q_{\bar{y}} \left( \frac{\bar{x}^2}{2} - \frac{L \bar{x}}{2} + \frac{L^2}{12} \right)\end{split}
\end{equation*}\begin{equation*}
\begin{split}V_p(\bar{x}) = -q_{\bar{y}} \left( \bar{x} - \frac{L}{2} \right)\end{split}
\end{equation*}
\sphinxAtStartPar
in which \(D_{EI}\), \(L\), and \(q_{\bar{y}}\)
are defined in \sphinxcode{\sphinxupquote{beam1e}} and
\begin{equation*}
\begin{split}\mathbf{C}^{-1} =
\begin{bmatrix}
1 & 0 & 0 & 0 \\
0 & 1 & 0 & 0 \\
-\frac{3}{L^2} & -\frac{2}{L} & \frac{3}{L^2} & -\frac{1}{L} \\
\frac{2}{L^3} & \frac{1}{L^2} & -\frac{2}{L^3} & \frac{1}{L^2}
\end{bmatrix}\end{split}
\end{equation*}
\end{description}\end{quote}


\subsection{beam1we}
\label{\detokenize{beam_functions:beam1we}}\begin{quote}\begin{description}
\sphinxlineitem{Purpose}
\sphinxAtStartPar
Compute element stiffness matrix for a one dimensional beam element on elastic support.

\begin{figure}[htbp]
\centering

\noindent\sphinxincludegraphics[width=0.700\linewidth]{{beam1w}.png}
\end{figure}

\sphinxlineitem{Syntax}
\begin{sphinxVerbatim}[commandchars=\\\{\}]
\PYG{n}{Ke}\PYG{+w}{ }\PYG{p}{=}\PYG{+w}{ }\PYG{n}{beam1we}\PYG{p}{(}\PYG{n}{ex}\PYG{p}{,}\PYG{+w}{ }\PYG{n}{ep}\PYG{p}{)}
\PYG{p}{[}\PYG{n}{Ke}\PYG{p}{,}\PYG{+w}{ }\PYG{n}{fe}\PYG{p}{]}\PYG{+w}{ }\PYG{p}{=}\PYG{+w}{ }\PYG{n}{beam1we}\PYG{p}{(}\PYG{n}{ex}\PYG{p}{,}\PYG{+w}{ }\PYG{n}{ep}\PYG{p}{,}\PYG{+w}{ }\PYG{n+nb}{eq}\PYG{p}{)}
\end{sphinxVerbatim}

\sphinxlineitem{Description}
\sphinxAtStartPar
\sphinxstylestrong{beam1we} provides the global element stiffness matrix \sphinxstylestrong{Ke} for a one dimensional beam element with elastic support.

\sphinxAtStartPar
The input variables
\begin{equation*}
\begin{split}\mathrm{ex} = [x_1 \;\; x_2]
\qquad
\mathrm{ep} = [E\;\; I \;\; k_{\bar{y}}]\end{split}
\end{equation*}
\sphinxAtStartPar
supply the element nodal coordinates \(x_1\) and \(x_2\), the modulus of elasticity \(E\), the moment of inertia \(I\), and the spring stiffness in the transverse direction \(k_{\bar{y}}\).

\sphinxAtStartPar
The element load vector \sphinxstylestrong{fe} can also be computed if a uniformly distributed load is applied to the element. The optional input variable
\begin{equation*}
\begin{split}\mathrm{eq} = \begin{bmatrix} q_{\bar{y}} \end{bmatrix}\end{split}
\end{equation*}
\sphinxAtStartPar
contains the distributed load per unit length, \(q_{\bar{y}}\).

\begin{figure}[htbp]
\centering

\noindent\sphinxincludegraphics{{beam1e_2}.png}
\end{figure}

\sphinxlineitem{Theory}
\sphinxAtStartPar
The element stiffness matrix \(\bar{\mathbf{K}}^e\), stored in \sphinxstylestrong{Ke}, is computed according to
\begin{equation*}
\begin{split}\bar{\mathbf{K}}^e = \bar{\mathbf{K}}^e_0 + \bar{\mathbf{K}}^e_s\end{split}
\end{equation*}
\sphinxAtStartPar
where
\begin{equation*}
\begin{split}\bar{\mathbf{K}}^e_0 = \frac{D_{EI}}{L^3}
\begin{bmatrix}
    12 & 6L & -12 & 6L \\
    6L & 4L^2 & -6L & 2L^2 \\
    -12 & -6L & 12 & -6L \\
    6L & 2L^2 & -6L & 4L^2
\end{bmatrix}\end{split}
\end{equation*}
\sphinxAtStartPar
and
\begin{equation*}
\begin{split}\bar{\mathbf{K}}^e_s = \frac{k_{\bar{y}} L}{420}
\begin{bmatrix}
    156 & 22L & 54 & -13L \\
    22L & 4L^2 & 13L & -3L^2 \\
    54 & 13L & 156 & -22L \\
    -13L & -3L^2 & -22L & 4L^2
\end{bmatrix}\end{split}
\end{equation*}
\sphinxAtStartPar
where the bending stiffness \(D_{EI}\) and the length \(L\) are given by
\begin{equation*}
\begin{split}D_{EI} = EI \qquad L = x_2 - x_1\end{split}
\end{equation*}
\sphinxAtStartPar
The element loads \(\bar{\mathbf{f}}_l^e\) stored in the variable \sphinxstylestrong{fe} are computed according to
\begin{equation*}
\begin{split}\bar{\mathbf{f}}_l^e = q_{\bar{y}}
\begin{bmatrix}
    \dfrac{L}{2} \\
    \dfrac{L^2}{12} \\
    \dfrac{L}{2} \\
    -\dfrac{L^2}{12}
\end{bmatrix}\end{split}
\end{equation*}
\end{description}\end{quote}


\subsection{beam1ws}
\label{\detokenize{beam_functions:beam1ws}}\begin{quote}\begin{description}
\sphinxlineitem{Purpose}
\sphinxAtStartPar
Compute section forces in a one dimensional beam element with elastic support.

\begin{figure}[htbp]
\centering

\noindent\sphinxincludegraphics[width=0.700\linewidth]{{beam1s}.png}
\end{figure}

\sphinxlineitem{Syntax}
\begin{sphinxVerbatim}[commandchars=\\\{\}]
\PYG{n}{es}\PYG{+w}{ }\PYG{p}{=}\PYG{+w}{ }\PYG{n}{beam1ws}\PYG{p}{(}\PYG{n}{ex}\PYG{p}{,}\PYG{+w}{ }\PYG{n}{ep}\PYG{p}{,}\PYG{+w}{ }\PYG{n}{ed}\PYG{p}{)}
\PYG{n}{es}\PYG{+w}{ }\PYG{p}{=}\PYG{+w}{ }\PYG{n}{beam1ws}\PYG{p}{(}\PYG{n}{ex}\PYG{p}{,}\PYG{+w}{ }\PYG{n}{ep}\PYG{p}{,}\PYG{+w}{ }\PYG{n}{ed}\PYG{p}{,}\PYG{+w}{ }\PYG{n+nb}{eq}\PYG{p}{)}
\PYG{p}{[}\PYG{n}{es}\PYG{p}{,}\PYG{+w}{ }\PYG{n}{edi}\PYG{p}{,}\PYG{+w}{ }\PYG{n}{eci}\PYG{p}{]}\PYG{+w}{ }\PYG{p}{=}\PYG{+w}{ }\PYG{n}{beam1ws}\PYG{p}{(}\PYG{n}{ex}\PYG{p}{,}\PYG{+w}{ }\PYG{n}{ep}\PYG{p}{,}\PYG{+w}{ }\PYG{n}{ed}\PYG{p}{,}\PYG{+w}{ }\PYG{n+nb}{eq}\PYG{p}{,}\PYG{+w}{ }\PYG{n}{n}\PYG{p}{)}
\end{sphinxVerbatim}

\sphinxlineitem{Description}
\sphinxAtStartPar
\sphinxcode{\sphinxupquote{beam1ws}} computes the section forces and displacements in local directions
along the beam element \sphinxcode{\sphinxupquote{beam1we}}.

\sphinxAtStartPar
The input variables \sphinxcode{\sphinxupquote{ex}}, \sphinxcode{\sphinxupquote{ep}} and \sphinxcode{\sphinxupquote{eq}} are defined in
\sphinxcode{\sphinxupquote{beam1we}}, and the element displacements, stored
in \sphinxcode{\sphinxupquote{ed}}, are obtained by the function \sphinxcode{\sphinxupquote{extract}}.
If distributed loads are applied to the element, the variable \sphinxcode{\sphinxupquote{eq}} must be
included.
The number of evaluation points for section forces and displacements are
determined by \sphinxcode{\sphinxupquote{n}}. If \sphinxcode{\sphinxupquote{n}} is omitted, only the ends of the
beam are evaluated.

\sphinxAtStartPar
The output variables
\begin{align*}\!\begin{aligned}
\mathrm{es} =
\left[
\begin{array}{cc}
V(0)  & M(0) \\
V(\bar{x}_{2}) & M(\bar{x}_{2})  \\
\vdots & \vdots \\
V(\bar{x}_{n-1}) & M(\bar{x}_{n-1})\\
V(L) & M(L)
\end{array}
\right]\\
\quad
\mathrm{edi} =
\left[
\begin{array}{c}
v(0)   \\
v(\bar{x}_{2})   \\
\vdots \\
v(\bar{x}_{n-1})\\
v(L)
\end{array}
\right]\\
\quad
\mathrm{eci} =
\left[
\begin{array}{c}
0  \\
\bar x_{2} \\
\vdots   \\
\bar x_{n-1} \\
L
\end{array}
\right]\\
\end{aligned}\end{align*}
\sphinxAtStartPar
contain the section forces, the displacements, and the evaluation points on the local \(\bar{x}\)\sphinxhyphen{}axis.
\(L\) is the length of the beam element.

\sphinxlineitem{Theory}
\sphinxAtStartPar
The nodal displacements in local coordinates are given by
\begin{equation*}
\begin{split}\mathbf{\bar{a}}^e = \begin{bmatrix} \bar{u}_1 \\ \bar{u}_2 \\ \bar{u}_3 \\ \bar{u}_4 \end{bmatrix}\end{split}
\end{equation*}
\sphinxAtStartPar
where the transpose of \(\mathbf{a}^e\) is stored in \sphinxcode{\sphinxupquote{ed}}.

\sphinxAtStartPar
The displacement \(v(\bar{x})\), the bending moment \(M(\bar{x})\) and the shear force \(V(\bar{x})\) are computed from
\begin{equation*}
\begin{split}v(\bar{x}) = \mathbf{N} \mathbf{\bar{a}}^e + v_p(\bar{x})\end{split}
\end{equation*}\begin{equation*}
\begin{split}M(\bar{x}) = D_{EI} \mathbf{B} \mathbf{\bar{a}}^e + M_p(\bar{x})\end{split}
\end{equation*}\begin{equation*}
\begin{split}V(\bar{x}) = -D_{EI} \frac{d\mathbf{B}}{dx} \mathbf{\bar{a}}^e + V_p(\bar{x})\end{split}
\end{equation*}
\sphinxAtStartPar
where
\begin{equation*}
\begin{split}\mathbf{N} = \begin{bmatrix} 1 & \bar{x} & \bar{x}^2 & \bar{x}^3 \end{bmatrix} \mathbf{C}^{-1}\end{split}
\end{equation*}\begin{equation*}
\begin{split}\mathbf{B} = \begin{bmatrix} 0 & 0 & 2 & 6\bar{x} \end{bmatrix} \mathbf{C}^{-1}\end{split}
\end{equation*}\begin{equation*}
\begin{split}\frac{d\mathbf{B}}{dx} = \begin{bmatrix} 0 & 0 & 0 & 6 \end{bmatrix} \mathbf{C}^{-1}\end{split}
\end{equation*}\begin{equation*}
\begin{split}v_p(\bar{x}) = -\frac{k_{\bar{y}}}{D_{EI}}
\begin{bmatrix}
\frac{\bar{x}^4 - 2L\bar{x}^3 + L^2\bar{x}^2}{24} \\
\frac{\bar{x}^5 - 3L^2\bar{x}^3 + 2L^3\bar{x}^2}{120} \\
\frac{\bar{x}^6 - 4L^3\bar{x}^3 + 3L^4\bar{x}^2}{360} \\
\frac{\bar{x}^7 - 5L^4\bar{x}^3 + 4L^5\bar{x}^2}{840}
\end{bmatrix}^T \mathbf{C}^{-1} \mathbf{\bar{a}}^e
+ \frac{q_{\bar{y}}}{D_{EI}}\left(\frac{\bar{x}^4}{24} - \frac{L\bar{x}^3}{12} + \frac{L^2\bar{x}^2}{24}\right)\end{split}
\end{equation*}\begin{equation*}
\begin{split}M_p(\bar{x}) = -k_{\bar{y}}
\begin{bmatrix}
\frac{6\bar{x}^2 - 6L\bar{x} + L^2}{12} \\
\frac{10\bar{x}^3 - 9L^2\bar{x} + 2L^3}{60} \\
\frac{5\bar{x}^4 - 4L^3\bar{x} + L^4}{60} \\
\frac{21\bar{x}^5 - 15L^4\bar{x} + 4L^5}{420}
\end{bmatrix}^T \mathbf{C}^{-1} \mathbf{\bar{a}}^e
+ q_{\bar{y}}\left(\frac{\bar{x}^2}{2} - \frac{L\bar{x}}{2} + \frac{L^2}{12}\right)\end{split}
\end{equation*}\begin{equation*}
\begin{split}V_p(\bar{x}) = k_{\bar{y}}
\begin{bmatrix}
\frac{2\bar{x} - L}{2} \\
\frac{10\bar{x}^2 - 3L^2}{20} \\
\frac{5\bar{x}^3 - L^3}{15} \\
\frac{7\bar{x}^4 - L^4}{28}
\end{bmatrix}^T \mathbf{C}^{-1} \mathbf{\bar{a}}^e
- q_{\bar{y}}\left(\bar{x} - \frac{L}{2}\right)\end{split}
\end{equation*}
\sphinxAtStartPar
in which \(D_{EI}\), \(k_{\bar{y}}\), \(L\), and \(q_{\bar{y}}\)
are defined in \sphinxcode{\sphinxupquote{beam1we}} and
\begin{equation*}
\begin{split}\mathbf{C}^{-1} =
\begin{bmatrix}
1 & 0 & 0 & 0 \\
0 & 1 & 0 & 0 \\
-\frac{3}{L^2} & -\frac{2}{L} & \frac{3}{L^2} & -\frac{1}{L} \\
\frac{2}{L^3} & \frac{1}{L^2} & -\frac{2}{L^3} & \frac{1}{L^2}
\end{bmatrix}\end{split}
\end{equation*}
\end{description}\end{quote}


\section{2D beam elements}
\label{\detokenize{beam_functions:id1}}

\subsection{beam2e}
\label{\detokenize{beam_functions:beam2e}}
\index{beam2e@\spxentry{beam2e}}\ignorespaces \begin{quote}\begin{description}
\sphinxlineitem{Purpose}
\sphinxAtStartPar
Compute element stiffness matrix for a two\sphinxhyphen{}dimensional beam element.

\begin{figure}[htbp]
\centering

\noindent\sphinxincludegraphics[width=0.700\linewidth]{{BEAM2E}.png}
\end{figure}

\sphinxlineitem{Syntax}
\end{description}\end{quote}

\begin{sphinxVerbatim}[commandchars=\\\{\}]
\PYG{n}{Ke}\PYG{+w}{ }\PYG{p}{=}\PYG{+w}{ }\PYG{n}{beam2e}\PYG{p}{(}\PYG{n}{ex}\PYG{p}{,}\PYG{+w}{ }\PYG{n}{ey}\PYG{p}{,}\PYG{+w}{ }\PYG{n}{ep}\PYG{p}{)}
\PYG{p}{[}\PYG{n}{Ke}\PYG{p}{,}\PYG{+w}{ }\PYG{n}{fe}\PYG{p}{]}\PYG{+w}{ }\PYG{p}{=}\PYG{+w}{ }\PYG{n}{beam2e}\PYG{p}{(}\PYG{n}{ex}\PYG{p}{,}\PYG{+w}{ }\PYG{n}{ey}\PYG{p}{,}\PYG{+w}{ }\PYG{n}{ep}\PYG{p}{,}\PYG{+w}{ }\PYG{n+nb}{eq}\PYG{p}{)}
\end{sphinxVerbatim}
\begin{quote}\begin{description}
\sphinxlineitem{Description}
\sphinxAtStartPar
\sphinxstylestrong{beam2e} provides the global element stiffness matrix \sphinxstylestrong{Ke} for a two\sphinxhyphen{}dimensional beam element.

\sphinxAtStartPar
The input variables:
\begin{equation*}
\begin{split}\begin{aligned}
&\text{ex} = [x_1, x_2] \\
&\text{ey} = [y_1, y_2] \\
&\text{ep} = [E, A, I]
\end{aligned}\end{split}
\end{equation*}
\sphinxAtStartPar
supply the element nodal coordinates \(x_1\), \(y_1\), \(x_2\), and \(y_2\), the modulus of elasticity \(E\), the cross\sphinxhyphen{}section area \(A\), and the moment of inertia \(I\).

\sphinxAtStartPar
The element load vector \sphinxstylestrong{fe} can also be computed if a uniformly distributed transverse load is applied to the element. The optional input variable:
\begin{equation*}
\begin{split}\text{eq} = [q_{\bar{x}}, q_{\bar{y}}]\end{split}
\end{equation*}
\sphinxAtStartPar
contains the distributed loads per unit length, \(q_{\bar{x}}\) and \(q_{\bar{y}}\).

\begin{figure}[htbp]
\centering

\noindent\sphinxincludegraphics[width=0.700\linewidth]{{BEAM2LOA}.png}
\end{figure}

\sphinxlineitem{Theory}
\sphinxAtStartPar
The element stiffness matrix \(\mathbf{K}^e\), stored in \sphinxstylestrong{Ke}, is computed according to:
\begin{equation*}
\begin{split}\mathbf{K}^e = \mathbf{G}^T \bar{\mathbf{K}}^e \mathbf{G}\end{split}
\end{equation*}
\sphinxAtStartPar
where:
\begin{equation*}
\begin{split}\bar{\mathbf{K}}^e =
\begin{bmatrix}
\frac{D_{EA}}{L} & 0 & 0 & -\frac{D_{EA}}{L} & 0 & 0 \\
0 & \frac{12D_{EI}}{L^3} & \frac{6D_{EI}}{L^2} & 0 & -\frac{12D_{EI}}{L^3} & \frac{6D_{EI}}{L^2} \\
0 & \frac{6D_{EI}}{L^2} & \frac{4D_{EI}}{L} & 0 & -\frac{6D_{EI}}{L^2} & \frac{2D_{EI}}{L} \\
-\frac{D_{EA}}{L} & 0 & 0 & \frac{D_{EA}}{L} & 0 & 0 \\
0 & -\frac{12D_{EI}}{L^3} & -\frac{6D_{EI}}{L^2} & 0 & \frac{12D_{EI}}{L^3} & -\frac{6D_{EI}}{L^2} \\
0 & \frac{6D_{EI}}{L^2} & \frac{2D_{EI}}{L} & 0 & -\frac{6D_{EI}}{L^2} & \frac{4D_{EI}}{L}
\end{bmatrix}\end{split}
\end{equation*}\begin{equation*}
\begin{split}\mathbf{G} =
\begin{bmatrix}
n_{x\bar{x}} & n_{y\bar{x}} & 0 & 0 & 0 & 0 \\
n_{x\bar{y}} & n_{y\bar{y}} & 0 & 0 & 0 & 0 \\
0 & 0 & 1 & 0 & 0 & 0 \\
0 & 0 & 0 & n_{x\bar{x}} & n_{y\bar{x}} & 0 \\
0 & 0 & 0 & n_{x\bar{y}} & n_{y\bar{y}} & 0 \\
0 & 0 & 0 & 0 & 0 & 1
\end{bmatrix}\end{split}
\end{equation*}
\sphinxAtStartPar
where the axial stiffness \(D_{EA}\), the bending stiffness \(D_{EI}\), and the length \(L\) are given by:
\begin{equation*}
\begin{split}D_{EA} = EA, \quad D_{EI} = EI, \quad L = \sqrt{(x_2 - x_1)^2 + (y_2 - y_1)^2}\end{split}
\end{equation*}
\sphinxAtStartPar
The transformation matrix \(\mathbf{G}\) contains the direction cosines:
\begin{equation*}
\begin{split}n_{x\bar{x}} = n_{y\bar{y}} = \frac{x_2 - x_1}{L}, \quad
n_{y\bar{x}} = -n_{x\bar{y}} = \frac{y_2 - y_1}{L}\end{split}
\end{equation*}
\sphinxAtStartPar
The element loads \(\mathbf{f}^e_l\), stored in the variable \sphinxstylestrong{fe}, are computed according to:
\begin{equation*}
\begin{split}\mathbf{f}^e_l = \mathbf{G}^T \bar{\mathbf{f}}^e_l\end{split}
\end{equation*}
\sphinxAtStartPar
where:
\begin{equation*}
\begin{split}\bar{\mathbf{f}}^e_l =
\begin{bmatrix}
\frac{q_{\bar{x}}L}{2} \\
\frac{q_{\bar{y}}L}{2} \\
\frac{q_{\bar{y}}L^2}{12} \\
\frac{q_{\bar{x}}L}{2} \\
\frac{q_{\bar{y}}L}{2} \\
-\frac{q_{\bar{y}}L^2}{12}
\end{bmatrix}\end{split}
\end{equation*}
\end{description}\end{quote}


\subsection{beam2s}
\label{\detokenize{beam_functions:beam2s}}
\index{beam2s@\spxentry{beam2s}}\ignorespaces \begin{quote}\begin{description}
\sphinxlineitem{Purpose}
\sphinxAtStartPar
Compute section forces in a two\sphinxhyphen{}dimensional beam element.

\begin{figure}[htbp]
\centering

\noindent\sphinxincludegraphics[width=0.700\linewidth]{{beam2s}.png}
\end{figure}

\sphinxlineitem{Syntax}
\begin{sphinxVerbatim}[commandchars=\\\{\}]
\PYG{p}{[}\PYG{n}{es}\PYG{p}{]}\PYG{+w}{ }\PYG{p}{=}\PYG{+w}{ }\PYG{n}{beam2s}\PYG{p}{(}\PYG{n}{ex}\PYG{p}{,}\PYG{+w}{ }\PYG{n}{ey}\PYG{p}{,}\PYG{+w}{ }\PYG{n}{ep}\PYG{p}{,}\PYG{+w}{ }\PYG{n}{ed}\PYG{p}{)}
\PYG{p}{[}\PYG{n}{es}\PYG{p}{]}\PYG{+w}{ }\PYG{p}{=}\PYG{+w}{ }\PYG{n}{beam2s}\PYG{p}{(}\PYG{n}{ex}\PYG{p}{,}\PYG{+w}{ }\PYG{n}{ey}\PYG{p}{,}\PYG{+w}{ }\PYG{n}{ep}\PYG{p}{,}\PYG{+w}{ }\PYG{n}{ed}\PYG{p}{,}\PYG{+w}{ }\PYG{n+nb}{eq}\PYG{p}{)}
\PYG{p}{[}\PYG{n}{es}\PYG{p}{,}\PYG{+w}{ }\PYG{n}{edi}\PYG{p}{]}\PYG{+w}{ }\PYG{p}{=}\PYG{+w}{ }\PYG{n}{beam2s}\PYG{p}{(}\PYG{n}{ex}\PYG{p}{,}\PYG{+w}{ }\PYG{n}{ey}\PYG{p}{,}\PYG{+w}{ }\PYG{n}{ep}\PYG{p}{,}\PYG{+w}{ }\PYG{n}{ed}\PYG{p}{,}\PYG{+w}{ }\PYG{n+nb}{eq}\PYG{p}{,}\PYG{+w}{ }\PYG{n}{n}\PYG{p}{)}
\PYG{p}{[}\PYG{n}{es}\PYG{p}{,}\PYG{+w}{ }\PYG{n}{edi}\PYG{p}{,}\PYG{+w}{ }\PYG{n}{eci}\PYG{p}{]}\PYG{+w}{ }\PYG{p}{=}\PYG{+w}{ }\PYG{n}{beam2s}\PYG{p}{(}\PYG{n}{ex}\PYG{p}{,}\PYG{+w}{ }\PYG{n}{ey}\PYG{p}{,}\PYG{+w}{ }\PYG{n}{ep}\PYG{p}{,}\PYG{+w}{ }\PYG{n}{ed}\PYG{p}{,}\PYG{+w}{ }\PYG{n+nb}{eq}\PYG{p}{,}\PYG{+w}{ }\PYG{n}{n}\PYG{p}{)}
\end{sphinxVerbatim}

\sphinxlineitem{Description}
\sphinxAtStartPar
\sphinxstylestrong{beam2s} computes the section forces and displacements in local directions along the beam element \sphinxstylestrong{beam2e}.

\sphinxAtStartPar
The input variables \sphinxstylestrong{ex}, \sphinxstylestrong{ey}, \sphinxstylestrong{ep}, and \sphinxstylestrong{eq} are defined in \sphinxstylestrong{beam2e}.

\sphinxAtStartPar
The element displacements, stored in \sphinxstylestrong{ed}, are obtained by the function \sphinxstylestrong{extract}. If a distributed load is applied to the element, the variable \sphinxstylestrong{eq} must be included. The number of evaluation points for section forces and displacements is determined by \sphinxstylestrong{n}. If \sphinxstylestrong{n} is omitted, only the ends of the beam are evaluated.

\sphinxAtStartPar
The output variables:
\begin{align*}\!\begin{aligned}
es =
\begin{bmatrix}
N(0) & V(0) & M(0) \\
N(\bar{x}_2) & V(\bar{x}_2) & M(\bar{x}_2) \\
\vdots & \vdots & \vdots \\
N(\bar{x}_{n-1}) & V(\bar{x}_{n-1}) & M(\bar{x}_{n-1}) \\
N(L) & V(L) & M(L)
\end{bmatrix}\\
edi =
\begin{bmatrix}
u(0) & v(0) \\
u(\bar{x}_2) & v(\bar{x}_2) \\
\vdots & \vdots \\
u(\bar{x}_{n-1}) & v(\bar{x}_{n-1}) \\
u(L) & v(L)
\end{bmatrix}\\
eci =
\begin{bmatrix}
0 \\
\bar{x}_2 \\
\vdots \\
\bar{x}_{n-1} \\
L
\end{bmatrix}\\
\end{aligned}\end{align*}
\sphinxAtStartPar
contain the section forces, the displacements, and the evaluation points on the local \(\bar{x}\)\sphinxhyphen{}axis. \(L\) is the length of the beam element.

\sphinxlineitem{Theory}
\sphinxAtStartPar
The nodal displacements in local coordinates are given by:
\begin{equation*}
\begin{split}\mathbf{\bar{a}}^e =
\begin{bmatrix}
\bar{u}_1 \\
\bar{u}_2 \\
\bar{u}_3 \\
\bar{u}_4 \\
\bar{u}_5 \\
\bar{u}_6
\end{bmatrix}
= \mathbf{G} \mathbf{a}^e\end{split}
\end{equation*}
\sphinxAtStartPar
where \(\mathbf{G}\) is described in \sphinxstylestrong{beam2e} and the transpose of \(\mathbf{a}^e\) is stored in \sphinxstylestrong{ed}.

\sphinxAtStartPar
The displacements associated with bar action and beam action are determined as:
\begin{equation*}
\begin{split}\mathbf{\bar{a}}^e_{\text{bar}} =
\begin{bmatrix}
\bar{u}_1 \\
\bar{u}_4
\end{bmatrix},
\quad
\mathbf{\bar{a}}^e_{\text{beam}} =
\begin{bmatrix}
\bar{u}_2 \\
\bar{u}_3 \\
\bar{u}_5 \\
\bar{u}_6
\end{bmatrix}\end{split}
\end{equation*}
\sphinxAtStartPar
The displacement \(u(\bar{x})\) and the normal force \(N(\bar{x})\) are computed from:
\begin{equation*}
\begin{split}u(\bar{x}) = \mathbf{N}_{\text{bar}} \mathbf{\bar{a}}^e_{\text{bar}} + u_p(\bar{x})\end{split}
\end{equation*}\begin{equation*}
\begin{split}N(\bar{x}) = D_{EA} \mathbf{B}_{\text{bar}} \mathbf{\bar{a}}^e + N_p(\bar{x})\end{split}
\end{equation*}
\sphinxAtStartPar
where:
\begin{equation*}
\begin{split}\mathbf{N}_{\text{bar}} =
\begin{bmatrix}
1 & \bar{x}
\end{bmatrix}
\mathbf{C}^{-1}_{\text{bar}} =
\begin{bmatrix}
1 - \frac{\bar{x}}{L} & \frac{\bar{x}}{L}
\end{bmatrix}\end{split}
\end{equation*}\begin{equation*}
\begin{split}\mathbf{B}_{\text{bar}} =
\begin{bmatrix}
0 & 1
\end{bmatrix}
\mathbf{C}^{-1}_{\text{bar}} =
\begin{bmatrix}
-\frac{1}{L} & \frac{1}{L}
\end{bmatrix}\end{split}
\end{equation*}\begin{equation*}
\begin{split}u_p(\bar{x}) = -\frac{q_{\bar{x}}}{D_{EA}}
\left(\frac{\bar{x}^2}{2} - \frac{L \bar{x}}{2}\right)\end{split}
\end{equation*}\begin{equation*}
\begin{split}N_p(\bar{x}) = -q_{\bar{x}}
\left(\bar{x} - \frac{L}{2}\right)\end{split}
\end{equation*}
\sphinxAtStartPar
where \(D_{EA}\), \(L\), and \(q_{\bar{x}}\) are defined in \sphinxstylestrong{beam2e}, and:
\begin{equation*}
\begin{split}\mathbf{C}^{-1}_{\text{bar}} =
\begin{bmatrix}
1 & 0 \\
-\frac{1}{L} & \frac{1}{L}
\end{bmatrix}\end{split}
\end{equation*}
\sphinxAtStartPar
The displacement \(v(\bar{x})\), the bending moment \(M(\bar{x})\), and the shear force \(V(\bar{x})\) are computed from:
\begin{equation*}
\begin{split}v(\bar{x}) = \mathbf{N}_{\text{beam}} \mathbf{\bar{a}}^e_{\text{beam}} + v_p(\bar{x})\end{split}
\end{equation*}\begin{equation*}
\begin{split}M(\bar{x}) = D_{EI} \mathbf{B}_{\text{beam}} \mathbf{\bar{a}}^e_{\text{beam}} + M_p(\bar{x})\end{split}
\end{equation*}\begin{equation*}
\begin{split}V(\bar{x}) = -D_{EI} \frac{d\mathbf{B}_{\text{beam}}}{d\bar{x}} \mathbf{\bar{a}}^e_{\text{beam}} + V_p(\bar{x})\end{split}
\end{equation*}
\sphinxAtStartPar
where:
\begin{equation*}
\begin{split}\mathbf{N}_{\text{beam}} =
\begin{bmatrix}
1 & \bar{x} & \bar{x}^2 & \bar{x}^3
\end{bmatrix}
\mathbf{C}^{-1}_{\text{beam}}\end{split}
\end{equation*}\begin{equation*}
\begin{split}\mathbf{B}_{\text{beam}} =
\begin{bmatrix}
0 & 0 & 2 & 6\bar{x}
\end{bmatrix}
\mathbf{C}^{-1}_{\text{beam}}\end{split}
\end{equation*}\begin{equation*}
\begin{split}\frac{d\mathbf{B}_{\text{beam}}}{d\bar{x}} =
\begin{bmatrix}
0 & 0 & 0 & 6
\end{bmatrix}
\mathbf{C}^{-1}_{\text{beam}}\end{split}
\end{equation*}\begin{equation*}
\begin{split}v_p(\bar{x}) = \frac{q_{\bar{y}}}{D_{EI}}
\left(\frac{\bar{x}^4}{24} - \frac{L \bar{x}^3}{12} + \frac{L^2 \bar{x}^2}{24}\right)\end{split}
\end{equation*}\begin{equation*}
\begin{split}M_p(\bar{x}) = q_{\bar{y}}
\left(\frac{\bar{x}^2}{2} - \frac{L \bar{x}}{2} + \frac{L^2}{12}\right)\end{split}
\end{equation*}\begin{equation*}
\begin{split}V_p(\bar{x}) = -q_{\bar{y}}
\left(\bar{x} - \frac{L}{2}\right)\end{split}
\end{equation*}
\sphinxAtStartPar
where \(D_{EI}\), \(L\), and \(q_{\bar{y}}\) are defined in \sphinxstylestrong{beam2e}, and:
\begin{equation*}
\begin{split}\mathbf{C}^{-1}_{\text{beam}} =
\begin{bmatrix}
1 & 0 & 0 & 0 \\
0 & 1 & 0 & 0 \\
-\frac{3}{L^2} & -\frac{2}{L} & \frac{3}{L^2} & -\frac{1}{L} \\
\frac{2}{L^3} & \frac{1}{L^2} & -\frac{2}{L^3} & \frac{1}{L^2}
\end{bmatrix}\end{split}
\end{equation*}
\end{description}\end{quote}


\subsection{beam2te}
\label{\detokenize{beam_functions:beam2te}}\begin{quote}\begin{description}
\sphinxlineitem{Purpose}
\sphinxAtStartPar
Compute element stiffness matrix for a two dimensional Timoshenko beam element.

\begin{figure}[htbp]
\centering
\capstart

\noindent\sphinxincludegraphics[width=0.700\linewidth]{{BEAM2T}.png}
\caption{Two dimensional beam element}\label{\detokenize{beam_functions:id3}}\end{figure}

\sphinxlineitem{Syntax}
\begin{sphinxVerbatim}[commandchars=\\\{\}]
\PYG{n}{Ke}\PYG{+w}{ }\PYG{p}{=}\PYG{+w}{ }\PYG{n}{beam2te}\PYG{p}{(}\PYG{n}{ex}\PYG{p}{,}\PYG{+w}{ }\PYG{n}{ey}\PYG{p}{,}\PYG{+w}{ }\PYG{n}{ep}\PYG{p}{)}
\PYG{p}{[}\PYG{n}{Ke}\PYG{p}{,}\PYG{+w}{ }\PYG{n}{fe}\PYG{p}{]}\PYG{+w}{ }\PYG{p}{=}\PYG{+w}{ }\PYG{n}{beam2te}\PYG{p}{(}\PYG{n}{ex}\PYG{p}{,}\PYG{+w}{ }\PYG{n}{ey}\PYG{p}{,}\PYG{+w}{ }\PYG{n}{ep}\PYG{p}{,}\PYG{+w}{ }\PYG{n+nb}{eq}\PYG{p}{)}
\end{sphinxVerbatim}

\sphinxlineitem{Description}
\sphinxAtStartPar
\sphinxcode{\sphinxupquote{beam2te}} provides the global element stiffness matrix \sphinxcode{\sphinxupquote{Ke}} for a
two dimensional Timoshenko beam element.

\sphinxAtStartPar
The input variables
\begin{equation*}
\begin{split}\begin{aligned}
\mathbf{ex} &= [x_1 \;\; x_2] \\
\mathbf{ey} &= [y_1 \;\; y_2] \\
\mathbf{ep} &= [E \;\; G \;\; A \;\; I \;\; k_s]
\end{aligned}\end{split}
\end{equation*}
\sphinxAtStartPar
supply the element nodal coordinates
\(x_1\), \(y_1\), \(x_2\), and \(y_2\), the modulus of elasticity \(E\), the
shear modulus \(G\), the cross section area \(A\), the moment of inertia \(I\)
and the shear correction factor \(k_s\).

\sphinxAtStartPar
The element load vector \sphinxcode{\sphinxupquote{fe}} can also be computed if uniformly
distributed loads are applied to the element.
The optional input variable
\begin{equation*}
\begin{split}\mathrm{eq} = [q_{\bar{x}} \;\; q_{\bar{y}}]\end{split}
\end{equation*}
\sphinxAtStartPar
contains the distributed loads per unit length, \(q_{\bar{x}}\) and \(q_{\bar{y}}\).

\begin{figure}[htbp]
\centering
\capstart

\noindent\sphinxincludegraphics[width=0.700\linewidth]{{BEAM2LOA}.png}
\caption{Uniformly distributed load}\label{\detokenize{beam_functions:id4}}\end{figure}

\sphinxlineitem{Theory}
\sphinxAtStartPar
The element stiffness matrix \(\mathbf{K}^e\), stored in \sphinxcode{\sphinxupquote{Ke}}, is computed
according to
\begin{equation*}
\begin{split}\mathbf{K}^e = \mathbf{G}^T \bar{\mathbf{K}}^e \mathbf{G}\end{split}
\end{equation*}
\sphinxAtStartPar
where \(\mathbf{G}\) is described in \sphinxcode{\sphinxupquote{beam2e}},
and \(\bar{\mathbf{K}}^e\) is given by
\begin{equation*}
\begin{split}\bar{\mathbf{K}}^e = \begin{bmatrix}
\frac{D_{EA}}{L} & 0 & 0 & -\frac{D_{EA}}{L} & 0 & 0 \\
0 & \frac{12D_{EI}}{L^3(1+\mu)} & \frac{6D_{EI}}{L^2(1+\mu)} & 0 & -\frac{12D_{EI}}{L^3(1+\mu)} & \frac{6D_{EI}}{L^2(1+\mu)} \\
0 & \frac{6D_{EI}}{L^2(1+\mu)} & \frac{4D_{EI}(1+\frac{\mu}{4})}{L(1+\mu)} & 0 & -\frac{6D_{EI}}{L^2(1+\mu)} & \frac{2D_{EI}(1-\frac{\mu}{2})}{L(1+\mu)} \\
-\frac{D_{EA}}{L} & 0 & 0 & \frac{D_{EA}}{L} & 0 & 0 \\
0 & -\frac{12D_{EI}}{L^3(1+\mu)} & -\frac{6D_{EI}}{L^2(1+\mu)} & 0 & \frac{12D_{EI}}{L^3(1+\mu)} & -\frac{6D_{EI}}{L^2(1+\mu)} \\
0 & \frac{6D_{EI}}{L^2(1+\mu)} & \frac{2D_{EI}(1-\frac{\mu}{2})}{L(1+\mu)} & 0 & -\frac{6D_{EI}}{L^2(1+\mu)} & \frac{4D_{EI}(1+\frac{\mu}{4})}{L(1+\mu)}
\end{bmatrix}\end{split}
\end{equation*}
\sphinxAtStartPar
where the axial stiffness \(D_{EA}\), the bending stiffness \(D_{EI}\), and the length \(L\) are given by
\begin{equation*}
\begin{split}D_{EA} = EA \qquad D_{EI} = EI \qquad L = \sqrt{(x_2-x_1)^2 + (y_2-y_1)^2}\end{split}
\end{equation*}
\sphinxAtStartPar
and where
\begin{equation*}
\begin{split}\mu = \frac{12 D_{EI}}{L^2 G A k_s}\end{split}
\end{equation*}
\sphinxAtStartPar
The element loads \(\mathbf{f}^e_l\) stored in
the variable \sphinxcode{\sphinxupquote{fe}} are computed according to
\begin{equation*}
\begin{split}\mathbf{f}^e_l = \mathbf{G}^T \bar{\mathbf{f}}^e_l\end{split}
\end{equation*}
\sphinxAtStartPar
where
\begin{equation*}
\begin{split}\bar{\mathbf{f}}^e_l =
\begin{bmatrix}
\frac{q_{\bar{x}} L}{2} \\
\frac{q_{\bar{y}} L}{2} \\
\frac{q_{\bar{y}} L^2}{12} \\
\frac{q_{\bar{x}} L}{2} \\
\frac{q_{\bar{y}} L}{2} \\
-\frac{q_{\bar{y}} L^2}{12}
\end{bmatrix}\end{split}
\end{equation*}
\end{description}\end{quote}


\subsection{beam2ts}
\label{\detokenize{beam_functions:beam2ts}}\begin{quote}\begin{description}
\sphinxlineitem{Purpose}
\sphinxAtStartPar
Compute section forces in a two dimensional Timoshenko beam element.

\begin{figure}[htbp]
\centering

\noindent\sphinxincludegraphics[width=0.700\linewidth]{{beam2s}.png}
\end{figure}

\sphinxlineitem{Syntax}
\begin{sphinxVerbatim}[commandchars=\\\{\}]
\PYG{n}{es}\PYG{+w}{ }\PYG{p}{=}\PYG{+w}{ }\PYG{n}{beam2ts}\PYG{p}{(}\PYG{n}{ex}\PYG{p}{,}\PYG{+w}{ }\PYG{n}{ey}\PYG{p}{,}\PYG{+w}{ }\PYG{n}{ep}\PYG{p}{,}\PYG{+w}{ }\PYG{n}{ed}\PYG{p}{)}
\PYG{n}{es}\PYG{+w}{ }\PYG{p}{=}\PYG{+w}{ }\PYG{n}{beam2ts}\PYG{p}{(}\PYG{n}{ex}\PYG{p}{,}\PYG{+w}{ }\PYG{n}{ey}\PYG{p}{,}\PYG{+w}{ }\PYG{n}{ep}\PYG{p}{,}\PYG{+w}{ }\PYG{n}{ed}\PYG{p}{,}\PYG{+w}{ }\PYG{n+nb}{eq}\PYG{p}{)}
\PYG{p}{[}\PYG{n}{es}\PYG{p}{,}\PYG{+w}{ }\PYG{n}{edi}\PYG{p}{,}\PYG{+w}{ }\PYG{n}{eci}\PYG{p}{]}\PYG{+w}{ }\PYG{p}{=}\PYG{+w}{ }\PYG{n}{beam2ts}\PYG{p}{(}\PYG{n}{ex}\PYG{p}{,}\PYG{+w}{ }\PYG{n}{ey}\PYG{p}{,}\PYG{+w}{ }\PYG{n}{ep}\PYG{p}{,}\PYG{+w}{ }\PYG{n}{ed}\PYG{p}{,}\PYG{+w}{ }\PYG{n+nb}{eq}\PYG{p}{,}\PYG{+w}{ }\PYG{n}{n}\PYG{p}{)}
\end{sphinxVerbatim}

\sphinxlineitem{Description}
\sphinxAtStartPar
\sphinxcode{\sphinxupquote{beam2ts}} computes the section forces and displacements in local directions
along the beam element \sphinxcode{\sphinxupquote{beam2te}}.

\sphinxAtStartPar
The input variables \sphinxcode{\sphinxupquote{ex}}, \sphinxcode{\sphinxupquote{ey}}, \sphinxcode{\sphinxupquote{ep}} and \sphinxcode{\sphinxupquote{eq}} are defined in
\sphinxcode{\sphinxupquote{beam2te}}. The element displacements, stored
in \sphinxcode{\sphinxupquote{ed}}, are obtained by the function \sphinxcode{\sphinxupquote{extract}}.
If distributed loads are applied to the element, the variable \sphinxcode{\sphinxupquote{eq}} must be
included.
The number of evaluation points for section forces and displacements are
determined by \sphinxcode{\sphinxupquote{n}}. If \sphinxcode{\sphinxupquote{n}} is omitted, only the ends of the
beam are evaluated.

\sphinxAtStartPar
The output variables
\begin{equation*}
\begin{split}\begin{aligned}
\mathrm{es} &= \left[\; \mathbf{N} \; \mathbf{V} \; \mathbf{M}\; \right] \\
\mathrm{edi} &= \left[\; \mathbf{u} \; \mathbf{v} \; \boldsymbol{\theta} \; \right] \\
\mathrm{eci} &= \left[ \mathbf{\bar{x}} \right]
\end{aligned}\end{split}
\end{equation*}
\sphinxAtStartPar
consist of column matrices that contain
the section forces, the displacements and rotation of the cross section
(note that the rotation \(\theta\) is not equal to
\(\frac{d\bar v}{d\bar x}\)),
and the evaluation points on the local \(\bar{x}\)\sphinxhyphen{}axis.
The explicit matrix expressions are
\begin{equation*}
\begin{split}\mathrm{es} =
\begin{bmatrix}
N_{1} & V_{1}  & M_{1}  \\
N_{2} & V_{2}  & M_{2}  \\
\vdots    &\vdots   &\vdots   \\
N_{n}  & V_{n}  & M_{n}
\end{bmatrix}
\qquad
\mathrm{edi} =
\begin{bmatrix}
u_{1} & v_{1} & \theta_1   \\
u_{2} & v_{2} & \theta_2 \\
\vdots  & \vdots & \vdots   \\
u_{n} & v_{n} & \theta_n
\end{bmatrix}
\qquad
\mathrm{eci} =
\begin{bmatrix}
0  \\
\bar x_{2} \\
\vdots   \\
\bar x_{n-1} \\
L
\end{bmatrix}\end{split}
\end{equation*}
\sphinxAtStartPar
where \(L\) is the length of the beam element.

\sphinxlineitem{Theory}
\sphinxAtStartPar
The nodal displacements in local coordinates are given by
\begin{equation*}
\begin{split}\mathbf{\bar{a}}^e =
\begin{bmatrix}
\bar{u}_1 \\ \bar{u}_2 \\ \bar{u}_3 \\ \bar{u}_4 \\ \bar{u}_5 \\ \bar{u}_6
\end{bmatrix}
= \mathbf{G} \mathbf{a}^e\end{split}
\end{equation*}
\sphinxAtStartPar
where \(\mathbf{G}\) is described in \sphinxcode{\sphinxupquote{beam2e}}
and the transpose of \(\mathbf{a}^e\) is stored in \sphinxcode{\sphinxupquote{ed}}.
The displacements associated with bar action and beam action are determined as
\begin{equation*}
\begin{split}\mathbf{\bar{a}}^e_{\mathrm{bar}} =
\begin{bmatrix} \bar{u}_1 \\ \bar{u}_4 \end{bmatrix}
\qquad
\mathbf{\bar{a}}^e_{\mathrm{beam}} =
\begin{bmatrix} \bar{u}_2 \\ \bar{u}_3 \\ \bar{u}_5 \\ \bar{u}_6 \end{bmatrix}\end{split}
\end{equation*}
\sphinxAtStartPar
The displacement \(u(\bar{x})\) and the normal force \(N(\bar{x})\) are computed from
\begin{equation*}
\begin{split}u(\bar{x}) = \mathbf{N}_{\mathrm{bar}} \mathbf{\bar{a}}^e_{\mathrm{bar}} + u_p(\bar{x})\end{split}
\end{equation*}\begin{equation*}
\begin{split}N(\bar{x}) = D_{EA} \mathbf{B}_{\mathrm{bar}} \mathbf{\bar{a}}^e + N_p(\bar{x})\end{split}
\end{equation*}
\sphinxAtStartPar
where
\begin{equation*}
\begin{split}\mathbf{N}_{\mathrm{bar}} = \begin{bmatrix} 1 & \bar{x} \end{bmatrix} \mathbf{C}^{-1}_{\mathrm{bar}} = \begin{bmatrix} 1-\frac{\bar{x}}{L} & \frac{\bar{x}}{L} \end{bmatrix}\end{split}
\end{equation*}\begin{equation*}
\begin{split}\mathbf{B}_{\mathrm{bar}} = \begin{bmatrix} 0 & 1 \end{bmatrix} \mathbf{C}^{-1}_{\mathrm{bar}} = \begin{bmatrix} -\frac{1}{L} & \frac{1}{L} \end{bmatrix}\end{split}
\end{equation*}\begin{equation*}
\begin{split}u_p(\bar{x}) = -\frac{q_{\bar{x}}}{D_{EA}}\left(\frac{\bar{x}^2}{2}-\frac{L\bar{x}}{2}\right)\end{split}
\end{equation*}\begin{equation*}
\begin{split}N_p(\bar{x}) = -q_{\bar{x}}\left(\bar{x}-\frac{L}{2}\right)\end{split}
\end{equation*}
\sphinxAtStartPar
in which \(D_{EA}\), \(L\), and \(q_{\bar{x}}\)
are defined in \sphinxcode{\sphinxupquote{beam2te}} and
\begin{equation*}
\begin{split}\mathbf{C}^{-1}_{\mathrm{bar}} =
\begin{bmatrix}
1 & 0 \\
-\frac{1}{L} & \frac{1}{L}
\end{bmatrix}\end{split}
\end{equation*}
\sphinxAtStartPar
The displacement \(v(\bar{x})\), the rotation \(\theta(\bar{x})\), the bending moment \(M(\bar{x})\) and the shear force \(V(\bar{x})\) are computed from
\begin{equation*}
\begin{split}v(\bar{x}) = \mathbf{N}_{\mathrm{beam},v} \mathbf{\bar{a}}^e_{\mathrm{beam}} + v_p(\bar{x})\end{split}
\end{equation*}\begin{equation*}
\begin{split}\theta(\bar{x}) = \mathbf{N}_{\mathrm{beam},\theta} \mathbf{\bar{a}}^e_{\mathrm{beam}} + \theta_p(\bar{x})\end{split}
\end{equation*}\begin{equation*}
\begin{split}M(\bar{x}) = D_{EI} \frac{d\theta}{dx} = D_{EI} \frac{d\mathbf{N}_{\mathrm{beam},\theta}}{d\bar{x}} \mathbf{\bar{a}}^e_{\mathrm{beam}} + M_p(\bar{x})\end{split}
\end{equation*}\begin{equation*}
\begin{split}V(\bar{x}) = D_{GA} k_s \left(\frac{d v}{dx} - \theta \right) = D_{GA} k_s \left(\frac{d\mathbf{N}_{\mathrm{beam},v}}{d\bar{x}} - \mathbf{N}_{\mathrm{beam},\theta} \right) \mathbf{\bar{a}}^e_{\mathrm{beam}} + V_p(\bar{x})\end{split}
\end{equation*}
\sphinxAtStartPar
where
\begin{equation*}
\begin{split}\mathbf{N}_{\mathrm{beam},v} = \begin{bmatrix} 1 & \bar{x} & \bar{x}^2 & \bar{x}^3 \end{bmatrix} \mathbf{C}^{-1}_{\mathrm{beam}}\end{split}
\end{equation*}\begin{equation*}
\begin{split}\frac{d\mathbf{N}_{\mathrm{beam},v}}{d\bar{x}} = \begin{bmatrix} 0 & 1 & 2\bar{x} & 3\bar{x}^2 \end{bmatrix} \mathbf{C}^{-1}_{\mathrm{beam}}\end{split}
\end{equation*}\begin{equation*}
\begin{split}\mathbf{N}_{\mathrm{beam},\theta} = \begin{bmatrix} 0 & 1 & 2\bar{x} & 3\bar{x}^2 + 6\alpha \end{bmatrix} \mathbf{C}^{-1}_{\mathrm{beam}}\end{split}
\end{equation*}\begin{equation*}
\begin{split}\frac{d\mathbf{N}_{\mathrm{beam},\theta}}{d\bar{x}} = \begin{bmatrix} 0 & 0 & 2 & 6\bar{x} \end{bmatrix} \mathbf{C}^{-1}_{\mathrm{beam}}\end{split}
\end{equation*}\begin{equation*}
\begin{split}v_p(\bar{x}) = \frac{q_{\bar{y}}}{D_{EI}}\left(\frac{\bar{x}^4}{24} - \frac{L\bar{x}^3}{12} + \frac{L^2\bar{x}^2}{2}\right) + \frac{q_{\bar{y}}}{D_{GA}k_s}\left(-\frac{\bar{x}^2}{2} + \frac{L\bar{x}}{2}\right)\end{split}
\end{equation*}\begin{equation*}
\begin{split}\theta_p(\bar{x}) = \frac{q_{\bar{y}}}{D_{EI}}\left(\frac{\bar{x}^3}{6} - \frac{L\bar{x}^2}{4} + \frac{L^2\bar{x}}{12}\right)\end{split}
\end{equation*}\begin{equation*}
\begin{split}M_p(\bar{x}) = q_{\bar{y}}\left(\frac{\bar{x}^2}{2} - \frac{L\bar{x}}{2} + \frac{L^2}{12}\right)\end{split}
\end{equation*}\begin{equation*}
\begin{split}V_p(\bar{x}) = -q_{\bar{y}}\left(\bar{x} - \frac{L}{2}\right)\end{split}
\end{equation*}
\sphinxAtStartPar
in which \(D_{EI}\), \(D_{GA}\), \(k_s\), \(L\), and \(q_{\bar{y}}\)
are defined in \sphinxcode{\sphinxupquote{beam2te}} and
\begin{equation*}
\begin{split}\mathbf{C}^{-1}_{\mathrm{beam}} = \frac{1}{L^2 + 12\alpha}
\begin{bmatrix}
L^2 + 12\alpha & 0 & 0 & 0 \\
-\frac{12\alpha}{L} & L^2 + 6\alpha & \frac{12\alpha}{L} & -6\alpha \\
-3 & -2L - \frac{6\alpha}{L} & 3 & -L + \frac{6\alpha}{L} \\
\frac{2}{L} & 1 & -\frac{2}{L} & 1
\end{bmatrix}\end{split}
\end{equation*}
\sphinxAtStartPar
with
\begin{equation*}
\begin{split}\alpha = \frac{D_{EI}}{D_{GA} k_s}\end{split}
\end{equation*}
\end{description}\end{quote}


\subsection{beam2we}
\label{\detokenize{beam_functions:beam2we}}
\index{beam2we@\spxentry{beam2we}}\ignorespaces \begin{quote}\begin{description}
\sphinxlineitem{Purpose}
\sphinxAtStartPar
Compute element stiffness matrix for a two dimensional beam element on elastic support.

\begin{figure}[htbp]
\centering

\noindent\sphinxincludegraphics[width=0.700\linewidth]{{beam2w}.png}
\end{figure}

\sphinxlineitem{Syntax}
\begin{sphinxVerbatim}[commandchars=\\\{\}]
\PYG{n}{Ke}\PYG{+w}{ }\PYG{p}{=}\PYG{+w}{ }\PYG{n}{beam2we}\PYG{p}{(}\PYG{n}{ex}\PYG{p}{,}\PYG{+w}{ }\PYG{n}{ey}\PYG{p}{,}\PYG{+w}{ }\PYG{n}{ep}\PYG{p}{)}
\PYG{p}{[}\PYG{n}{Ke}\PYG{p}{,}\PYG{+w}{ }\PYG{n}{fe}\PYG{p}{]}\PYG{+w}{ }\PYG{p}{=}\PYG{+w}{ }\PYG{n}{beam2we}\PYG{p}{(}\PYG{n}{ex}\PYG{p}{,}\PYG{+w}{ }\PYG{n}{ey}\PYG{p}{,}\PYG{+w}{ }\PYG{n}{ep}\PYG{p}{,}\PYG{+w}{ }\PYG{n+nb}{eq}\PYG{p}{)}
\end{sphinxVerbatim}

\sphinxlineitem{Description}
\sphinxAtStartPar
\sphinxcode{\sphinxupquote{beam2we}} provides the global element stiffness matrix \sphinxcode{\sphinxupquote{Ke}} for a two dimensional beam element with elastic support.

\sphinxAtStartPar
The input variables
\begin{equation*}
\begin{split}\begin{array}{l}
\text{ex} = [x_1\;\; x_2] \\
\text{ey} = [y_1\;\; y_2] \\
\text{ep} = [E\;\; A\;\; I\;\; k_{\bar{x}}\;\; k_{\bar{y}}]
\end{array}\end{split}
\end{equation*}
\sphinxAtStartPar
supply the element nodal coordinates \(x_1\), \(x_2\), \(y_1\), and \(y_2\), the modulus of elasticity \(E\), the cross section area \(A\), the moment of inertia \(I\), the spring stiffness in the axial direction \(k_{\bar{x}}\), and the spring stiffness in the transverse direction \(k_{\bar{y}}\).

\sphinxAtStartPar
The element load vector \sphinxcode{\sphinxupquote{fe}} can also be computed if uniformly distributed loads are applied to the element. The optional input variable
\begin{equation*}
\begin{split}\text{eq} = [q_{\bar{x}}\;\; q_{\bar{y}}]\end{split}
\end{equation*}
\sphinxAtStartPar
contains the distributed load per unit length, \(q_{\bar{x}}\) and \(q_{\bar{y}}\).

\sphinxlineitem{Theory}
\sphinxAtStartPar
The element stiffness matrix \(\mathbf{K}^e\), stored in \sphinxcode{\sphinxupquote{Ke}}, is computed according to
\begin{equation*}
\begin{split}\mathbf{K}^e = \mathbf{G}^T \bar{\mathbf{K}}^e \mathbf{G}\end{split}
\end{equation*}
\sphinxAtStartPar
where
\begin{equation*}
\begin{split}\bar{\mathbf{K}}^e = \bar{\mathbf{K}}^e_0 + \bar{\mathbf{K}}^e_s\end{split}
\end{equation*}\begin{equation*}
\begin{split}\bar{\mathbf{K}}^e_0 =
\begin{bmatrix}
\frac{D_{EA}}{L} & 0 & 0 & -\frac{D_{EA}}{L} & 0 & 0 \\
0 & \frac{12 D_{EI}}{L^3} & \frac{6 D_{EI}}{L^2} & 0 & -\frac{12 D_{EI}}{L^3} & \frac{6 D_{EI}}{L^2} \\
0 & \frac{6 D_{EI}}{L^2} & \frac{4 D_{EI}}{L} & 0 & -\frac{6 D_{EI}}{L^2} & \frac{2 D_{EI}}{L} \\
-\frac{D_{EA}}{L} & 0 & 0 & \frac{D_{EA}}{L} & 0 & 0 \\
0 & -\frac{12 D_{EI}}{L^3} & -\frac{6 D_{EI}}{L^2} & 0 & \frac{12 D_{EI}}{L^3} & -\frac{6 D_{EI}}{L^2} \\
0 & \frac{6 D_{EI}}{L^2} & \frac{2 D_{EI}}{L} & 0 & -\frac{6 D_{EI}}{L^2} & \frac{4 D_{EI}}{L}
\end{bmatrix}\end{split}
\end{equation*}\begin{equation*}
\begin{split}\bar{\mathbf{K}}^e_s = \frac{L}{420}
\begin{bmatrix}
140k_{\bar{x}} & 0 & 0 & 70k_{\bar{x}} & 0 & 0 \\
0 & 156k_{\bar{y}} & 22k_{\bar{y}}L & 0 & 54k_{\bar{y}} & -13k_{\bar{y}}L \\
0 & 22k_{\bar{y}}L & 4k_{\bar{y}}L^2 & 0 & 13k_{\bar{y}}L & -3k_{\bar{y}}L^2 \\
70k_{\bar{x}} & 0 & 0 & 140k_{\bar{x}} & 0 & 0 \\
0 & 54k_{\bar{y}} & 13k_{\bar{y}}L & 0 & 156k_{\bar{y}} & -22k_{\bar{y}}L \\
0 & -13k_{\bar{y}}L & -3k_{\bar{y}}L^2 & 0 & -22k_{\bar{y}}L & 4k_{\bar{y}}L^2
\end{bmatrix}\end{split}
\end{equation*}\begin{equation*}
\begin{split}\mathbf{G} =
\begin{bmatrix}
n_{x\bar{x}} & n_{y\bar{x}} & 0 & 0 & 0 & 0 \\
n_{x\bar{y}} & n_{y\bar{y}} & 0 & 0 & 0 & 0 \\
0 & 0 & 1 & 0 & 0 & 0 \\
0 & 0 & 0 & n_{x\bar{x}} & n_{y\bar{x}} & 0 \\
0 & 0 & 0 & n_{x\bar{y}} & n_{y\bar{y}} & 0 \\
0 & 0 & 0 & 0 & 0 & 1
\end{bmatrix}\end{split}
\end{equation*}
\sphinxAtStartPar
where the axial stiffness \(D_{EA}\), the bending stiffness \(D_{EI}\) and the length \(L\) are given by
\begin{equation*}
\begin{split}D_{EA} = EA;\quad D_{EI} = EI;\quad L = \sqrt{(x_2 - x_1)^2 + (y_2 - y_1)^2}\end{split}
\end{equation*}
\sphinxAtStartPar
The transformation matrix \(\mathbf{G}\) contains the direction cosines
\begin{equation*}
\begin{split}n_{x\bar{x}} = n_{y\bar{y}} = \frac{x_2 - x_1}{L} \qquad
n_{y\bar{x}} = -n_{x\bar{y}} = \frac{y_2 - y_1}{L}\end{split}
\end{equation*}
\sphinxAtStartPar
The element loads \(\mathbf{f}^e_l\) stored in the variable \sphinxcode{\sphinxupquote{fe}} are computed according to
\begin{equation*}
\begin{split}\mathbf{f}^e_l = \mathbf{G}^T \bar{\mathbf{f}}^e_l\end{split}
\end{equation*}
\sphinxAtStartPar
where
\begin{equation*}
\begin{split}\bar{\mathbf{f}}^e_l =
\begin{bmatrix}
\dfrac{q_{\bar{x}}L}{2} \\
\dfrac{q_{\bar{y}}L}{2} \\
\dfrac{q_{\bar{y}}L^2}{12} \\
\dfrac{q_{\bar{x}}L}{2} \\
\dfrac{q_{\bar{y}}L}{2} \\
-\dfrac{q_{\bar{y}}L^2}{12}
\end{bmatrix}\end{split}
\end{equation*}
\end{description}\end{quote}


\subsection{beam2ws}
\label{\detokenize{beam_functions:beam2ws}}\begin{quote}\begin{description}
\sphinxlineitem{Purpose}
\sphinxAtStartPar
Compute section forces in a two dimensional beam element with elastic support.

\begin{figure}[htbp]
\centering

\noindent\sphinxincludegraphics[width=0.700\linewidth]{{beam2s}.png}
\end{figure}

\sphinxlineitem{Syntax}
\begin{sphinxVerbatim}[commandchars=\\\{\}]
\PYG{n}{es}\PYG{+w}{ }\PYG{p}{=}\PYG{+w}{ }\PYG{n}{beam2ws}\PYG{p}{(}\PYG{n}{ex}\PYG{p}{,}\PYG{+w}{ }\PYG{n}{ey}\PYG{p}{,}\PYG{+w}{ }\PYG{n}{ep}\PYG{p}{,}\PYG{+w}{ }\PYG{n}{ed}\PYG{p}{)}
\PYG{n}{es}\PYG{+w}{ }\PYG{p}{=}\PYG{+w}{ }\PYG{n}{beam2ws}\PYG{p}{(}\PYG{n}{ex}\PYG{p}{,}\PYG{+w}{ }\PYG{n}{ey}\PYG{p}{,}\PYG{+w}{ }\PYG{n}{ep}\PYG{p}{,}\PYG{+w}{ }\PYG{n}{ed}\PYG{p}{,}\PYG{+w}{ }\PYG{n+nb}{eq}\PYG{p}{)}
\PYG{n}{es}\PYG{p}{,}\PYG{+w}{ }\PYG{n}{edi}\PYG{p}{,}\PYG{+w}{ }\PYG{n}{eci}\PYG{+w}{ }\PYG{p}{=}\PYG{+w}{ }\PYG{n}{beam2ws}\PYG{p}{(}\PYG{n}{ex}\PYG{p}{,}\PYG{+w}{ }\PYG{n}{ey}\PYG{p}{,}\PYG{+w}{ }\PYG{n}{ep}\PYG{p}{,}\PYG{+w}{ }\PYG{n}{ed}\PYG{p}{,}\PYG{+w}{ }\PYG{n+nb}{eq}\PYG{p}{,}\PYG{+w}{ }\PYG{n}{n}\PYG{p}{)}
\end{sphinxVerbatim}

\sphinxlineitem{Description}
\sphinxAtStartPar
\sphinxcode{\sphinxupquote{beam2ws}} computes the section forces and displacements in local directions
along the beam element \sphinxcode{\sphinxupquote{beam2we}}.

\sphinxAtStartPar
The input variables \sphinxcode{\sphinxupquote{ex}}, \sphinxcode{\sphinxupquote{ey}}, \sphinxcode{\sphinxupquote{ep}} and \sphinxcode{\sphinxupquote{eq}} are defined in
\sphinxcode{\sphinxupquote{beam2we}}, and the element displacements, stored
in \sphinxcode{\sphinxupquote{ed}}, are obtained by the function \sphinxcode{\sphinxupquote{extract}}.
If distributed loads are applied to the element, the variable \sphinxcode{\sphinxupquote{eq}} must be
included.
The number of evaluation points for section forces and displacements are
determined by \sphinxcode{\sphinxupquote{n}}. If \sphinxcode{\sphinxupquote{n}} is omitted, only the ends of the
beam are evaluated.

\sphinxAtStartPar
The output variables
\begin{align*}\!\begin{aligned}
\mathrm{es} =
\begin{bmatrix}
N(0) & V(0)  & M(0) \\
N(\bar{x}_{2}) & V(\bar{x}_{2}) & M(\bar{x}_{2})  \\
\vdots & \vdots & \vdots \\
N(\bar{x}_{n-1}) & V(\bar{x}_{n-1}) & M(\bar{x}_{n-1})\\
N(L) & V(L) & M(L)
\end{bmatrix}\\
\quad
\mathrm{edi} =
\begin{bmatrix}
u(0) & v(0)   \\
u(\bar{x}_{2}) & v(\bar{x}_{2})   \\
\vdots & \vdots \\
u(\bar{x}_{n-1}) & v(\bar{x}_{n-1})\\
u(L) & v(L)
\end{bmatrix}\\
\quad
\mathrm{eci} =
\begin{bmatrix}
0  \\
\bar x_{2} \\
\vdots   \\
\bar x_{n-1} \\
L
\end{bmatrix}\\
\end{aligned}\end{align*}
\sphinxAtStartPar
contain the section forces, the displacements, and the evaluation points on the local \(\bar{x}\)\sphinxhyphen{}axis.
\(L\) is the length of the beam element.

\sphinxlineitem{Theory}
\sphinxAtStartPar
The nodal displacements in local coordinates are given by
\begin{equation*}
\begin{split}\bar{\mathbf{a}}^e =
\begin{bmatrix}
\bar{u}_1 \\ \bar{u}_2 \\ \bar{u}_3 \\ \bar{u}_4 \\ \bar{u}_5 \\ \bar{u}_6
\end{bmatrix}
= \mathbf{G} \mathbf{a}^e\end{split}
\end{equation*}
\sphinxAtStartPar
where \(\mathbf{G}\) is described in \sphinxcode{\sphinxupquote{beam2we}}
and the transpose of \(\mathbf{a}^e\) is stored in \sphinxcode{\sphinxupquote{ed}}.
The displacements associated with bar action and beam action are determined as
\begin{equation*}
\begin{split}\bar{\mathbf{a}}^e_{\text{bar}} =
\begin{bmatrix}
\bar{u}_1 \\
\bar{u}_4
\end{bmatrix}
\qquad
\bar{\mathbf{a}}^e_{\text{beam}} =
\begin{bmatrix}
\bar{u}_2 \\ \bar{u}_3 \\ \bar{u}_5 \\ \bar{u}_6
\end{bmatrix}\end{split}
\end{equation*}
\sphinxAtStartPar
The displacement \(u(\bar{x})\) and the normal force \(N(\bar{x})\) are computed from
\begin{equation*}
\begin{split}u(\bar{x}) = \mathbf{N}_{\text{bar}} \bar{\mathbf{a}}^e_{\text{bar}} + u_p(\bar{x})\end{split}
\end{equation*}\begin{equation*}
\begin{split}N(\bar{x}) = D_{EA} \mathbf{B}_{\text{bar}} \bar{\mathbf{a}}^e + N_p(\bar{x})\end{split}
\end{equation*}
\sphinxAtStartPar
where
\begin{equation*}
\begin{split}\mathbf{N}_{\text{bar}} = \begin{bmatrix} 1 & \bar{x} \end{bmatrix} \mathbf{C}^{-1}_{\text{bar}} = \begin{bmatrix} 1-\frac{\bar{x}}{L} & \frac{\bar{x}}{L} \end{bmatrix}\end{split}
\end{equation*}\begin{equation*}
\begin{split}\mathbf{B}_{\text{bar}} = \begin{bmatrix} 0 & 1 \end{bmatrix} \mathbf{C}^{-1}_{\text{bar}} = \begin{bmatrix} -\frac{1}{L} & \frac{1}{L} \end{bmatrix}\end{split}
\end{equation*}\begin{equation*}
\begin{split}u_p(\bar{x}) = \frac{k_{\bar{x}}}{D_{EA}}
\begin{bmatrix}
\frac{\bar{x}^2-L\bar{x}}{2} & \frac{\bar{x}^3-L^2\bar{x}}{6}
\end{bmatrix}
\mathbf{C}^{-1}_{\text{bar}} \bar{\mathbf{a}}^e_{\text{bar}}
- \frac{q_{\bar{x}}}{D_{EA}}\left(\frac{\bar{x}^2}{2}-\frac{L\bar{x}}{2}\right)\end{split}
\end{equation*}\begin{equation*}
\begin{split}N_p(\bar{x}) = k_{\bar{x}}
\begin{bmatrix}
\frac{2\bar{x}-L}{2} & \frac{3\bar{x}^2-L^2}{6}
\end{bmatrix}
\mathbf{C}^{-1}_{\text{bar}} \bar{\mathbf{a}}^e_{\text{bar}}
- q_{\bar{x}}\left(\bar{x}-\frac{L}{2}\right)\end{split}
\end{equation*}
\sphinxAtStartPar
in which \(D_{EA}\), \(k_{\bar{x}}\), \(L\), and \(q_{\bar{x}}\)
are defined in \sphinxcode{\sphinxupquote{beam2we}} and
\begin{equation*}
\begin{split}\mathbf{C}^{-1}_{\text{bar}} =
\begin{bmatrix}
1 & 0 \\
-\frac{1}{L} & \frac{1}{L}
\end{bmatrix}\end{split}
\end{equation*}
\sphinxAtStartPar
The displacement \(v(\bar{x})\), the bending moment \(M(\bar{x})\) and the shear force \(V(\bar{x})\) are computed from
\begin{equation*}
\begin{split}v(\bar{x}) = \mathbf{N}_{\text{beam}} \bar{\mathbf{a}}^e_{\text{beam}} + v_p(\bar{x})\end{split}
\end{equation*}\begin{equation*}
\begin{split}M(\bar{x}) = D_{EI} \mathbf{B}_{\text{beam}} \bar{\mathbf{a}}^e_{\text{beam}} + M_p(\bar{x})\end{split}
\end{equation*}\begin{equation*}
\begin{split}V(\bar{x}) = -D_{EI} \frac{d\mathbf{B}_{\text{beam}}}{dx} \bar{\mathbf{a}}^e_{\text{beam}} + V_p(\bar{x})\end{split}
\end{equation*}
\sphinxAtStartPar
where
\begin{equation*}
\begin{split}\mathbf{N}_{\text{beam}} = \begin{bmatrix} 1 & \bar{x} & \bar{x}^2 & \bar{x}^3 \end{bmatrix} \mathbf{C}^{-1}_{\text{beam}}\end{split}
\end{equation*}\begin{equation*}
\begin{split}\mathbf{B}_{\text{beam}} = \begin{bmatrix} 0 & 0 & 2 & 6\bar{x} \end{bmatrix} \mathbf{C}^{-1}_{\text{beam}}\end{split}
\end{equation*}\begin{equation*}
\begin{split}\frac{d\mathbf{B}_{\text{beam}}}{dx} = \begin{bmatrix} 0 & 0 & 0 & 6 \end{bmatrix} \mathbf{C}^{-1}_{\text{beam}}\end{split}
\end{equation*}\begin{equation*}
\begin{split}v_p(\bar{x}) = -\frac{k_{\bar{y}}}{D_{EI}}
\begin{bmatrix}
\frac{\bar{x}^4-2L\bar{x}^3+L^2\bar{x}^2}{24} \\
\frac{\bar{x}^5-3L^2\bar{x}^3+2L^3\bar{x}^2}{120} \\
\frac{\bar{x}^6-4L^3\bar{x}^3+3L^4\bar{x}^2}{360} \\
\frac{\bar{x}^7-5L^4\bar{x}^3+4L^5\bar{x}^2}{840}
\end{bmatrix}^T
\mathbf{C}^{-1}_{\text{beam}} \bar{\mathbf{a}}^e_{\text{beam}}
+ \frac{q_{\bar{y}}}{D_{EI}}\left(\frac{\bar{x}^4}{24}-\frac{L\bar{x}^3}{12}+\frac{L^2\bar{x}^2}{24}\right)\end{split}
\end{equation*}\begin{equation*}
\begin{split}M_p(\bar{x}) = -k_{\bar{y}}
\begin{bmatrix}
\frac{6\bar{x}^2-6L\bar{x}+L^2}{12} \\
\frac{10\bar{x}^3-9L^2\bar{x}+2L^3}{60} \\
\frac{5\bar{x}^4-4L^3\bar{x}+L^4}{60} \\
\frac{21\bar{x}^5-15L^4\bar{x}+4L^5}{420}
\end{bmatrix}^T
\mathbf{C}^{-1}_{\text{beam}} \bar{\mathbf{a}}^e_{\text{beam}}
+ q_{\bar{y}}\left(\frac{\bar{x}^2}{2}-\frac{L\bar{x}}{2}+\frac{L^2}{12}\right)\end{split}
\end{equation*}\begin{equation*}
\begin{split}V_p(\bar{x}) = k_{\bar{y}}
\begin{bmatrix}
\frac{2\bar{x}-L}{2} \\
\frac{10\bar{x}^2-3L^2}{20} \\
\frac{5\bar{x}^3-L^3}{15} \\
\frac{7\bar{x}^4-L^4}{28}
\end{bmatrix}^T
\mathbf{C}^{-1}_{\text{beam}} \bar{\mathbf{a}}^e_{\text{beam}}
- q_{\bar{y}}\left(\bar{x}-\frac{L}{2}\right)\end{split}
\end{equation*}
\sphinxAtStartPar
in which \(D_{EI}\), \(k_{\bar{y}}\), \(L\), and \(q_{\bar{y}}\)
are defined in \sphinxcode{\sphinxupquote{beam2we}} and
\begin{equation*}
\begin{split}\mathbf{C}^{-1}_{\text{beam}} =
\begin{bmatrix}
1 & 0 & 0 & 0 \\
0 & 1 & 0 & 0 \\
-\frac{3}{L^2} & -\frac{2}{L} & \frac{3}{L^2} & -\frac{1}{L} \\
\frac{2}{L^3} & \frac{1}{L^2} & -\frac{2}{L^3} & \frac{1}{L^2}
\end{bmatrix}\end{split}
\end{equation*}
\end{description}\end{quote}


\subsection{beam2ge}
\label{\detokenize{beam_functions:beam2ge}}\begin{quote}\begin{description}
\sphinxlineitem{Purpose}
\sphinxAtStartPar
Compute element stiffness matrix for a two dimensional nonlinear beam element with respect to geometrical nonlinearity.

\begin{figure}[htbp]
\centering

\noindent\sphinxincludegraphics[width=0.700\linewidth]{{beam2g}.png}
\end{figure}

\sphinxlineitem{Syntax}
\begin{sphinxVerbatim}[commandchars=\\\{\}]
\PYG{n}{Ke}\PYG{+w}{ }\PYG{p}{=}\PYG{+w}{ }\PYG{n}{beam2ge}\PYG{p}{(}\PYG{n}{ex}\PYG{p}{,}\PYG{+w}{ }\PYG{n}{ey}\PYG{p}{,}\PYG{+w}{ }\PYG{n}{ep}\PYG{p}{,}\PYG{+w}{ }\PYG{n}{Qx}\PYG{p}{)}
\PYG{p}{[}\PYG{n}{Ke}\PYG{p}{,}\PYG{+w}{ }\PYG{n}{fe}\PYG{p}{]}\PYG{+w}{ }\PYG{p}{=}\PYG{+w}{ }\PYG{n}{beam2ge}\PYG{p}{(}\PYG{n}{ex}\PYG{p}{,}\PYG{+w}{ }\PYG{n}{ey}\PYG{p}{,}\PYG{+w}{ }\PYG{n}{ep}\PYG{p}{,}\PYG{+w}{ }\PYG{n}{Qx}\PYG{p}{,}\PYG{+w}{ }\PYG{n+nb}{eq}\PYG{p}{)}
\end{sphinxVerbatim}

\sphinxlineitem{Description}
\sphinxAtStartPar
\sphinxcode{\sphinxupquote{beam2ge}} provides the global element stiffness matrix \sphinxcode{\sphinxupquote{Ke}} for a two dimensional beam element with respect to geometrical nonlinearity.

\sphinxAtStartPar
The input variables:
\begin{itemize}
\item {} 
\sphinxAtStartPar
\sphinxcode{\sphinxupquote{ex = {[}x1 x2{]}}}

\item {} 
\sphinxAtStartPar
\sphinxcode{\sphinxupquote{ey = {[}y1 y2{]}}}

\item {} 
\sphinxAtStartPar
\sphinxcode{\sphinxupquote{ep = {[}E A I{]}}}

\end{itemize}

\sphinxAtStartPar
supply the element nodal coordinates \sphinxcode{\sphinxupquote{x1}}, \sphinxcode{\sphinxupquote{y1}}, \sphinxcode{\sphinxupquote{x2}}, and \sphinxcode{\sphinxupquote{y2}}, the modulus of elasticity \sphinxcode{\sphinxupquote{E}}, the cross section area \sphinxcode{\sphinxupquote{A}}, and the moment of inertia \sphinxcode{\sphinxupquote{I}}.
\begin{itemize}
\item {} 
\sphinxAtStartPar
\sphinxcode{\sphinxupquote{Qx = {[}Q\_xbar{]}}}

\end{itemize}

\sphinxAtStartPar
contains the value of the predefined axial force \sphinxcode{\sphinxupquote{Q\_xbar}}, which is positive in tension.

\sphinxAtStartPar
The element load vector \sphinxcode{\sphinxupquote{fe}} can also be computed if a uniformly distributed transverse load is applied to the element. The optional input variable
\begin{itemize}
\item {} 
\sphinxAtStartPar
\sphinxcode{\sphinxupquote{eq = {[}q\_ybar{]}}}

\end{itemize}

\sphinxAtStartPar
contains the distributed transverse load per unit length, \sphinxcode{\sphinxupquote{q\_ybar}}. Note that \sphinxcode{\sphinxupquote{eq}} is a scalar and not a vector as in \sphinxcode{\sphinxupquote{beam2e}}.

\sphinxlineitem{Theory}
\sphinxAtStartPar
The element stiffness matrix \(\mathbf{K}^e\), stored in the variable \sphinxcode{\sphinxupquote{Ke}}, is computed according to
\begin{equation*}
\begin{split}\mathbf{K}^e = \mathbf{G}^T \bar{\mathbf{K}}^e \mathbf{G}\end{split}
\end{equation*}
\sphinxAtStartPar
where \(\bar{\mathbf{K}}^e\) is given by
\begin{equation*}
\begin{split}\bar{\mathbf{K}}^e = \bar{\mathbf{K}}^e_0 + \bar{\mathbf{K}}^e_{\sigma}\end{split}
\end{equation*}
\sphinxAtStartPar
with
\begin{equation*}
\begin{split}\bar{\mathbf{K}}^e_0 = \begin{bmatrix}
\frac{D_{EA}}{L} & 0 & 0 & -\frac{D_{EA}}{L} & 0 & 0 \\
0 & \frac{12 D_{EI}}{L^3} & \frac{6 D_{EI}}{L^2} & 0 & -\frac{12 D_{EI}}{L^3} & \frac{6 D_{EI}}{L^2} \\
0 & \frac{6 D_{EI}}{L^2} & \frac{4 D_{EI}}{L} & 0 & -\frac{6 D_{EI}}{L^2} & \frac{2 D_{EI}}{L} \\
-\frac{D_{EA}}{L} & 0 & 0 & \frac{D_{EA}}{L} & 0 & 0 \\
0 & -\frac{12 D_{EI}}{L^3} & -\frac{6 D_{EI}}{L^2} & 0 & \frac{12 D_{EI}}{L^3} & -\frac{6 D_{EI}}{L^2} \\
0 & \frac{6 D_{EI}}{L^2} & \frac{2 D_{EI}}{L} & 0 & -\frac{6 D_{EI}}{L^2} & \frac{4 D_{EI}}{L}
\end{bmatrix}\end{split}
\end{equation*}\begin{equation*}
\begin{split}\bar{\mathbf{K}}^e_{\sigma} = Q_{\bar{x}} \begin{bmatrix}
0 & 0 & 0 & 0 & 0 & 0 \\
0 & \frac{6}{5L} & \frac{1}{10} & 0 & -\frac{6}{5L} & \frac{1}{10} \\
0 & \frac{1}{10} & \frac{2L}{15} & 0 & -\frac{1}{10} & -\frac{L}{30} \\
0 & 0 & 0 & 0 & 0 & 0 \\
0 & -\frac{6}{5L} & -\frac{1}{10} & 0 & \frac{6}{5L} & -\frac{1}{10} \\
0 & \frac{1}{10} & -\frac{L}{30} & 0 & -\frac{1}{10} & \frac{2L}{15}
\end{bmatrix}\end{split}
\end{equation*}\begin{equation*}
\begin{split}\mathbf{G} = \begin{bmatrix}
n_{x\bar{x}} & n_{y\bar{x}} & 0 & 0 & 0 & 0 \\
n_{x\bar{y}} & n_{y\bar{y}} & 0 & 0 & 0 & 0 \\
0 & 0 & 1 & 0 & 0 & 0 \\
0 & 0 & 0 & n_{x\bar{x}} & n_{y\bar{x}} & 0 \\
0 & 0 & 0 & n_{x\bar{y}} & n_{y\bar{y}} & 0 \\
0 & 0 & 0 & 0 & 0 & 1
\end{bmatrix}\end{split}
\end{equation*}
\sphinxAtStartPar
where the axial stiffness \(D_{EA}\), the bending stiffness \(D_{EI}\) and the length \(L\) are given by
\begin{equation*}
\begin{split}D_{EA} = EA;\quad D_{EI} = EI;\quad L = \sqrt{(x_2 - x_1)^2 + (y_2 - y_1)^2}\end{split}
\end{equation*}
\sphinxAtStartPar
The transformation matrix \(\mathbf{G}\) contains the direction cosines
\begin{equation*}
\begin{split}n_{x\bar{x}} = n_{y\bar{y}} = \frac{x_2 - x_1}{L} \qquad
n_{y\bar{x}} = -n_{x\bar{y}} = \frac{y_2 - y_1}{L}\end{split}
\end{equation*}
\sphinxAtStartPar
The element loads \(\mathbf{f}^e_l\) stored in \sphinxcode{\sphinxupquote{fe}} are computed according to
\begin{equation*}
\begin{split}\mathbf{f}^e_l = \mathbf{G}^T \bar{\mathbf{f}}^e_l\end{split}
\end{equation*}
\sphinxAtStartPar
where
\begin{equation*}
\begin{split}\bar{\mathbf{f}}^e_l = q_{\bar{y}} \begin{bmatrix} 0 \\ \frac{L}{2} \\ \frac{L^2}{12} \\ 0 \\ \frac{L}{2} \\ -\frac{L^2}{12} \end{bmatrix}\end{split}
\end{equation*}
\end{description}\end{quote}


\subsection{beam2gs}
\label{\detokenize{beam_functions:beam2gs}}\begin{quote}\begin{description}
\sphinxlineitem{Purpose}
\sphinxAtStartPar
Compute section forces in a two dimensional nonlinear beam element with geometrical nonlinearity.

\begin{figure}[htbp]
\centering

\noindent\sphinxincludegraphics[width=0.700\linewidth]{{beam2s}.png}
\end{figure}

\sphinxlineitem{Syntax}
\begin{sphinxVerbatim}[commandchars=\\\{\}]
\PYG{p}{[}\PYG{n}{es}\PYG{p}{,}\PYG{+w}{ }\PYG{n}{Qx}\PYG{p}{]}\PYG{+w}{ }\PYG{p}{=}\PYG{+w}{ }\PYG{n}{beam2gs}\PYG{p}{(}\PYG{n}{ex}\PYG{p}{,}\PYG{+w}{ }\PYG{n}{ey}\PYG{p}{,}\PYG{+w}{ }\PYG{n}{ep}\PYG{p}{,}\PYG{+w}{ }\PYG{n}{ed}\PYG{p}{,}\PYG{+w}{ }\PYG{n}{Qx}\PYG{p}{)}
\PYG{p}{[}\PYG{n}{es}\PYG{p}{,}\PYG{+w}{ }\PYG{n}{Qx}\PYG{p}{]}\PYG{+w}{ }\PYG{p}{=}\PYG{+w}{ }\PYG{n}{beam2gs}\PYG{p}{(}\PYG{n}{ex}\PYG{p}{,}\PYG{+w}{ }\PYG{n}{ey}\PYG{p}{,}\PYG{+w}{ }\PYG{n}{ep}\PYG{p}{,}\PYG{+w}{ }\PYG{n}{ed}\PYG{p}{,}\PYG{+w}{ }\PYG{n}{Qx}\PYG{p}{,}\PYG{+w}{ }\PYG{n+nb}{eq}\PYG{p}{)}
\PYG{p}{[}\PYG{n}{es}\PYG{p}{,}\PYG{+w}{ }\PYG{n}{Qx}\PYG{p}{,}\PYG{+w}{ }\PYG{n}{edi}\PYG{p}{]}\PYG{+w}{ }\PYG{p}{=}\PYG{+w}{ }\PYG{n}{beam2gs}\PYG{p}{(}\PYG{n}{ex}\PYG{p}{,}\PYG{+w}{ }\PYG{n}{ey}\PYG{p}{,}\PYG{+w}{ }\PYG{n}{ep}\PYG{p}{,}\PYG{+w}{ }\PYG{n}{ed}\PYG{p}{,}\PYG{+w}{ }\PYG{n}{Qx}\PYG{p}{,}\PYG{+w}{ }\PYG{n+nb}{eq}\PYG{p}{,}\PYG{+w}{ }\PYG{n}{n}\PYG{p}{)}
\PYG{p}{[}\PYG{n}{es}\PYG{p}{,}\PYG{+w}{ }\PYG{n}{Qx}\PYG{p}{,}\PYG{+w}{ }\PYG{n}{edi}\PYG{p}{,}\PYG{+w}{ }\PYG{n}{eci}\PYG{p}{]}\PYG{+w}{ }\PYG{p}{=}\PYG{+w}{ }\PYG{n}{beam2gs}\PYG{p}{(}\PYG{n}{ex}\PYG{p}{,}\PYG{+w}{ }\PYG{n}{ey}\PYG{p}{,}\PYG{+w}{ }\PYG{n}{ep}\PYG{p}{,}\PYG{+w}{ }\PYG{n}{ed}\PYG{p}{,}\PYG{+w}{ }\PYG{n}{Qx}\PYG{p}{,}\PYG{+w}{ }\PYG{n+nb}{eq}\PYG{p}{,}\PYG{+w}{ }\PYG{n}{n}\PYG{p}{)}
\end{sphinxVerbatim}

\sphinxlineitem{Description}
\sphinxAtStartPar
\sphinxcode{\sphinxupquote{beam2gs}} computes the section forces and displacements in local directions along the geometric nonlinear beam element \sphinxcode{\sphinxupquote{beam2ge}}.

\sphinxAtStartPar
The input variables \sphinxcode{\sphinxupquote{ex}}, \sphinxcode{\sphinxupquote{ey}}, \sphinxcode{\sphinxupquote{ep}}, \sphinxcode{\sphinxupquote{Qx}}, and \sphinxcode{\sphinxupquote{eq}} are described in \sphinxcode{\sphinxupquote{beam2ge}}. The element displacements, stored in \sphinxcode{\sphinxupquote{ed}}, are obtained by the function \sphinxcode{\sphinxupquote{extract}}. If a distributed transversal load is applied to the element, the variable \sphinxcode{\sphinxupquote{eq}} must be included. The number of evaluation points for section forces and displacements are determined by \sphinxcode{\sphinxupquote{n}}. If \sphinxcode{\sphinxupquote{n}} is omitted, only the ends of the beam are evaluated.

\sphinxAtStartPar
The output variable \sphinxcode{\sphinxupquote{Qx}} contains \(Q_{\bar{x}}\) and the output variables
\begin{align*}\!\begin{aligned}
\mathrm{es} =
\begin{bmatrix}
N(0) & V(0)  & M(0) \\
N(\bar{x}_{2}) & V(\bar{x}_{2}) & M(\bar{x}_{2})  \\
\vdots & \vdots & \vdots \\
N(\bar{x}_{n-1}) & V(\bar{x}_{n-1}) & M(\bar{x}_{n-1})\\
N(L) & V(L) & M(L)
\end{bmatrix}\\
\quad
\mathrm{edi} =
\begin{bmatrix}
u(0) & v(0)   \\
u(\bar{x}_{2}) & v(\bar{x}_{2})   \\
\vdots & \vdots \\
u(\bar{x}_{n-1}) & v(\bar{x}_{n-1})\\
u(L) & v(L)
\end{bmatrix}\\
\quad
\mathrm{eci} =
\begin{bmatrix}
0  \\
\bar x_{2} \\
\vdots   \\
\bar x_{n-1} \\
L
\end{bmatrix}\\
\end{aligned}\end{align*}
\sphinxAtStartPar
contain the section forces, the displacements, and the evaluation points on the local \(\bar{x}\)\sphinxhyphen{}axis. \(L\) is the length of the beam element.

\sphinxlineitem{Theory}
\sphinxAtStartPar
The nodal displacements in local coordinates are given by
\begin{equation*}
\begin{split}\mathbf{\bar{a}}^e =
\begin{bmatrix}
\bar{u}_1 \\ \bar{u}_2 \\ \bar{u}_3 \\ \bar{u}_4 \\ \bar{u}_5 \\ \bar{u}_6
\end{bmatrix}
= \mathbf{G} \mathbf{a}^e\end{split}
\end{equation*}
\sphinxAtStartPar
where \(\mathbf{G}\) is described in \sphinxcode{\sphinxupquote{beam2ge}} and the transpose of \(\mathbf{a}^e\) is stored in \sphinxcode{\sphinxupquote{ed}}.

\sphinxAtStartPar
The displacements associated with bar action and beam action are determined as
\begin{equation*}
\begin{split}\mathbf{\bar{a}}^e_{\mathrm{bar}} =
\begin{bmatrix}
\bar{u}_1 \\
\bar{u}_4
\end{bmatrix};
\qquad
\mathbf{\bar{a}}^e_{\mathrm{beam}} =
\begin{bmatrix}
\bar{u}_2 \\ \bar{u}_3 \\ \bar{u}_5 \\ \bar{u}_6
\end{bmatrix}\end{split}
\end{equation*}
\sphinxAtStartPar
The displacement \(u(\bar{x})\) is computed from
\begin{equation*}
\begin{split}u(\bar{x}) = \mathbf{N}_{\mathrm{bar}} \mathbf{\bar{a}}^e_{\mathrm{bar}}\end{split}
\end{equation*}
\sphinxAtStartPar
where
\begin{equation*}
\begin{split}\mathbf{N}_{\mathrm{bar}} = \begin{bmatrix} 1 & \bar{x} \end{bmatrix} \mathbf{C}^{-1}_{\mathrm{bar}} = \begin{bmatrix} 1-\frac{\bar{x}}{L} & \frac{\bar{x}}{L} \end{bmatrix}\end{split}
\end{equation*}
\sphinxAtStartPar
where \(L\) is defined in \sphinxcode{\sphinxupquote{beam2ge}} and
\begin{equation*}
\begin{split}\mathbf{C}^{-1}_{\mathrm{bar}} =
\begin{bmatrix}
1 & 0 \\
-\frac{1}{L} & \frac{1}{L}
\end{bmatrix}\end{split}
\end{equation*}
\sphinxAtStartPar
The displacement \(v(\bar{x})\), the rotation \(\theta(\bar{x})\), the bending moment \(M(\bar{x})\) and the shear force \(V(\bar{x})\) are computed from
\begin{equation*}
\begin{split}v(\bar{x}) = \mathbf{N}_{\mathrm{beam}} \mathbf{\bar{a}}^e_{\mathrm{beam}} + v_p(\bar{x})\end{split}
\end{equation*}\begin{equation*}
\begin{split}\theta(\bar{x}) = \frac{d\mathbf{N}_{\mathrm{beam}}}{dx} \mathbf{\bar{a}}^e_{\mathrm{beam}} + \theta_p(\bar{x})\end{split}
\end{equation*}\begin{equation*}
\begin{split}M(\bar{x}) = D_{EI} \mathbf{B}_{\mathrm{beam}} \mathbf{\bar{a}}^e_{\mathrm{beam}} + M_p(\bar{x})\end{split}
\end{equation*}\begin{equation*}
\begin{split}V(\bar{x}) = -D_{EI} \frac{d\mathbf{B}_{\mathrm{beam}}}{dx} \mathbf{\bar{a}}^e_{\mathrm{beam}} + V_p(\bar{x})\end{split}
\end{equation*}
\sphinxAtStartPar
where
\begin{equation*}
\begin{split}\mathbf{N}_{\mathrm{beam}} = \begin{bmatrix} 1 & \bar{x} & \bar{x}^2 & \bar{x}^3 \end{bmatrix} \mathbf{C}^{-1}_{\mathrm{beam}}\end{split}
\end{equation*}\begin{equation*}
\begin{split}\frac{d\mathbf{N}_{\mathrm{beam}}}{dx} = \begin{bmatrix} 0 & 1 & 2\bar{x} & 3\bar{x}^2 \end{bmatrix} \mathbf{C}^{-1}_{\mathrm{beam}}\end{split}
\end{equation*}\begin{equation*}
\begin{split}\mathbf{B}_{\mathrm{beam}} = \begin{bmatrix} 0 & 0 & 2 & 6\bar{x} \end{bmatrix} \mathbf{C}^{-1}_{\mathrm{beam}}\end{split}
\end{equation*}\begin{equation*}
\begin{split}\frac{d\mathbf{B}_{\mathrm{beam}}}{dx} = \begin{bmatrix} 0 & 0 & 0 & 6 \end{bmatrix} \mathbf{C}^{-1}_{\mathrm{beam}}\end{split}
\end{equation*}\begin{equation*}
\begin{split}v_p(\bar{x}) =
-\frac{Q_{\bar{x}}}{D_{EI}}
\begin{bmatrix}
0 \\
0 \\
\frac{\bar{x}^4}{12}-\frac{L \bar{x}^3}{6}+\frac{L^2 \bar{x}^2}{12} \\
\frac{\bar{x}^5}{20}-\frac{3L^2 \bar{x}^3}{20}+\frac{L^3 \bar{x}^2}{10}
\end{bmatrix}^T
\mathbf{C}^{-1}_{\mathrm{beam}} \mathbf{\bar{a}}^e_{\mathrm{beam}}
+ \frac{q_{\bar{y}}}{D_{EI}}\left(\frac{\bar{x}^4}{24}-\frac{L \bar{x}^3}{12}+\frac{L^2 \bar{x}^2}{24}\right)\end{split}
\end{equation*}\begin{equation*}
\begin{split}\theta_p(\bar{x}) =
-\frac{Q_{\bar{x}}}{D_{EI}}
\begin{bmatrix}
0 \\
0 \\
\frac{\bar{x}^3}{3}-\frac{L \bar{x}^2}{2}+\frac{L^2 \bar{x}}{6} \\
\frac{\bar{x}^4}{4}-\frac{9L^2 \bar{x}^2}{20}+\frac{L^3 \bar{x}}{5}
\end{bmatrix}^T
\mathbf{C}^{-1}_{\mathrm{beam}} \mathbf{\bar{a}}^e_{\mathrm{beam}}
+ \frac{q_{\bar{y}}}{D_{EI}}\left(\frac{\bar{x}^3}{6}-\frac{L \bar{x}^2}{4}+\frac{L^2 \bar{x}}{12}\right)\end{split}
\end{equation*}\begin{equation*}
\begin{split}M_p(\bar{x}) =
-Q_{\bar{x}}
\begin{bmatrix}
0 \\
0 \\
\bar{x}^2 - L\bar{x} + \frac{L^2}{6} \\
\bar{x}^3 - \frac{9L^2 \bar{x}}{10} + \frac{L^3}{5}
\end{bmatrix}^T
\mathbf{C}^{-1}_{\mathrm{beam}} \mathbf{\bar{a}}^e_{\mathrm{beam}}
+ q_{\bar{y}}\left(\frac{\bar{x}^2}{2}-\frac{L \bar{x}}{2}+\frac{L^2}{12}\right)\end{split}
\end{equation*}\begin{equation*}
\begin{split}V_p(\bar{x}) =
Q_{\bar{x}}
\begin{bmatrix}
0 \\
0 \\
2\bar{x} - L \\
3\bar{x}^2 - \frac{9L^2}{10}
\end{bmatrix}^T
\mathbf{C}^{-1}_{\mathrm{beam}} \mathbf{\bar{a}}^e_{\mathrm{beam}}
- q_{\bar{y}}\left(\bar{x} - \frac{L}{2}\right)\end{split}
\end{equation*}
\sphinxAtStartPar
in which \(D_{EI}\), \(L\), and \(q_{\bar{y}}\) are defined in \sphinxcode{\sphinxupquote{beam2ge}} and
\begin{equation*}
\begin{split}\mathbf{C}^{-1}_{\mathrm{beam}} =
\begin{bmatrix}
1 & 0 & 0 & 0 \\
0 & 1 & 0 & 0 \\
-\frac{3}{L^2} & -\frac{2}{L} & \frac{3}{L^2} & -\frac{1}{L} \\
\frac{2}{L^3} & \frac{1}{L^2} & -\frac{2}{L^3} & \frac{1}{L^2}
\end{bmatrix}\end{split}
\end{equation*}
\sphinxAtStartPar
An updated value of the axial force is computed as
\begin{equation*}
\begin{split}Q_{\bar{x}} = D_{EA} \begin{bmatrix} 0 & 1 \end{bmatrix} \mathbf{C}^{-1}_{\mathrm{bar}} \mathbf{\bar{a}}^e_{\mathrm{bar}}\end{split}
\end{equation*}
\sphinxAtStartPar
The normal force \(N(\bar{x})\) is then computed as
\begin{equation*}
\begin{split}N(\bar{x}) = Q_{\bar{x}} + \theta(\bar{x}) V(\bar{x})\end{split}
\end{equation*}
\end{description}\end{quote}


\subsection{beam2gxe}
\label{\detokenize{beam_functions:beam2gxe}}\begin{quote}\begin{description}
\sphinxlineitem{Purpose}
\sphinxAtStartPar
Compute element stiffness matrix for a two dimensional nonlinear beam element with exact solution.

\begin{figure}[htbp]
\centering

\noindent\sphinxincludegraphics[width=0.700\linewidth]{{BEAM2G}.png}
\end{figure}

\sphinxlineitem{Syntax}
\begin{sphinxVerbatim}[commandchars=\\\{\}]
\PYG{n}{Ke}\PYG{+w}{ }\PYG{p}{=}\PYG{+w}{ }\PYG{n}{beam2gxe}\PYG{p}{(}\PYG{n}{ex}\PYG{p}{,}\PYG{+w}{ }\PYG{n}{ey}\PYG{p}{,}\PYG{+w}{ }\PYG{n}{ep}\PYG{p}{,}\PYG{+w}{ }\PYG{n}{Qx}\PYG{p}{)}
\PYG{p}{[}\PYG{n}{Ke}\PYG{p}{,}\PYG{+w}{ }\PYG{n}{fe}\PYG{p}{]}\PYG{+w}{ }\PYG{p}{=}\PYG{+w}{ }\PYG{n}{beam2gxe}\PYG{p}{(}\PYG{n}{ex}\PYG{p}{,}\PYG{+w}{ }\PYG{n}{ey}\PYG{p}{,}\PYG{+w}{ }\PYG{n}{ep}\PYG{p}{,}\PYG{+w}{ }\PYG{n}{Qx}\PYG{p}{,}\PYG{+w}{ }\PYG{n+nb}{eq}\PYG{p}{)}
\end{sphinxVerbatim}

\sphinxlineitem{Description}
\sphinxAtStartPar
\sphinxcode{\sphinxupquote{beam2gxe}} provides the global element stiffness matrix \sphinxcode{\sphinxupquote{Ke}} for a two dimensional beam element with respect to geometrical nonlinearity considering exact solution.

\sphinxAtStartPar
The input variables:
\begin{itemize}
\item {} 
\sphinxAtStartPar
\sphinxcode{\sphinxupquote{ex = {[}x1 x2{]}}}

\item {} 
\sphinxAtStartPar
\sphinxcode{\sphinxupquote{ey = {[}y1 y2{]}}}

\item {} 
\sphinxAtStartPar
\sphinxcode{\sphinxupquote{ep = {[}E A I{]}}}

\end{itemize}

\sphinxAtStartPar
supply the element nodal coordinates \(x_1\), \(y_1\), \(x_2\), and \(y_2\), the modulus of elasticity \(E\), the cross section area \(A\), and the moment of inertia \(I\).
\begin{itemize}
\item {} 
\sphinxAtStartPar
\sphinxcode{\sphinxupquote{Qx = {[}Q\_xbar{]}}}

\end{itemize}

\sphinxAtStartPar
contains the value of the predefined axial force \(Q_{\bar{x}}\), which is positive in tension.

\sphinxAtStartPar
The element load vector \sphinxcode{\sphinxupquote{fe}} can also be computed if a uniformly distributed transverse load is applied to the element. The optional input variable
\begin{itemize}
\item {} 
\sphinxAtStartPar
\sphinxcode{\sphinxupquote{eq = {[}q\_ybar{]}}}

\end{itemize}

\sphinxAtStartPar
then contains the distributed transverse load per unit length, \(q_{\bar{y}}\). Note that \sphinxcode{\sphinxupquote{eq}} is a scalar and not a vector as in \sphinxcode{\sphinxupquote{beam2e}}.

\sphinxlineitem{Theory}
\sphinxAtStartPar
The element stiffness matrix \(\mathbf{K}^e\), stored in the variable \sphinxcode{\sphinxupquote{Ke}}, is computed according to
\begin{equation*}
\begin{split}\mathbf{K}^e = \mathbf{G}^T \bar{\mathbf{K}}^e \mathbf{G}\end{split}
\end{equation*}
\sphinxAtStartPar
with
\begin{equation*}
\begin{split}\bar{\mathbf{K}}^e = \begin{bmatrix}
\frac{D_{EA}}{L} & 0 & 0 & -\frac{D_{EA}}{L} & 0 & 0 \\
0 & \frac{12 D_{EI}}{L^3} \phi_5 & \frac{6 D_{EI}}{L^2} \phi_2 & 0 & -\frac{12 D_{EI}}{L^3} \phi_5 & \frac{6 D_{EI}}{L^2} \phi_2 \\
0 & \frac{6 D_{EI}}{L^2} \phi_2 & \frac{4 D_{EI}}{L} \phi_3 & 0 & -\frac{6 D_{EI}}{L^2} \phi_2 & \frac{2 D_{EI}}{L} \phi_4 \\
-\frac{D_{EA}}{L} & 0 & 0 & \frac{D_{EA}}{L} & 0 & 0 \\
0 & -\frac{12 D_{EI}}{L^3} \phi_5 & -\frac{6 D_{EI}}{L^2} \phi_2 & 0 & \frac{12 D_{EI}}{L^3} \phi_5 & -\frac{6 D_{EI}}{L^2} \phi_2 \\
0 & \frac{6 D_{EI}}{L^2} \phi_2 & \frac{2 D_{EI}}{L} \phi_4 & 0 & -\frac{6 D_{EI}}{L^2} \phi_2 & \frac{4 D_{EI}}{L} \phi_3
\end{bmatrix}\end{split}
\end{equation*}\begin{equation*}
\begin{split}\mathbf{G} = \begin{bmatrix}
n_{x\bar{x}} & n_{y\bar{x}} & 0 & 0 & 0 & 0 \\
n_{x\bar{y}} & n_{y\bar{y}} & 0 & 0 & 0 & 0 \\
0 & 0 & 1 & 0 & 0 & 0 \\
0 & 0 & 0 & n_{x\bar{x}} & n_{y\bar{x}} & 0 \\
0 & 0 & 0 & n_{x\bar{y}} & n_{y\bar{y}} & 0 \\
0 & 0 & 0 & 0 & 0 & 1
\end{bmatrix}\end{split}
\end{equation*}
\sphinxAtStartPar
where the axial stiffness \(D_{EA}\), the bending stiffness \(D_{EI}\) and the length \(L\) are given by
\begin{equation*}
\begin{split}D_{EA} = EA; \quad D_{EI} = EI; \quad L = \sqrt{(x_2 - x_1)^2 + (y_2 - y_1)^2}\end{split}
\end{equation*}
\sphinxAtStartPar
The transformation matrix \(\mathbf{G}\) contains the direction cosines
\begin{equation*}
\begin{split}n_{x\bar{x}} = n_{y\bar{y}} = \frac{x_2 - x_1}{L} \qquad
n_{y\bar{x}} = -n_{x\bar{y}} = \frac{y_2 - y_1}{L}\end{split}
\end{equation*}
\sphinxAtStartPar
For axial compression (\(Q_{\bar{x}} < 0\)):
\begin{equation*}
\begin{split}\phi_2 = \frac{1}{12} \frac{k^2 L^2}{1 - \phi_1} \qquad
\phi_3 = \frac{1}{4} \phi_1 + \frac{3}{4} \phi_2\end{split}
\end{equation*}\begin{equation*}
\begin{split}\phi_4 = -\frac{1}{2} \phi_1 + \frac{3}{2} \phi_2 \qquad
\phi_5 = \phi_1 \phi_2\end{split}
\end{equation*}
\sphinxAtStartPar
with
\begin{equation*}
\begin{split}k = \sqrt{\frac{-Q_{\bar{x}}}{D_{EI}}} \qquad \phi_1 = \frac{kL}{2} \cot \frac{kL}{2}\end{split}
\end{equation*}
\sphinxAtStartPar
For axial tension (\(Q_{\bar{x}} > 0\)):
\begin{equation*}
\begin{split}\phi_2 = -\frac{1}{12} \frac{k^2 L^2}{1 - \phi_1} \qquad
\phi_3 = \frac{1}{4} \phi_1 + \frac{3}{4} \phi_2\end{split}
\end{equation*}\begin{equation*}
\begin{split}\phi_4 = -\frac{1}{2} \phi_1 + \frac{3}{2} \phi_2 \qquad
\phi_5 = \phi_1 \phi_2\end{split}
\end{equation*}
\sphinxAtStartPar
with
\begin{equation*}
\begin{split}k = \sqrt{\frac{Q_{\bar{x}}}{D_{EI}}} \qquad \phi_1 = \frac{kL}{2} \coth \frac{kL}{2}\end{split}
\end{equation*}
\sphinxAtStartPar
The element loads \(\mathbf{f}^e_l\) stored in the variable \sphinxcode{\sphinxupquote{fe}} are computed according to
\begin{equation*}
\begin{split}\mathbf{f}^e_l = \mathbf{G}^T \bar{\mathbf{f}}^e_l\end{split}
\end{equation*}
\sphinxAtStartPar
where
\begin{equation*}
\begin{split}\bar{\mathbf{f}}^e_l = qL \begin{bmatrix} 0 \\ \frac{1}{2} \\ \frac{L}{12} \psi \\ 0 \\ \frac{1}{2} \\ -\frac{L}{12} \psi \end{bmatrix}\end{split}
\end{equation*}
\sphinxAtStartPar
For an axial compressive force (\(Q_{\bar{x}} < 0\)):
\begin{equation*}
\begin{split}\psi = 6 \left( \frac{2}{(kL)^2} - \frac{1 + \cos kL}{kL \sin kL} \right)\end{split}
\end{equation*}
\sphinxAtStartPar
and for an axial tensile force (\(Q_{\bar{x}} > 0\)):
\begin{equation*}
\begin{split}\psi = -6 \left( \frac{2}{(kL)^2} - \frac{1 + \cosh kL}{kL \sinh kL} \right)\end{split}
\end{equation*}
\end{description}\end{quote}


\subsection{beam2gxs}
\label{\detokenize{beam_functions:beam2gxs}}\begin{quote}\begin{description}
\sphinxlineitem{Purpose}
\sphinxAtStartPar
Compute section forces in a two dimensional geometric nonlinear beam element with exact solution.

\begin{figure}[htbp]
\centering

\noindent\sphinxincludegraphics[width=0.700\linewidth]{{beam2s}.png}
\end{figure}

\sphinxlineitem{Syntax}
\begin{sphinxVerbatim}[commandchars=\\\{\}]
\PYG{p}{[}\PYG{n}{es}\PYG{p}{,}\PYG{n}{Qx}\PYG{p}{]}\PYG{+w}{ }\PYG{p}{=}\PYG{+w}{ }\PYG{n}{beam2gxs}\PYG{p}{(}\PYG{n}{ex}\PYG{p}{,}\PYG{+w}{ }\PYG{n}{ey}\PYG{p}{,}\PYG{+w}{ }\PYG{n}{ep}\PYG{p}{,}\PYG{+w}{ }\PYG{n}{ed}\PYG{p}{,}\PYG{+w}{ }\PYG{n}{Qx}\PYG{p}{)}
\PYG{p}{[}\PYG{n}{es}\PYG{p}{,}\PYG{n}{Qx}\PYG{p}{]}\PYG{+w}{ }\PYG{p}{=}\PYG{+w}{ }\PYG{n}{beam2gxs}\PYG{p}{(}\PYG{n}{ex}\PYG{p}{,}\PYG{+w}{ }\PYG{n}{ey}\PYG{p}{,}\PYG{+w}{ }\PYG{n}{ep}\PYG{p}{,}\PYG{+w}{ }\PYG{n}{ed}\PYG{p}{,}\PYG{+w}{ }\PYG{n}{Qx}\PYG{p}{,}\PYG{+w}{ }\PYG{n+nb}{eq}\PYG{p}{)}
\PYG{p}{[}\PYG{n}{es}\PYG{p}{,}\PYG{n}{Qx}\PYG{p}{,}\PYG{n}{edi}\PYG{p}{]}\PYG{+w}{ }\PYG{p}{=}\PYG{+w}{ }\PYG{n}{beam2gxs}\PYG{p}{(}\PYG{n}{ex}\PYG{p}{,}\PYG{+w}{ }\PYG{n}{ey}\PYG{p}{,}\PYG{+w}{ }\PYG{n}{ep}\PYG{p}{,}\PYG{+w}{ }\PYG{n}{ed}\PYG{p}{,}\PYG{+w}{ }\PYG{n}{Qx}\PYG{p}{,}\PYG{+w}{ }\PYG{n+nb}{eq}\PYG{p}{,}\PYG{+w}{ }\PYG{n}{n}\PYG{p}{)}
\PYG{p}{[}\PYG{n}{es}\PYG{p}{,}\PYG{n}{Qx}\PYG{p}{,}\PYG{n}{edi}\PYG{p}{,}\PYG{n}{eci}\PYG{p}{]}\PYG{+w}{ }\PYG{p}{=}\PYG{+w}{ }\PYG{n}{beam2gxs}\PYG{p}{(}\PYG{n}{ex}\PYG{p}{,}\PYG{+w}{ }\PYG{n}{ey}\PYG{p}{,}\PYG{+w}{ }\PYG{n}{ep}\PYG{p}{,}\PYG{+w}{ }\PYG{n}{ed}\PYG{p}{,}\PYG{+w}{ }\PYG{n}{Qx}\PYG{p}{,}\PYG{+w}{ }\PYG{n+nb}{eq}\PYG{p}{,}\PYG{+w}{ }\PYG{n}{n}\PYG{p}{)}
\end{sphinxVerbatim}

\sphinxlineitem{Description}
\sphinxAtStartPar
\sphinxcode{\sphinxupquote{beam2gxs}} computes the section forces and displacements in local directions along the geometric nonlinear beam element \sphinxcode{\sphinxupquote{beam2gxe}}.

\sphinxAtStartPar
The input variables \sphinxcode{\sphinxupquote{ex}}, \sphinxcode{\sphinxupquote{ey}}, \sphinxcode{\sphinxupquote{ep}}, \sphinxcode{\sphinxupquote{Qx}}, and \sphinxcode{\sphinxupquote{eq}} are described in \sphinxcode{\sphinxupquote{beam2gxe}}. The element displacements, stored in \sphinxcode{\sphinxupquote{ed}}, are obtained by the function \sphinxcode{\sphinxupquote{extract}}. If a distributed transversal load is applied to the element, the variable \sphinxcode{\sphinxupquote{eq}} must be included. The number of evaluation points for section forces and displacements are determined by \sphinxcode{\sphinxupquote{n}}. If \sphinxcode{\sphinxupquote{n}} is omitted, only the ends of the beam are evaluated.

\sphinxAtStartPar
The output variable \sphinxcode{\sphinxupquote{Qx}} contains \(Q_{\bar{x}}\) and the output variables
\begin{align*}\!\begin{aligned}
\mathrm{es} =
\begin{bmatrix}
N(0) & V(0)  & M(0) \\
N(\bar{x}_{2}) & V(\bar{x}_{2}) & M(\bar{x}_{2})  \\
\vdots & \vdots & \vdots \\
N(\bar{x}_{n-1}) & V(\bar{x}_{n-1}) & M(\bar{x}_{n-1})\\
N(L) & V(L) & M(L)
\end{bmatrix}\\
\quad
\mathrm{edi} =
\begin{bmatrix}
u(0) & v(0)   \\
u(\bar{x}_{2}) & v(\bar{x}_{2})   \\
\vdots & \vdots \\
u(\bar{x}_{n-1}) & v(\bar{x}_{n-1})\\
u(L) & v(L)
\end{bmatrix}\\
\quad
\mathrm{eci} =
\begin{bmatrix}
0  \\
\bar x_{2} \\
\vdots   \\
\bar x_{n-1} \\
L
\end{bmatrix}\\
\end{aligned}\end{align*}
\sphinxAtStartPar
contain the section forces, the displacements, and the evaluation points on the local \(\bar{x}\)\sphinxhyphen{}axis. \(L\) is the length of the beam element.

\sphinxlineitem{Theory}
\sphinxAtStartPar
The nodal displacements in local coordinates are given by
\begin{equation*}
\begin{split}\mathbf{\bar{a}}^e =
\begin{bmatrix}
\bar{u}_1 \\ \bar{u}_2 \\ \bar{u}_3 \\ \bar{u}_4 \\ \bar{u}_5 \\ \bar{u}_6
\end{bmatrix}
= \mathbf{G} \mathbf{a}^e\end{split}
\end{equation*}
\sphinxAtStartPar
where \(\mathbf{G}\) is described in \sphinxcode{\sphinxupquote{beam2ge}} and the transpose of \(\mathbf{a}^e\) is stored in \sphinxcode{\sphinxupquote{ed}}. The displacements associated with bar action and beam action are determined as
\begin{equation*}
\begin{split}\mathbf{\bar{a}}^e_{\text{bar}} =
\begin{bmatrix}
\bar{u}_1 \\
\bar{u}_4
\end{bmatrix}
; \quad
\mathbf{\bar{a}}^e_{\text{beam}} =
\begin{bmatrix}
\bar{u}_2 \\ \bar{u}_3 \\ \bar{u}_5 \\ \bar{u}_6
\end{bmatrix}\end{split}
\end{equation*}
\sphinxAtStartPar
The displacement \(u(\bar{x})\) is computed from
\begin{equation*}
\begin{split}u(\bar{x}) = \mathbf{N}_{\text{bar}} \mathbf{\bar{a}}^e_{\text{bar}}\end{split}
\end{equation*}
\sphinxAtStartPar
where
\begin{equation*}
\begin{split}\mathbf{N}_{\text{bar}} =
\begin{bmatrix}
1 & \bar{x}
\end{bmatrix}
\mathbf{C}^{-1}_{\text{bar}} =
\begin{bmatrix}
1-\frac{\bar{x}}{L} & \frac{\bar{x}}{L}
\end{bmatrix}\end{split}
\end{equation*}
\sphinxAtStartPar
where \(L\) is defined in \sphinxcode{\sphinxupquote{beam2gxe}} and
\begin{equation*}
\begin{split}\mathbf{C}^{-1}_{\text{bar}} =
\begin{bmatrix}
1 & 0 \\
-\frac{1}{L} & \frac{1}{L}
\end{bmatrix}\end{split}
\end{equation*}
\sphinxAtStartPar
The displacement \(v(\bar{x})\), the rotation \(\theta(\bar{x})\), the bending moment \(M(\bar{x})\) and the shear force \(V(\bar{x})\) are computed from
\begin{equation*}
\begin{split}v(\bar{x}) = \mathbf{N}_{\text{beam}} \mathbf{\bar{a}}^e_{\text{beam}} + v_p(\bar{x})\end{split}
\end{equation*}\begin{equation*}
\begin{split}\theta(\bar{x}) = \frac{d\mathbf{N}_{\text{beam}}}{dx} \mathbf{\bar{a}}^e_{\text{beam}} + \theta_p(\bar{x})\end{split}
\end{equation*}\begin{equation*}
\begin{split}M(\bar{x}) = D_{EI} \mathbf{B}_{\text{beam}} \mathbf{\bar{a}}^e_{\text{beam}} + M_p(\bar{x})\end{split}
\end{equation*}\begin{equation*}
\begin{split}V(\bar{x}) = -D_{EI} \frac{d\mathbf{B}_{\text{beam}}}{dx} \mathbf{\bar{a}}^e_{\text{beam}} + V_p(\bar{x})\end{split}
\end{equation*}
\sphinxAtStartPar
For an axial compressive force (\(Q_{\bar{x}} < 0\)) we have
\begin{equation*}
\begin{split}\mathbf{N}_{\text{beam}} =
\begin{bmatrix}
1 & \bar{x} & \cos k \bar{x} & \sin k \bar{x}
\end{bmatrix}
\mathbf{C}^{-1}_{\text{beam}}\end{split}
\end{equation*}\begin{equation*}
\begin{split}\frac{d\mathbf{N}_{\text{beam}}}{dx} =
\begin{bmatrix}
0 & 1 & -k \sin k \bar{x} & k \cos k \bar{x}
\end{bmatrix}
\mathbf{C}^{-1}_{\text{beam}}\end{split}
\end{equation*}\begin{equation*}
\begin{split}\mathbf{B}_{\text{beam}} =
\begin{bmatrix}
0 & 0 & -k^2 \cos k \bar{x} & -k^2 \sin k \bar{x}
\end{bmatrix}
\mathbf{C}^{-1}_{\text{beam}}\end{split}
\end{equation*}\begin{equation*}
\begin{split}\frac{d\mathbf{B}_{\text{beam}}}{dx} =
\begin{bmatrix}
0 & 0 & k^3 \sin k \bar{x} & -k^3 \cos k \bar{x}
\end{bmatrix}
\mathbf{C}^{-1}_{\text{beam}}\end{split}
\end{equation*}\begin{equation*}
\begin{split}v_p(\bar{x}) =
\frac{q_{\bar{y}}L^4}{2D_{EI}}
\left[
    \frac{1 + \cos kL}{(kL)^3 \sin kL}(-1 + \cos k \bar{x})
    -\frac{1}{(kL)^3} \sin k \bar{x}
    + \frac{1}{(kL)^2} \left(\frac{\bar{x}^2}{L^2}-\frac{\bar{x}}{L}\right)
\right]\end{split}
\end{equation*}\begin{equation*}
\begin{split}\theta_p(\bar{x}) =
\frac{q_{\bar{y}}L^3}{2D_{EI}}
\left[
    -\frac{1 + \cos kL}{(kL)^2 \sin kL} \sin k \bar{x}
    -\frac{1}{(kL)^2} \cos k \bar{x}
    + \frac{1}{(kL)^2} \left(\frac{2\bar{x}}{L}-1\right)
\right]\end{split}
\end{equation*}\begin{equation*}
\begin{split}M_p(\bar{x}) =
\frac{q_{\bar{y}}L^2}{2}
\left[
    -\frac{1 + \cos kL}{(kL) \sin kL} \cos k \bar{x}
    +\frac{1}{(kL)} \sin k \bar{x}
    + \frac{2}{(kL)^2}
\right]\end{split}
\end{equation*}\begin{equation*}
\begin{split}V_p(\bar{x}) = Q_{\bar{x}}
\begin{bmatrix}
0 \\
0 \\
2\bar{x} - L \\
3\bar{x}^2 - \frac{9L^2}{10}
\end{bmatrix}^T
\mathbf{C}^{-1}_{\text{beam}}
\mathbf{\bar{a}}^e_{\text{beam}}
- q_{\bar{y}}\left(\bar{x} - \frac{L}{2}\right)\end{split}
\end{equation*}
\sphinxAtStartPar
in which \(D_{EI}\), \(L\), and \(q_{\bar{y}}\) are defined in \sphinxcode{\sphinxupquote{beam2gxe}} and
\begin{equation*}
\begin{split}\mathbf{C}^{-1}_{\text{beam}} =
\begin{bmatrix}
k (kL \sin kL+\cos kL-1) & -kL \cos kL+\sin kL & -k (1-\cos kL) & -\sin kL+kL \\
- k^2  \sin kL & -k (1-\cos kL) &  k^2  \sin kL & -k (1-\cos kL) \\
-k(1-\cos kL) & kL \cos kL-\sin kL & k (1-\cos kL) & \sin kL-kL \\
k\sin kL & kL \sin kL+\cos kL-1 & -k \sin kL & 1-\cos kL
\end{bmatrix}\end{split}
\end{equation*}
\sphinxAtStartPar
An updated value of the axial force is computed as
\begin{equation*}
\begin{split}Q_{\bar{x}} = D_{EA}
\begin{bmatrix}
0 & 1
\end{bmatrix}
\mathbf{C}^{-1}_{\text{bar}}
\mathbf{\bar{a}}^e_{\text{bar}}\end{split}
\end{equation*}
\sphinxAtStartPar
The normal force \(N(\bar{x})\) is then computed as
\begin{equation*}
\begin{split}N(\bar{x}) = Q_{\bar{x}} + \theta(\bar{x}) V(\bar{x})\end{split}
\end{equation*}
\end{description}\end{quote}


\subsection{beam2de}
\label{\detokenize{beam_functions:beam2de}}\begin{quote}\begin{description}
\sphinxlineitem{Purpose}
\sphinxAtStartPar
Compute element stiffness, mass and damping matrices for a two dimensional beam element.

\begin{figure}[htbp]
\centering

\noindent\sphinxincludegraphics[width=0.700\linewidth]{{BEAM2D}.png}
\end{figure}

\sphinxlineitem{Syntax}
\begin{sphinxVerbatim}[commandchars=\\\{\}]
\PYG{p}{[}\PYG{n}{Ke}\PYG{p}{,}\PYG{+w}{ }\PYG{n}{Me}\PYG{p}{]}\PYG{+w}{ }\PYG{p}{=}\PYG{+w}{ }\PYG{n}{beam2de}\PYG{p}{(}\PYG{n}{ex}\PYG{p}{,}\PYG{+w}{ }\PYG{n}{ey}\PYG{p}{,}\PYG{+w}{ }\PYG{n}{ep}\PYG{p}{)}
\PYG{p}{[}\PYG{n}{Ke}\PYG{p}{,}\PYG{+w}{ }\PYG{n}{Me}\PYG{p}{,}\PYG{+w}{ }\PYG{n}{Ce}\PYG{p}{]}\PYG{+w}{ }\PYG{p}{=}\PYG{+w}{ }\PYG{n}{beam2de}\PYG{p}{(}\PYG{n}{ex}\PYG{p}{,}\PYG{+w}{ }\PYG{n}{ey}\PYG{p}{,}\PYG{+w}{ }\PYG{n}{ep}\PYG{p}{)}
\end{sphinxVerbatim}

\sphinxlineitem{Description}
\sphinxAtStartPar
\sphinxcode{\sphinxupquote{beam2de}} provides the global element stiffness matrix \sphinxcode{\sphinxupquote{Ke}}, the global element mass matrix \sphinxcode{\sphinxupquote{Me}}, and the global element damping matrix \sphinxcode{\sphinxupquote{Ce}}, for a two dimensional beam element.

\sphinxAtStartPar
The input variables \sphinxcode{\sphinxupquote{ex}} and \sphinxcode{\sphinxupquote{ey}} are described in \sphinxcode{\sphinxupquote{beam2e}}, and
\begin{equation*}
\begin{split}ep = [ E,\; A,\; I,\; m,\; [a_0,\; a_1] ]\end{split}
\end{equation*}
\sphinxAtStartPar
contains the modulus of elasticity \(E\), the cross section area \(A\), the moment of inertia \(I\), the mass per unit length \(m\), and the Rayleigh damping coefficients \(a_0\) and \(a_1\).
If \(a_0\) and \(a_1\) are omitted, the element damping matrix \sphinxcode{\sphinxupquote{Ce}} is not computed.

\sphinxlineitem{Theory}
\sphinxAtStartPar
The element stiffness matrix \(\mathbf{K}^e\), the element mass matrix \(\mathbf{M}^e\) and the element damping matrix \(\mathbf{C}^e\), stored in the variables \sphinxcode{\sphinxupquote{Ke}}, \sphinxcode{\sphinxupquote{Me}} and \sphinxcode{\sphinxupquote{Ce}}, respectively, are computed according to
\begin{equation*}
\begin{split}\mathbf{K}^e = \mathbf{G}^T \bar{\mathbf{K}}^e \mathbf{G} \qquad
\mathbf{M}^e = \mathbf{G}^T \bar{\mathbf{M}}^e \mathbf{G} \qquad
\mathbf{C}^e = \mathbf{G}^T \bar{\mathbf{C}}^e \mathbf{G}\end{split}
\end{equation*}
\sphinxAtStartPar
where \(\mathbf{G}\) and \(\bar{\mathbf{K}}^e\) are described in \sphinxcode{\sphinxupquote{beam2e}}.

\sphinxAtStartPar
The matrix \(\bar{\mathbf{M}}^e\) is given by
\begin{equation*}
\begin{split}\bar{\mathbf{M}}^e = \frac{mL}{420}
\begin{bmatrix}
140 & 0 & 0 & 70 & 0 & 0 \\
0 & 156 & 22L & 0 & 54 & -13L \\
0 & 22L & 4L^2 & 0 & 13L & -3L^2 \\
70 & 0 & 0 & 140 & 0 & 0 \\
0 & 54 & 13L & 0 & 156 & -22L \\
0 & -13L & -3L^2 & 0 & -22L & 4L^2
\end{bmatrix}\end{split}
\end{equation*}
\sphinxAtStartPar
and the matrix \(\bar{\mathbf{C}}^e\) is computed by combining \(\bar{\mathbf{K}}^e\) and \(\bar{\mathbf{M}}^e\):
\begin{equation*}
\begin{split}\bar{\mathbf{C}}^e = a_0 \bar{\mathbf{M}}^e + a_1 \bar{\mathbf{K}}^e\end{split}
\end{equation*}
\end{description}\end{quote}


\subsection{beam2ds}
\label{\detokenize{beam_functions:beam2ds}}\begin{quote}\begin{description}
\sphinxlineitem{Purpose}
\sphinxAtStartPar
Compute section forces for a two dimensional beam element in dynamic analysis.

\begin{figure}[htbp]
\centering

\noindent\sphinxincludegraphics[width=0.700\linewidth]{{BEAM2S}.png}
\end{figure}

\sphinxlineitem{Syntax}
\begin{sphinxVerbatim}[commandchars=\\\{\}]
\PYG{n}{es}\PYG{+w}{ }\PYG{p}{=}\PYG{+w}{ }\PYG{n}{beam2ds}\PYG{p}{(}\PYG{n}{ex}\PYG{p}{,}\PYG{+w}{ }\PYG{n}{ey}\PYG{p}{,}\PYG{+w}{ }\PYG{n}{ep}\PYG{p}{,}\PYG{+w}{ }\PYG{n}{ed}\PYG{p}{,}\PYG{+w}{ }\PYG{n}{ev}\PYG{p}{,}\PYG{+w}{ }\PYG{n}{ea}\PYG{p}{)}
\PYG{p}{[}\PYG{n}{es}\PYG{p}{,}\PYG{+w}{ }\PYG{n}{edi}\PYG{p}{,}\PYG{+w}{ }\PYG{n}{eci}\PYG{p}{]}\PYG{+w}{ }\PYG{p}{=}\PYG{+w}{ }\PYG{n}{beam2gs}\PYG{p}{(}\PYG{n}{ex}\PYG{p}{,}\PYG{+w}{ }\PYG{n}{ey}\PYG{p}{,}\PYG{+w}{ }\PYG{n}{ep}\PYG{p}{,}\PYG{+w}{ }\PYG{n}{ed}\PYG{p}{,}\PYG{+w}{ }\PYG{n}{ev}\PYG{p}{,}\PYG{+w}{ }\PYG{n}{ea}\PYG{p}{,}\PYG{+w}{ }\PYG{n}{n}\PYG{p}{)}
\end{sphinxVerbatim}

\sphinxlineitem{Description}
\sphinxAtStartPar
\sphinxcode{\sphinxupquote{beam2ds}} computes the section forces at the ends of the dynamic beam element \sphinxcode{\sphinxupquote{beam2de}}.

\sphinxAtStartPar
The input variables \sphinxcode{\sphinxupquote{ex}}, \sphinxcode{\sphinxupquote{ey}}, and \sphinxcode{\sphinxupquote{ep}} are defined in \sphinxcode{\sphinxupquote{beam2de}}. The element displacements, velocities, and accelerations, stored in \sphinxcode{\sphinxupquote{ed}}, \sphinxcode{\sphinxupquote{ev}}, and \sphinxcode{\sphinxupquote{ea}} respectively, are obtained by the function \sphinxcode{\sphinxupquote{extract}}.

\sphinxAtStartPar
The output variable \sphinxcode{\sphinxupquote{es}} contains the section forces at the ends of the beam:
\begin{equation*}
\begin{split}es = \begin{bmatrix}
N_1 & V_1 & M_1 \\
N_2 & V_2 & M_2
\end{bmatrix}\end{split}
\end{equation*}
\sphinxlineitem{Theory}
\sphinxAtStartPar
The section forces at the ends of the beam are obtained from the element force vector:
\begin{equation*}
\begin{split}\bar{\mathbf{P}} =
\begin{bmatrix}
-N_1 & -V_1 & -M_1 & N_2 & V_2 & M_2
\end{bmatrix}^T\end{split}
\end{equation*}
\sphinxAtStartPar
computed according to:
\begin{equation*}
\begin{split}\bar{\mathbf{P}} =
    \bar{\mathbf{K}}^e \mathbf{G} \mathbf{a}^e
+ \bar{\mathbf{C}}^e \mathbf{G} \dot{\mathbf{a}}^e
+ \bar{\mathbf{M}}^e \mathbf{G} \ddot{\mathbf{a}}^e\end{split}
\end{equation*}
\sphinxAtStartPar
The matrices \(\bar{\mathbf{K}}^e\) and \(\mathbf{G}\) are described in \sphinxcode{\sphinxupquote{beam2e}}, and the matrices \(\bar{\mathbf{M}}^e\) and \(\bar{\mathbf{C}}^e\) are described in \sphinxcode{\sphinxupquote{beam2d}}.

\sphinxAtStartPar
The nodal displacements:
\begin{equation*}
\begin{split}\mathbf{a}^e = \begin{bmatrix}
u_1 & u_2 & u_3 & u_4 & u_5 & u_6
\end{bmatrix}^T\end{split}
\end{equation*}
\sphinxAtStartPar
shown in \sphinxcode{\sphinxupquote{beam2de}} also define the directions of the nodal velocities:
\begin{equation*}
\begin{split}\dot{\mathbf{a}}^e = \begin{bmatrix}
\dot{u}_1 & \dot{u}_2 & \dot{u}_3 & \dot{u}_4 & \dot{u}_5 & \dot{u}_6
\end{bmatrix}^T\end{split}
\end{equation*}
\sphinxAtStartPar
and the nodal accelerations:
\begin{equation*}
\begin{split}\ddot{\mathbf{a}}^e = \begin{bmatrix}
\ddot{u}_1 & \ddot{u}_2 & \ddot{u}_3 & \ddot{u}_4 & \ddot{u}_5 & \ddot{u}_6
\end{bmatrix}^T\end{split}
\end{equation*}
\sphinxAtStartPar
Note that the transposes of \(\mathbf{a}^e\), \(\dot{\mathbf{a}}^e\), and \(\ddot{\mathbf{a}}^e\) are stored in \sphinxcode{\sphinxupquote{ed}}, \sphinxcode{\sphinxupquote{ev}}, and \sphinxcode{\sphinxupquote{ea}} respectively.

\end{description}\end{quote}


\section{3D beam elements}
\label{\detokenize{beam_functions:id2}}

\subsection{beam3e}
\label{\detokenize{beam_functions:beam3e}}\begin{quote}\begin{description}
\sphinxlineitem{Purpose}
\sphinxAtStartPar
Compute element stiffness matrix for a three dimensional beam element.

\begin{figure}[htbp]
\centering

\noindent\sphinxincludegraphics[width=0.700\linewidth]{{BEAM3E}.png}
\end{figure}

\sphinxlineitem{Syntax}
\begin{sphinxVerbatim}[commandchars=\\\{\}]
\PYG{n}{Ke}\PYG{+w}{ }\PYG{p}{=}\PYG{+w}{ }\PYG{n}{beam3e}\PYG{p}{(}\PYG{n}{ex}\PYG{p}{,}\PYG{+w}{ }\PYG{n}{ey}\PYG{p}{,}\PYG{+w}{ }\PYG{n}{ez}\PYG{p}{,}\PYG{+w}{ }\PYG{n}{eo}\PYG{p}{,}\PYG{+w}{ }\PYG{n}{ep}\PYG{p}{)}
\PYG{p}{[}\PYG{n}{Ke}\PYG{p}{,}\PYG{+w}{ }\PYG{n}{fe}\PYG{p}{]}\PYG{+w}{ }\PYG{p}{=}\PYG{+w}{ }\PYG{n}{beam3e}\PYG{p}{(}\PYG{n}{ex}\PYG{p}{,}\PYG{+w}{ }\PYG{n}{ey}\PYG{p}{,}\PYG{+w}{ }\PYG{n}{ez}\PYG{p}{,}\PYG{+w}{ }\PYG{n}{eo}\PYG{p}{,}\PYG{+w}{ }\PYG{n}{ep}\PYG{p}{,}\PYG{+w}{ }\PYG{n+nb}{eq}\PYG{p}{)}
\end{sphinxVerbatim}

\sphinxlineitem{Description}
\sphinxAtStartPar
\sphinxcode{\sphinxupquote{beam3e}} provides the global element stiffness matrix \sphinxcode{\sphinxupquote{Ke}} for a three dimensional beam element.

\sphinxAtStartPar
The input variables
\begin{equation*}
\begin{split}\begin{aligned}
\mathrm{ex} &= [x_1 \;\; x_2] \\
\mathrm{ey} &= [y_1 \;\; y_2] \\
\mathrm{ez} &= [z_1 \;\; z_2]
\end{aligned}
\qquad
\mathrm{eo} = [x_{\bar{z}} \;\; y_{\bar{z}} \;\; z_{\bar{z}}]\end{split}
\end{equation*}
\sphinxAtStartPar
supply the element nodal coordinates \(x_1\), \(y_1\), etc. as well as the direction of the local beam coordinate system \((\bar{x}, \bar{y}, \bar{z})\). By giving a global vector \((x_{\bar{z}}, y_{\bar{z}}, z_{\bar{z}})\) parallel with the positive local \(\bar{z}\) axis of the beam, the local beam coordinate system is defined.

\sphinxAtStartPar
The variable
\begin{equation*}
\begin{split}\mathrm{ep} = [E \;\; G \;\; A \;\; I_{\bar{y}} \;\; I_{\bar{z}} \;\; K_v]\end{split}
\end{equation*}
\sphinxAtStartPar
supplies the modulus of elasticity \(E\), the shear modulus \(G\), the cross section area \(A\), the moment of inertia with respect to the \(\bar{y}\) axis \(I_{\bar{y}}\), the moment of inertia with respect to the \(\bar{z}\) axis \(I_{\bar{z}}\), and St. Venant torsion constant \(K_v\).

\sphinxAtStartPar
The element load vector \sphinxcode{\sphinxupquote{fe}} can also be computed if uniformly distributed loads are applied to the element. The optional input variable
\begin{equation*}
\begin{split}\mathrm{eq} = [q_{\bar{x}} \;\; q_{\bar{y}} \;\; q_{\bar{z}} \;\; q_{\bar{\omega}}]\end{split}
\end{equation*}
\sphinxAtStartPar
then contains the distributed loads. The positive directions of \(q_{\bar{x}}\), \(q_{\bar{y}}\), and \(q_{\bar{z}}\) follow the local beam coordinate system. The distributed torque \(q_{\bar{\omega}}\) is positive if directed in the local \(\bar{x}\)\sphinxhyphen{}direction, i.e. from local \(\bar{y}\) to local \(\bar{z}\). All the loads are per unit length.

\sphinxlineitem{Theory}
\sphinxAtStartPar
The element stiffness matrix \(\mathbf{K}^e\) is computed according to
\begin{equation*}
\begin{split}\mathbf{K}^e = \mathbf{G}^T \bar{\mathbf{K}}^e \mathbf{G}\end{split}
\end{equation*}
\sphinxAtStartPar
where
\begin{equation*}
\begin{split}\bar{\mathbf{K}}^e = \left[
\begin{array}{cccccccccccc}
\frac{D_{EA}}{L} & 0 & 0 & 0 & 0 & 0 & -\frac{D_{EA}}{L} & 0 & 0 & 0 & 0 & 0 \\
0 & \frac{12D_{EI_{\bar z}}}{L^3} & 0 & 0 & 0 & \frac{6D_{EI_{\bar z}}}{L^2} & 0 & -\frac{12D_{EI_{\bar z}}}{L^3} & 0 & 0 & 0 & \frac{6D_{EI_{\bar z}}}{L^2} \\
0 & 0 & \frac{12D_{EI_{\bar y}}}{L^3} & 0 & -\frac{6D_{EI_{\bar y}}}{L^2} & 0 & 0 & 0 & -\frac{12D_{EI_{\bar y}}}{L^3} & 0 & -\frac{6D_{EI_{\bar y}}}{L^2} & 0 \\
0 & 0 & 0 & \frac{D_{GK}}{L} & 0 & 0 & 0 & 0 & 0 & -\frac{D_{GK}}{L} & 0 & 0 \\
0 & 0 & -\frac{6D_{EI_{\bar y}}}{L^2} & 0 & \frac{4D_{EI_{\bar y}}}{L} & 0 & 0 & 0 & \frac{6D_{EI_{\bar y}}}{L^2} & 0 & \frac{2D_{EI_{\bar y}}}{L} & 0 \\
0 & \frac{6D_{EI_{\bar z}}}{L^2} & 0 & 0 & 0 & \frac{4D_{EI_{\bar z}}}{L} & 0 & -\frac{6D_{EI_{\bar z}}}{L^2} & 0 & 0 & 0 & \frac{2D_{EI_{\bar z}}}{L} \\
-\frac{D_{EA}}{L} & 0 & 0 & 0 & 0 & 0 & \frac{D_{EA}}{L} & 0 & 0 & 0 & 0 & 0 \\
0 & -\frac{12D_{EI_{\bar z}}}{L^3} & 0 & 0 & 0 & -\frac{6D_{EI_{\bar z}}}{L^2} & 0 & \frac{12D_{EI_{\bar z}}}{L^3} & 0 & 0 & 0 & -\frac{6D_{EI_{\bar z}}}{L^2} \\
0 & 0 & -\frac{12D_{EI_{\bar y}}}{L^3} & 0 & \frac{6D_{EI_{\bar y}}}{L^2} & 0 & 0 & 0 & \frac{12D_{EI_{\bar y}}}{L^3} & 0 & \frac{6D_{EI_{\bar y}}}{L^2} & 0 \\
0 & 0 & 0 & -\frac{D_{GK}}{L} & 0 & 0 & 0 & 0 & 0 & \frac{D_{GK}}{L} & 0 & 0 \\
0 & 0 & -\frac{6D_{EI_{\bar y}}}{L^2} & 0 & \frac{2D_{EI_{\bar y}}}{L} & 0 & 0 & 0 & \frac{6D_{EI_{\bar y}}}{L^2} & 0 & \frac{4D_{EI_{\bar y}}}{L} & 0 \\
0 & \frac{6D_{EI_{\bar z}}}{L^2} & 0 & 0 & 0 & \frac{2D_{EI_{\bar z}}}{L} & 0 & -\frac{6D_{EI_{\bar z}}}{L^2} & 0 & 0 & 0 & \frac{4D_{EI_{\bar z}}}{L}
\end{array}
\right]\end{split}
\end{equation*}
\sphinxAtStartPar
and
\begin{equation*}
\begin{split}\mathbf{G} = \left[
\begin{array}{cccccccccccc}
n_{x\bar{x}} & n_{y\bar{x}} & n_{z\bar{x}} & 0 & 0 & 0 & 0 & 0 & 0 & 0 & 0 & 0 \\
n_{x\bar{y}} & n_{y\bar{y}} & n_{z\bar{y}} & 0 & 0 & 0 & 0 & 0 & 0 & 0 & 0 & 0 \\
n_{x\bar{z}} & n_{y\bar{z}} & n_{z\bar{z}} & 0 & 0 & 0 & 0 & 0 & 0 & 0 & 0 & 0 \\
0 & 0 & 0 & n_{x\bar{x}} & n_{y\bar{x}} & n_{z\bar{x}} & 0 & 0 & 0 & 0 & 0 & 0 \\
0 & 0 & 0 & n_{x\bar{y}} & n_{y\bar{y}} & n_{z\bar{y}} & 0 & 0 & 0 & 0 & 0 & 0 \\
0 & 0 & 0 & n_{x\bar{z}} & n_{y\bar{z}} & n_{z\bar{z}} & 0 & 0 & 0 & 0 & 0 & 0 \\
0 & 0 & 0 & 0 & 0 & 0 & n_{x\bar{x}} & n_{y\bar{x}} & n_{z\bar{x}} & 0 & 0 & 0 \\
0 & 0 & 0 & 0 & 0 & 0 & n_{x\bar{y}} & n_{y\bar{y}} & n_{z\bar{y}} & 0 & 0 & 0 \\
0 & 0 & 0 & 0 & 0 & 0 & n_{x\bar{z}} & n_{y\bar{z}} & n_{z\bar{z}} & 0 & 0 & 0 \\
0 & 0 & 0 & 0 & 0 & 0 & 0 & 0 & 0 & n_{x\bar{x}} & n_{y\bar{x}} & n_{z\bar{x}} \\
0 & 0 & 0 & 0 & 0 & 0 & 0 & 0 & 0 & n_{x\bar{y}} & n_{y\bar{y}} & n_{z\bar{y}} \\
0 & 0 & 0 & 0 & 0 & 0 & 0 & 0 & 0 & n_{x\bar{z}} & n_{y\bar{z}} & n_{z\bar{z}}
\end{array}
\right]\end{split}
\end{equation*}
\sphinxAtStartPar
where the axial stiffness \(D_{EA}\), the bending stiffness \(D_{EI_{\bar z}}\), the bending stiffness \(D_{EI_{\bar y}}\), and the St. Venant torsion stiffness \(D_{GK}\) are given by
\begin{equation*}
\begin{split}D_{EA} = EA; \quad D_{EI_{\bar z}} = EI_{\bar z}; \quad D_{EI_{\bar y}} = EI_{\bar y}; \quad D_{GK} = GK_v\end{split}
\end{equation*}
\sphinxAtStartPar
The length \(L\) is given by
\begin{equation*}
\begin{split}L = \sqrt{(x_2 - x_1)^2 + (y_2 - y_1)^2 + (z_2 - z_1)^2}\end{split}
\end{equation*}
\sphinxAtStartPar
The transformation matrix \(\mathbf{G}\) contains direction cosines computed as
\begin{equation*}
\begin{split}\begin{aligned}
n_{x\bar{x}} &= \frac{x_2 - x_1}{L} \qquad n_{y\bar{x}} = \frac{y_2 - y_1}{L} \qquad n_{z\bar{x}} = \frac{z_2 - z_1}{L} \\
n_{x\bar{z}} &= \frac{x_{\bar{z}}}{L_{\bar{z}}} \qquad n_{y\bar{z}} = \frac{y_{\bar{z}}}{L_{\bar{z}}} \qquad n_{z\bar{z}} = \frac{z_{\bar{z}}}{L_{\bar{z}}} \\
n_{x\bar{y}} &= n_{y\bar{z}} n_{z\bar{x}} - n_{z\bar{z}} n_{y\bar{x}} \\
n_{y\bar{y}} &= n_{z\bar{z}} n_{x\bar{x}} - n_{x\bar{z}} n_{z\bar{x}} \\
n_{z\bar{y}} &= n_{x\bar{z}} n_{y\bar{x}} - n_{y\bar{z}} n_{x\bar{x}}
\end{aligned}\end{split}
\end{equation*}
\sphinxAtStartPar
where
\begin{equation*}
\begin{split}L_{\bar{z}} = \sqrt{x_{\bar{z}}^2 + y_{\bar{z}}^2 + z_{\bar{z}}^2}\end{split}
\end{equation*}
\sphinxAtStartPar
The element load vector \(\mathbf{f}_l^e\), stored in \sphinxcode{\sphinxupquote{fe}}, is computed according to
\begin{equation*}
\begin{split}\mathbf{f}_l^e = \mathbf{G}^T \bar{\mathbf{f}}_l^e\end{split}
\end{equation*}
\sphinxAtStartPar
where
\begin{equation*}
\begin{split}\bar{\mathbf{f}}_l^e =
\begin{bmatrix}
\dfrac{q_{\bar{x}}L}{2} \\
\dfrac{q_{\bar{y}}L}{2} \\
\dfrac{q_{\bar{z}}L}{2} \\
\dfrac{q_{\bar{\omega}}L}{2} \\
-\dfrac{q_{\bar{z}}L^2}{12} \\
\dfrac{q_{\bar{y}}L^2}{12} \\
\dfrac{q_{\bar{x}}L}{2} \\
\dfrac{q_{\bar{y}}L}{2} \\
\dfrac{q_{\bar{z}}L}{2} \\
\dfrac{q_{\bar{\omega}}L}{2} \\
\dfrac{q_{\bar{z}}L^2}{12} \\
-\dfrac{q_{\bar{y}}L^2}{12}
\end{bmatrix}\end{split}
\end{equation*}
\end{description}\end{quote}


\subsection{beam3s}
\label{\detokenize{beam_functions:beam3s}}\begin{quote}\begin{description}
\sphinxlineitem{Purpose}
\sphinxAtStartPar
Compute section forces in a three dimensional beam element.

\begin{figure}[htbp]
\centering

\noindent\sphinxincludegraphics[width=0.700\linewidth]{{BEAM3S}.png}
\end{figure}

\sphinxlineitem{Syntax}
\begin{sphinxVerbatim}[commandchars=\\\{\}]
\PYG{p}{[}\PYG{n}{es}\PYG{p}{]}\PYG{p}{=}\PYG{n}{beam3s}\PYG{p}{(}\PYG{n}{ex}\PYG{p}{,}\PYG{n}{ey}\PYG{p}{,}\PYG{n}{ez}\PYG{p}{,}\PYG{n}{eo}\PYG{p}{,}\PYG{n}{ep}\PYG{p}{,}\PYG{n}{ed}\PYG{p}{)}
\PYG{p}{[}\PYG{n}{es}\PYG{p}{]}\PYG{p}{=}\PYG{n}{beam3s}\PYG{p}{(}\PYG{n}{ex}\PYG{p}{,}\PYG{n}{ey}\PYG{p}{,}\PYG{n}{ez}\PYG{p}{,}\PYG{n}{eo}\PYG{p}{,}\PYG{n}{ep}\PYG{p}{,}\PYG{n}{ed}\PYG{p}{,}\PYG{n+nb}{eq}\PYG{p}{)}
\PYG{p}{[}\PYG{n}{es}\PYG{p}{,}\PYG{n}{edi}\PYG{p}{]}\PYG{p}{=}\PYG{n}{beam3s}\PYG{p}{(}\PYG{n}{ex}\PYG{p}{,}\PYG{n}{ey}\PYG{p}{,}\PYG{n}{ez}\PYG{p}{,}\PYG{n}{eo}\PYG{p}{,}\PYG{n}{ep}\PYG{p}{,}\PYG{n}{ed}\PYG{p}{,}\PYG{n+nb}{eq}\PYG{p}{,}\PYG{n}{n}\PYG{p}{)}
\PYG{p}{[}\PYG{n}{es}\PYG{p}{,}\PYG{n}{edi}\PYG{p}{,}\PYG{n}{eci}\PYG{p}{]}\PYG{p}{=}\PYG{n}{beam3s}\PYG{p}{(}\PYG{n}{ex}\PYG{p}{,}\PYG{n}{ey}\PYG{p}{,}\PYG{n}{ez}\PYG{p}{,}\PYG{n}{eo}\PYG{p}{,}\PYG{n}{ep}\PYG{p}{,}\PYG{n}{ed}\PYG{p}{,}\PYG{n+nb}{eq}\PYG{p}{,}\PYG{n}{n}\PYG{p}{)}
\end{sphinxVerbatim}

\sphinxlineitem{Description}
\sphinxAtStartPar
\sphinxcode{\sphinxupquote{beam3s}} computes the section forces and displacements in local directions along the beam element \sphinxcode{\sphinxupquote{beam3e}}.

\sphinxAtStartPar
The input variables \sphinxcode{\sphinxupquote{ex}}, \sphinxcode{\sphinxupquote{ey}}, \sphinxcode{\sphinxupquote{ez}}, \sphinxcode{\sphinxupquote{eo}}, \sphinxcode{\sphinxupquote{ep}}, and \sphinxcode{\sphinxupquote{eq}} are defined in \sphinxcode{\sphinxupquote{beam3e}}.

\sphinxAtStartPar
The element displacements, stored in \sphinxcode{\sphinxupquote{ed}}, are obtained by the function \sphinxcode{\sphinxupquote{extract}}. If a distributed load is applied to the element, the variable \sphinxcode{\sphinxupquote{eq}} must be included. The number of evaluation points for section forces and displacements are determined by \sphinxcode{\sphinxupquote{n}}. If \sphinxcode{\sphinxupquote{n}} is omitted, only the ends of the beam are evaluated.

\sphinxAtStartPar
The output variables:
\begin{equation*}
\begin{split}\mathrm{es} =
\begin{bmatrix}
N(0) & V_{\bar{y}}(0)  & V_{\bar{z}}(0)  & T(0)  & M_{\bar{y}}(0) & M_{\bar{z}}(0) \\
N(\bar{x}_{2}) & V_{\bar{y}}(\bar{x}_{2}) & V_{\bar{z}}(\bar{x}_{2}) & T(\bar{x}_{2}) & M_{\bar{y}}(\bar{x}_{2}) & M_{\bar{z}}(\bar{x}_{2}) \\
\vdots & \vdots & \vdots & \vdots & \vdots & \vdots  \\
N(\bar{x}_{n-1}) & V_{\bar{y}}(\bar{x}_{n-1}) & V_{\bar{z}}(\bar{x}_{n-1}) & T(\bar{x}_{n-1}) & M_{\bar{y}}(\bar{x}_{n-1}) & M_{\bar{z}}(\bar{x}_{n-1}) \\
N(L) & V_{\bar{y}}(L) & V_{\bar{z}}(L) & T(\bar{x}_{n-1}) & M_{\bar{y}}(L) & M_{\bar{z}}(L)
\end{bmatrix}\end{split}
\end{equation*}\begin{equation*}
\begin{split}\mathrm{edi} =
\begin{bmatrix}
u(0) & v(0) & w(0)  & \varphi(0) \\
u(\bar{x}_{2}) & v(\bar{x}_{2}) & w(\bar{x}_{2})  & \varphi(\bar{x}_{2})    \\
\vdots & \vdots & \vdots & \vdots \\
u(\bar{x}_{n-1}) & v(\bar{x}_{n-1}) & w(\bar{x}_{n-1}) & \varphi(\bar{x}_{n-1})\\
u(L) & v(L) & w(L) & \varphi(L)
\end{bmatrix}
\qquad
\mathrm{eci} =
\begin{bmatrix}
0  \\
\bar x_{2} \\
\vdots   \\
\bar x_{n-1} \\
L
\end{bmatrix}\end{split}
\end{equation*}
\sphinxAtStartPar
contain the section forces, the displacements, and the evaluation points on the local \(\bar{x}\)\sphinxhyphen{}axis.
\(L\) is the length of the beam element.

\sphinxlineitem{Theory}
\sphinxAtStartPar
The nodal displacements in local coordinates are given by
\begin{equation*}
\begin{split}\mathbf{\bar{a}}^e=
\begin{bmatrix}
\bar{u}_1 \\ \bar{u}_2 \\ \bar{u}_3 \\ \bar{u}_4 \\ \bar{u}_5 \\
\bar{u}_6 \\ \bar{u}_7 \\ \bar{u}_8 \\ \bar{u}_9 \\ \bar{u}_{10} \\ \bar{u}_{11} \\
\bar{u}_{12}
\end{bmatrix}
= \mathbf{G} \mathbf{a}^e\end{split}
\end{equation*}
\sphinxAtStartPar
where \(\mathbf{G}\) is described in \sphinxcode{\sphinxupquote{beam3e}} and the transpose of \(\mathbf{a}^e\) is stored in \sphinxcode{\sphinxupquote{ed}}.

\sphinxAtStartPar
The displacements associated with bar action, beam action in the \(\bar{x}\bar{y}\)\sphinxhyphen{}plane, beam action in the \(\bar{x}\bar{z}\)\sphinxhyphen{}plane, and torsion are determined as
\begin{equation*}
\begin{split}\mathbf{\bar{a}}^e_{\text{bar}}=
\begin{bmatrix}
\bar{u}_1 \\
\bar{u}_7
\end{bmatrix};
\quad
\mathbf{\bar{a}}^e_{\text{beam},\bar{z}}=
\begin{bmatrix}
\bar{u}_2 \\ \bar{u}_6 \\ \bar{u}_8 \\ \bar{u}_{12}
\end{bmatrix};
\quad
\mathbf{\bar{a}}^e_{\text{beam},\bar{y}}=
\begin{bmatrix}
\bar{u}_3 \\ -\bar{u}_5 \\ \bar{u}_9 \\ -\bar{u}_{11}
\end{bmatrix};
\quad
\mathbf{\bar{a}}^e_{\text{torsion}}=
\begin{bmatrix}
\bar{u}_4 \\
\bar{u}_{10}
\end{bmatrix}\end{split}
\end{equation*}
\sphinxAtStartPar
The displacement \(u(\bar{x})\) and the normal force \(N(\bar{x})\) are computed from
\begin{equation*}
\begin{split}u(\bar{x}) = \mathbf{N}_{\text{bar}} \mathbf{\bar{a}}^e_{\text{bar}} + u_p(\bar{x})\end{split}
\end{equation*}\begin{equation*}
\begin{split}N(\bar{x}) = D_{EA} \mathbf{B}_{\text{bar}} \mathbf{\bar{a}}^e + N_p(\bar{x})\end{split}
\end{equation*}
\sphinxAtStartPar
where
\begin{equation*}
\begin{split}\mathbf{N}_{\text{bar}} = \begin{bmatrix} 1 & \bar{x} \end{bmatrix} \mathbf{C}^{-1}_{\text{bar}} = \begin{bmatrix} 1-\frac{\bar{x}}{L} & \frac{\bar{x}}{L} \end{bmatrix}\end{split}
\end{equation*}\begin{equation*}
\begin{split}\mathbf{B}_{\text{bar}} = \begin{bmatrix} 0 & 1 \end{bmatrix} \mathbf{C}^{-1}_{\text{bar}} = \begin{bmatrix} -\frac{1}{L} & \frac{1}{L} \end{bmatrix}\end{split}
\end{equation*}\begin{equation*}
\begin{split}u_p(\bar{x}) = -\frac{q_{\bar{x}}}{D_{EA}}\left(\frac{\bar{x}^2}{2}-\frac{L\bar{x}}{2}\right)\end{split}
\end{equation*}\begin{equation*}
\begin{split}N_p(\bar{x}) = -q_{\bar{x}}\left(\bar{x}-\frac{L}{2}\right)\end{split}
\end{equation*}
\sphinxAtStartPar
in which \(D_{EA}\), \(L\), and \(q_{\bar{x}}\) are defined in \sphinxcode{\sphinxupquote{beam3e}} and
\begin{equation*}
\begin{split}\mathbf{C}^{-1}_{\text{bar}} =
\begin{bmatrix}
1 & 0 \\
-\frac{1}{L} & \frac{1}{L}
\end{bmatrix}\end{split}
\end{equation*}
\sphinxAtStartPar
The displacement \(v(\bar{x})\), the bending moment \(M_{\bar{z}}(\bar{x})\) and the shear force \(V_{\bar{y}}(\bar{x})\) are computed from
\begin{equation*}
\begin{split}v(\bar{x}) = \mathbf{N}_{\text{beam}} \mathbf{\bar{a}}^e_{\text{beam},\bar{z}} + v_p(\bar{x})\end{split}
\end{equation*}\begin{equation*}
\begin{split}M_{\bar{z}}(\bar{x}) = D_{EI_{\bar{z}}} \mathbf{B}_{\text{beam}} \mathbf{\bar{a}}^e_{\text{beam},\bar{z}} + M_{\bar{z},p}(\bar{x})\end{split}
\end{equation*}\begin{equation*}
\begin{split}V_{\bar{y}}(\bar{x}) = -D_{EI_{\bar{z}}} \frac{d\mathbf{B}_{\text{beam}}}{dx} \mathbf{\bar{a}}^e_{\text{beam},\bar{z}} + V_{\bar{y},p}(\bar{x})\end{split}
\end{equation*}
\sphinxAtStartPar
where
\begin{equation*}
\begin{split}\mathbf{N}_{\text{beam}} = \begin{bmatrix} 1 & \bar{x} & \bar{x}^2 & \bar{x}^3 \end{bmatrix} \mathbf{C}^{-1}_{\text{beam}}\end{split}
\end{equation*}\begin{equation*}
\begin{split}\mathbf{B}_{\text{beam}} = \begin{bmatrix} 0 & 0 & 2 & 6\bar{x} \end{bmatrix} \mathbf{C}^{-1}_{\text{beam}}\end{split}
\end{equation*}\begin{equation*}
\begin{split}\frac{d\mathbf{B}_{\text{beam}}}{dx} = \begin{bmatrix} 0 & 0 & 0 & 6 \end{bmatrix} \mathbf{C}^{-1}_{\text{beam}}\end{split}
\end{equation*}\begin{equation*}
\begin{split}v_p(\bar{x}) = \frac{q_{\bar{y}}}{D_{EI_{\bar{z}}}}\left(\frac{\bar{x}^4}{24}-\frac{L \bar{x}^3}{12}+\frac{L^2 \bar{x}^2}{24}\right)\end{split}
\end{equation*}\begin{equation*}
\begin{split}M_{\bar{z},p}(\bar{x}) = q_{\bar{y}}\left(\frac{\bar{x}^2}{2}-\frac{L \bar{x}}{2}+\frac{L^2}{12}\right)\end{split}
\end{equation*}\begin{equation*}
\begin{split}V_{\bar{y},p}(\bar{x}) = -q_{\bar{y}}\left(\bar{x}-\frac{L}{2}\right)\end{split}
\end{equation*}
\sphinxAtStartPar
in which \(D_{EI_{\bar{z}}}\), \(L\), and \(q_{\bar{y}}\) are defined in \sphinxcode{\sphinxupquote{beam3e}} and
\begin{equation*}
\begin{split}\mathbf{C}^{-1}_{\text{beam}} =
\begin{bmatrix}
1 & 0 & 0 & 0 \\
0 & 1 & 0 & 0 \\
-\frac{3}{L^2} & -\frac{2}{L} & \frac{3}{L^2} & -\frac{1}{L} \\
\frac{2}{L^3} & \frac{1}{L^2} & -\frac{2}{L^3} & \frac{1}{L^2}
\end{bmatrix}\end{split}
\end{equation*}
\sphinxAtStartPar
The displacement \(w(\bar{x})\), the bending moment \(M_{\bar{y}}(\bar{x})\) and the shear force \(V_{\bar{z}}(\bar{x})\) are computed from
\begin{equation*}
\begin{split}w(\bar{x}) = \mathbf{N}_{\text{beam}} \mathbf{\bar{a}}^e_{\text{beam},\bar{y}} + w_p(\bar{x})\end{split}
\end{equation*}\begin{equation*}
\begin{split}M_{\bar{y}}(\bar{x}) = -D_{EI_{\bar{y}}} \mathbf{B}_{\text{beam}} \mathbf{\bar{a}}^e_{\text{beam},\bar{y}} + M_{\bar{y},p}(\bar{x})\end{split}
\end{equation*}\begin{equation*}
\begin{split}V_{\bar{z}}(\bar{x}) = -D_{EI_{\bar{y}}} \frac{d\mathbf{B}_{\text{beam}}}{dx} \mathbf{\bar{a}}^e_{\text{beam},\bar{y}} + V_{\bar{z},p}(\bar{x})\end{split}
\end{equation*}
\sphinxAtStartPar
where
\begin{equation*}
\begin{split}w_p(\bar{x}) = \frac{q_{\bar{z}}}{D_{EI_{\bar{y}}}}\left(\frac{\bar{x}^4}{24}-\frac{L \bar{x}^3}{12}+\frac{L^2 \bar{x}^2}{24}\right)\end{split}
\end{equation*}\begin{equation*}
\begin{split}M_{\bar{y},p}(\bar{x}) = -q_{\bar{z}}\left(\frac{\bar{x}^2}{2}-\frac{L \bar{x}}{2}+\frac{L^2}{12}\right)\end{split}
\end{equation*}\begin{equation*}
\begin{split}V_{\bar{z},p}(\bar{x}) = -q_{\bar{z}}\left(\bar{x}-\frac{L}{2}\right)\end{split}
\end{equation*}
\sphinxAtStartPar
in which \(D_{EI_{\bar{y}}}\), \(L\), and \(q_{\bar{z}}\) are defined in \sphinxcode{\sphinxupquote{beam3e}} and \(\mathbf{N}_{\text{beam}}\), \(\mathbf{B}_{\text{beam}}\), and \(\frac{d\mathbf{B}_{\text{beam}}}{dx}\) are given above.

\sphinxAtStartPar
The displacement \(\varphi(\bar{x})\) and the torque \(T(\bar{x})\) are computed from
\begin{equation*}
\begin{split}\varphi(\bar{x}) = \mathbf{N}_{\text{torsion}} \mathbf{\bar{a}}^e_{\text{torsion}} + \varphi_p(\bar{x})\end{split}
\end{equation*}\begin{equation*}
\begin{split}T(\bar{x}) = D_{GK} \mathbf{B}_{\text{torsion}} \mathbf{\bar{a}}^e + T_p(\bar{x})\end{split}
\end{equation*}
\sphinxAtStartPar
where
\begin{equation*}
\begin{split}\mathbf{N}_{\text{torsion}} = \mathbf{N}_{\text{bar}}\end{split}
\end{equation*}\begin{equation*}
\begin{split}\mathbf{B}_{\text{torsion}} = \mathbf{B}_{\text{bar}}\end{split}
\end{equation*}\begin{equation*}
\begin{split}\varphi_p(\bar{x}) = -\frac{q_{\omega}}{D_{GK}}\left(\frac{\bar{x}^2}{2}-\frac{L\bar{x}}{2}\right)\end{split}
\end{equation*}\begin{equation*}
\begin{split}T_p(\bar{x}) = -q_{\omega}\left(\bar{x}-\frac{L}{2}\right)\end{split}
\end{equation*}
\sphinxAtStartPar
in which \(D_{GK}\), \(L\), and \(q_{\omega}\) are defined in \sphinxcode{\sphinxupquote{beam3e}}.

\end{description}\end{quote}

\sphinxstepscope


\chapter{Plate element functions}
\label{\detokenize{plate_functions:plate-element-functions}}\label{\detokenize{plate_functions::doc}}
\sphinxAtStartPar
Only one plate element is currently available, a rectangular 12 dof element.
The element presumes a linear elastic material which can be isotropic or anisotropic.


\section{platre}
\label{\detokenize{plate_functions:platre}}\begin{quote}\begin{description}
\sphinxlineitem{Purpose}
\sphinxAtStartPar
Compute element stiffness matrix for a rectangular plate element.

\begin{figure}[htbp]
\centering

\noindent\sphinxincludegraphics[width=0.700\linewidth]{{PLATRE}.png}
\end{figure}

\sphinxlineitem{Syntax}
\begin{sphinxVerbatim}[commandchars=\\\{\}]
\PYG{n}{Ke}\PYG{+w}{ }\PYG{p}{=}\PYG{+w}{ }\PYG{n}{platre}\PYG{p}{(}\PYG{n}{ex}\PYG{p}{,}\PYG{+w}{ }\PYG{n}{ey}\PYG{p}{,}\PYG{+w}{ }\PYG{n}{ep}\PYG{p}{,}\PYG{+w}{ }\PYG{n}{D}\PYG{p}{)}
\PYG{p}{[}\PYG{n}{Ke}\PYG{p}{,}\PYG{+w}{ }\PYG{n}{fe}\PYG{p}{]}\PYG{+w}{ }\PYG{p}{=}\PYG{+w}{ }\PYG{n}{platre}\PYG{p}{(}\PYG{n}{ex}\PYG{p}{,}\PYG{+w}{ }\PYG{n}{ey}\PYG{p}{,}\PYG{+w}{ }\PYG{n}{ep}\PYG{p}{,}\PYG{+w}{ }\PYG{n}{D}\PYG{p}{,}\PYG{+w}{ }\PYG{n+nb}{eq}\PYG{p}{)}
\end{sphinxVerbatim}

\sphinxlineitem{Description}
\sphinxAtStartPar
\sphinxcode{\sphinxupquote{platre}} provides an element stiffness matrix \(Ke\), and an element load
vector \(fe\), for a rectangular plate element. This element can only be used if the element edges are parallel
to the coordinate axes.

\sphinxAtStartPar
The element nodal coordinates \(x_1, y_1, x_2\) etc. are supplied to the function
by \(\mathbf{ex}\) and \(\mathbf{ey}\), the element thickness \(t\)
by \(\mathbf{ep}\), and the material properties by the
constitutive matrix \(D\). Any arbitrary \(\mathbf{D}\)\sphinxhyphen{}matrix with
dimensions \((3 \times 3)\) and valid for plane stress may be given.
For an isotropic elastic material the constitutive matrix can be formed
by the function \sphinxcode{\sphinxupquote{hooke}} (see Section {\hyperref[\detokenize{material_functions:material-functions}]{\sphinxcrossref{\DUrole{std}{\DUrole{std-ref}{Material functions}}}}}).
\begin{equation*}
\begin{split}\begin{array}{l}
\mathbf{ex} = [\, x_1 \;\; x_2 \;\; x_3 \;\; x_4 \,] \\
\mathbf{ey} = [\, y_1 \;\; y_2 \;\; y_3 \;\; y_4 \,]
\end{array}
\qquad
\mathbf{ep} = [\, t \,]
\qquad
\mathbf{D} = \begin{bmatrix}
    D_{11} & D_{12} & D_{13} \\
    D_{21} & D_{22} & D_{23} \\
    D_{31} & D_{32} & D_{33}
\end{bmatrix}\end{split}
\end{equation*}
\sphinxAtStartPar
If a uniformly distributed load is applied to the element,
the element load vector \(fe\) is computed. The input variable
\begin{equation*}
\begin{split}\mathbf{eq} = [\, q_z \,]\end{split}
\end{equation*}
\sphinxAtStartPar
then contains the load \(q_z\) per unit area in the \(z\)\sphinxhyphen{}direction.

\sphinxlineitem{Theory}
\sphinxAtStartPar
The element stiffness matrix \(\mathbf{K}^e\) and the
element load vector \(\mathbf{f}_l^e\), stored in \(Ke\) and \(fe\)
respectively, are computed according to
\begin{equation*}
\begin{split}\mathbf{K}^e = (\mathbf{C}^{-1})^T \int_A \bar{\mathbf{B}}^T
\tilde{\mathbf{D}} \, \bar{\mathbf{B}} \, dA \, \mathbf{C}^{-1}\end{split}
\end{equation*}\begin{equation*}
\begin{split}\mathbf{f}_l^e = (\mathbf{C}^{-1})^T \int_A \bar{\mathbf{N}}^T q_z \, dA\end{split}
\end{equation*}
\sphinxAtStartPar
where the constitutive matrix
\begin{equation*}
\begin{split}\tilde{\mathbf{D}} = \frac{t^3}{12} \mathbf{D}\end{split}
\end{equation*}
\sphinxAtStartPar
and where \(\mathbf{D}\) is defined by \(D\).

\sphinxAtStartPar
The evaluation of the integrals for the rectangular plate element is based
on the displacement approximation \(w(x, y)\) and is expressed
in terms of the nodal variables \(u_1, u_2, \ldots, u_{12}\) as
\begin{equation*}
\begin{split}w(x, y) = \mathbf{N}^e \mathbf{a}^e = \bar{\mathbf{N}} \mathbf{C}^{-1} \mathbf{a}^e\end{split}
\end{equation*}
\sphinxAtStartPar
where
\begin{equation*}
\begin{split}\bar{\mathbf{N}} = [\, 1 \;\; x \;\; y \;\; x^2 \;\; x y \;\; y^2 \;\; x^3 \;\; x^2 y \;\; x y^2 \;\; y^3 \;\; x^3 y \;\; x y^3 \,]\end{split}
\end{equation*}\begin{equation*}
\begin{split}\mathbf{C} = \left[
\begin{array}{cccccccccccc}
1 & -a & -b & a^2 & ab & b^2 & -a^3 & -a^2b & -ab^2 & -b^3 & a^3b & ab^3 \\
0 & 0 & 1 & 0 & -a & -2b & 0 & a^2 & 2ab & 3b^2 & -a^3 & -3ab^2 \\
0 & -1 & 0 & 2a & b & 0 & -3a^2 & -2ab & -b^2 & 0 & 3a^2b & b^3 \\
1 & a & -b & a^2 & -ab & b^2 & a^3 & -a^2b & ab^2 & -b^3 & -a^3b & -ab^3 \\
0 & 0 & 1 & 0 & a & -2b & 0 & a^2 & -2ab & 3b^2 & a^3 & 3ab^2 \\
0 & -1 & 0 & -2a & b & 0 & -3a^2 & 2ab & -b^2 & 0 & 3a^2b & b^3 \\
1 & a & b & a^2 & ab & b^2 & a^3 & a^2b & ab^2 & b^3 & a^3b & ab^3 \\
0 & 0 & 1 & 0 & a & 2b & 0 & a^2 & 2ab & 3b^2 & a^3 & 3ab^2 \\
0 & -1 & 0 & -2a & -b & 0 & -3a^2 & -2ab & -b^2 & 0 & -3a^2b & -b^3 \\
1 & -a & b & a^2 & -ab & b^2 & -a^3 & a^2b & -ab^2 & b^3 & -a^3b & -ab^3 \\
0 & 0 & 1 & 0 & -a & 2b & 0 & a^2 & -2ab & 3b^2 & -a^3 & -3ab^2 \\
0 & -1 & 0 & 2a & -b & 0 & -3a^2 & 2ab & -b^2 & 0 & -3a^2b & -b^3
\end{array}
\right]\end{split}
\end{equation*}\begin{equation*}
\begin{split}\mathbf{a}^e = [\, u_1 \;\; u_2 \;\; \ldots \;\; u_{12} \,]^T\end{split}
\end{equation*}
\sphinxAtStartPar
and where
\begin{equation*}
\begin{split}a = \frac{1}{2}(x_3 - x_1) \qquad b = \frac{1}{2}(y_3 - y_1)\end{split}
\end{equation*}
\sphinxAtStartPar
The matrix \(\bar{\mathbf{B}}\) is obtained as
\begin{equation*}
\begin{split}\bar{\mathbf{B}} = \stackrel{*}{\nabla} \bar{\mathbf{N}} =
\left[
\begin{array}{cccccccccccc}
0 & 0 & 0 & 2 & 0 & 0 & 6x & 2y & 0 & 0 & 6xy & 0 \\
0 & 0 & 0 & 0 & 0 & 2 & 0 & 0 & 2x & 6y & 0 & 6xy \\
0 & 0 & 0 & 0 & 2 & 0 & 0 & 4x & 4y & 0 & 6x^2 & 6y^2 \\
\end{array}
\right]\end{split}
\end{equation*}
\sphinxAtStartPar
where
\begin{equation*}
\begin{split}\stackrel{*}{\nabla} = \begin{bmatrix}
\frac{\partial^2}{\partial x^2} \\
\frac{\partial^2}{\partial y^2} \\
2 \frac{\partial^2}{\partial x \partial y}
\end{bmatrix}\end{split}
\end{equation*}
\sphinxAtStartPar
Evaluation of the integrals for the rectangular plate element is
done analytically. Computation of the integrals for the element
load vector \(\mathbf{f}_l^e\) yields
\begin{equation*}
\begin{split}\mathbf{f}_l^e = q_z L_x L_y \begin{bmatrix}
\frac{1}{4} \\ \frac{L_y}{24} \\ -\frac{L_x}{24} \\ \frac{1}{4} \\ \frac{L_y}{24} \\ \frac{L_x}{24} \\
\frac{1}{4} \\ -\frac{L_y}{24} \\ \frac{L_x}{24} \\ \frac{1}{4} \\ -\frac{L_y}{24} \\ -\frac{L_x}{24}
\end{bmatrix}\end{split}
\end{equation*}
\sphinxAtStartPar
where
\begin{equation*}
\begin{split}L_x = x_3 - x_1 \qquad L_y = y_3 - y_1\end{split}
\end{equation*}
\end{description}\end{quote}


\section{platrs}
\label{\detokenize{plate_functions:platrs}}\begin{quote}\begin{description}
\sphinxlineitem{Purpose}
\sphinxAtStartPar
Compute section forces in a rectangular plate element.

\begin{figure}[htbp]
\centering

\noindent\sphinxincludegraphics[width=0.700\linewidth]{{PLATRS}.png}
\end{figure}

\sphinxlineitem{Syntax}
\begin{sphinxVerbatim}[commandchars=\\\{\}]
\PYG{p}{[}\PYG{n}{es}\PYG{p}{,}\PYG{n}{et}\PYG{p}{]}\PYG{p}{=}\PYG{n}{platrs}\PYG{p}{(}\PYG{n}{ex}\PYG{p}{,}\PYG{n}{ey}\PYG{p}{,}\PYG{n}{ep}\PYG{p}{,}\PYG{n}{D}\PYG{p}{,}\PYG{n}{ed}\PYG{p}{)}
\end{sphinxVerbatim}

\sphinxlineitem{Description}
\sphinxAtStartPar
\(\mathtt{platrs}\) computes the section forces \(\mathtt{es}\) and the curvature matrix \(\mathtt{et}\) in a rectangular plate element.
The section forces and the curvatures are computed at the center of the element.

\sphinxAtStartPar
The input variables \(\mathbf{ex}\), \(\mathbf{ey}\), \(\mathbf{ep}\) and \(\mathbf{D}\) are defined in \(\mathtt{platre}\).
The vector \(\mathbf{ed}\) contains the nodal displacements \(\mathbf{a}^e\) of the element and is obtained by the function \(\mathtt{extract}\) as
\begin{equation*}
\begin{split}\mathbf{ed} = (\mathbf{a}^e)^T = [\,u_1\;\; u_2\;\; \ldots \;\; u_{12}\;]\end{split}
\end{equation*}
\sphinxAtStartPar
The output variables
\begin{equation*}
\begin{split}\mathtt{es} = \left[\,\mathbf{M}^T\; \mathbf{V}^T\,\right] = \left[\, M_{xx}\; M_{yy}\; M_{xy}\; V_{xz}\; V_{yz}\; \right]\end{split}
\end{equation*}\begin{equation*}
\begin{split}\mathtt{et} = \boldsymbol{\kappa}^T = \left[\, \kappa_{xx}\; \kappa_{yy}\; \kappa_{xy}\; \right]\end{split}
\end{equation*}
\sphinxAtStartPar
contain the section forces and curvatures in global directions.

\sphinxlineitem{Theory}
\sphinxAtStartPar
The curvatures and the section forces are computed according to
\begin{equation*}
\begin{split}\boldsymbol{\kappa} = \left[
\begin{array}{c}
\kappa_{xx} \\
\kappa_{yy} \\
\kappa_{xy}
\end{array}
\right] =
\bar{\mathbf{B}}\,\mathbf{C}^{-1}\,\mathbf{a}^e\end{split}
\end{equation*}\begin{equation*}
\begin{split}\mathbf{M} =
\left[
\begin{array}{c}
M_{xx} \\
M_{yy} \\
M_{xy}
\end{array}
\right] =
\tilde{\mathbf{D}}\,\boldsymbol{\kappa}\end{split}
\end{equation*}\begin{equation*}
\begin{split}\mathbf{V} =
\left[
\begin{array}{c}
V_{xz} \\
V_{yz}
\end{array}
\right] =
\tilde{\nabla}\,\mathbf{M}\end{split}
\end{equation*}
\sphinxAtStartPar
where the matrices \(\tilde{\mathbf{D}}\), \(\bar{\mathbf{B}}\), \(\mathbf{C}\) and \(\mathbf{a}^e\) are described in \(\mathtt{platre}\), and where
\begin{equation*}
\begin{split}\tilde{\nabla} = \left[
\begin{array}{ccc}
\dfrac{\partial}{\partial x} & 0 & \dfrac{\partial}{\partial y} \\
0 & \dfrac{\partial}{\partial y} & \dfrac{\partial}{\partial x}
\end{array}
\right]\end{split}
\end{equation*}
\end{description}\end{quote}

\sphinxstepscope


\chapter{System functions}
\label{\detokenize{system_functions:system-functions}}\label{\detokenize{system_functions::doc}}
\sphinxAtStartPar
The group of system functions comprises functions for the setting up, solving,
and elimination of systems of equations. The functions are separated
in two groups:
\begin{itemize}
\item {} 
\sphinxAtStartPar
Static system functions

\item {} 
\sphinxAtStartPar
Dynamic system functions

\end{itemize}

\sphinxAtStartPar
Static system functions concern the linear system of equations
\begin{equation*}
\begin{split}\mathbf{K} \mathbf{a} = \mathbf{f}\end{split}
\end{equation*}
\sphinxAtStartPar
where \(\mathbf{K}\) is the global stiffness matrix and \(\mathbf{f}\) is the global load
vector. Often used static system functions are \sphinxcode{\sphinxupquote{assem}} and \sphinxcode{\sphinxupquote{solveq}}.
The function \sphinxcode{\sphinxupquote{assem}} assembles the global stiffness matrix and \sphinxcode{\sphinxupquote{solveq}}
computes the global displacement vector \(\mathbf{a}\) considering the
boundary conditions. It should be noted that \(\mathbf{K}\), \(\mathbf{f}\), and \(\mathbf{a}\)
also represent analogous quantities in systems other than structural mechanical systems.
For example, in a heat flow problem \(\mathbf{K}\) represents the conductivity matrix,
\(\mathbf{f}\) the heat flow, and \(\mathbf{a}\) the temperature.

\sphinxAtStartPar
Dynamic system functions are related to different aspects of linear
dynamic systems of coupled ordinary differential equations according to
\begin{equation*}
\begin{split}\mathbf{C} \dot{\mathbf{a}} + \mathbf{K} \mathbf{a} = \mathbf{f}\end{split}
\end{equation*}
\sphinxAtStartPar
for first\sphinxhyphen{}order systems and
\begin{equation*}
\begin{split}\mathbf{M} \ddot{\mathbf{a}} + \mathbf{C} \dot{\mathbf{a}} + \mathbf{K} \mathbf{a} = \mathbf{f}\end{split}
\end{equation*}
\sphinxAtStartPar
for second\sphinxhyphen{}order systems. First\sphinxhyphen{}order systems occur typically in transient
heat conduction and second\sphinxhyphen{}order systems occur in structural dynamics.


\section{Static system functions}
\label{\detokenize{system_functions:static-system-functions}}
\sphinxAtStartPar
The group of static system functions comprises functions for setting up and solving
the global system of equations. It also contains a function for eigenvalue
analysis, a function for static condensation, a function for extraction of element
displacements from the global displacement vector and a function for extraction of element
coordinates.


\subsection{assem}
\label{\detokenize{system_functions:assem}}\begin{quote}\begin{description}
\sphinxlineitem{Purpose}
\sphinxAtStartPar
Assemble element matrices.

\begin{figure}[htbp]
\centering

\noindent\sphinxincludegraphics[width=0.700\linewidth]{{ASSEM}.png}
\end{figure}

\sphinxlineitem{Syntax}
\begin{sphinxVerbatim}[commandchars=\\\{\}]
\PYG{n}{K}\PYG{+w}{ }\PYG{p}{=}\PYG{+w}{ }\PYG{n}{assem}\PYG{p}{(}\PYG{n}{edof}\PYG{p}{,}\PYG{+w}{ }\PYG{n}{K}\PYG{p}{,}\PYG{+w}{ }\PYG{n}{Ke}\PYG{p}{)}
\PYG{p}{[}\PYG{n}{K}\PYG{p}{,}\PYG{+w}{ }\PYG{n}{f}\PYG{p}{]}\PYG{+w}{ }\PYG{p}{=}\PYG{+w}{ }\PYG{n}{assem}\PYG{p}{(}\PYG{n}{edof}\PYG{p}{,}\PYG{+w}{ }\PYG{n}{K}\PYG{p}{,}\PYG{+w}{ }\PYG{n}{Ke}\PYG{p}{,}\PYG{+w}{ }\PYG{n}{f}\PYG{p}{,}\PYG{+w}{ }\PYG{n}{fe}\PYG{p}{)}
\end{sphinxVerbatim}

\sphinxlineitem{Description}
\sphinxAtStartPar
\sphinxcode{\sphinxupquote{assem}} adds the element stiffness matrix \(\mathbf{K}^e\), stored in \sphinxcode{\sphinxupquote{Ke}}, to the structure stiffness matrix \(\mathbf{K}\), stored in \sphinxcode{\sphinxupquote{K}}, according to the topology matrix \(\mathbf{edof}\).

\sphinxAtStartPar
The element topology matrix \(\mathbf{edof}\) is defined as
\begin{equation*}
\begin{split}\mathbf{edof} = [el \quad  \underbrace{dof_1\quad dof_2\quad\ldots \quad dof_{ned}}_{\text{global dof.}} ]\end{split}
\end{equation*}
\sphinxAtStartPar
where the first column contains the element number, and columns 2 to \((ned+1)\) contain the corresponding global degrees of freedom (\(ned\) = number of element degrees of freedom).

\sphinxAtStartPar
In the case where the matrix \(\mathbf{K}^e\) is identical for several elements, assembling of these can be carried out simultaneously. Each row in \(\mathbf{Edof}\) then represents one element, i.e. \(nel\) is the total number of considered elements.
\begin{equation*}
\begin{split}\mathbf{Edof} = \left. \left[
\begin{array}{c}
el_1 \\
el_2 \\
\vdots \\
el_{nel}
\end{array}
\quad
\begin{array}{cccccc}
dof_1 & dof_2 & \cdots & \cdots & \cdots & dof_{ned} \\
dof_1 & dof_2 & \cdots & \cdots & \cdots & dof_{ned} \\
\vdots & \vdots &  &  &  & \vdots \\
dof_1 & dof_2 & \cdots & \cdots & \cdots & dof_{ned}
\end{array}
\right] \right\} \text{one row for each element}\end{split}
\end{equation*}
\sphinxAtStartPar
If \(\mathbf{fe}\) and \(\mathbf{f}\) are given in the function, the element load vector \(\mathbf{f}^e\) is also added to the global load vector \(\mathbf{f}\).

\end{description}\end{quote}


\subsection{coordxtr}
\label{\detokenize{system_functions:coordxtr}}
\index{coordxtr@\spxentry{coordxtr}}\ignorespaces \begin{quote}\begin{description}
\sphinxlineitem{Purpose}
\sphinxAtStartPar
Extract element coordinates from a global coordinate matrix.

\begin{figure}[htbp]
\centering

\noindent\sphinxincludegraphics[width=0.700\linewidth]{{COORD}.png}
\end{figure}

\sphinxlineitem{Syntax}
\begin{sphinxVerbatim}[commandchars=\\\{\}]
\PYG{p}{[}\PYG{n}{Ex}\PYG{p}{,}\PYG{+w}{ }\PYG{n}{Ey}\PYG{p}{,}\PYG{+w}{ }\PYG{n}{Ez}\PYG{p}{]}\PYG{+w}{ }\PYG{p}{=}\PYG{+w}{ }\PYG{n}{coordxtr}\PYG{p}{(}\PYG{n}{Edof}\PYG{p}{,}\PYG{+w}{ }\PYG{n}{Coord}\PYG{p}{,}\PYG{+w}{ }\PYG{n}{Dof}\PYG{p}{,}\PYG{+w}{ }\PYG{n}{nen}\PYG{p}{)}
\end{sphinxVerbatim}

\sphinxlineitem{Description}
\sphinxAtStartPar
\sphinxcode{\sphinxupquote{coordxtr}} extracts element nodal coordinates from the global coordinate matrix \sphinxcode{\sphinxupquote{Coord}} for elements with equal numbers of element nodes and dof’s.

\sphinxAtStartPar
Input variables are the element topology matrix \sphinxcode{\sphinxupquote{Edof}}, defined in \sphinxcode{\sphinxupquote{assem}}, the global coordinate matrix \sphinxcode{\sphinxupquote{Coord}}, the global topology matrix \sphinxcode{\sphinxupquote{Dof}}, and the number of element nodes \sphinxcode{\sphinxupquote{nen}} in each element.
\begin{equation*}
\begin{split}\mathbf{Coord} = \begin{bmatrix}
x_1 & y_1 & [z_1] \\
x_2 & y_2 & [z_2] \\
x_3 & y_3 & [z_3] \\
\vdots & \vdots & \vdots \\
x_n & y_n & [z_n]
\end{bmatrix}
\qquad
\mathbf{Dof} = \begin{bmatrix}
k_1 & l_1 & \ldots & m_1 \\
k_2 & l_2 & \ldots & m_2 \\
k_3 & l_3 & \ldots & m_3 \\
\vdots & \vdots & \ldots & \vdots \\
k_n & l_n & \ldots & m_n
\end{bmatrix}
\qquad
nen = [\;nen\;]\end{split}
\end{equation*}
\sphinxAtStartPar
The nodal coordinates defined in row \sphinxstyleemphasis{i} of \sphinxcode{\sphinxupquote{Coord}} correspond to the degrees of freedom of row \sphinxstyleemphasis{i} in \sphinxcode{\sphinxupquote{Dof}}. The components \(k_i\), \(l_i\) and \(m_i\) define the degrees of freedom of node \sphinxstyleemphasis{i}, and \sphinxstyleemphasis{n} is the number of global nodes for the considered part of the FE\sphinxhyphen{}model.

\sphinxAtStartPar
The output variables \sphinxcode{\sphinxupquote{Ex}}, \sphinxcode{\sphinxupquote{Ey}}, and \sphinxcode{\sphinxupquote{Ez}} are matrices defined according to
\begin{equation*}
\begin{split}\mathbf{Ex} = \begin{bmatrix}
x_1^1 & x_2^1 & x_3^1 & \ldots & x_{nen}^1 \\
x_1^2 & x_2^2 & x_3^2 & \ldots & x_{nen}^2 \\
\vdots & \vdots & \vdots & \vdots & \vdots \\
x_1^{nel} & x_2^{nel} & x_3^{nel} & \ldots & x_{nen}^{nel}
\end{bmatrix}\end{split}
\end{equation*}
\sphinxAtStartPar
where row \sphinxstyleemphasis{i} gives the \sphinxstyleemphasis{x}\sphinxhyphen{}coordinates of the element defined in row \sphinxstyleemphasis{i} of \sphinxcode{\sphinxupquote{Edof}}, and where \sphinxstyleemphasis{nel} is the number of considered elements.

\sphinxAtStartPar
The element coordinate data extracted by the function \sphinxcode{\sphinxupquote{coordxtr}} can be used for plotting purposes and to create input data for the element stiffness functions.

\begin{sphinxadmonition}{note}{Note}

\sphinxAtStartPar
For the two dimensional beam element, the \sphinxcode{\sphinxupquote{extract}} function will extract six nodal displacements for each element given in the first column vector in \sphinxcode{\sphinxupquote{Edof}} and store them in the variable \sphinxcode{\sphinxupquote{ed}} as
\begin{equation*}
\begin{split}\mathbf{ed} = \begin{bmatrix}
u_1 & u_2 & u_3 & u_4 & u_5 & u_6 \\
u_1 & u_2 & u_3 & u_4 & u_5 & u_6 \\
\vdots & \vdots & \vdots & \vdots & \vdots & \vdots \\
u_1 & u_2 & u_3 & u_4 & u_5 & u_6
\end{bmatrix}\end{split}
\end{equation*}\end{sphinxadmonition}

\end{description}\end{quote}


\subsection{eigen}
\label{\detokenize{system_functions:eigen}}\begin{quote}\begin{description}
\sphinxlineitem{Purpose}
\sphinxAtStartPar
Solve the generalized eigenvalue problem.

\sphinxlineitem{Syntax}
\begin{sphinxVerbatim}[commandchars=\\\{\}]
\PYG{n}{L}\PYG{+w}{ }\PYG{p}{=}\PYG{+w}{ }\PYG{n}{eigen}\PYG{p}{(}\PYG{n}{K}\PYG{p}{,}\PYG{+w}{ }\PYG{n}{M}\PYG{p}{)}
\PYG{n}{L}\PYG{+w}{ }\PYG{p}{=}\PYG{+w}{ }\PYG{n}{eigen}\PYG{p}{(}\PYG{n}{K}\PYG{p}{,}\PYG{+w}{ }\PYG{n}{M}\PYG{p}{,}\PYG{+w}{ }\PYG{n}{b}\PYG{p}{)}
\PYG{p}{[}\PYG{n}{L}\PYG{p}{,}\PYG{+w}{ }\PYG{n}{X}\PYG{p}{]}\PYG{+w}{ }\PYG{p}{=}\PYG{+w}{ }\PYG{n}{eigen}\PYG{p}{(}\PYG{n}{K}\PYG{p}{,}\PYG{+w}{ }\PYG{n}{M}\PYG{p}{)}
\PYG{p}{[}\PYG{n}{L}\PYG{p}{,}\PYG{+w}{ }\PYG{n}{X}\PYG{p}{]}\PYG{+w}{ }\PYG{p}{=}\PYG{+w}{ }\PYG{n}{eigen}\PYG{p}{(}\PYG{n}{K}\PYG{p}{,}\PYG{+w}{ }\PYG{n}{M}\PYG{p}{,}\PYG{+w}{ }\PYG{n}{b}\PYG{p}{)}
\end{sphinxVerbatim}

\sphinxlineitem{Description}
\sphinxAtStartPar
\sphinxcode{\sphinxupquote{eigen}} solves the eigenvalue problem
\begin{equation*}
\begin{split}| \mathbf{K} - \lambda \mathbf{M} | = 0\end{split}
\end{equation*}
\sphinxAtStartPar
where \(\mathbf{K}\) and \(\mathbf{M}\) are square matrices. The eigenvalues \(\lambda\) are stored in the vector \(\mathbf{L}\) and the corresponding eigenvectors in the matrix \(\mathbf{X}\).

\sphinxAtStartPar
If certain rows and columns in matrices \(\mathbf{K}\) and \(\mathbf{M}\) are to be eliminated in computing the eigenvalues, \(\mathbf{b}\) must be given in the function. The rows (and columns) that are to be eliminated are described in the vector \(\mathbf{b}\) defined as
\begin{equation*}
\begin{split}\mathbf{b} = \begin{bmatrix}
dof_1 \\
dof_2 \\
\vdots \\
dof_{nb}
\end{bmatrix}\end{split}
\end{equation*}
\sphinxAtStartPar
The computed eigenvalues are given in order ranging from the smallest to the largest. The eigenvectors are normalized so that
\begin{equation*}
\begin{split}\mathbf{X}^T \mathbf{M} \mathbf{X} = \mathbf{I}\end{split}
\end{equation*}
\sphinxAtStartPar
where \(\mathbf{I}\) is the identity matrix.

\end{description}\end{quote}


\subsection{extract\_ed}
\label{\detokenize{system_functions:extract-ed}}\begin{quote}\begin{description}
\sphinxlineitem{Purpose}
\sphinxAtStartPar
Extract element nodal quantities from a global solution vector.

\begin{figure}[htbp]
\centering

\noindent\sphinxincludegraphics[width=0.700\linewidth]{{EXTRA}.png}
\end{figure}

\sphinxlineitem{Syntax}
\begin{sphinxVerbatim}[commandchars=\\\{\}]
\PYG{n}{ed}\PYG{+w}{ }\PYG{p}{=}\PYG{+w}{ }\PYG{n}{extract\PYGZus{}ed}\PYG{p}{(}\PYG{n}{edof}\PYG{p}{,}\PYG{+w}{ }\PYG{n}{a}\PYG{p}{)}
\end{sphinxVerbatim}

\sphinxlineitem{Description}
\sphinxAtStartPar
The \sphinxcode{\sphinxupquote{extract\_ed}} function extracts element displacements or corresponding quantities \(\mathbf{a}^e\) from the global solution vector \(\mathbf{a}\), stored in \sphinxcode{\sphinxupquote{a}}.

\sphinxAtStartPar
Input variables are the element topology matrix \(\mathbf{edof}\), defined in \sphinxcode{\sphinxupquote{assem}}, and the global solution vector \sphinxcode{\sphinxupquote{a}}.

\sphinxAtStartPar
The output variable
\begin{equation*}
\begin{split}\mathbf{ed} = (\mathbf{a}^e)^T\end{split}
\end{equation*}
\sphinxAtStartPar
contains the element displacement vector.

\sphinxAtStartPar
If \(\mathbf{Edof}\) contains more than one element, \(\mathbf{Ed}\) will be a matrix
\begin{equation*}
\begin{split}\mathbf{Ed} = \begin{bmatrix}
    (\mathbf{a}^e)_1^T \\
    (\mathbf{a}^e)_2^T \\
    \vdots \\
    (\mathbf{a}^e)_{nel}^T
\end{bmatrix}\end{split}
\end{equation*}
\sphinxAtStartPar
where row \sphinxstyleemphasis{i} gives the element displacements for the element defined in row \sphinxstyleemphasis{i} of \sphinxcode{\sphinxupquote{Edof}}, and \sphinxstyleemphasis{nel} is the total number of considered elements.

\sphinxlineitem{Example}
\sphinxAtStartPar
For the two\sphinxhyphen{}dimensional beam element, the \sphinxcode{\sphinxupquote{extract}} function will extract six nodal displacements for each element given in \(\mathbf{Edof}\), and create a matrix \sphinxcode{\sphinxupquote{Ed}} of size \sphinxstyleemphasis{(nel × 6)}.
\begin{equation*}
\begin{split}\mathbf{Ed} = \begin{bmatrix}
    u_1 & u_2 & u_3 & u_4 & u_5 & u_6 \\
    u_1 & u_2 & u_3 & u_4 & u_5 & u_6 \\
    \vdots & \vdots & \vdots & \vdots & \vdots & \vdots \\
    u_1 & u_2 & u_3 & u_4 & u_5 & u_6
\end{bmatrix}\end{split}
\end{equation*}
\end{description}\end{quote}


\subsection{red}
\label{\detokenize{system_functions:red}}\begin{quote}\begin{description}
\sphinxlineitem{Purpose}
\sphinxAtStartPar
Reduce the size of a square matrix by omitting rows and columns.

\sphinxlineitem{Syntax}
\begin{sphinxVerbatim}[commandchars=\\\{\}]
\PYG{n}{B}\PYG{+w}{ }\PYG{p}{=}\PYG{+w}{ }\PYG{n}{red}\PYG{p}{(}\PYG{n}{A}\PYG{p}{,}\PYG{+w}{ }\PYG{n}{b}\PYG{p}{)}
\PYG{p}{[}\PYG{n}{B}\PYG{p}{,}\PYG{+w}{ }\PYG{n}{b}\PYG{p}{]}\PYG{+w}{ }\PYG{p}{=}\PYG{+w}{ }\PYG{n}{red}\PYG{p}{(}\PYG{n}{A}\PYG{p}{,}\PYG{+w}{ }\PYG{n}{b}\PYG{p}{)}
\end{sphinxVerbatim}

\sphinxlineitem{Description}
\sphinxAtStartPar
\sphinxcode{\sphinxupquote{B = red(A, b)}} reduces the square matrix \sphinxcode{\sphinxupquote{A}} to a smaller matrix \sphinxcode{\sphinxupquote{B}} by omitting rows and columns of \sphinxcode{\sphinxupquote{A}}. The indices for rows and columns to be omitted are specified by the column vector \sphinxcode{\sphinxupquote{b}}.

\sphinxlineitem{Example}
\sphinxAtStartPar
Assume that the matrix \sphinxcode{\sphinxupquote{A}} is defined as
\begin{equation*}
\begin{split}A = \begin{bmatrix}
1 & 2 & 3 & 4 \\
5 & 6 & 7 & 8 \\
9 & 10 & 11 & 12 \\
13 & 14 & 15 & 16
\end{bmatrix}\end{split}
\end{equation*}
\sphinxAtStartPar
and \sphinxcode{\sphinxupquote{b}} as
\begin{equation*}
\begin{split}b = \begin{bmatrix}
2 \\
4
\end{bmatrix}\end{split}
\end{equation*}
\sphinxAtStartPar
The statement \sphinxcode{\sphinxupquote{B = red(A, b)}} results in the matrix
\begin{equation*}
\begin{split}B = \begin{bmatrix}
1 & 3 \\
9 & 11
\end{bmatrix}\end{split}
\end{equation*}
\end{description}\end{quote}


\subsection{solveq}
\label{\detokenize{system_functions:solveq}}\begin{quote}\begin{description}
\sphinxlineitem{Purpose}
\sphinxAtStartPar
Solve equation system.

\sphinxlineitem{Syntax}
\begin{sphinxVerbatim}[commandchars=\\\{\}]
\PYG{n}{a}\PYG{+w}{ }\PYG{p}{=}\PYG{+w}{ }\PYG{n}{solveq}\PYG{p}{(}\PYG{n}{K}\PYG{p}{,}\PYG{+w}{ }\PYG{n}{f}\PYG{p}{)}
\PYG{n}{a}\PYG{+w}{ }\PYG{p}{=}\PYG{+w}{ }\PYG{n}{solveq}\PYG{p}{(}\PYG{n}{K}\PYG{p}{,}\PYG{+w}{ }\PYG{n}{f}\PYG{p}{,}\PYG{+w}{ }\PYG{n}{bc}\PYG{p}{)}
\PYG{p}{[}\PYG{n}{a}\PYG{p}{,}\PYG{+w}{ }\PYG{n}{r}\PYG{p}{]}\PYG{+w}{ }\PYG{p}{=}\PYG{+w}{ }\PYG{n}{solveq}\PYG{p}{(}\PYG{n}{K}\PYG{p}{,}\PYG{+w}{ }\PYG{n}{f}\PYG{p}{,}\PYG{+w}{ }\PYG{n}{bc}\PYG{p}{)}
\end{sphinxVerbatim}

\sphinxlineitem{Description}
\sphinxAtStartPar
The function \sphinxcode{\sphinxupquote{solveq}} solves the equation system
\begin{equation*}
\begin{split}\mathbf{K}\;\mathbf{a} = \mathbf{f}\end{split}
\end{equation*}
\sphinxAtStartPar
where \(\mathbf{K}\) is a matrix and \(\mathbf{a}\) and \(\mathbf{f}\) are vectors.

\sphinxAtStartPar
The matrix \(\mathbf{K}\) and the vector \(\mathbf{f}\) must be predefined. The solution of the system of equations is stored in a vector \(\mathbf{a}\) which is created by the function.

\sphinxAtStartPar
If some values of \(\mathbf{a}\) are to be prescribed, the row number and the corresponding values are given in the boundary condition matrix
\begin{equation*}
\begin{split}\mathbf{bc} = \left[
\begin{array}{c}
dof_1 \\
dof_2 \\
\vdots \\
dof_{nbc}
\end{array}
\quad
\begin{array}{c}
u_1 \\
u_2 \\
\vdots \\
u_{nbc}
\end{array}
\right]\end{split}
\end{equation*}
\sphinxAtStartPar
where the first column contains the row numbers and the second column the corresponding prescribed values.

\sphinxAtStartPar
If \sphinxcode{\sphinxupquote{r}} is given in the function, support forces are computed according to
\begin{equation*}
\begin{split}\mathbf{r} = \mathbf{K}\;\mathbf{a} - \mathbf{f}\end{split}
\end{equation*}
\end{description}\end{quote}


\subsection{statcon}
\label{\detokenize{system_functions:statcon}}\begin{quote}\begin{description}
\sphinxlineitem{Purpose}
\sphinxAtStartPar
Reduce system of equations by static condensation.

\sphinxlineitem{Syntax}
\begin{sphinxVerbatim}[commandchars=\\\{\}]
\PYG{p}{[}\PYG{n}{K1}\PYG{p}{,}\PYG{+w}{ }\PYG{n}{f1}\PYG{p}{]}\PYG{+w}{ }\PYG{p}{=}\PYG{+w}{ }\PYG{n}{statcon}\PYG{p}{(}\PYG{n}{K}\PYG{p}{,}\PYG{+w}{ }\PYG{n}{f}\PYG{p}{)}
\PYG{p}{[}\PYG{n}{K1}\PYG{p}{,}\PYG{+w}{ }\PYG{n}{f1}\PYG{p}{]}\PYG{+w}{ }\PYG{p}{=}\PYG{+w}{ }\PYG{n}{statcon}\PYG{p}{(}\PYG{n}{K}\PYG{p}{,}\PYG{+w}{ }\PYG{n}{f}\PYG{p}{,}\PYG{+w}{ }\PYG{n}{b}\PYG{p}{)}
\PYG{p}{[}\PYG{n}{K1}\PYG{p}{,}\PYG{+w}{ }\PYG{n}{f1}\PYG{p}{]}\PYG{+w}{ }\PYG{p}{=}\PYG{+w}{ }\PYG{n}{statcon}\PYG{p}{(}\PYG{n}{K}\PYG{p}{,}\PYG{+w}{ }\PYG{n}{f}\PYG{p}{,}\PYG{+w}{ }\PYG{n}{b}\PYG{p}{,}\PYG{+w}{ }\PYG{n}{bc}\PYG{p}{)}
\end{sphinxVerbatim}

\sphinxlineitem{Description}
\sphinxAtStartPar
\sphinxcode{\sphinxupquote{statcon}} reduces a system of equations
\begin{equation*}
\begin{split}\mathbf{K}\;\mathbf{a} = \mathbf{f}\end{split}
\end{equation*}
\sphinxAtStartPar
by static condensation.

\sphinxAtStartPar
The degrees of freedom to be eliminated are supplied to the function by the vector
\begin{equation*}
\begin{split}\mathbf{b} = \begin{bmatrix}
dof_1 \\
dof_2 \\
\vdots \\
dof_{nb}
\end{bmatrix}\end{split}
\end{equation*}
\sphinxAtStartPar
where each row in \sphinxcode{\sphinxupquote{b}} contains one degree of freedom to be eliminated.

\sphinxAtStartPar
The elimination gives the reduced system of equations
\begin{equation*}
\begin{split}\mathbf{K}_1\;\mathbf{a}_1 = \mathbf{f}_1\end{split}
\end{equation*}
\sphinxAtStartPar
where \(\mathbf{K}_1\) and \(\mathbf{f}_1\) are stored in \sphinxcode{\sphinxupquote{K1}} and \sphinxcode{\sphinxupquote{f1}} respectively.

\end{description}\end{quote}


\section{Dynamic system functions}
\label{\detokenize{system_functions:dynamic-system-functions}}
\sphinxAtStartPar
The group of system functions comprises functions for solving linear dynamic systems
by time stepping or modal analysis, functions for frequency domain analysis, etc.


\subsection{dyna2}
\label{\detokenize{system_functions:dyna2}}
\sphinxAtStartPar
Compute the dynamic solution to a set of uncoupled second\sphinxhyphen{}order differential equations.
\begin{quote}\begin{description}
\sphinxlineitem{Syntax}
\begin{sphinxVerbatim}[commandchars=\\\{\}]
\PYG{n}{X}\PYG{+w}{ }\PYG{p}{=}\PYG{+w}{ }\PYG{n}{dyna2}\PYG{p}{(}\PYG{n}{w2}\PYG{p}{,}\PYG{+w}{ }\PYG{n}{xi}\PYG{p}{,}\PYG{+w}{ }\PYG{n}{f}\PYG{p}{,}\PYG{+w}{ }\PYG{n}{g}\PYG{p}{,}\PYG{+w}{ }\PYG{n}{dt}\PYG{p}{)}
\end{sphinxVerbatim}

\sphinxlineitem{Description}
\sphinxAtStartPar
\sphinxcode{\sphinxupquote{dyna2}} computes the solution to the set
\begin{equation*}
\begin{split}\ddot{x}_i + 2 \xi_i \omega_i \dot{x}_i + \omega^{2}_i x_i = f_i g(t), \qquad i=1,\ldots,m\end{split}
\end{equation*}
\sphinxAtStartPar
of differential equations, where \(g(t)\) is a piecewise linear time function.

\sphinxAtStartPar
The vectors \sphinxcode{\sphinxupquote{w2}}, \sphinxcode{\sphinxupquote{xi}}, and \sphinxcode{\sphinxupquote{f}} contain the squared circular frequencies \(\omega_i^2\), the damping ratios \(\xi_i\), and the applied forces \(f_i\), respectively.

\sphinxAtStartPar
The vector \sphinxcode{\sphinxupquote{g}} defines the load function in terms of straight line segments between equally spaced points in time. This function may have been formed by the command \sphinxcode{\sphinxupquote{gfunc}}.

\sphinxAtStartPar
The dynamic solution is computed at equal time increments defined by \sphinxcode{\sphinxupquote{dt}}. Including the initial zero vector as the first column vector, the result is stored in the \(m \times n\) matrix \sphinxcode{\sphinxupquote{X}}, where \(n-1\) is the number of time steps.

\end{description}\end{quote}

\begin{sphinxadmonition}{note}{Note}

\sphinxAtStartPar
The accuracy of the solution is \sphinxstyleemphasis{not} a function of the output time increment \sphinxcode{\sphinxupquote{dt}}, since the command produces the exact solution for straight line segments in the loading time function.
\end{sphinxadmonition}
\begin{quote}\begin{description}
\sphinxlineitem{See also}
\sphinxAtStartPar
\DUrole{xref}{\DUrole{std}{\DUrole{std-ref}{gfunc}}}

\end{description}\end{quote}


\subsection{dyna2f}
\label{\detokenize{system_functions:dyna2f}}\begin{quote}\begin{description}
\sphinxlineitem{Purpose}
\sphinxAtStartPar
Compute the dynamic solution to a set of uncoupled second\sphinxhyphen{}order differential equations.

\sphinxlineitem{Syntax}
\begin{sphinxVerbatim}[commandchars=\\\{\}]
\PYG{n}{Y}\PYG{+w}{ }\PYG{p}{=}\PYG{+w}{ }\PYG{n}{dyna2f}\PYG{p}{(}\PYG{n}{w2}\PYG{p}{,}\PYG{+w}{ }\PYG{n}{xi}\PYG{p}{,}\PYG{+w}{ }\PYG{n}{f}\PYG{p}{,}\PYG{+w}{ }\PYG{n}{p}\PYG{p}{,}\PYG{+w}{ }\PYG{n}{dt}\PYG{p}{)}
\end{sphinxVerbatim}

\sphinxlineitem{Description}
\sphinxAtStartPar
\sphinxcode{\sphinxupquote{dyna2f}} computes the solution to the set of differential equations:
\begin{equation*}
\begin{split}\ddot{x}_i + 2 \xi_i\omega_i \dot{x}_i + \omega^{2}_i x_i = f_i g(t), \qquad i=1,\ldots,m\end{split}
\end{equation*}
\sphinxAtStartPar
The vectors \sphinxcode{\sphinxupquote{w2}}, \sphinxcode{\sphinxupquote{xi}} and \sphinxcode{\sphinxupquote{f}} are the squared circular frequencies \(\omega_i^2\), the damping ratios \(\xi_i\), and the applied forces \(f_i\), respectively. The force vector \sphinxcode{\sphinxupquote{p}} contains the Fourier coefficients \(p(k)\) formed by the command \sphinxcode{\sphinxupquote{fft}}.

\sphinxAtStartPar
The solution in the frequency domain is computed at equal time increments defined by \sphinxcode{\sphinxupquote{dt}}. The result is stored in the \(m \times n\) matrix \sphinxcode{\sphinxupquote{Y}}, where \sphinxcode{\sphinxupquote{m}} is the number of equations and \sphinxcode{\sphinxupquote{n}} is the number of frequencies resulting from the \sphinxcode{\sphinxupquote{fft}} command. The dynamic solution in the time domain is achieved by the use of the command \sphinxcode{\sphinxupquote{ifft}}.

\sphinxlineitem{Example}
\sphinxAtStartPar
The dynamic solution to a set of uncoupled second\sphinxhyphen{}order differential equations can be computed by the following sequence of commands:

\begin{sphinxVerbatim}[commandchars=\\\{\}]
\PYG{o}{\PYGZgt{}}\PYG{o}{\PYGZgt{}}\PYG{+w}{ }\PYG{n}{g}\PYG{+w}{ }\PYG{p}{=}\PYG{+w}{ }\PYG{n}{gfunc}\PYG{p}{(}\PYG{n}{G}\PYG{p}{,}\PYG{+w}{ }\PYG{n}{dt}\PYG{p}{)}\PYG{p}{;}
\PYG{o}{\PYGZgt{}}\PYG{o}{\PYGZgt{}}\PYG{+w}{ }\PYG{n}{p}\PYG{+w}{ }\PYG{p}{=}\PYG{+w}{ }\PYG{n+nb}{fft}\PYG{p}{(}\PYG{n}{g}\PYG{p}{)}\PYG{p}{;}
\PYG{o}{\PYGZgt{}}\PYG{o}{\PYGZgt{}}\PYG{+w}{ }\PYG{n}{Y}\PYG{+w}{ }\PYG{p}{=}\PYG{+w}{ }\PYG{n}{dyna2f}\PYG{p}{(}\PYG{n}{w2}\PYG{p}{,}\PYG{+w}{ }\PYG{n}{xi}\PYG{p}{,}\PYG{+w}{ }\PYG{n}{f}\PYG{p}{,}\PYG{+w}{ }\PYG{n}{p}\PYG{p}{,}\PYG{+w}{ }\PYG{n}{dt}\PYG{p}{)}\PYG{p}{;}
\PYG{o}{\PYGZgt{}}\PYG{o}{\PYGZgt{}}\PYG{+w}{ }\PYG{n}{X}\PYG{+w}{ }\PYG{p}{=}\PYG{+w}{ }\PYG{p}{(}\PYG{n+nb}{real}\PYG{p}{(}\PYG{n+nb}{ifft}\PYG{p}{(}\PYG{n}{Y}\PYG{p}{.}\PYG{o}{\PYGZsq{}}\PYG{p}{)}\PYG{p}{)}\PYG{p}{)}\PYG{o}{\PYGZsq{}}\PYG{p}{;}
\end{sphinxVerbatim}

\sphinxAtStartPar
where it is assumed that the input variables \sphinxcode{\sphinxupquote{G}}, \sphinxcode{\sphinxupquote{dt}}, \sphinxcode{\sphinxupquote{w2}}, \sphinxcode{\sphinxupquote{xi}} and \sphinxcode{\sphinxupquote{f}} are properly defined. Note that the \sphinxcode{\sphinxupquote{ifft}} command operates on column vectors if \sphinxcode{\sphinxupquote{Y}} is a matrix; therefore use the transpose of \sphinxcode{\sphinxupquote{Y}}. The output from the \sphinxcode{\sphinxupquote{ifft}} command is complex. Therefore use \sphinxcode{\sphinxupquote{Y.\textquotesingle{}}} to transpose rows and columns in \sphinxcode{\sphinxupquote{Y}} in order to avoid the complex conjugate transpose of \sphinxcode{\sphinxupquote{Y}}.

\sphinxAtStartPar
The time response is represented by the real part of the output from the \sphinxcode{\sphinxupquote{ifft}} command. If the transpose is used and the result is stored in a matrix \sphinxcode{\sphinxupquote{X}}, each row will represent the time response for each equation as the output from the command \sphinxcode{\sphinxupquote{dyna2}}.

\sphinxlineitem{See also}
\sphinxAtStartPar
\DUrole{xref}{\DUrole{std}{\DUrole{std-ref}{gfunc}}}, \DUrole{xref}{\DUrole{std}{\DUrole{std-ref}{fft}}}, \DUrole{xref}{\DUrole{std}{\DUrole{std-ref}{ifft}}}

\end{description}\end{quote}


\subsection{fft}
\label{\detokenize{system_functions:fft}}\begin{quote}\begin{description}
\sphinxlineitem{Purpose}
\sphinxAtStartPar
Transform functions in time domain to frequency domain.

\sphinxlineitem{Syntax}
\begin{sphinxVerbatim}[commandchars=\\\{\}]
\PYG{n}{p}\PYG{+w}{ }\PYG{p}{=}\PYG{+w}{ }\PYG{n+nb}{fft}\PYG{p}{(}\PYG{n}{g}\PYG{p}{)}
\PYG{n}{p}\PYG{+w}{ }\PYG{p}{=}\PYG{+w}{ }\PYG{n+nb}{fft}\PYG{p}{(}\PYG{n}{g}\PYG{p}{,}\PYG{+w}{ }\PYG{n}{N}\PYG{p}{)}
\end{sphinxVerbatim}

\sphinxlineitem{Description}
\sphinxAtStartPar
The \sphinxcode{\sphinxupquote{fft}} function transforms a time dependent function to the frequency domain.

\sphinxAtStartPar
The function to be transformed is stored in the vector \sphinxcode{\sphinxupquote{g}}.
Each row in \sphinxcode{\sphinxupquote{g}} contains the value of the function at equal time intervals.
The function represents a span \(-\infty \leq t \leq +\infty\); however, only the values within a typical period are specified by \sphinxcode{\sphinxupquote{g}}.

\sphinxAtStartPar
The \sphinxcode{\sphinxupquote{fft}} command can be used with one or two input arguments.
If \sphinxcode{\sphinxupquote{N}} is not specified, the number of frequencies used in the transformation is equal to the number of points in the time domain (i.e., the length of the variable \sphinxcode{\sphinxupquote{g}}), and the output will be a vector of the same size containing complex values representing the frequency content of the input signal.

\sphinxAtStartPar
The scalar variable \sphinxcode{\sphinxupquote{N}} can be used to specify the number of frequencies used in the Fourier transform. The size of the output vector in this case will be equal to \sphinxcode{\sphinxupquote{N}}. It should be remembered that the highest harmonic component in the time signal that can be identified by the Fourier transform corresponds to half the sampling frequency. The sampling frequency is equal to \(1/dt\), where \(dt\) is the time increment of the time signal.

\sphinxAtStartPar
The complex Fourier coefficients \(p(k)\) are stored in the vector \sphinxcode{\sphinxupquote{p}}, and are computed according to
\begin{equation*}
\begin{split}p(k) = \sum_{j=1}^N x(j) \omega_N^{(j-1)(k-1)},\end{split}
\end{equation*}
\sphinxAtStartPar
where
\begin{equation*}
\begin{split}\omega_N = e^{-2 \pi i / N}.\end{split}
\end{equation*}
\begin{sphinxadmonition}{note}{Note}

\sphinxAtStartPar
This is a MATLAB built\sphinxhyphen{}in function.
\end{sphinxadmonition}

\end{description}\end{quote}


\subsection{freqresp}
\label{\detokenize{system_functions:freqresp}}\begin{quote}\begin{description}
\sphinxlineitem{Purpose}
\sphinxAtStartPar
Compute frequency response of a known discrete time response.

\sphinxlineitem{Syntax}
\begin{sphinxVerbatim}[commandchars=\\\{\}]
\PYG{p}{[}\PYG{n}{Freq}\PYG{p}{,}\PYG{+w}{ }\PYG{n}{Resp}\PYG{p}{]}\PYG{+w}{ }\PYG{p}{=}\PYG{+w}{ }\PYG{n}{freqresp}\PYG{p}{(}\PYG{n}{D}\PYG{p}{,}\PYG{+w}{ }\PYG{n}{dt}\PYG{p}{)}
\end{sphinxVerbatim}

\sphinxlineitem{Description}
\sphinxAtStartPar
\sphinxcode{\sphinxupquote{freqresp}} computes the frequency response of a discrete dynamic system.
\begin{itemize}
\item {} 
\sphinxAtStartPar
\sphinxcode{\sphinxupquote{D}} is the time history function.

\item {} 
\sphinxAtStartPar
\sphinxcode{\sphinxupquote{dt}} is the sampling time increment, i.e., the time increment used in the time integration procedure.

\item {} 
\sphinxAtStartPar
\sphinxcode{\sphinxupquote{Resp}} contains the computed response as a function of frequency.

\item {} 
\sphinxAtStartPar
\sphinxcode{\sphinxupquote{Freq}} contains the corresponding frequencies.

\end{itemize}

\sphinxlineitem{Example}
\sphinxAtStartPar
The result can be visualized by:

\begin{sphinxVerbatim}[commandchars=\\\{\}]
\PYG{n}{plot}\PYG{p}{(}\PYG{n}{Freq}\PYG{p}{,} \PYG{n}{Resp}\PYG{p}{)}
\PYG{n}{xlabel}\PYG{p}{(}\PYG{l+s+s1}{\PYGZsq{}}\PYG{l+s+s1}{frequency (Hz)}\PYG{l+s+s1}{\PYGZsq{}}\PYG{p}{)}
\end{sphinxVerbatim}

\sphinxAtStartPar
or:

\begin{sphinxVerbatim}[commandchars=\\\{\}]
\PYG{n}{semilogy}\PYG{p}{(}\PYG{n}{Freq}\PYG{p}{,} \PYG{n}{Resp}\PYG{p}{)}
\PYG{n}{xlabel}\PYG{p}{(}\PYG{l+s+s1}{\PYGZsq{}}\PYG{l+s+s1}{frequency (Hz)}\PYG{l+s+s1}{\PYGZsq{}}\PYG{p}{)}
\end{sphinxVerbatim}

\sphinxAtStartPar
The dimension of \sphinxcode{\sphinxupquote{Resp}} is the same as that of the original time history function.

\begin{sphinxadmonition}{note}{Note}

\sphinxAtStartPar
The time history function of a discrete system computed by direct integration often behaves in an unstructured manner. The reason for this is that the time history is a mixture of several participating eigenmodes at different eigenfrequencies. By using a Fourier transform, however, the response as a function of frequency can be computed efficiently. In particular, it is possible to identify the participating frequencies.
\end{sphinxadmonition}

\end{description}\end{quote}


\section{gfunc}
\label{\detokenize{system_functions:gfunc}}\begin{quote}\begin{description}
\sphinxlineitem{Purpose}
\sphinxAtStartPar
Form vector with function values at equally spaced points by linear interpolation.

\begin{figure}[htbp]
\centering
\capstart

\noindent\sphinxincludegraphics[width=0.700\linewidth]{{F32}.png}
\caption{Piecewise linear time dependent function}\label{\detokenize{system_functions:id1}}\end{figure}

\sphinxlineitem{Syntax}
\begin{sphinxVerbatim}[commandchars=\\\{\}]
\PYG{p}{[}\PYG{n}{t}\PYG{p}{,}\PYG{+w}{ }\PYG{n}{g}\PYG{p}{]}\PYG{+w}{ }\PYG{p}{=}\PYG{+w}{ }\PYG{n}{gfunc}\PYG{p}{(}\PYG{n}{G}\PYG{p}{,}\PYG{+w}{ }\PYG{n}{dt}\PYG{p}{)}
\end{sphinxVerbatim}

\sphinxlineitem{Description}
\sphinxAtStartPar
\sphinxcode{\sphinxupquote{gfunc}} uses linear interpolation to compute values at equally spaced points for a discrete function \(g\) given by
\begin{equation*}
\begin{split}G = \left[
\begin{array}{cc}
t_1 & g(t_1)\\
t_2 & g(t_2)\\
\vdots \\
t_N & g(t_N)
\end{array}
\right],\end{split}
\end{equation*}
\sphinxAtStartPar
as shown in the figure above.

\sphinxAtStartPar
Function values are computed in the range \(t_1 \leq t \leq t_N\), at equal increments, \sphinxcode{\sphinxupquote{dt}} being defined by the variable \sphinxcode{\sphinxupquote{dt}}. The number of linear segments (steps) is \((t_N-t_1)/dt\). The corresponding vector \sphinxcode{\sphinxupquote{t}} is also computed. The result can be plotted by using the command \sphinxcode{\sphinxupquote{plot(t, g)}}.

\end{description}\end{quote}


\subsection{ifft}
\label{\detokenize{system_functions:ifft}}\begin{quote}\begin{description}
\sphinxlineitem{Purpose}
\sphinxAtStartPar
Transform function in frequency domain to time domain.

\sphinxlineitem{Syntax}
\begin{sphinxVerbatim}[commandchars=\\\{\}]
\PYG{n}{x}\PYG{+w}{ }\PYG{p}{=}\PYG{+w}{ }\PYG{n+nb}{ifft}\PYG{p}{(}\PYG{n}{y}\PYG{p}{)}
\PYG{n}{x}\PYG{+w}{ }\PYG{p}{=}\PYG{+w}{ }\PYG{n+nb}{ifft}\PYG{p}{(}\PYG{n}{y}\PYG{p}{,}\PYG{+w}{ }\PYG{n}{N}\PYG{p}{)}
\end{sphinxVerbatim}

\sphinxlineitem{Description}
\sphinxAtStartPar
\sphinxcode{\sphinxupquote{ifft}} transforms a function in the frequency domain to a function in the time domain.

\sphinxAtStartPar
The function to be transformed is given in the vector \sphinxcode{\sphinxupquote{y}}. Each row in \sphinxcode{\sphinxupquote{y}} contains Fourier terms in the interval \(-\infty \leq \omega \leq +\infty\).

\sphinxAtStartPar
The \sphinxcode{\sphinxupquote{ifft}} command can be used with one or two input arguments. The scalar variable \sphinxcode{\sphinxupquote{N}} can be used to specify the number of frequencies used in the Fourier transform. The size of the output vector in this case will be equal to \sphinxcode{\sphinxupquote{N}}. See also the description of the command \sphinxcode{\sphinxupquote{fft}}.

\sphinxAtStartPar
The inverse Fourier coefficients \(x(j)\), stored in the variable \sphinxcode{\sphinxupquote{x}}, are computed according to
\begin{equation*}
\begin{split}x(j) = \frac{1}{N} \sum_{k=1}^N y(k) \omega_N^{-(j-1)(k-1)},\end{split}
\end{equation*}
\sphinxAtStartPar
where
\begin{equation*}
\begin{split}\omega_N = e^{-2 \pi i / N}.\end{split}
\end{equation*}
\begin{sphinxadmonition}{note}{Note}

\sphinxAtStartPar
This is a MATLAB built\sphinxhyphen{}in function.
\end{sphinxadmonition}

\sphinxlineitem{See also}
\sphinxAtStartPar
\DUrole{xref}{\DUrole{std}{\DUrole{std-ref}{fft}}}

\end{description}\end{quote}


\subsection{ritz}
\label{\detokenize{system_functions:ritz}}\begin{quote}\begin{description}
\sphinxlineitem{Purpose}
\sphinxAtStartPar
Compute approximative eigenvalues and eigenvectors by the Lanczos method.

\sphinxlineitem{Syntax}
\begin{sphinxVerbatim}[commandchars=\\\{\}]
\PYG{n}{L}\PYG{+w}{ }\PYG{p}{=}\PYG{+w}{ }\PYG{n}{ritz}\PYG{p}{(}\PYG{n}{K}\PYG{p}{,}\PYG{+w}{ }\PYG{n}{M}\PYG{p}{,}\PYG{+w}{ }\PYG{n}{f}\PYG{p}{,}\PYG{+w}{ }\PYG{n}{m}\PYG{p}{)}
\PYG{n}{L}\PYG{+w}{ }\PYG{p}{=}\PYG{+w}{ }\PYG{n}{ritz}\PYG{p}{(}\PYG{n}{K}\PYG{p}{,}\PYG{+w}{ }\PYG{n}{M}\PYG{p}{,}\PYG{+w}{ }\PYG{n}{f}\PYG{p}{,}\PYG{+w}{ }\PYG{n}{m}\PYG{p}{,}\PYG{+w}{ }\PYG{n}{b}\PYG{p}{)}
\PYG{p}{[}\PYG{n}{L}\PYG{p}{,}\PYG{+w}{ }\PYG{n}{X}\PYG{p}{]}\PYG{+w}{ }\PYG{p}{=}\PYG{+w}{ }\PYG{n}{ritz}\PYG{p}{(}\PYG{n}{K}\PYG{p}{,}\PYG{+w}{ }\PYG{n}{M}\PYG{p}{,}\PYG{+w}{ }\PYG{n}{f}\PYG{p}{,}\PYG{+w}{ }\PYG{n}{m}\PYG{p}{)}
\PYG{p}{[}\PYG{n}{L}\PYG{p}{,}\PYG{+w}{ }\PYG{n}{X}\PYG{p}{]}\PYG{+w}{ }\PYG{p}{=}\PYG{+w}{ }\PYG{n}{ritz}\PYG{p}{(}\PYG{n}{K}\PYG{p}{,}\PYG{+w}{ }\PYG{n}{M}\PYG{p}{,}\PYG{+w}{ }\PYG{n}{f}\PYG{p}{,}\PYG{+w}{ }\PYG{n}{m}\PYG{p}{,}\PYG{+w}{ }\PYG{n}{b}\PYG{p}{)}
\end{sphinxVerbatim}

\sphinxlineitem{Description}
\sphinxAtStartPar
\sphinxcode{\sphinxupquote{ritz}} computes, by the use of the Lanczos algorithm, \sphinxcode{\sphinxupquote{m}} approximative eigenvalues and \sphinxcode{\sphinxupquote{m}} corresponding eigenvectors for a given pair of \sphinxstyleemphasis{n}\sphinxhyphen{}by\sphinxhyphen{}\sphinxstyleemphasis{n} matrices \sphinxcode{\sphinxupquote{K}} and \sphinxcode{\sphinxupquote{M}} and a given non\sphinxhyphen{}zero starting vector \sphinxcode{\sphinxupquote{f}}.

\sphinxAtStartPar
If certain rows and columns in matrices \(\mathbf{K}\) and \(\mathbf{M}\) are to be eliminated in computing the eigenvalues, \(\mathbf{b}\) must be given in the command. The rows (and columns) to be eliminated are described in the vector \(\mathbf{b}\) defined as
\begin{equation*}
\begin{split}\mathbf{b} = \begin{bmatrix}
dof_1 \\
dof_2 \\
\vdots \\
dof_{nb}
\end{bmatrix}\end{split}
\end{equation*}
\begin{sphinxadmonition}{note}{Note}

\sphinxAtStartPar
If the number of vectors, \sphinxcode{\sphinxupquote{m}}, is chosen less than the total number of degrees\sphinxhyphen{}of\sphinxhyphen{}freedom, \(n\), only about the first \sphinxcode{\sphinxupquote{m/2}} Ritz vectors are good approximations of the true eigenvectors. Recall that the Ritz vectors satisfy the \sphinxcode{\sphinxupquote{M}}\sphinxhyphen{}orthonormality condition
\end{sphinxadmonition}
\begin{equation*}
\begin{split}\mathbf{X}^T \mathbf{M} \mathbf{X} = \mathbf{I}\end{split}
\end{equation*}
\sphinxAtStartPar
where \(\mathbf{I}\) is the identity matrix.

\end{description}\end{quote}


\subsection{spectra}
\label{\detokenize{system_functions:spectra}}\begin{quote}\begin{description}
\sphinxlineitem{Purpose}
\sphinxAtStartPar
Compute seismic response spectra for elastic design.

\sphinxlineitem{Syntax}
\begin{sphinxVerbatim}[commandchars=\\\{\}]
\PYG{n}{s}\PYG{+w}{ }\PYG{p}{=}\PYG{+w}{ }\PYG{n}{spectra}\PYG{p}{(}\PYG{n}{a}\PYG{p}{,}\PYG{+w}{ }\PYG{n}{xi}\PYG{p}{,}\PYG{+w}{ }\PYG{n}{dt}\PYG{p}{,}\PYG{+w}{ }\PYG{n}{f}\PYG{p}{)}
\end{sphinxVerbatim}

\sphinxlineitem{Description}
\sphinxAtStartPar
The \sphinxcode{\sphinxupquote{spectra}} function computes the seismic response spectrum for a known acceleration history function.
\begin{itemize}
\item {} 
\sphinxAtStartPar
The computation is based on the vector \sphinxcode{\sphinxupquote{a}}, which contains an acceleration time history function defined at equal time steps.

\item {} 
\sphinxAtStartPar
The time step is specified by the variable \sphinxcode{\sphinxupquote{dt}}.

\item {} 
\sphinxAtStartPar
The value of the damping ratio is given by the variable \sphinxcode{\sphinxupquote{xi}}.

\item {} 
\sphinxAtStartPar
Output from the computation, stored in the vector \sphinxcode{\sphinxupquote{s}}, is achieved at frequencies specified by the column vector \sphinxcode{\sphinxupquote{f}}.

\end{itemize}

\sphinxlineitem{Example}
\sphinxAtStartPar
The following procedure can be used to produce a seismic response spectrum for a damping ratio \(\xi = 0.05\), defined at 34 logarithmically spaced frequency points. The acceleration time history \sphinxcode{\sphinxupquote{a}} has been sampled at a frequency of 50 Hz, corresponding to a time increment \sphinxcode{\sphinxupquote{dt = 0.02}} between collected points:

\begin{sphinxVerbatim}[commandchars=\\\{\}]
\PYG{n}{freq}\PYG{+w}{ }\PYG{p}{=}\PYG{+w}{ }\PYG{n+nb}{logspace}\PYG{p}{(}\PYG{l+m+mi}{0}\PYG{p}{,}\PYG{+w}{ }\PYG{n+nb}{log10}\PYG{p}{(}\PYG{l+m+mi}{2}\PYGZca{}\PYG{p}{(}\PYG{l+m+mi}{33}\PYG{o}{/}\PYG{l+m+mi}{6}\PYG{p}{)}\PYG{p}{)}\PYG{p}{,}\PYG{+w}{ }\PYG{l+m+mi}{34}\PYG{p}{)}\PYG{p}{;}
\PYG{n}{xi}\PYG{+w}{ }\PYG{p}{=}\PYG{+w}{ }\PYG{l+m+mf}{0.05}\PYG{p}{;}
\PYG{n}{dt}\PYG{+w}{ }\PYG{p}{=}\PYG{+w}{ }\PYG{l+m+mf}{0.02}\PYG{p}{;}
\PYG{n}{s}\PYG{+w}{ }\PYG{p}{=}\PYG{+w}{ }\PYG{n}{spectra}\PYG{p}{(}\PYG{n}{a}\PYG{p}{,}\PYG{+w}{ }\PYG{n}{xi}\PYG{p}{,}\PYG{+w}{ }\PYG{n}{dt}\PYG{p}{,}\PYG{+w}{ }\PYG{n}{freq}\PYG{o}{\PYGZsq{}}\PYG{p}{)}\PYG{p}{;}
\end{sphinxVerbatim}

\sphinxAtStartPar
The resulting spectrum can be plotted by the command:

\begin{sphinxVerbatim}[commandchars=\\\{\}]
\PYG{n+nb}{loglog}\PYG{p}{(}\PYG{n}{freq}\PYG{p}{,}\PYG{+w}{ }\PYG{n}{s}\PYG{p}{,}\PYG{+w}{ }\PYG{l+s}{\PYGZsq{}}\PYG{l+s}{*\PYGZsq{}}\PYG{p}{)}
\end{sphinxVerbatim}

\end{description}\end{quote}


\subsection{step1}
\label{\detokenize{system_functions:step1}}\begin{quote}\begin{description}
\sphinxlineitem{Purpose}
\sphinxAtStartPar
Compute the dynamic solution to a set of first order differential equations.

\sphinxlineitem{Syntax}
\begin{sphinxVerbatim}[commandchars=\\\{\}]
\PYG{p}{[}\PYG{n}{a}\PYG{p}{,}\PYG{n}{da}\PYG{p}{]}\PYG{+w}{ }\PYG{p}{=}\PYG{+w}{ }\PYG{n}{step1}\PYG{p}{(}\PYG{n}{K}\PYG{p}{,}\PYG{+w}{ }\PYG{n}{C}\PYG{p}{,}\PYG{+w}{ }\PYG{n}{f}\PYG{p}{,}\PYG{+w}{ }\PYG{n}{a0}\PYG{p}{,}\PYG{+w}{ }\PYG{n}{bc}\PYG{p}{,}\PYG{+w}{ }\PYG{n}{ip}\PYG{p}{)}
\PYG{p}{[}\PYG{n}{a}\PYG{p}{,}\PYG{n}{da}\PYG{p}{]}\PYG{+w}{ }\PYG{p}{=}\PYG{+w}{ }\PYG{n}{step1}\PYG{p}{(}\PYG{n}{K}\PYG{p}{,}\PYG{+w}{ }\PYG{n}{C}\PYG{p}{,}\PYG{+w}{ }\PYG{n}{f}\PYG{p}{,}\PYG{+w}{ }\PYG{n}{a0}\PYG{p}{,}\PYG{+w}{ }\PYG{n}{bc}\PYG{p}{,}\PYG{+w}{ }\PYG{n}{ip}\PYG{p}{,}\PYG{+w}{ }\PYG{n}{times}\PYG{p}{)}
\PYG{p}{[}\PYG{n}{a}\PYG{p}{,}\PYG{n}{da}\PYG{p}{,}\PYG{+w}{ }\PYG{n}{ahist}\PYG{p}{,}\PYG{+w}{ }\PYG{n}{dahist}\PYG{p}{]}\PYG{+w}{ }\PYG{p}{=}\PYG{+w}{ }\PYG{n}{step1}\PYG{p}{(}\PYG{n}{K}\PYG{p}{,}\PYG{+w}{ }\PYG{n}{C}\PYG{p}{,}\PYG{+w}{ }\PYG{n}{f}\PYG{p}{,}\PYG{+w}{ }\PYG{n}{a0}\PYG{p}{,}\PYG{+w}{ }\PYG{n}{bc}\PYG{p}{,}\PYG{+w}{ }\PYG{n}{ip}\PYG{p}{,}\PYG{+w}{ }\PYG{n}{times}\PYG{p}{,}\PYG{+w}{ }\PYG{n}{dofs}\PYG{p}{)}
\end{sphinxVerbatim}

\sphinxlineitem{Description}
\sphinxAtStartPar
\sphinxcode{\sphinxupquote{step1}} computes at equal time steps the solution to a set of first order differential equations of the form:
\begin{equation*}
\begin{split}\mathbf{C} \dot{\mathbf{a}} + \mathbf{K}\mathbf{a} = \mathbf{f}(x, t), \\
\mathbf{a}(0) = \mathbf{a}_0\end{split}
\end{equation*}
\sphinxAtStartPar
The command solves transient field problems. In the case of heat conduction, \sphinxcode{\sphinxupquote{K}} and \sphinxcode{\sphinxupquote{C}} represent the \(n \times n\) conductivity and capacity matrices, respectively. \sphinxcode{\sphinxupquote{a}} is the temperature and \sphinxcode{\sphinxupquote{da}} (i.e., \(\dot{\mathbf{a}}\)) is the time derivative of the temperature.

\sphinxAtStartPar
The matrix \sphinxcode{\sphinxupquote{f}} contains the time\sphinxhyphen{}dependent load vectors. If no external loads are active, use \sphinxcode{\sphinxupquote{{[}{]}}} for \sphinxcode{\sphinxupquote{f}}. The matrix \sphinxcode{\sphinxupquote{f}} is organized as follows:

\begin{sphinxVerbatim}[commandchars=\\\{\}]
\PYG{n}{f}\PYG{+w}{ }\PYG{p}{=}\PYG{+w}{ }\PYG{p}{[}
\PYG{n+nb}{time}\PYG{+w}{ }\PYG{n}{history}\PYG{+w}{ }\PYG{n}{of}\PYG{+w}{ }\PYG{n}{the}\PYG{+w}{ }\PYG{n+nb}{load}\PYG{+w}{ }\PYG{n}{at}\PYG{+w}{ }\PYG{n}{dof\PYGZus{}1}
\PYG{n+nb}{time}\PYG{+w}{ }\PYG{n}{history}\PYG{+w}{ }\PYG{n}{of}\PYG{+w}{ }\PYG{n}{the}\PYG{+w}{ }\PYG{n+nb}{load}\PYG{+w}{ }\PYG{n}{at}\PYG{+w}{ }\PYG{n}{dof\PYGZus{}2}
\PYG{k}{...}
\PYG{n+nb}{time}\PYG{+w}{ }\PYG{n}{history}\PYG{+w}{ }\PYG{n}{of}\PYG{+w}{ }\PYG{n}{the}\PYG{+w}{ }\PYG{n+nb}{load}\PYG{+w}{ }\PYG{n}{at}\PYG{+w}{ }\PYG{n}{dof\PYGZus{}n}
\PYG{p}{]}
\end{sphinxVerbatim}

\sphinxAtStartPar
The dimension of \sphinxcode{\sphinxupquote{f}} is:
\begin{quote}

\sphinxAtStartPar
(number of degrees\sphinxhyphen{}of\sphinxhyphen{}freedom) × (number of timesteps + 1)
\end{quote}

\sphinxAtStartPar
The initial conditions are given by the vector \sphinxcode{\sphinxupquote{a0}} containing initial values of \sphinxcode{\sphinxupquote{a}}.

\sphinxAtStartPar
The matrix \sphinxcode{\sphinxupquote{bc}} contains the time\sphinxhyphen{}dependent prescribed values of the field variable \sphinxcode{\sphinxupquote{a}}. If no field variables are prescribed, use \sphinxcode{\sphinxupquote{{[}{]}}} for \sphinxcode{\sphinxupquote{bc}}. The matrix \sphinxcode{\sphinxupquote{bc}} is organized as follows:

\begin{sphinxVerbatim}[commandchars=\\\{\}]
\PYG{n}{bc}\PYG{+w}{ }\PYG{p}{=}\PYG{+w}{ }\PYG{p}{[}
\PYG{n}{dof\PYGZus{}1}\PYG{+w}{   }\PYG{l+s}{time}\PYG{+w}{ }\PYG{l+s}{history}\PYG{+w}{ }\PYG{l+s}{of}\PYG{+w}{ }\PYG{l+s}{the}\PYG{+w}{ }\PYG{l+s}{field}\PYG{+w}{ }\PYG{l+s}{variable}
\PYG{n}{dof\PYGZus{}2}\PYG{+w}{   }\PYG{l+s}{time}\PYG{+w}{ }\PYG{l+s}{history}\PYG{+w}{ }\PYG{l+s}{of}\PYG{+w}{ }\PYG{l+s}{the}\PYG{+w}{ }\PYG{l+s}{field}\PYG{+w}{ }\PYG{l+s}{variable}
\PYG{k}{...}
\PYG{n}{dof\PYGZus{}m2}\PYG{+w}{  }\PYG{l+s}{time}\PYG{+w}{ }\PYG{l+s}{history}\PYG{+w}{ }\PYG{l+s}{of}\PYG{+w}{ }\PYG{l+s}{the}\PYG{+w}{ }\PYG{l+s}{field}\PYG{+w}{ }\PYG{l+s}{variable}
\PYG{p}{]}
\end{sphinxVerbatim}

\sphinxAtStartPar
The dimension of \sphinxcode{\sphinxupquote{bc}} is:
\begin{quote}

\sphinxAtStartPar
(number of dofs with prescribed field values) × (number of timesteps + 2)
\end{quote}

\sphinxAtStartPar
The time integration procedure is governed by the parameters given in the vector \sphinxcode{\sphinxupquote{ip}} defined as:

\begin{sphinxVerbatim}[commandchars=\\\{\}]
\PYG{n}{ip}\PYG{+w}{ }\PYG{p}{=}\PYG{+w}{ }\PYG{p}{[}\PYG{n}{dt}\PYG{p}{,}\PYG{+w}{ }\PYG{n}{T}\PYG{p}{,}\PYG{+w}{ }\PYG{n+nb}{alpha}\PYG{p}{]}
\end{sphinxVerbatim}

\sphinxAtStartPar
where \sphinxcode{\sphinxupquote{dt}} specifies the length of the time increment, \sphinxcode{\sphinxupquote{T}} is the total time, and \sphinxcode{\sphinxupquote{alpha}} is the time integration constant. Frequently used values of \sphinxcode{\sphinxupquote{alpha}} are:


\begin{savenotes}\sphinxattablestart
\sphinxthistablewithglobalstyle
\centering
\begin{tabulary}{\linewidth}[t]{TT}
\sphinxtoprule
\sphinxstyletheadfamily 
\sphinxAtStartPar
alpha
&\sphinxstyletheadfamily 
\sphinxAtStartPar
Method
\\
\sphinxmidrule
\sphinxtableatstartofbodyhook
\sphinxAtStartPar
0
&
\sphinxAtStartPar
Forward difference; forward Euler
\\
\sphinxhline
\sphinxAtStartPar
0.5
&
\sphinxAtStartPar
Trapezoidal rule; Crank\sphinxhyphen{}Nicholson
\\
\sphinxhline
\sphinxAtStartPar
1
&
\sphinxAtStartPar
Backward difference; backward Euler
\\
\sphinxbottomrule
\end{tabulary}
\sphinxtableafterendhook\par
\sphinxattableend\end{savenotes}

\sphinxAtStartPar
The computed values of \sphinxcode{\sphinxupquote{a}} and \sphinxcode{\sphinxupquote{da}} are stored in \sphinxcode{\sphinxupquote{a}} and \sphinxcode{\sphinxupquote{da}}, respectively. The first column contains the initial values, and the following columns contain the values for each time step. The dimension is:
\begin{quote}

\sphinxAtStartPar
(number of degrees\sphinxhyphen{}of\sphinxhyphen{}freedom) × (number of time steps + 1)
\end{quote}

\sphinxAtStartPar
If the values are to be stored only for specific times, the parameter \sphinxcode{\sphinxupquote{times}} specifies at which times the solution will be stored. The values are stored in \sphinxcode{\sphinxupquote{a}} and \sphinxcode{\sphinxupquote{da}}, one column for each requested time according to \sphinxcode{\sphinxupquote{times}}. The dimension is then:
\begin{quote}

\sphinxAtStartPar
(number of degrees\sphinxhyphen{}of\sphinxhyphen{}freedom) × (number of requested times + 1)
\end{quote}

\sphinxAtStartPar
If the history is to be stored in \sphinxcode{\sphinxupquote{ahist}} and \sphinxcode{\sphinxupquote{dahist}} for some degrees of freedom, the parameter \sphinxcode{\sphinxupquote{dofs}} specifies for which degrees of freedom the history is to be stored. The computed time histories are stored in \sphinxcode{\sphinxupquote{ahist}} and \sphinxcode{\sphinxupquote{dahist}}, respectively, with one row for each requested degree of freedom. The dimension is:
\begin{quote}

\sphinxAtStartPar
(number of specified degrees of freedom) × (number of timesteps + 1)
\end{quote}

\sphinxAtStartPar
The time history functions can be generated using the command \sphinxcode{\sphinxupquote{gfunc}}. If all the values of the time histories of \sphinxcode{\sphinxupquote{f}} or \sphinxcode{\sphinxupquote{bc}} are kept constant, these values need to be stated only once. In this case, the number of columns in \sphinxcode{\sphinxupquote{f}} is one and in \sphinxcode{\sphinxupquote{bc}} two.

\sphinxAtStartPar
In most cases, only a few degrees\sphinxhyphen{}of\sphinxhyphen{}freedom are affected by the exterior load, and hence the matrix contains only a few non\sphinxhyphen{}zero entries. In such cases, it is possible to save space by defining \sphinxcode{\sphinxupquote{f}} as \sphinxcode{\sphinxupquote{sparse}} (a MATLAB built\sphinxhyphen{}in function).

\begin{sphinxadmonition}{note}{Note}

\sphinxAtStartPar
Reference: Bathe, K.J.: \sphinxstyleemphasis{Finite Element Procedures in Engineering Analysis}, Prentice\sphinxhyphen{}Hall, Englewood Cliffs, New Jersey, pp. 511\sphinxhyphen{}514, 1982.
\end{sphinxadmonition}

\end{description}\end{quote}


\subsection{step2}
\label{\detokenize{system_functions:step2}}\begin{quote}\begin{description}
\sphinxlineitem{Purpose}
\sphinxAtStartPar
Compute the dynamic solution to a set of second order differential equations.

\sphinxlineitem{Syntax}
\begin{sphinxVerbatim}[commandchars=\\\{\}]
\PYG{p}{[}\PYG{n}{a}\PYG{p}{,}\PYG{+w}{ }\PYG{n}{da}\PYG{p}{,}\PYG{+w}{ }\PYG{n}{d2a}\PYG{p}{]}\PYG{+w}{ }\PYG{p}{=}\PYG{+w}{ }\PYG{n}{step2}\PYG{p}{(}\PYG{n}{K}\PYG{p}{,}\PYG{+w}{ }\PYG{n}{C}\PYG{p}{,}\PYG{+w}{ }\PYG{n}{M}\PYG{p}{,}\PYG{+w}{ }\PYG{n}{f}\PYG{p}{,}\PYG{+w}{ }\PYG{n}{a0}\PYG{p}{,}\PYG{+w}{ }\PYG{n}{da0}\PYG{p}{,}\PYG{+w}{ }\PYG{n}{bc}\PYG{p}{,}\PYG{+w}{ }\PYG{n}{ip}\PYG{p}{)}
\PYG{p}{[}\PYG{n}{a}\PYG{p}{,}\PYG{+w}{ }\PYG{n}{da}\PYG{p}{,}\PYG{+w}{ }\PYG{n}{d2a}\PYG{p}{]}\PYG{+w}{ }\PYG{p}{=}\PYG{+w}{ }\PYG{n}{step2}\PYG{p}{(}\PYG{n}{K}\PYG{p}{,}\PYG{+w}{ }\PYG{n}{C}\PYG{p}{,}\PYG{+w}{ }\PYG{n}{M}\PYG{p}{,}\PYG{+w}{ }\PYG{n}{f}\PYG{p}{,}\PYG{+w}{ }\PYG{n}{a0}\PYG{p}{,}\PYG{+w}{ }\PYG{n}{da0}\PYG{p}{,}\PYG{+w}{ }\PYG{n}{bc}\PYG{p}{,}\PYG{+w}{ }\PYG{n}{ip}\PYG{p}{,}\PYG{+w}{ }\PYG{n}{times}\PYG{p}{)}
\PYG{p}{[}\PYG{n}{a}\PYG{p}{,}\PYG{+w}{ }\PYG{n}{da}\PYG{p}{,}\PYG{+w}{ }\PYG{n}{d2a}\PYG{p}{,}\PYG{+w}{ }\PYG{n}{ahist}\PYG{p}{,}\PYG{+w}{ }\PYG{n}{dahist}\PYG{p}{,}\PYG{+w}{ }\PYG{n}{d2ahist}\PYG{p}{]}\PYG{+w}{ }\PYG{p}{=}\PYG{+w}{ }\PYG{n}{step2}\PYG{p}{(}\PYG{n}{K}\PYG{p}{,}\PYG{+w}{ }\PYG{n}{C}\PYG{p}{,}\PYG{+w}{ }\PYG{n}{M}\PYG{p}{,}\PYG{+w}{ }\PYG{n}{f}\PYG{p}{,}\PYG{+w}{ }\PYG{n}{a0}\PYG{p}{,}\PYG{+w}{ }\PYG{n}{da0}\PYG{p}{,}\PYG{+w}{ }\PYG{n}{bc}\PYG{p}{,}\PYG{+w}{ }\PYG{n}{ip}\PYG{p}{,}\PYG{+w}{ }\PYG{n}{times}\PYG{p}{,}\PYG{+w}{ }\PYG{n}{dofs}\PYG{p}{)}
\end{sphinxVerbatim}

\sphinxlineitem{Description}
\sphinxAtStartPar
\sphinxcode{\sphinxupquote{step2}} computes at equal time steps the solution to a set of second order differential equations of the form:
\begin{equation*}
\begin{split}\mathbf{M} \ddot{\mathbf{a}} + \mathbf{C} \dot{\mathbf{a}} + \mathbf{K} \mathbf{a} = \mathbf{f}(x, t), \\
\mathbf{a}(0) = \mathbf{a}_0, \\
\dot{\mathbf{a}}(0) = \mathbf{v}_0.\end{split}
\end{equation*}
\sphinxAtStartPar
In structural mechanics problems, \sphinxcode{\sphinxupquote{K}}, \sphinxcode{\sphinxupquote{C}} and \sphinxcode{\sphinxupquote{M}} represent the \(n \times n\) stiffness, damping and mass matrices, respectively. \sphinxcode{\sphinxupquote{a}} is the displacement, \sphinxcode{\sphinxupquote{da}} ( = \(\dot{\mathbf{a}}\) ) is the velocity and \sphinxcode{\sphinxupquote{d2a}} ( = \(\ddot{\mathbf{a}}\) ) is the acceleration.

\sphinxAtStartPar
The matrix \sphinxcode{\sphinxupquote{f}} contains the time\sphinxhyphen{}dependent load vectors. If no external loads are active, use \sphinxcode{\sphinxupquote{{[}{]}}} for \sphinxcode{\sphinxupquote{f}}. The matrix \sphinxcode{\sphinxupquote{f}} is organized as:
\begin{equation*}
\begin{split}f = \begin{bmatrix}
\text{time history of the load at } dof_1 \\
\text{time history of the load at } dof_2 \\
\vdots \\
\text{time history of the load at } dof_n
\end{bmatrix}\end{split}
\end{equation*}
\sphinxAtStartPar
The dimension of \sphinxcode{\sphinxupquote{f}} is:
\begin{quote}

\sphinxAtStartPar
(number of degrees\sphinxhyphen{}of\sphinxhyphen{}freedom) × (number of timesteps + 1)
\end{quote}

\sphinxAtStartPar
The initial conditions are given by the vectors \sphinxcode{\sphinxupquote{a0}} and \sphinxcode{\sphinxupquote{da0}}, containing initial displacements and initial velocities.

\sphinxAtStartPar
The matrix \sphinxcode{\sphinxupquote{bc}} contains the time\sphinxhyphen{}dependent prescribed displacement. If no displacements are prescribed, use \sphinxcode{\sphinxupquote{{[}{]}}} for \sphinxcode{\sphinxupquote{bc}}. The matrix \sphinxcode{\sphinxupquote{bc}} is organized as:
\begin{equation*}
\begin{split}bc = \begin{bmatrix}
dof_1 & \text{time history of the displacement} \\
dof_2 & \text{time history of the displacement} \\
\vdots & \vdots \\
dof_{m_2} & \text{time history of the displacement}
\end{bmatrix}\end{split}
\end{equation*}
\sphinxAtStartPar
The dimension of \sphinxcode{\sphinxupquote{bc}} is:
\begin{quote}

\sphinxAtStartPar
(number of dofs with prescribed displacement) × (number of timesteps + 2)
\end{quote}

\sphinxAtStartPar
The time integration procedure is governed by the parameters given in the vector \sphinxcode{\sphinxupquote{ip}} defined as:
\begin{equation*}
\begin{split}ip = [dt, T, \alpha, \delta]\end{split}
\end{equation*}
\sphinxAtStartPar
where \sphinxcode{\sphinxupquote{dt}} specifies the time increment, \sphinxcode{\sphinxupquote{T}} the total time, and \sphinxcode{\sphinxupquote{alpha}} and \sphinxcode{\sphinxupquote{delta}} are time integration constants for the Newmark family of methods.

\sphinxAtStartPar
Frequently used values:


\begin{savenotes}\sphinxattablestart
\sphinxthistablewithglobalstyle
\centering
\begin{tabulary}{\linewidth}[t]{TTT}
\sphinxtoprule
\sphinxtableatstartofbodyhook
\sphinxAtStartPar
\(\alpha\)
&
\sphinxAtStartPar
\(\delta\)
&
\sphinxAtStartPar
Method
\\
\sphinxhline
\sphinxAtStartPar
\(\frac{1}{4}\)
&
\sphinxAtStartPar
\(\frac{1}{2}\)
&
\sphinxAtStartPar
Average acceleration (trapezoidal) rule
\\
\sphinxhline
\sphinxAtStartPar
\(\frac{1}{6}\)
&
\sphinxAtStartPar
\(\frac{1}{2}\)
&
\sphinxAtStartPar
Linear acceleration
\\
\sphinxhline
\sphinxAtStartPar
0
&
\sphinxAtStartPar
\(\frac{1}{2}\)
&
\sphinxAtStartPar
Central difference
\\
\sphinxbottomrule
\end{tabulary}
\sphinxtableafterendhook\par
\sphinxattableend\end{savenotes}

\sphinxAtStartPar
The computed values of \(\mathbf{a}\), \(\dot{\mathbf{a}}\) and \(\ddot{\mathbf{a}}\) are stored in \sphinxcode{\sphinxupquote{a}}, \sphinxcode{\sphinxupquote{da}} and \sphinxcode{\sphinxupquote{d2a}}, respectively. The first column contains the initial values and the following columns contain the values for each time step.

\sphinxAtStartPar
The dimension of \sphinxcode{\sphinxupquote{a}}, \sphinxcode{\sphinxupquote{da}} and \sphinxcode{\sphinxupquote{d2a}} is:
\begin{quote}

\sphinxAtStartPar
(number of degrees\sphinxhyphen{}of\sphinxhyphen{}freedom) × (number of time steps + 1)
\end{quote}

\sphinxAtStartPar
If the values are to be stored only for specific times, the parameter \sphinxcode{\sphinxupquote{times}} specifies at which times the solution will be stored. The values are stored in \sphinxcode{\sphinxupquote{a}}, \sphinxcode{\sphinxupquote{da}} and \sphinxcode{\sphinxupquote{d2a}}, one column for each requested time according to \sphinxcode{\sphinxupquote{times}}. The dimension is then:
\begin{quote}

\sphinxAtStartPar
(number of degrees\sphinxhyphen{}of\sphinxhyphen{}freedom) × (number of requested times + 1)
\end{quote}

\sphinxAtStartPar
If the history is to be stored in \sphinxcode{\sphinxupquote{ahist}}, \sphinxcode{\sphinxupquote{dahist}} and \sphinxcode{\sphinxupquote{d2ahist}} for some degrees of freedom, the parameter \sphinxcode{\sphinxupquote{dofs}} specifies for which degrees of freedom the history is to be stored. The computed time histories are stored in \sphinxcode{\sphinxupquote{ahist}}, \sphinxcode{\sphinxupquote{dahist}} and \sphinxcode{\sphinxupquote{d2ahist}}, one row for each requested degree of freedom according to \sphinxcode{\sphinxupquote{dofs}}. The dimension is:
\begin{quote}

\sphinxAtStartPar
(number of specified degrees of freedom) × (number of timesteps + 1)
\end{quote}

\sphinxAtStartPar
In most cases only a few degrees\sphinxhyphen{}of\sphinxhyphen{}freedom are affected by the exterior load, and hence the matrix contains only few non\sphinxhyphen{}zero entries. In such cases it is possible to save space by defining \sphinxcode{\sphinxupquote{f}} as sparse (a MATLAB built\sphinxhyphen{}in function).

\end{description}\end{quote}


\subsection{sweep}
\label{\detokenize{system_functions:sweep}}\begin{quote}\begin{description}
\sphinxlineitem{Purpose}
\sphinxAtStartPar
Compute complex frequency response functions.

\sphinxlineitem{Syntax}
\begin{sphinxVerbatim}[commandchars=\\\{\}]
\PYG{n}{Y}\PYG{+w}{ }\PYG{p}{=}\PYG{+w}{ }\PYG{n}{sweep}\PYG{p}{(}\PYG{n}{K}\PYG{p}{,}\PYG{+w}{ }\PYG{n}{C}\PYG{p}{,}\PYG{+w}{ }\PYG{n}{M}\PYG{p}{,}\PYG{+w}{ }\PYG{n}{p}\PYG{p}{,}\PYG{+w}{ }\PYG{n}{w}\PYG{p}{)}
\end{sphinxVerbatim}

\sphinxlineitem{Description}
\sphinxAtStartPar
\sphinxcode{\sphinxupquote{sweep}} computes the complex frequency response function for a system of the form:
\begin{equation*}
\begin{split}[\mathbf{K} + i\omega\mathbf{C} - \omega^2 \mathbf{M} ]\mathbf{y}(\omega) = \mathbf{p}\end{split}
\end{equation*}
\sphinxAtStartPar
Here, \sphinxcode{\sphinxupquote{K}}, \sphinxcode{\sphinxupquote{C}}, and \sphinxcode{\sphinxupquote{M}} represent the \sphinxstyleemphasis{m}\sphinxhyphen{}by\sphinxhyphen{}\sphinxstyleemphasis{m} stiffness, damping, and mass matrices, respectively. The vector \sphinxcode{\sphinxupquote{p}} defines the amplitude of the force. The frequency response function is computed for the values of \(\omega\) given by the vector \sphinxcode{\sphinxupquote{w}}.

\sphinxAtStartPar
The complex frequency response function is stored in the matrix \sphinxcode{\sphinxupquote{Y}} with dimension \sphinxstyleemphasis{m}\sphinxhyphen{}by\sphinxhyphen{}\sphinxstyleemphasis{n}, where \sphinxstyleemphasis{n} is equal to the number of circular frequencies defined in \sphinxcode{\sphinxupquote{w}}.

\sphinxlineitem{Example}
\sphinxAtStartPar
The steady\sphinxhyphen{}state response can be computed by:

\begin{sphinxVerbatim}[commandchars=\\\{\}]
\PYG{n}{X}\PYG{+w}{ }\PYG{p}{=}\PYG{+w}{ }\PYG{n+nb}{real}\PYG{p}{(}\PYG{n}{Y}\PYG{+w}{ }\PYG{o}{*}\PYG{+w}{ }\PYG{n+nb}{exp}\PYG{p}{(}\PYG{n+nb}{i}\PYG{+w}{ }\PYG{o}{*}\PYG{+w}{ }\PYG{n}{w}\PYG{+w}{ }\PYG{o}{*}\PYG{+w}{ }\PYG{n}{t}\PYG{p}{)}\PYG{p}{)}\PYG{p}{;}
\end{sphinxVerbatim}

\sphinxAtStartPar
and the amplitude by:

\begin{sphinxVerbatim}[commandchars=\\\{\}]
\PYG{n}{Z}\PYG{+w}{ }\PYG{p}{=}\PYG{+w}{ }\PYG{n+nb}{abs}\PYG{p}{(}\PYG{n}{Y}\PYG{p}{)}
\end{sphinxVerbatim}

\end{description}\end{quote}

\sphinxstepscope


\chapter{Statements}
\label{\detokenize{statements:statements}}\label{\detokenize{statements::doc}}
\sphinxstepscope


\chapter{Graphics functions}
\label{\detokenize{graphics_functions:graphics-functions}}\label{\detokenize{graphics_functions::doc}}
\sphinxAtStartPar
The group of graphics functions comprises functions
for element based graphics. Mesh plots, displacements, section forces, flows, iso lines and
principal stresses can be displayed. The functions are divided into two dimensional,
and general graphics functions.


\section{Two dimensional graphics functions}
\label{\detokenize{graphics_functions:two-dimensional-graphics-functions}}

\subsection{dispbeam2}
\label{\detokenize{graphics_functions:dispbeam2}}\begin{quote}\begin{description}
\sphinxlineitem{Purpose}
\sphinxAtStartPar
Draw the displacements for a two dimensional beam element.

\sphinxlineitem{Syntax}
\begin{sphinxVerbatim}[commandchars=\\\{\}]
\PYG{p}{[}\PYG{n}{sfac}\PYG{p}{]}\PYG{+w}{ }\PYG{p}{=}\PYG{+w}{ }\PYG{n}{dispbeam2}\PYG{p}{(}\PYG{n}{ex}\PYG{p}{,}\PYG{+w}{ }\PYG{n}{ey}\PYG{p}{,}\PYG{+w}{ }\PYG{n}{edi}\PYG{p}{)}
\PYG{p}{[}\PYG{n}{sfac}\PYG{p}{]}\PYG{+w}{ }\PYG{p}{=}\PYG{+w}{ }\PYG{n}{dispbeam2}\PYG{p}{(}\PYG{n}{ex}\PYG{p}{,}\PYG{+w}{ }\PYG{n}{ey}\PYG{p}{,}\PYG{+w}{ }\PYG{n}{edi}\PYG{p}{,}\PYG{+w}{ }\PYG{n}{plotpar}\PYG{p}{)}
\PYG{n}{dispbeam2}\PYG{p}{(}\PYG{n}{ex}\PYG{p}{,}\PYG{+w}{ }\PYG{n}{ey}\PYG{p}{,}\PYG{+w}{ }\PYG{n}{edi}\PYG{p}{,}\PYG{+w}{ }\PYG{n}{plotpar}\PYG{p}{,}\PYG{+w}{ }\PYG{n}{sfac}\PYG{p}{)}
\end{sphinxVerbatim}

\sphinxlineitem{Description}
\sphinxAtStartPar
Input variables are the coordinate matrices \(ex\) and \(ey\), see e.g. \(beam2e\), and the element displacements \(edi\) obtained by e.g. \(beam2s\).

\sphinxAtStartPar
The variable \(plotpar\) sets plot parameters for linetype, linecolour and node marker:
\begin{equation*}
\begin{split}plotpar = [\, linetype \;\; linecolor \;\; nodemark \,]\end{split}
\end{equation*}
\sphinxAtStartPar
where


\begin{savenotes}\sphinxattablestart
\sphinxthistablewithglobalstyle
\centering
\begin{tabular}[t]{\X{20}{100}\X{30}{100}\X{20}{100}\X{30}{100}}
\sphinxtoprule
\sphinxtableatstartofbodyhook
\sphinxAtStartPar
\(linetype = 1\)
&
\sphinxAtStartPar
solid line
&
\sphinxAtStartPar
\(linecolor = 1\)
&
\sphinxAtStartPar
black
\\
\sphinxhline
\sphinxAtStartPar
2
&
\sphinxAtStartPar
dashed line
&
\sphinxAtStartPar
2
&
\sphinxAtStartPar
blue
\\
\sphinxhline
\sphinxAtStartPar
3
&
\sphinxAtStartPar
dotted line
&
\sphinxAtStartPar
3
&
\sphinxAtStartPar
magenta
\\
\sphinxhline&&
\sphinxAtStartPar
4
&
\sphinxAtStartPar
red
\\
\sphinxbottomrule
\end{tabular}
\sphinxtableafterendhook\par
\sphinxattableend\end{savenotes}


\begin{savenotes}\sphinxattablestart
\sphinxthistablewithglobalstyle
\centering
\begin{tabular}[t]{\X{20}{50}\X{30}{50}}
\sphinxtoprule
\sphinxtableatstartofbodyhook
\sphinxAtStartPar
\(nodemark = 1\)
&
\sphinxAtStartPar
circle
\\
\sphinxhline
\sphinxAtStartPar
2
&
\sphinxAtStartPar
star
\\
\sphinxhline
\sphinxAtStartPar
0
&
\sphinxAtStartPar
no mark
\\
\sphinxbottomrule
\end{tabular}
\sphinxtableafterendhook\par
\sphinxattableend\end{savenotes}

\sphinxAtStartPar
Default is dashed black lines with circles at nodes.

\sphinxAtStartPar
The scale factor \(sfac\) is a scalar that the element displacements are multiplied with to get a suitable geometrical representation. If \(sfac\) is omitted in the input list, the scale factor is set automatically.

\end{description}\end{quote}


\subsection{eldraw2}
\label{\detokenize{graphics_functions:eldraw2}}\begin{quote}\begin{description}
\sphinxlineitem{Purpose}
\sphinxAtStartPar
Draw the undeformed mesh for a two dimensional structure.

\sphinxlineitem{Syntax}
\begin{sphinxVerbatim}[commandchars=\\\{\}]
\PYG{n}{eldraw2}\PYG{p}{(}\PYG{n}{Ex}\PYG{p}{,}\PYG{+w}{ }\PYG{n}{Ey}\PYG{p}{)}
\PYG{n}{eldraw2}\PYG{p}{(}\PYG{n}{Ex}\PYG{p}{,}\PYG{+w}{ }\PYG{n}{Ey}\PYG{p}{,}\PYG{+w}{ }\PYG{n}{plotpar}\PYG{p}{)}
\PYG{n}{eldraw2}\PYG{p}{(}\PYG{n}{Ex}\PYG{p}{,}\PYG{+w}{ }\PYG{n}{Ey}\PYG{p}{,}\PYG{+w}{ }\PYG{n}{plotpar}\PYG{p}{,}\PYG{+w}{ }\PYG{n}{elnum}\PYG{p}{)}
\end{sphinxVerbatim}

\sphinxlineitem{Description}
\sphinxAtStartPar
\(\texttt{eldraw2}\) displays the undeformed mesh for a two dimensional structure.

\sphinxAtStartPar
Input variables are the coordinate matrices \(\texttt{Ex}\) and \(\texttt{Ey}\) formed by the function \(\texttt{coordxtr}\).

\sphinxAtStartPar
The variable \(\texttt{plotpar}\) sets plot parameters for linetype, linecolor and node marker:
\begin{equation*}
\begin{split}\texttt{plotpar} = [\, \text{linetype} \;\; \text{linecolor} \;\; \text{nodemark} \,]\end{split}
\end{equation*}

\begin{savenotes}\sphinxattablestart
\sphinxthistablewithglobalstyle
\centering
\begin{tabulary}{\linewidth}[t]{TTTT}
\sphinxtoprule
\sphinxstyletheadfamily 
\sphinxAtStartPar
linetype
&\sphinxstyletheadfamily 
\sphinxAtStartPar
meaning
&\sphinxstyletheadfamily 
\sphinxAtStartPar
linecolor
&\sphinxstyletheadfamily 
\sphinxAtStartPar
meaning
\\
\sphinxmidrule
\sphinxtableatstartofbodyhook
\sphinxAtStartPar
1
&
\sphinxAtStartPar
solid line
&
\sphinxAtStartPar
1
&
\sphinxAtStartPar
black
\\
\sphinxhline
\sphinxAtStartPar
2
&
\sphinxAtStartPar
dashed line
&
\sphinxAtStartPar
2
&
\sphinxAtStartPar
blue
\\
\sphinxhline
\sphinxAtStartPar
3
&
\sphinxAtStartPar
dotted line
&
\sphinxAtStartPar
3
&
\sphinxAtStartPar
magenta
\\
\sphinxhline&&
\sphinxAtStartPar
4
&
\sphinxAtStartPar
red
\\
\sphinxbottomrule
\end{tabulary}
\sphinxtableafterendhook\par
\sphinxattableend\end{savenotes}


\begin{savenotes}\sphinxattablestart
\sphinxthistablewithglobalstyle
\centering
\begin{tabulary}{\linewidth}[t]{TT}
\sphinxtoprule
\sphinxstyletheadfamily 
\sphinxAtStartPar
nodemark
&\sphinxstyletheadfamily 
\sphinxAtStartPar
meaning
\\
\sphinxmidrule
\sphinxtableatstartofbodyhook
\sphinxAtStartPar
1
&
\sphinxAtStartPar
circle
\\
\sphinxhline
\sphinxAtStartPar
2
&
\sphinxAtStartPar
star
\\
\sphinxhline
\sphinxAtStartPar
0
&
\sphinxAtStartPar
no mark
\\
\sphinxbottomrule
\end{tabulary}
\sphinxtableafterendhook\par
\sphinxattableend\end{savenotes}

\sphinxAtStartPar
Default is solid black lines with circles at nodes.

\sphinxAtStartPar
Element numbers can be displayed at the center of the element if a column vector \(\texttt{elnum}\) with the element numbers is supplied. This column vector can be derived from the element topology matrix \(\texttt{Edof}\):
\begin{equation*}
\begin{split}\texttt{elnum} = \texttt{Edof}(:,1)\end{split}
\end{equation*}
\sphinxAtStartPar
i.e. the first column of the topology matrix.

\sphinxlineitem{Limitations}
\sphinxAtStartPar
Supported elements are bar, beam, triangular three node, and quadrilateral four node elements.

\end{description}\end{quote}


\subsection{eldisp2}
\label{\detokenize{graphics_functions:eldisp2}}\begin{quote}\begin{description}
\sphinxlineitem{Purpose}
\sphinxAtStartPar
Draw the deformed mesh for a two dimensional structure.

\sphinxlineitem{Syntax}
\begin{sphinxVerbatim}[commandchars=\\\{\}]
\PYG{p}{[}\PYG{n}{sfac}\PYG{p}{]}\PYG{+w}{ }\PYG{p}{=}\PYG{+w}{ }\PYG{n}{eldisp2}\PYG{p}{(}\PYG{n}{Ex}\PYG{p}{,}\PYG{+w}{ }\PYG{n}{Ey}\PYG{p}{,}\PYG{+w}{ }\PYG{n}{Ed}\PYG{p}{)}
\PYG{p}{[}\PYG{n}{sfac}\PYG{p}{]}\PYG{+w}{ }\PYG{p}{=}\PYG{+w}{ }\PYG{n}{eldisp2}\PYG{p}{(}\PYG{n}{Ex}\PYG{p}{,}\PYG{+w}{ }\PYG{n}{Ey}\PYG{p}{,}\PYG{+w}{ }\PYG{n}{Ed}\PYG{p}{,}\PYG{+w}{ }\PYG{n}{plotpar}\PYG{p}{)}
\PYG{n}{eldisp2}\PYG{p}{(}\PYG{n}{Ex}\PYG{p}{,}\PYG{+w}{ }\PYG{n}{Ey}\PYG{p}{,}\PYG{+w}{ }\PYG{n}{Ed}\PYG{p}{,}\PYG{+w}{ }\PYG{n}{plotpar}\PYG{p}{,}\PYG{+w}{ }\PYG{n}{sfac}\PYG{p}{)}
\end{sphinxVerbatim}

\sphinxlineitem{Description}
\sphinxAtStartPar
\(\text{eldisp2}\) displays the deformed mesh for a two dimensional structure.

\sphinxAtStartPar
Input variables are the coordinate matrices \(\text{Ex}\) and \(\text{Ey}\) formed by the function \(\text{coordxtr}\), and the element displacements \(\text{Ed}\) formed by the function \(\text{extract}\).

\sphinxAtStartPar
The variable \(\text{plotpar}\) sets plot parameters for linetype, linecolor and node marker:
\begin{equation*}
\begin{split}\text{plotpar} = [\, \text{linetype} \quad \text{linecolor} \quad \text{nodemark} \,]\end{split}
\end{equation*}
\sphinxAtStartPar
where


\begin{savenotes}\sphinxattablestart
\sphinxthistablewithglobalstyle
\centering
\begin{tabular}[t]{\X{20}{100}\X{30}{100}\X{20}{100}\X{30}{100}}
\sphinxtoprule
\sphinxstyletheadfamily 
\sphinxAtStartPar
linetype
&\sphinxstyletheadfamily 
\sphinxAtStartPar
line style
&\sphinxstyletheadfamily 
\sphinxAtStartPar
linecolor
&\sphinxstyletheadfamily 
\sphinxAtStartPar
color
\\
\sphinxmidrule
\sphinxtableatstartofbodyhook
\sphinxAtStartPar
1
&
\sphinxAtStartPar
solid line
&
\sphinxAtStartPar
1
&
\sphinxAtStartPar
black
\\
\sphinxhline
\sphinxAtStartPar
2
&
\sphinxAtStartPar
dashed line
&
\sphinxAtStartPar
2
&
\sphinxAtStartPar
blue
\\
\sphinxhline
\sphinxAtStartPar
3
&
\sphinxAtStartPar
dotted line
&
\sphinxAtStartPar
3
&
\sphinxAtStartPar
magenta
\\
\sphinxhline&&
\sphinxAtStartPar
4
&
\sphinxAtStartPar
red
\\
\sphinxbottomrule
\end{tabular}
\sphinxtableafterendhook\par
\sphinxattableend\end{savenotes}


\begin{savenotes}\sphinxattablestart
\sphinxthistablewithglobalstyle
\centering
\begin{tabular}[t]{\X{20}{50}\X{30}{50}}
\sphinxtoprule
\sphinxstyletheadfamily 
\sphinxAtStartPar
nodemark
&\sphinxstyletheadfamily 
\sphinxAtStartPar
marker
\\
\sphinxmidrule
\sphinxtableatstartofbodyhook
\sphinxAtStartPar
1
&
\sphinxAtStartPar
circle
\\
\sphinxhline
\sphinxAtStartPar
2
&
\sphinxAtStartPar
star
\\
\sphinxhline
\sphinxAtStartPar
0
&
\sphinxAtStartPar
no mark
\\
\sphinxbottomrule
\end{tabular}
\sphinxtableafterendhook\par
\sphinxattableend\end{savenotes}

\sphinxAtStartPar
Default is dashed black lines with circles at nodes.

\sphinxAtStartPar
The scale factor \(\text{sfac}\) is a scalar that the element displacements are multiplied with to get a suitable geometrical representation. The scale factor is set automatically if it is omitted in the input list.

\sphinxlineitem{Limitations}
\sphinxAtStartPar
Supported elements are bar, beam, triangular three node, and quadrilateral four node elements.

\end{description}\end{quote}


\subsection{elflux2}
\label{\detokenize{graphics_functions:elflux2}}\begin{quote}\begin{description}
\sphinxlineitem{Purpose}
\sphinxAtStartPar
Draw element flow arrows for two dimensional elements.

\sphinxlineitem{Syntax}
\begin{sphinxVerbatim}[commandchars=\\\{\}]
\PYG{p}{[}\PYG{n}{sfac}\PYG{p}{]}\PYG{+w}{ }\PYG{p}{=}\PYG{+w}{ }\PYG{n}{elflux2}\PYG{p}{(}\PYG{n}{Ex}\PYG{p}{,}\PYG{+w}{ }\PYG{n}{Ey}\PYG{p}{,}\PYG{+w}{ }\PYG{n}{Es}\PYG{p}{)}
\PYG{p}{[}\PYG{n}{sfac}\PYG{p}{]}\PYG{+w}{ }\PYG{p}{=}\PYG{+w}{ }\PYG{n}{elflux2}\PYG{p}{(}\PYG{n}{Ex}\PYG{p}{,}\PYG{+w}{ }\PYG{n}{Ey}\PYG{p}{,}\PYG{+w}{ }\PYG{n}{Es}\PYG{p}{,}\PYG{+w}{ }\PYG{n}{plotpar}\PYG{p}{)}
\PYG{n}{elflux2}\PYG{p}{(}\PYG{n}{Ex}\PYG{p}{,}\PYG{+w}{ }\PYG{n}{Ey}\PYG{p}{,}\PYG{+w}{ }\PYG{n}{Es}\PYG{p}{,}\PYG{+w}{ }\PYG{n}{plotpar}\PYG{p}{,}\PYG{+w}{ }\PYG{n}{sfac}\PYG{p}{)}
\end{sphinxVerbatim}

\sphinxlineitem{Description}
\sphinxAtStartPar
\(\mathrm{elflux2}\) displays element heat flux vectors (or corresponding quantities) for a number of elements of the same type.
The flux vectors are displayed as arrows at the element centroids.
Note that only the flux vectors are displayed. To display the element mesh, use \(\mathrm{eldraw2}\).

\sphinxAtStartPar
Input variables are the coordinate matrices \(\mathrm{Ex}\) and \(\mathrm{Ey}\), and the element flux matrix \(\mathrm{Es}\) defined in \(\mathrm{flw2ts}\) or \(\mathrm{flw2qs}\).

\sphinxAtStartPar
The variable \(\mathrm{plotpar}\) sets plot parameters for the flux arrows:

\sphinxAtStartPar
\(\mathrm{plotpar} = [\, \mathrm{arrowtype} \;\; \mathrm{arrowcolor} \,]\)


\begin{savenotes}\sphinxattablestart
\sphinxthistablewithglobalstyle
\centering
\begin{tabulary}{\linewidth}[t]{TTTT}
\sphinxtoprule
\sphinxstyletheadfamily 
\sphinxAtStartPar
\(arrowtype\)
&\sphinxstyletheadfamily 
\sphinxAtStartPar
Line style
&\sphinxstyletheadfamily 
\sphinxAtStartPar
\(arrowcolor\)
&\sphinxstyletheadfamily 
\sphinxAtStartPar
Color
\\
\sphinxmidrule
\sphinxtableatstartofbodyhook
\sphinxAtStartPar
1
&
\sphinxAtStartPar
solid
&
\sphinxAtStartPar
1
&
\sphinxAtStartPar
black
\\
\sphinxhline
\sphinxAtStartPar
2
&
\sphinxAtStartPar
dashed
&
\sphinxAtStartPar
2
&
\sphinxAtStartPar
blue
\\
\sphinxhline
\sphinxAtStartPar
3
&
\sphinxAtStartPar
dotted
&
\sphinxAtStartPar
3
&
\sphinxAtStartPar
magenta
\\
\sphinxhline&&
\sphinxAtStartPar
4
&
\sphinxAtStartPar
red
\\
\sphinxbottomrule
\end{tabulary}
\sphinxtableafterendhook\par
\sphinxattableend\end{savenotes}

\sphinxAtStartPar
Default, if \(\mathrm{plotpar}\) is omitted, is solid black arrows.

\sphinxAtStartPar
The scale factor \(\mathrm{sfac}\) is a scalar that the values are multiplied with to get a suitable arrow size in relation to the element size. The scale factor is set automatically if it is omitted in the input list.

\sphinxlineitem{Limitations}
\sphinxAtStartPar
Supported elements are triangular 3 node and quadrilateral 4 node elements.

\end{description}\end{quote}


\subsection{eliso2}
\label{\detokenize{graphics_functions:eliso2}}\begin{quote}\begin{description}
\sphinxlineitem{Purpose}
\sphinxAtStartPar
Display element iso lines for two dimensional elements.

\sphinxlineitem{Syntax}
\begin{sphinxVerbatim}[commandchars=\\\{\}]
\PYG{n}{eliso2}\PYG{p}{(}\PYG{n}{Ex}\PYG{p}{,}\PYG{+w}{ }\PYG{n}{Ey}\PYG{p}{,}\PYG{+w}{ }\PYG{n}{Ed}\PYG{p}{,}\PYG{+w}{ }\PYG{n}{isov}\PYG{p}{)}
\PYG{n}{eliso2}\PYG{p}{(}\PYG{n}{Ex}\PYG{p}{,}\PYG{+w}{ }\PYG{n}{Ey}\PYG{p}{,}\PYG{+w}{ }\PYG{n}{Ed}\PYG{p}{,}\PYG{+w}{ }\PYG{n}{isov}\PYG{p}{,}\PYG{+w}{ }\PYG{n}{plotpar}\PYG{p}{)}
\end{sphinxVerbatim}

\sphinxlineitem{Description}
\sphinxAtStartPar
\(\mathtt{eliso2}\) displays element iso lines for a number of elements of the same type.
Note that only the iso lines are displayed. To display the element mesh, use \(\mathtt{eldraw2}\).

\sphinxAtStartPar
Input variables are the coordinate matrices \(\mathtt{Ex}\) and \(\mathtt{Ey}\) formed by the function \(\mathtt{coordxtr}\),
and the element nodal quantities (e.g., displacement or energy potential) matrix \(\mathtt{Ed}\) defined in \(\mathtt{extract}\).

\sphinxAtStartPar
If \(\mathtt{isov}\) is a scalar, it determines the number of iso lines to be displayed.
If \(\mathtt{isov}\) is a vector, it determines the values of the iso lines to be displayed
(number of iso lines equal to the length of vector \(\mathtt{isov}\)):

\sphinxAtStartPar
\(\mathtt{isov} = [\, \text{iso lines} \,]\)

\sphinxAtStartPar
\(\mathtt{isov} = [\, \text{isovalue}(1) \; \ldots \; \text{isovalue}(n) \,]\)

\sphinxAtStartPar
The variable \(\mathtt{plotpar}\) sets plot parameters for the iso lines:

\sphinxAtStartPar
\(\mathtt{plotpar} = [\, \text{linetype} \;\; \text{linecolor} \;\; \text{textfcn} \,]\)
\begin{itemize}
\item {} 
\sphinxAtStartPar
\(\text{linetype}\): 1 = solid, 2 = dashed, 3 = dotted

\item {} 
\sphinxAtStartPar
\(\text{linecolor}\): 1 = black, 2 = blue, 3 = magenta, 4 = red

\item {} 
\sphinxAtStartPar
\(\text{textfcn}\): 0 = iso values not printed, 1 = iso values printed at the iso lines, 2 = iso values printed where the cursor indicates

\end{itemize}

\sphinxAtStartPar
Default is solid, black lines and no iso values printed.

\sphinxlineitem{Limitations}
\sphinxAtStartPar
Supported elements are triangular 3 node and quadrilateral 4 node elements.

\end{description}\end{quote}


\subsection{elprinc2}
\label{\detokenize{graphics_functions:elprinc2}}\begin{quote}\begin{description}
\sphinxlineitem{Purpose}
\sphinxAtStartPar
Draw element principal stresses as arrows for two dimensional elements.

\sphinxlineitem{Syntax}
\begin{sphinxVerbatim}[commandchars=\\\{\}]
\PYG{p}{[}\PYG{n}{sfac}\PYG{p}{]}\PYG{+w}{ }\PYG{p}{=}\PYG{+w}{ }\PYG{n}{elprinc2}\PYG{p}{(}\PYG{n}{Ex}\PYG{p}{,}\PYG{+w}{ }\PYG{n}{Ey}\PYG{p}{,}\PYG{+w}{ }\PYG{n}{Es}\PYG{p}{)}
\PYG{p}{[}\PYG{n}{sfac}\PYG{p}{]}\PYG{+w}{ }\PYG{p}{=}\PYG{+w}{ }\PYG{n}{elprinc2}\PYG{p}{(}\PYG{n}{Ex}\PYG{p}{,}\PYG{+w}{ }\PYG{n}{Ey}\PYG{p}{,}\PYG{+w}{ }\PYG{n}{Es}\PYG{p}{,}\PYG{+w}{ }\PYG{n}{plotpar}\PYG{p}{)}
\PYG{n}{elprinc2}\PYG{p}{(}\PYG{n}{Ex}\PYG{p}{,}\PYG{+w}{ }\PYG{n}{Ey}\PYG{p}{,}\PYG{+w}{ }\PYG{n}{Es}\PYG{p}{,}\PYG{+w}{ }\PYG{n}{plotpar}\PYG{p}{,}\PYG{+w}{ }\PYG{n}{sfac}\PYG{p}{)}
\end{sphinxVerbatim}

\sphinxlineitem{Description}
\sphinxAtStartPar
\(\text{elprinc2}\) displays element principal stresses for a number of elements of the same type. The principal stresses are displayed as arrows at the element centroids. Note that only the principal stresses are displayed. To display the element mesh, use \(\text{eldraw2}\).

\sphinxAtStartPar
Input variables are the coordinate matrices \(\text{Ex}\) and \(\text{Ey}\), and the element stresses matrix \(\text{Es}\) defined in \(\text{plants}\) or \(\text{planqs}\).

\sphinxAtStartPar
The variable \(\text{plotpar}\) sets plot parameters for the principal stress arrows:
\begin{equation*}
\begin{split}\text{plotpar} = [\, \text{arrowtype} \;\; \text{arrowcolor} \,]\end{split}
\end{equation*}
\sphinxAtStartPar
where


\begin{savenotes}\sphinxattablestart
\sphinxthistablewithglobalstyle
\centering
\begin{tabular}[t]{\X{20}{100}\X{30}{100}\X{20}{100}\X{30}{100}}
\sphinxtoprule
\sphinxtableatstartofbodyhook
\sphinxAtStartPar
\(\text{arrowtype} = 1\)
&
\sphinxAtStartPar
solid
&
\sphinxAtStartPar
\(\text{arrowcolor} = 1\)
&
\sphinxAtStartPar
black
\\
\sphinxhline
\sphinxAtStartPar
2
&
\sphinxAtStartPar
dashed
&
\sphinxAtStartPar
2
&
\sphinxAtStartPar
blue
\\
\sphinxhline
\sphinxAtStartPar
3
&
\sphinxAtStartPar
dotted
&
\sphinxAtStartPar
3
&
\sphinxAtStartPar
magenta
\\
\sphinxhline&&
\sphinxAtStartPar
4
&
\sphinxAtStartPar
red
\\
\sphinxbottomrule
\end{tabular}
\sphinxtableafterendhook\par
\sphinxattableend\end{savenotes}

\sphinxAtStartPar
Default, if \(\text{plotpar}\) is omitted, is solid black arrows.

\sphinxAtStartPar
The scale factor \(\text{sfac}\) is a scalar that values are multiplied with to get a suitable arrow size in relation to the element size. The scale factor is set automatically if it is omitted in the input list.

\sphinxlineitem{Limitations}
\sphinxAtStartPar
Supported elements are triangular 3 node and quadrilateral 4 node elements.

\end{description}\end{quote}


\subsection{scalfact2}
\label{\detokenize{graphics_functions:scalfact2}}\begin{quote}\begin{description}
\sphinxlineitem{Purpose}
\sphinxAtStartPar
Determine scale factor for drawing computational results.

\sphinxlineitem{Syntax}
\begin{sphinxVerbatim}[commandchars=\\\{\}]
\PYG{p}{[}\PYG{n}{sfac}\PYG{p}{]}\PYG{+w}{ }\PYG{p}{=}\PYG{+w}{ }\PYG{n}{scalfact2}\PYG{p}{(}\PYG{n}{ex}\PYG{p}{,}\PYG{+w}{ }\PYG{n}{ey}\PYG{p}{,}\PYG{+w}{ }\PYG{n}{ed}\PYG{p}{)}
\PYG{p}{[}\PYG{n}{sfac}\PYG{p}{]}\PYG{+w}{ }\PYG{p}{=}\PYG{+w}{ }\PYG{n}{scalfact2}\PYG{p}{(}\PYG{n}{ex}\PYG{p}{,}\PYG{+w}{ }\PYG{n}{ey}\PYG{p}{,}\PYG{+w}{ }\PYG{n}{ed}\PYG{p}{,}\PYG{+w}{ }\PYG{n+nb}{rat}\PYG{p}{)}
\end{sphinxVerbatim}

\sphinxlineitem{Description}
\sphinxAtStartPar
\sphinxcode{\sphinxupquote{scalfact2()}} determines a scale factor \(sfac\) for drawing computational results, such as displacements, section forces, or flux.

\sphinxAtStartPar
Input variables are the coordinate matrices \(ex\) and \(ey\), and the matrix \(ed\) containing the quantity to be displayed.
The scalar \(rat\) defines the ratio between the geometric representation of the largest quantity to be displayed and the element size.
If \(rat\) is not specified, \(0.2\) is used.

\sphinxlineitem{Theory}
\sphinxAtStartPar
The scale factor \(sfac\) is computed so that the largest value in \(ed\) is represented as a fraction \(rat\) of the element size, ensuring a visually appropriate scaling of computational results.

\end{description}\end{quote}


\subsection{scalgraph2}
\label{\detokenize{graphics_functions:scalgraph2}}\begin{quote}\begin{description}
\sphinxlineitem{Purpose}
\sphinxAtStartPar
Draw a graphic scale.

\sphinxlineitem{Syntax}
\begin{sphinxVerbatim}[commandchars=\\\{\}]
\PYG{n}{scalgraph2}\PYG{p}{(}\PYG{n}{sfac}\PYG{p}{,}\PYG{+w}{ }\PYG{n}{magnitude}\PYG{p}{)}
\PYG{n}{scalgraph2}\PYG{p}{(}\PYG{n}{sfac}\PYG{p}{,}\PYG{+w}{ }\PYG{n}{magnitude}\PYG{p}{,}\PYG{+w}{ }\PYG{n}{plotpar}\PYG{p}{)}
\end{sphinxVerbatim}

\sphinxlineitem{Description}
\sphinxAtStartPar
\sphinxcode{\sphinxupquote{scalgraph2}} draws a graphic scale to visualize the magnitude of displayed computational results. The input variable \(\mathit{sfac}\) is a scale factor determined by the function \sphinxcode{\sphinxupquote{scalfact2}}. The variable \(\mathit{magnitude}\) is defined as \([S\;\;x\;\;y]\), where \(S\) specifies the value corresponding to the length of the graphic scale, and \((x, y)\) are the coordinates of the starting point. If no coordinates are given, the starting point will be \((0, -0.5)\).

\sphinxlineitem{Theory}
\sphinxAtStartPar
The variable \sphinxcode{\sphinxupquote{plotpar}} sets the graphic scale color:

\sphinxAtStartPar
\(\mathrm{plotpar} = [\mathrm{color}]\)

\sphinxAtStartPar
where


\begin{savenotes}\sphinxattablestart
\sphinxthistablewithglobalstyle
\centering
\begin{tabular}[t]{\X{10}{30}\X{20}{30}}
\sphinxtoprule
\sphinxtableatstartofbodyhook
\sphinxAtStartPar
\(\mathrm{color} = 1\)
&
\sphinxAtStartPar
black
\\
\sphinxhline
\sphinxAtStartPar
2
&
\sphinxAtStartPar
blue
\\
\sphinxhline
\sphinxAtStartPar
3
&
\sphinxAtStartPar
magenta
\\
\sphinxhline
\sphinxAtStartPar
4
&
\sphinxAtStartPar
red
\\
\sphinxbottomrule
\end{tabular}
\sphinxtableafterendhook\par
\sphinxattableend\end{savenotes}

\end{description}\end{quote}


\subsection{secforce2}
\label{\detokenize{graphics_functions:secforce2}}\begin{quote}\begin{description}
\sphinxlineitem{Purpose}
\sphinxAtStartPar
Draw the section force diagrams of a two dimensional bar or beam element in its global position.

\sphinxlineitem{Syntax}
\begin{sphinxVerbatim}[commandchars=\\\{\}]
\PYG{n}{secforce2}\PYG{p}{(}\PYG{n}{ex}\PYG{p}{,}\PYG{+w}{ }\PYG{n}{ey}\PYG{p}{,}\PYG{+w}{ }\PYG{n}{es}\PYG{p}{,}\PYG{+w}{ }\PYG{n}{plotpar}\PYG{p}{,}\PYG{+w}{ }\PYG{n}{sfac}\PYG{p}{)}
\PYG{n}{secforce2}\PYG{p}{(}\PYG{n}{ex}\PYG{p}{,}\PYG{+w}{ }\PYG{n}{ey}\PYG{p}{,}\PYG{+w}{ }\PYG{n}{es}\PYG{p}{,}\PYG{+w}{ }\PYG{n}{plotpar}\PYG{p}{,}\PYG{+w}{ }\PYG{n}{sfac}\PYG{p}{,}\PYG{+w}{ }\PYG{n}{eci}\PYG{p}{)}
\PYG{p}{[}\PYG{n}{sfac}\PYG{p}{]}\PYG{+w}{ }\PYG{p}{=}\PYG{+w}{ }\PYG{n}{secforce2}\PYG{p}{(}\PYG{n}{ex}\PYG{p}{,}\PYG{+w}{ }\PYG{n}{ey}\PYG{p}{,}\PYG{+w}{ }\PYG{n}{es}\PYG{p}{)}
\PYG{p}{[}\PYG{n}{sfac}\PYG{p}{]}\PYG{+w}{ }\PYG{p}{=}\PYG{+w}{ }\PYG{n}{secforce2}\PYG{p}{(}\PYG{n}{ex}\PYG{p}{,}\PYG{+w}{ }\PYG{n}{ey}\PYG{p}{,}\PYG{+w}{ }\PYG{n}{es}\PYG{p}{,}\PYG{+w}{ }\PYG{n}{plotpar}\PYG{p}{)}
\end{sphinxVerbatim}

\sphinxlineitem{Description}
\sphinxAtStartPar
The input variables \(\mathtt{ex}\) and \(\mathtt{ey}\) are defined in \(\mathtt{bar2e}\) or \(\mathtt{beam2e}\). The input variable
\begin{equation*}
\begin{split}\mathtt{es} = \begin{bmatrix} S_1 \\ S_2 \\ \vdots \\ S_n \end{bmatrix}\end{split}
\end{equation*}
\sphinxAtStartPar
consists of a column matrix that contains section forces. The values in \(\mathtt{es}\) are computed in, e.g., \(\mathtt{bar2s}\) or \(\mathtt{beam2s}\).

\sphinxAtStartPar
The variable \(\mathtt{plotpar}\) sets plot parameters for the diagram:
\begin{equation*}
\begin{split}\mathtt{plotpar} = [\, \text{linecolor} \;\; \text{elementcolor} \,]\end{split}
\end{equation*}
\sphinxAtStartPar
where


\begin{savenotes}\sphinxattablestart
\sphinxthistablewithglobalstyle
\centering
\begin{tabular}[t]{\X{20}{100}\X{30}{100}\X{20}{100}\X{30}{100}}
\sphinxtoprule
\sphinxstyletheadfamily 
\sphinxAtStartPar
linecolor
&\sphinxstyletheadfamily 
\sphinxAtStartPar
color
&\sphinxstyletheadfamily 
\sphinxAtStartPar
elementcolor
&\sphinxstyletheadfamily 
\sphinxAtStartPar
color
\\
\sphinxmidrule
\sphinxtableatstartofbodyhook
\sphinxAtStartPar
1
&
\sphinxAtStartPar
black
&
\sphinxAtStartPar
1
&
\sphinxAtStartPar
black
\\
\sphinxhline
\sphinxAtStartPar
2
&
\sphinxAtStartPar
blue
&
\sphinxAtStartPar
2
&
\sphinxAtStartPar
blue
\\
\sphinxhline
\sphinxAtStartPar
3
&
\sphinxAtStartPar
magenta
&
\sphinxAtStartPar
3
&
\sphinxAtStartPar
magenta
\\
\sphinxhline
\sphinxAtStartPar
4
&
\sphinxAtStartPar
red
&
\sphinxAtStartPar
4
&
\sphinxAtStartPar
red
\\
\sphinxbottomrule
\end{tabular}
\sphinxtableafterendhook\par
\sphinxattableend\end{savenotes}

\sphinxAtStartPar
The scale factor \(\mathtt{sfac}\) is a scalar that the section forces are multiplied with to get a suitable graphical representation. If \(\mathtt{sfac}\) is omitted in the input list, the scale factor is set automatically.

\sphinxAtStartPar
The input variable
\begin{equation*}
\begin{split}\mathtt{eci} = \begin{bmatrix} \bar{x}_1 \\ \bar{x}_2 \\ \vdots \\ \bar{x}_n \end{bmatrix}\end{split}
\end{equation*}
\sphinxAtStartPar
specifies the local \(\bar{x}\)\sphinxhyphen{}coordinates of the quantities in \(\mathtt{es}\). If \(\mathtt{eci}\) is not given, uniform distance is assumed.

\end{description}\end{quote}

\sphinxstepscope


\chapter{Examples}
\label{\detokenize{examples:examples}}\label{\detokenize{examples::doc}}


\renewcommand{\indexname}{Index}
\printindex
\end{document}