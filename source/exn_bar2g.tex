\newpage
\lhead[\bf exn\_bar2g]{\bf Nonlinear analysis}
\rhead[\bf Nonlinear analysis]{\bf exn\_bar2g}
\begin{tagx}
\tag{Purpose:} Plane truss considering geometric nonlinearity.
\tag{Description:} Consider a plane truss consisting of two bars with the properties
$E=200$ GPa, $A_1=6.0\cdot10^{-4}$ m$^2$ and $A_2=3.0\cdot10^{-4}$
m$^2$. The truss is loaded by a force $P=10$ MN to the left and a force $F=0.2$ MN downwards. 
The corresponding finite element model consists
of two elements and six degrees of freedom.

\begin{figure}[h]
\centerline{\bild[140mm]{exn1.eps}}
\end{figure}

The element property vectors \textsf{ep1} and \textsf{ep2}
and the element coordinate vectors \textsf{ex1}, \textsf{ex2}, \textsf{ey1}, and \textsf{ey2}
are defined. Initial values are given to the variables axial forces \textsf{QX1} and \textsf{QX2}. The element stiffness matrices \textsf{Ke1} and \textsf{Ke2} are computed using \textsf{bar2ge}.

The computation is initialised by defining the topology matrix \textsf{Edof},
containing element numbers and global element degrees of freedom. The element property vectors \textsf{ep1} and \textsf{ep2}
and the element coordinate vectors \textsf{ex1}, \textsf{ex2}, \textsf{ey1}, and \textsf{ey2}
are also defined. 
\begin{verbatim}
>> Edof=[1  1  2  5  6;
>>       2  3  4  5  6]; 
>> E=10e9;  
>> A1=4e-2;     A2=1e-2;
>> ep1=[E A1];  ep2=[E A2];
>> ex1=[0 1.6]; ey1=[0 0];
>> ex2=[0 1.6]; ey2=[1.2 0];
\end{verbatim}

The bar element function considering geometric nonlinearity {\sf bar2ge} requires the value
axial force $Q_{\bar{x}}$. Since the axial forces are a result of the computation the computation procedure is iterative. Initially, the axial forces are set to zero, i.e.\ $Q_{\bar{x}}^{(1)}=0$ and
$Q_{\bar{x}}^{(2)}=0$ which are stored in {\sf QX1} and {\sf QX2}. This means that the first iteration is equivalent to a linear analysis using {\sf bar2e}. To make sure that the first iteration is performed the scalar used for storing the previous axial force in element 1 {\sf QX01} is set to 1.
To avoid dividing by 0 in the second convergence check, a nonzero but small value is assumed for the initial axial force in Element 1, i.e.\ $Q_{\bar{x}, 0}^{(1)}=0.0001$.
In each iteration the axial forces {\sf QX1} and {\sf QX2} are updated according to the computational result. The iterations continue until the difference in axial force {\sf QX1} of the two latest iterations is less than an accepted error {\sf eps} chosen as $1.0\cdot 10^{-6}$ ({\sf QX1}$-${\sf QX01})/{\sf QX01} $<$ {\sf eps}.  

\begin{verbatim}
>> QX1=0.0001;  QX2=0;
>> QX01=1;
>> eps=1e-6;		
>> n=0;			

>> while(abs((QX1-QX01)/QX01)>eps)
\end{verbatim}

In each iteration the global stiffness matrix {\sf K} (6$\times$6) 
and the load vector {\sf f} (6$\times$1) is initially filled with zeros. The nodal loads of $10.0$ MN and $0.2$ MN acting at lower right corner of the frame are placed in position 5 and 6 of the load vector, respectively. 
Element stiffness matrices are computed by {\sf bar2ge} and assembled using {\sf assem}, after which the system of equations is solved using {\sf solveq}. Based on the computed displacements {\sf a}, new values of section forces and axial forces are computed by {\sf beam2gs}. If {\sf QX1} does not converge in 20 iterations the analysis is interrupted.

\begin{verbatim}
>>   n=n+1
>>   K=zeros(6,6);
>>   f=zeros(6,1);	
>>   f(5)=-10e6;
>>   f(6)=-0.2e6;
>>   
>>   Ke1=bar2ge(ex1,ey1,ep1,QX1);
>>   Ke2=bar2ge(ex2,ey2,ep2,QX2);
>>   K=assem(Edof(1,:),K,Ke1);
>>   K=assem(Edof(2,:),K,Ke2);
>>   bc=[1 0;2 0;3 0;4 0];	
>>   [a,r]=solveq(K,f,bc)
>>
>>   Ed=extract_ed(Edof,a);
>>
>>   QX01=QX1; 
>>   [es1,QX1]=bar2gs(ex1,ey1,ep1,Ed(1,:))
>>   [es2,QX2]=bar2gs(ex2,ey2,ep2,Ed(2,:))
>>   
>>   if(n>20)
>>      disp('The solution does not converge')
>>      break
>>   end
>> end
\end{verbatim}

After 7 iterations the computation has converged and the axial forces are

\begin{verbatim}
QX1 =

  -1.1136e+07

QX2 =

   1.4833e+06
	
\end{verbatim}	

The displacements according to the linear analysis and the analysis 
considering geometric nonlinearity are respectively:

\begin{verbatim}
a =                         a =

         0                           0
         0                           0
         0                           0
         0                           0
   -0.0411                     -0.0445
   -0.0659                     -0.1088
\end{verbatim}

the vertical displacement at the node to the right is 108.8 mm, 
which is 1.6 times larger than the result from a linear computation 
according to the first iteration. 
The axial force in Element 2 is 1.483 kN, which is 4.5 times larger than 
the value obtained in the linear computation.

\end{tagx}

\clearpage
